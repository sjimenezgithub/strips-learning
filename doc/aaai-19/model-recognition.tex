\def\year{2017}\relax
%File: formatting-instruction.tex
\documentclass[letterpaper]{article} %DO NOT CHANGE THIS
\usepackage{aaai18}  %Required
\usepackage{times}  %Required
\usepackage{helvet}  %Required
\usepackage{courier}  %Required
\usepackage{url}  %Required
\usepackage{graphicx}  %Required
\frenchspacing  %Required
\setlength{\pdfpagewidth}{8.5in}  %Required
\setlength{\pdfpageheight}{11in}  %Required
%PDF Info Is Required:
  \pdfinfo{}
\setcounter{secnumdepth}{0}

\usepackage{amsmath}
\usepackage{amssymb}
\usepackage{amsthm}
\usepackage{multirow}
\usepackage{tikz}
\usepackage{comment}

\usepackage{graphicx}
\usepackage{caption}
\usepackage{subcaption}
\usepackage{listings}

\lstset{
  basicstyle=\ttfamily,
  mathescape
}


\usepackage{multicol}
\usepackage{arydshln}
\usetikzlibrary{calc,backgrounds,positioning,fit}


\newcommand{\tup}[1]{{\langle #1 \rangle}}

\newcommand{\pre}{\mathsf{pre}}     % precondition
\newcommand{\del}{\mathsf{del}}     % effect
\newcommand{\add}{\mathsf{add}}     % effect
\newcommand{\eff}{\mathsf{eff}}     % effect
\newcommand{\cond}{\mathsf{cond}}   % conditional effect
\newcommand{\true}{\mathsf{true}}   % true
\newcommand{\false}{\mathsf{false}} % false
\newcommand{\PE}{\mathrm{PE}}     % precondition
\newcommand{\strips}{\textsc{Strips}}     % precondition


\newtheorem{theorem}{Theorem}
\newtheorem{lemma}[theorem]{Lemma}
\newtheorem{definition}[theorem]{Definition}


\begin{document}

\title{Model Recognition as Planning}

% Commented for blind submission
\author{Diego Aineto\and Sergio Jim\'enez\and Eva Onaindia \and Miquel Ram\'irez\\
{\small Departamento de Sistemas Inform\'aticos y Computaci\'on}\\
{\small Universitat Polit\`ecnica de Val\`encia.}\\
{\small Camino de Vera s/n. 46022 Valencia, Spain}\\
{\small \{dieaigar,serjice,onaindia\}@dsic.upv.es}}



\maketitle
\begin{abstract} 
Given a partially observed plan execution and a set of possible planning models, {\em model recognition} is the task of identifying the model in the set that produced the observed plan execution. The paper formalizes the {\em model recognition} task and proposes a novel method to assess the probability of a \strips\ model to produce a given partially observed plan execution. This method, that we called {\em model recognition as planning}, is robust to missing data in the intermediate states and actions of the observed plan execution besides, it is computable with an off-the-shelf classical planner. The effectiveness of {\em model recognition as planning} is shown in a set of \strips\ models encoding different {\em Turing Machines}. We show that {\em model recognition as planning} succeeds to identify the executed {\em Turing Machine} despite the actual applied transitions, internal machine state or the values of several tape cells, are unknown.
\end{abstract}


\section{Introduction}
\label{sec:introduction}
{\em Plan recognition} is the task of predicting the future actions of an agent provided observations of its current behaviour. Plan recognition is considered {\em automated planning} in reverse; while automated planning aims to compute a sequence of actions that accounts for a given goals, plan recognition aims to compute the goals that account for an observed sequence of actions~\cite{geffner:book:2013}.

Diverse approaches has been proposed for plan recognition such as {\em rule-based systems}, {\em parsing}, {\em graph-covering}, {\em Bayesian nets}, etc~\cite{carberry2001techniques}. {\em Plan recognition as planning} is the model-based approach for plan recognition~\cite{ramirez2012plan,ramirez2009plan}. This approach assumes that the action model of the observed agent is known and leverages it to compute the most likely goal of the agent according to the observed plan execution.

This paper introduces the task of {\em model recognition}.  Given a partially observed plan execution and a set of possible planning models, {\em model recognition} is the task of identifying which model in the set has the highest probability of producing the input observation. {\em Model recognition} is of interest because:
\begin{itemize}
\item Once the planning model is recognized with certainty, then the model-based machinery for automated planning becomes applicable~\cite{ghallab2004automated}.
\item It enables the recognition of algorithms by observing their execution given that diverse algorithm representations, like {\em {\sc GOLOG} programs}, {\em finite state controllers} or {\em push-down automata} can be encoded as classical planning models~\cite{baier2007exploiting,Geffner:FSM:AAAI10,segovia2017generating}.
\end{itemize}

The paper introduces also {\em model recognition as planning}; a novel method to assess the probability of a given \strips\ model to produce an observed plan execution. The method is robust to missing data in the intermediate states and actions of the observed plan execution besides, it is computable with an off-the-shelf classical planner. The paper evaluates the effectiveness of {\em model recognition as planning} with a set of \strips\ models that represent different {\em Turing Machines} (all of them defined within the same {\em tape alphabet} and same {\em machine states} but different {\em transition functions}). We show that {\em model recognition as planning} succeeds to identify the executed {\em Turing Machine} despite the actual applied transitions, the internal machine state or the values of several tape cells, are unknown.


\section{Background}
\label{sec:background}
This section formalizes classical planning and the observation of the execution of a classical plan.

\subsection{Classical planning}
We use $F$ to denote the set of {\em fluents} (propositional variables) describing a state. A {\em literal} $l$ is a valuation of a fluent $f\in F$; i.e. either~$l=f$ or $l=\neg f$. A set of literals $L$ represents a partial assignment of values to fluents (without loss of generality, we will assume that $L$ does not contain conflicting values). We use $\mathcal{L}(F)$ to denote the set of all literal sets on $F$; i.e.~all partial assignments of values to fluents.

A {\em state} $s$ is a full assignment of values to fluents; $|s|=|F|$, so the size of the state space is $2^{|F|}$. Explicitly including negative literals $\neg f$ in states simplifies subsequent definitions but often we will abuse of notation by defining a state $s$ only in terms of the fluents that are true in $s$, as it is common in \strips\ planning.

A {\em classical planning frame} is a tuple $\Phi=\tup{F,A}$, where $F$ is a set of fluents and $A$ is a set of actions. The dynamics of the actions in $A$ are specified with two functions: $App(s)$ that denotes the subset of actions applicable in a state $s$ and $\theta(s,a)$ that denotes the {\em successor state} that results of applying $a$ in $s$.

A {\em classical planning problem} is a tuple $P=\tup{F,A,I,G}$, where $I$ is an initial state and $G\subseteq\mathcal{L}(F)$ is a goal condition. A {\em plan} for $P$ is an action sequence $\pi=\tup{a_1, \ldots, a_n}$ that, when executed starting from the initial state $I$, induces the {\em state trajectory} $\tup{s_0, s_1, \ldots, s_n}$ such that $s_0=I$ and $a_i$ ({\small $1\leq i\leq n$}) is applicable in $s_{i-1}$ and generates the successor state $s_i=\theta(s_{i-1},a_i)$. The {\em plan length} is denoted with $|\pi|=n$.

A plan $\pi$ {\em solves} $P$ iff $G\subseteq s_n$; i.e.~if the goal condition is satisfied in the last state resulting from the execution of the plan $\pi$ in the initial state $I$. A solution plan is {\em optimal} if it has minimum length.

\subsection{The observation model}
Given classical planning problem $P=\tup{F,A,I,G}$, the observation of the execution of a plan $\pi$ that solves $P$ is an interleaved combination $\tau=\tup{s_1, a_1, \ldots , a_n, s_m}$ of a {\em partially-observed} sequence of actions and a sequence of {\em partially-observed} states. Because of the partial observability $0\leq n\leq |\pi|$ and $0\leq m\leq |\pi|+1$ and hence, the transitions between two consecutive observed states in $\tau$ may involve the execution of more than a single action. Formally $\theta(s_i,\tup{a_1,\ldots,a_k})=s_{i+1}$, where $k\geq 1$ is unknown and unbound. This means that having $\tau$ does not implies knowing the actual length of $\pi$.

We say that a given observation $\tau$ is {\em compliant} with a given plan $\pi$ iff:
\begin{itemize}
\item The sequence of actions $\tup{a_1, \ldots, a_n}$ in $\tau$ is a subsequence of $\pi$.
\item The sequence of states $\tup{s_1, \ldots, s_m}$ in $\tau$ is the same sequence of states traversed by $\pi$ but (1), with certain states omitted and/or (2), the value of certain fluents omitted in that states, i.e.~$|s_i|\leq |F|$ for every $0\leq i\leq m$.
\end{itemize}


\section{Model Recognition}
\label{sec:recognition}
The {\em model recognition} task is a tuple $\tup{P,M,\tau}$ where:
\begin{itemize}
\item $P=\tup{F,A,I,G}$ is a classical planning problem such that the dynamics of the actions in $A$ is unknown because the $App(s)$ and $\theta(s,a)$ functions are undefined.
\item $M=\{\mathcal{M}_1,\ldots,\mathcal{M}_m\}$ is a finite and non-empty set of action models, each defining a different dynamics (a different $App(s)$ and $\theta(s,a)$ pair) for the actions in $A$.
\item $\tau$ is the observation of the execution of a solution plan $\pi$ for $P$.
\end{itemize}
A solution to the {\em model recognition} task is the discrete probability distribution $P(\mathcal{M}|\tau)$ that expresses, for each model $\mathcal{M}\in M$, its probability of producing the observation $\tau$.

According to the {\em Bayes} rule, the probability of an hypothesis $\mathcal{H}$, provided the observation $\mathcal{O}$, is given by the expression $P(\mathcal{H}|\mathcal{O})=\frac{P(\mathcal{O}|\mathcal{H})P(\mathcal{H})}{P(\mathcal{O})}$. In the {\em model recognition} task, hypotheses are about the possible action models $\mathcal{M}\in M$ while the given observation is the partially observed plan execution $\tau$. The $P(\mathcal{M}|\tau)$ probability distribution can then be estimated in three steps:
\begin{enumerate}
\item Computing the {\em a priori} probabilities. $P(\tau)$, that measures how surprising is the given observation and $P(\mathcal{M})$, that expresses if one model is a priori more likely than the others. 
\item Computing the conditional probability $P(\tau|\mathcal{M})$.  Our approach is to estimate this value according to the minimum amount of {\em edition} that is required by $\mathcal{M}$ to produce a plan $\pi^*_\tau$ such that:
\begin{enumerate}
\item $\pi^*_\tau$ is an optimal solution for the classical planning problem $P$. 
\item $\pi^*_\tau$ is compliant with the given observation $\tau$.
\end{enumerate}
\item Applying the Bayes rule to obtain the normalized posterior probabilities (the $P(\mathcal{M}|\tau)$ probabilities must sum 1 for all the $\mathcal{M}\in M$).
\end{enumerate}

Note that the second step assumes that the observed agent is acting rationally, like in {\em plan recognition as planning}~\cite{ramirez2012plan,ramirez2009plan}. A related approach is recently followed for {\em model reconciliation}~\cite{Kambhampati:mreconciliation:ijcai17} where model edition is used to conform the different PDDL models of two different agents. 


\section{Recognition of \strips\ models}
\label{sec:asPlanning}
Here we analyze the particular instantiation of the {\em model recognition} task where the dynamics of the actions $A\in P$ (i.e. the $App(s)$ and $\theta(s,a)$ functions) are specified with \strips\ action schemas. Figure~\ref{fig:stack} shows the PDDL code for the {\em stack} actions, taken from a four-operator {\em blocksworld}~\cite{slaney2001blocks}. 

\subsection{The space of \strips\ models}
Like in PDDL~\cite{mcdermott1998pddl,fox2003pddl2}, we assume that fluents $F$ are instantiated from a set of {\em predicates} $\Psi$. Each predicate $p\in\Psi$ has an argument list of arity $ar(p)$. Given a set of {\em objects} $\Omega$, the set of fluents $F$ is induced by assigning objects in $\Omega$ to the arguments of predicates in $\Psi$, i.e.~$F=\{p(\omega):p\in\Psi,\omega\in\Omega^{ar(p)}\}$ s.t. $\Omega^k$ is the $k$-th Cartesian power of $\Omega$. 
\begin{figure}
\begin{scriptsize}
\begin{verbatim}
(:action stack
 :parameters (?v1 ?v2 - object)
 :precondition (and (holding ?v1) (clear ?v2))
 :effect (and (not (holding ?v1)) (not (clear ?v2))
              (handempty) (clear ?v1) (on ?v1 ?v2)))
\end{verbatim}
\end{scriptsize}
 \caption{\small \strips\ action schema coded in PDDL.}
\label{fig:stack}
\end{figure}

Let $\Omega_v=\{v_i\}_{i=1}^{\operatorname*{max}_{a\in A} ar(a)}$ be an additional set of objects ($\Omega\cap\Omega_v=\emptyset$), that we denote as {\em variable names}, and that is bound by the maximum arity of an action in a given planning frame. For instance, in a three-block four-operator {\em blocksworld} $\Omega=\{block_1, block_2, block_3\}$ while $\Omega_v=\{v_1, v_2\}$ because the {\small\tt stack} and {\small\tt unstack} actions have arity two. We define $F_v$, a new set of fluents, $F\cap F_v=\emptyset$, produced instantiating $\Psi$ using only {\em variable names}. This set contains eleven elements for the mentioned {\em blocksworld}, $F_v$={\small\tt\{handempty, holding($v_1$), holding($v_2$), clear($v_1$), clear($v_2$), ontable($v_1$), ontable($v_2$), on($v_1,v_1$), on($v_1,v_2$), on($v_2,v_1$), on($v_2,v_2$)\}}.

For a given action schema $\xi$, then $F_{\xi}\subseteq F_v$ defines the subset of elements that can appear in the preconditions and effects of that action schema. For instance $F_{\tt stack}=F_v$ while $F_{\tt pickup}$={\small\tt\{handempty, holding($v_1$), clear($v_1$), ontable($v_1$), on($v_1,v_1$)\}} excludes any element in $F_v$ that involves $v_2$ because {\small\tt pickup} actions have arity one. 

Now we are ready to define the fluents $pre_f(\xi)$, $del_f(\xi)$ and $add_f(\xi)$, for every $f\in F_{\xi}$, that represent the propositional encoding for the preconditions, negative and positive effects of an action schema $\xi$. If a fluent $pre_f(\xi)/del_f(\xi)/add_f(\xi)$ holds, it means that $f$ is a precondition/negative/positive effect in that schema. For instance, Figure~\ref{fig:encodedstack} shows the conjunction of fluents that represents he propositional encoding for the preconditions, negative and positive effects of the {\em stack} schema shown in Figure~\ref{fig:stack}. 

\begin{figure}
\begin{scriptsize}
\begin{verbatim}
(pre_holding_stack_v1) (pre_clear_stack_v2)
(del_holding_stack_v1) (del_clear_stack_v2)
(add_handempty_stack) (add_clear_stack_v1) 
(add_on_stack_v1_v2)
\end{verbatim}
\end{scriptsize}
 \caption{\small Propositional encoding for the {\em stack} schema from a four-operator {\em blocksworld}.}
\label{fig:encodedstack}
\end{figure}

The size of the space of possible \strips\ models for an action schema $\xi$ is given by the expression, $2^{2|F_{\xi}|}$ (\strips\ constraints require negative effects appearing as preconditions and that they cannot be positive effects and also, that positive effects cannot appear as preconditions). For the mentioned {\em blocksworld}, $2^{2|F_{\tt stack}|}=4194304$ while $2^{2|F_{\tt pickup}|}=1024$. We say that two \strips\ schemes $\xi$ and $\xi'$ are {\em comparable} iff both share the same space of possible \strips\ models. For instance, we claim that blocksworld schemas {\small\tt stack} and {\small\tt unstack} are {\em comparable} while  {\small\tt stack} and {\small\tt pickup} are not. Last but not least, two \strips\ models $\mathcal{M}$ and $\mathcal{M}'$ are {\em comparable} iff there exists a bijective function $\mathcal{M} \mapsto \mathcal{M}^*$ that maps every action schema $\xi\in\mathcal{M}$ to a comparable schema $\xi'\in\mathcal{M'}$ and vice versa.

\subsection{The \strips\ edit distance}
We define two edit \emph{operations} on a \strips\ model $\mathcal{M}\in M$:
\begin{itemize}
\item {\em Deletion}. A fluent $pre_f(\xi)/del_f(\xi)/add_f(\xi)$ is removed from the operator schema $\xi\in\mathcal{M}$, such that $f\in F_{\xi}$.
\item {\em Insertion}. A fluent $pre_f(\xi)/del_f(\xi)/add_f(\xi)$ is added to the operator schema $\xi\in\mathcal{M}$, s.t. $f\in F_{\xi}$.
\end{itemize}

We can now formalize an {\em edit distance} that quantifies how similar two given \strips\ action models are. The distance is symmetric and meets the {\em metric axioms} provided that the two {\em edit operations}, deletion and insertion, have the same positive cost.

\begin{definition}
  Let $\mathcal{M}$ and $\mathcal{M}'$ be two {\em comparable} \strips\ action models. The {\bf edit distance} $\delta(\mathcal{M},\mathcal{M}')$ is the minimum number of {\em edit operations} that is required to transform $\mathcal{M}$ into $\mathcal{M}'$.
\end{definition}

Since $F_v$ is a bound set, the maximum number of edits that can be introduced to a given action model defined within $F_v$ is bound as well. 
\begin{definition}
The \textbf{maximum edit distance} of an \strips\ model $\mathcal{M}$ built from the set of possible elements $F_v$ is $\delta(\mathcal{M},*)=\sum_{\xi\in\mathcal{M}} 3\times|F_{\xi}|$.
\end{definition}

We define now an edit distance to asses the matching of a \strips\ model with respect to an observation of a plan execution. 

\begin{definition}
  Given $\tau$, an observation of the execution of a plan for solving $P$ and $\mathcal{M}$, a \strips\ action model built from $F_v$. The {\bf observation edit distance}, $\delta(\mathcal{M},\tau)$, is the minimal edit distance from $\mathcal{M}$ to any {\em comparable} model $\mathcal{M}'$ s.t. $\mathcal{M}'$ produces a plan $\pi^*_\tau$ optimal for $P$ and compliant with $\tau$; \[\delta(\mathcal{M},\tau)=\min_{\forall \mathcal{M}' \rightarrow \tau} \delta(\mathcal{M},\mathcal{M}')\]
\end{definition}

Our $\delta(\mathcal{M},\tau)$ distance could also be defined assessing the edition required by the observed plan execution to match the given model. This implies defining {\em edit operations} that modify $\tau$ instead of $\mathcal{M}$~\cite{sohrabi:precognition:IJCAI2016}. Our definition of the {\em observation edit distance} is more practical since normally $F_v$ is smaller than $F$. In practice, the number of {\em variable objects} is normally smaller than the number of objects in a planning problem.

\subsection{The $P(\mathcal{M}|\tau)$ probability for \strips\ models}
Since we assume that the dynamics of the actions $A\in P$ is specified with \strips\ action schemas, we can estimate the $P(\mathcal{M}|\tau)$ probability as follows:
\begin{enumerate}
\item Assuming that a priori, all action models $\mathcal{M}\in M$ are equiprobable because there are no reasons to assume that one is a priori more likely than the others. Therefore, $P(\mathcal{M})=\frac{1}{\prod_{\xi\in\mathcal{M}} 2^{2|F_{\xi}|}}$.
\item Assuming that all the observations of plan executions with a maximum of $n$ observed actions and $m$ observed states are equiprobable then $P(\tau)=\frac{1}{2^{n\times|A|}\times2^{m\times|F|}}$.
\item To estimate the conditional probability $P(\tau|\mathcal{M})$ we compute the {\em observation edit distance} $\delta(\mathcal{M},\tau)$ and map it into a $[0,1]$ likelihood with the expression, $1-\frac{\delta(\mathcal{M},\tau)}{\delta(\mathcal{M},*)}$.
\end{enumerate}



\section{Model recognition as planning}
This section shows that, when the dynamics of the actions $A\in P$ are specified with \strips\ action schemas, then $\delta(\mathcal{M},\tau)$ can be computed with a compilation of a classical planning with conditional effects. The intuition behind this compilation is that a solution to the resulting classical planning task is a sequence of actions that:
\begin{enumerate}
\item {\bf Edits the action model $\mathcal{M}$ to build $\mathcal{M}'$}. A solution plan starts with a {\em prefix} that modifies the preconditions and effects of the action schemes in $\mathcal{M}$ using to the two {\em edit operations} defined above, {\em deletion} and {\em insertion}. 
\item {\bf Validates the edited model $\mathcal{M}'$ in the observation $\tau$}. The solution plan continues with a postfix that validates the edited model $\mathcal{M}'$ on the given observations $\tau$.
\end{enumerate}

Figure~\ref{fig:plan-pdistance} shows the plan for editing a given {\em blockswold} model where the positive effects {\tt\small (handempty)} and {\tt\small (clear ?v1)} of the {\tt\small stack} schema are missing. The edited action model is validated at the observation of a four action plan for inverting a two-block tower where the intermediate states, $s_1$, $s_2$ and $s_3$, are unobserved. 
\begin{figure}
{\tt\scriptsize
00 : (insert\_add\_handempty\_stack)\\
01 : (insert\_add\_clear\_stack\_var1)\\
02 : (apply\_unstack blockB blockA i1 i2)\\
03 : (apply\_putdown blockB i2 i3)\\
04 : (apply\_pickup blockA i3 i4)\\
05 : (apply\_stack blockA blockB i4 i5)\\
06 : (validate\_1)
}
 \caption{\small Plan for editing (steps [0-1]) and validating (steps [2-6]) a given \strips\ planning model for the {\em blockswold}.}
\label{fig:plan-pdistance}
\end{figure}

Note that our interest is not in $\mathcal{M}'$, the edited model resulting from the compilation, but in the number of required {\em edit operations} (insertions and deletions) required by $\mathcal{M}'$ to be validated in the given observation, e.g. $\delta(\mathcal{M},\tau)=2$ for the example in Figure~\ref{fig:plan-pdistance}. In this case $\delta(\mathcal{M},*)=3\times 2\times (11+5)$ since there are 4 action schemes ({\small\tt pickup}, {\small\tt putdown}, {\small\tt stack} and {\small\tt unstack}) and $|F_v|=|F_{\tt stack}|=|F_{\tt unstack}|=11$ while $|F_{\tt pickup}|=|F_{\tt putdown}|=5$. %The {\em observation edit distance} is exactly computed if the classical planning task resulting from our compilation is optimally solved (according to the number of edit actions); is approximated if it is solved with a satisfying planner; and is a less accurate estimate (but faster to be computed) if the solved task is a relaxation of the classical planning task that results from our compilation~\cite{bonet2001planning}.


\subsection{Conditional effects}
Conditional effects allow us to compactly define our compilation. An action $a\in A$ with conditional effects is defined as a set of {\em preconditions} $\pre(a)\in\mathcal{L}(F)$ and a set of {\em conditional effects} $\cond(a)$. Each conditional effect $C\rhd E\in\cond(a)$ is composed of two sets of literals $C\in\mathcal{L}(F)$, the {\em condition}, and $E\in\mathcal{L}(F)$, the {\em effect}. An action $a\in A$ is {\em applicable} in a state $s$ if and only if $\pre(a)\subseteq s$, and the {\em triggered effects} resulting from the action application are the effects whose conditions hold in $s$:
\[
triggered(s,a)=\bigcup_{C\rhd E\in\cond(a),C\subseteq s} E,
\]

The result of applying action $a$ in state $s$ is the {\em successor} state $\theta(s,a)=\{s\setminus\eff_c^-(s,a))\cup\eff_c^+(s,a)\}$ where $\eff_c^-(s,a)\subseteq triggered(s,a)$ and $\eff_c^+(s,a)\subseteq triggered(s,a)$ are, respectively, the triggered {\em negative} and {\em positive} effects.


\subsection{The compilation formalization}
Given a \strips\ model $\mathcal{M}\in M$ and the observation $\tau$ of the execution of a plan for solving $P=\tup{F,A,I,G}$, our compilation outputs a classical planning task $P'=\tup{F',A',I',G'}$ such that:
\begin{itemize}
\item $F'$ contains:
\begin{itemize}
\item The original fluents $F$. 
\item Fluents $pre_f(\xi)/del_f(\xi)/add_f(\xi)$ modeling the space of \strips\ models. 
\item The fluents $F_{\pi}=\{plan(name(a_i),\Omega^{ar(a_i)},i)\}_{\small 1\leq i\leq n}$ to code the $i^{th}$ action in $\tau$. The static facts $next_{i,i+1}$ and the fluents $at_i$, {\small $1\leq i< n$}, are also added to iterate through the $n$ steps of $\tau$.
\item The fluents $\{test_j\}_{1\leq j\leq m}$, indicating the state observation $s_j\in\tau$ where the action model is validated.
\item The fluents $mode_{edit}$ and $mode_{val}$ to indicate whether the operator schemas are edited or validated.
\end{itemize}
\item $I'$ extends the original initial state $I$ with the fluent $mode_{edit}$ set to true as well as the fluents $F_{\pi}$ plus fluents $at_1$ and $\{next_{i,i+1}\}$, {\small $1\leq i<n$}, for tracking the plan step where the action model is validated. Our compilation assumes that initially $\mathcal{M}'$ is defined as $\mathcal{M}$. Therefore fluents $pre_f(\xi)/del_f(\xi)/add_f(\xi)$ hold as given by $\mathcal{M}$.

\item $G'=G\bigcup\{at_n,test_m\}$.
\item $A'$ comprises three kinds of actions with conditional effects:
\begin{enumerate}
\item Actions for {\em editing} operator schema $\xi\in\mathcal{M}$:
\begin{itemize}
\item Actions for adding a {\em precondition} $f\in F_v(\xi)$ from the action schema $\xi\in\mathcal{M}$.

\begin{small}
\begin{align*}
\hspace*{7pt}\pre(\mathsf{programPre_{f,\xi}})=&\{\neg pre_{f}(\xi), \neg del_{f}(\xi),\neg add_{f}(\xi), mode_{edit}\},\\
\cond(\mathsf{programPre_{f,\xi}})=&\{\emptyset\}\rhd\{pre_{f}(\xi)\}.
\end{align*}
\end{small}

\item Actions for adding a {\em negative} or {\em positive} effect $f\in F_v(\xi)$ to the action schema $\xi\in\mathcal{M}$.

\begin{small}
\begin{align*}
\hspace*{7pt}\pre(\mathsf{programEff_{f,\xi}})=&\{\neg del_{f}(\xi),\neg add_{f}(\xi), mode_{edit}\},\\
\cond(\mathsf{programEff_{f,\xi}})=&\{pre_{f}(\xi)\}\rhd\{del_{f}(\xi)\},\\
&\{\neg pre_{f}(\xi)\}\rhd\{add_{f}(\xi)\}.
\end{align*}
\end{small}
\end{itemize}
Besides these actions, the $A'$ set also contains the actions for {\em deleting} a precondition and a negative/positive effect.


\item Actions for {\em applying} an edited operator schema $\xi\in\mathcal{M}$ bound with objects $\omega\subseteq\Omega^{ar(\xi)}$. Since operators headers are given as input, the variables $pars(\xi)$ are bound to the objects in $\omega$ that appear at the same position. Figure~\ref{fig:compilation} shows the PDDL encoding of the action for applying a programmed operator $stack$ from {\em blocksworld}.
\begin{small}
\begin{align*}
\hspace*{7pt}\pre(\mathsf{apply_{\xi,\omega}})=&\{pre_{f}(\xi)\implies p(\omega)\}_{\forall p\in\Psi,f=p(pars(\xi))}\\
&\cup \{\neg mode_{val}\},\\
\cond(\mathsf{apply_{\xi,\omega}})=&\{del_{f}(\xi)\}\rhd\{\neg p(\omega)\}_{\forall p\in\Psi,f=p(pars(\xi))},\\
&\{add_{f}(\xi)\}\rhd\{p(\omega)\}_{\forall p\in\Psi,f=p(pars(\xi))},\\
&\{mode_{edit}\}\rhd\{\neg mode_{edit}\},\\
&\{\emptyset\}\rhd\{mode_{val}\}.
\end{align*}
\end{small}

\begin{figure}
\begin{scriptsize}
\begin{verbatim}
(:action apply_stack
  :parameters (?o1 - object ?o2 - object)
  :precondition
   (and (or (not (pre_on_stack_v1_v1)) (on ?o1 ?o1))
        (or (not (pre_on_stack_v1_v2)) (on ?o1 ?o2))
        (or (not (pre_on_stack_v2_v1)) (on ?o2 ?o1))
        (or (not (pre_on_stack_v2_v2)) (on ?o2 ?o2))
        (or (not (pre_ontable_stack_v1)) (ontable ?o1))
        (or (not (pre_ontable_stack_v2)) (ontable ?o2))
        (or (not (pre_clear_stack_v1)) (clear ?o1))
        (or (not (pre_clear_stack_v2)) (clear ?o2))
        (or (not (pre_holding_stack_v1)) (holding ?o1))
        (or (not (pre_holding_stack_v2)) (holding ?o2))
        (or (not (pre_handempty_stack)) (handempty)))
  :effect
   (and (when (del_on_stack_v1_v1) (not (on ?o1 ?o1)))
        (when (del_on_stack_v1_v2) (not (on ?o1 ?o2)))
        (when (del_on_stack_v2_v1) (not (on ?o2 ?o1)))
        (when (del_on_stack_v2_v2) (not (on ?o2 ?o2)))
        (when (del_ontable_stack_v1) (not (ontable ?o1)))
        (when (del_ontable_stack_v2) (not (ontable ?o2)))
        (when (del_clear_stack_v1) (not (clear ?o1)))
        (when (del_clear_stack_v2) (not (clear ?o2)))
        (when (del_holding_stack_v1) (not (holding ?o1)))
        (when (del_holding_stack_v2) (not (holding ?o2)))
        (when (del_handempty_stack) (not (handempty)))
        (when (add_on_stack_v1_v1) (on ?o1 ?o1))
        (when (add_on_stack_v1_v2) (on ?o1 ?o2))
        (when (add_on_stack_v2_v1) (on ?o2 ?o1))
        (when (add_on_stack_v2_v2) (on ?o2 ?o2))
        (when (add_ontable_stack_v1) (ontable ?o1))
        (when (add_ontable_stack_v2) (ontable ?o2))
        (when (add_clear_stack_v1) (clear ?o1))
        (when (add_clear_stack_v2) (clear ?o2))
        (when (add_holding_stack_v1) (holding ?o1))
        (when (add_holding_stack_v2) (holding ?o2))
        (when (add_handempty_stack) (handempty))
        (when (modeProg) (not (modeProg)))))
\end{verbatim}
\end{scriptsize}
 \caption{\small PDDL action for applying an already programmed schema $stack$ (implications are coded as disjunctions).}
\label{fig:compilation}
\end{figure}
When the observation $\tau$ includes observed actions, then the extra conditional effects $\{at_{i},plan(name(a_i),\Omega^{ar(a_i)},i)\}\rhd\{\neg at_{i},at_{i+1}\}_{\forall i\in [1,n]}$ are included in the $\mathsf{apply_{\xi,\omega}}$ actions to validate that actions are applied, exclusively, in the same order as they appear in $\tau$.\\

\item Actions for {\em validating} the partially observed state $s_j\in\tau$, {\tt\small $1\leq j< m$}.
\begin{small}
\begin{align*}
\hspace*{7pt}\pre(\mathsf{validate_{j}})=&s_j\cup\{test_{j-1}\},\\
\cond(\mathsf{validate_{j}})=&\{\emptyset\}\rhd\{\neg test_{j-1}, test_j,\neg mode_{val}\}.
\end{align*}
\end{small}
\end{enumerate}
\end{itemize}


\subsection{Evaluation}
\label{sec:evaluation}
To evaluate the empirical performance of {\em model recognition as planning} we defined a set of possible \strips\ models, each representing a different {\em Turing Machine}, but all sharing the same set of {\em machine states} and same {\em tape alphabet}. 

\subsection{Modeling {\em Turing Machines} with \strips\ }
A {\em Turing machine} is a tuple $\mathcal{M}=\tup{Q,q_o,Q_{\bot},\Sigma,\Upsilon,\square,\delta}$:
\begin{itemize}
\item $Q$, is a finite and non-empty set of machine states such that $q_0\in Q$ is the initial state of the machine and $Q_{\bot}\subseteq Q$ is the subset of acceptor states.  
\item $\Sigma$ is the {\em tape alphabet}, that is a finite non-empty set of symbols that contains the {\em input alphabet} $\Upsilon\subseteq\Sigma$ (the subset of symbols allowed to initially appear in the tape) and the {\em blank symbol} $\square\in\Upsilon$ (the only symbol allowed to occur on the tape infinitely often).
\item $\delta: \Sigma\times (Q\setminus Q_{\bot}) \rightarrow \Sigma\times Q\times\{left,right\}$ is the {\em transition function}. For each possible pair of tape symbol and non-terminal machine state $\delta$ defines (1), the tape symbol to print at the current position of the header (2), the new state of the machine and (3), whether the header is shifted {\em left} or {\em right} after the print operation. If $\delta$ is not defined for the current pair of tape symbol and machine state, the machine halts.
\end{itemize}

Figure~\ref{tab:tm-anbncn} shows the $\delta$ function of a {\em Turing Machine} for recognizing the $\{a^nb^nc^n : n \geq 1 \}$ language. The {\em tape alphabet} is $\Sigma=\{a,b,c,x,y,z,\square\}$, the {\em input alphabet} $\Upsilon=\{a,b,c,\square\}$ and the possible machine states are $Q=\{q_0,q_1,q_2,q_3,q_4,\underline{q_5}\}$ where \underline{$q_5$} is the only acceptor state.

\begin{figure}
\begin{center}
    \begin{tabular}{| c || c | c | c | c | c | c |}
    \hline
      & $q_0$ & $q_1$ & $q_2$ & $q_3$ & $q_4$ & \underline{$q_5$} \\ \hline\hline
    a & x,r,$q_1$ & a,r,$q_1$ & - &  a,l,$q_3$ & - & - \\ \hline
    b & - & y,r,$q_2$ & b,r,$q_2$ & b,l,$q_3$ & - & - \\ \hline
    c & - & - & z,l,$q_3$ & - & - & - \\ \hline
    x & - & - & - & x,r,$q_0$ & - & - \\ \hline
    y & y,r,$q_4$ & y,r,$q_1$ & - & y,l,$q_3$ & y,r,$q_4$ & - \\ \hline
    z & - & - & z,r,$q_2$ & z,l,$q_3$ & z,r,$q_4$ & - \\\hline
    $\square$ & - & - & - & - & $\square$,r,$q_5$  & - \\                
    \hline
    \end{tabular}
\end{center}
  \caption{\small Seven-symbol six-state {\em Turing Machine} for recognizing the $\{a^nb^nc^n : n \geq 1 \}$ language (\underline{$q_5$} is the only acceptor state).}
  \label{tab:tm-anbncn}
\end{figure}

A classical planning frame $\Phi=\tup{F,A}$ can encode the {\em transition function} $\delta$ of a {\em Turing Machine} $\mathcal{M}$ as follows:
\begin{itemize}
\item Fluents $F$ are instantiated from a set of four {\em predicates} $\Psi$: {\small\tt (head ?x)} that encodes the current position of the header in the tape. {\small\tt (next ?x1 ?x2)} encoding that the cell {\tt ?x2} follows cell {\tt ?x1} in the tape. {\small\tt (symbol-$\sigma$ ?x)} encoding that the tape cell {\tt ?x} contains the symbol $\sigma\in\Sigma$. {\small\tt (state-$q$)} encoding that $q\in Q$ is the current machine state. Given a set of {\em objects} $\Omega$ that represent the cells in the tape of the given Turing Machine, the set of fluents $F$ is induced by assigning objects in $\Omega$ to the arguments of the predicates in $\Psi$.
\item Actions $A$ are instantiated from \strips\ operator schema. For each transition in $\delta$, a \strips\ schema is defined:
\begin{itemize}
\item The {\bf header} is {\small\tt transition-id(?xl ?x ?xr)} where $id$ uniquely identifies the transition in $\delta$. Parameters $?xl$, $?x$ and $?xr$ are tape cells.
\item The {\bf preconditions} are {\small\tt(head ?x)} and {\small\tt (next ?xl ?x) (next ?x ?xr)} to make $?x$ the tape cell pointed by the header and $?xl/?xr$ its left/right neighbors. Preconditions {\small\tt(symbol-$\sigma$ ?x)} and {\small\tt (state-$q$)} are also included to capture the symbol currently pointed by the header and the current machine state.
\item The {\bf delete effects} remove the symbol currently pointed by the header and the current machine state while the {\bf positive effects} set the new symbol pointed by the header and the new machine state according to $\delta$.
\end{itemize}
\end{itemize}

The \strips\ action schema of Figure~\ref{fig:update-rule} models the rule $a,q_0\rightarrow x,r,q_1$ of the {\em Turing Machine} defined in Figure~\ref{tab:tm-anbncn} (the full encoding of the {\em Turing Machine} of Figure~\ref{tab:tm-anbncn} produces a total of sixteen \strips\ action schema). 
\begin{figure}
\begin{scriptsize}
\begin{lstlisting}
(:action transition-1      ;;; a,$q_0\rightarrow$ x,r,$q_1$
  :parameters (?xl ?x ?xr)
  :precondition (and (head ?x) (symbol-a ?x) (state-$q_0$)
                     (next ?xl ?x) (next ?x ?xr))
  :effect (and (not (head ?x)) 
               (not (symbol-a ?x)) (not (state-$q_0$))
               (head ?xr) (symbol-x ?x) (state-$q_1$)))
\end{lstlisting}
\end{scriptsize}
 \caption{\small \strips\ action schema that models the transition $a,q_0\rightarrow x,r,q_1$ of the Turing Machine defined in Figure~\ref{tab:tm-anbncn}.}
\label{fig:update-rule}
\end{figure}

Since the {\em transition function} of a {\em Turing Machine} can be encoded as a classical planning frame $\Phi=\tup{F,A}$, executions of that {\em Turing Machine} are definable as an action sequence $\tup{a_1, \ldots, a_n}$ that induces the {\em state trajectory} $\tup{s_0, s_1, \ldots, s_n}$ such that $a_i$ ({\small $1\leq i\leq n$}) is applicable in $s_{i-1}$ and generates the successor state $s_i=\theta(s_{i-1},a_i)$.

\subsection{Experimental setup}
We randomly generated a $M=\{\mathcal{M}_1,\ldots,\mathcal{M}_{100}\}$ set of one-hundred different {\em Turing Machines} where each $\mathcal{M}\in M$ is a seven-symbol six-state {\em Turing Machine}. The classical planning frame $\Phi=\tup{F,A}$ is the same for all the {\em Turing Machines}, there is an $a\in A$ action for each pair of tape symbol and non-terminal state machine, while the $\theta(s,a)$ function is defined differently for each the machines (using a different \strips\ action model).

We randomly choose a machine $\mathcal{M}\in M$ and produce the observation $\tau$ of a fifty-step execution plan. Finally, we follow our {\em model recognition as planning} method to identify the {\em Turing Machine} that produced $\tau$. This experiment is repeated for different amounts of missing information in the input trace $\tau$: unknown applied transitions, unknown internal machine state and unknown values of several tape cells.

\subsubsection{Reproducibility}
{\sc Madagascar} is the classical planner we used to solve the instances that result from our compilations for its ability to deal with dead-ends~\cite{rintanen2014madagascar}. Due to its SAT-based nature, {\sc Madagascar} can apply the actions for editing preconditions in a single planning step (in parallel) because there is no interaction between them. Actions for editing effects can also be applied in a single planning step, thus significantly reducing the planning horizon.

The compilation source code, evaluation scripts and benchmarks (including the used training and test sets) are fully available at this anonymous repository {\em } so any experimental data reported in the paper can be reproduced.

\subsection{Results}

\subsection{Conclusions}
\label{sec:conclussions}



\bibliographystyle{aaai}
\bibliography{planlearnbibliography}

\end{document}
