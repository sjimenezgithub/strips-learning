%%%% ijcai19.tex

\typeout{IJCAI-19 Instructions for Authors}

% These are the instructions for authors for IJCAI-19.

\documentclass{article}
\pdfpagewidth=8.5in
\pdfpageheight=11in
% The file ijcai19.sty is NOT the same than previous years'
\usepackage{ijcai19}

% Use the postscript times font!
\usepackage{times}
\usepackage{soul}
\usepackage{url}
\usepackage[hidelinks]{hyperref}
\usepackage[utf8]{inputenc}
\usepackage[small]{caption}
\usepackage{graphicx}
\usepackage{amsmath}
\usepackage{booktabs}
\usepackage{algorithm}
\usepackage{algorithmic}
\urlstyle{same}

%%%%%%%%%%%%%%%%%% Added for this paper
\usepackage{amsthm}
\usepackage{amssymb}
\usepackage{listings}% http://ctan.org/pkg/listings
\lstset{
  basicstyle=\ttfamily,
  mathescape
}
\usepackage{ wasysym }
\newcommand{\tup}[1]{{\langle #1 \rangle}}
\newcommand{\pre}{\mathsf{pre}}     % precondition
\newcommand{\del}{\mathsf{del}}     % effect
\newcommand{\add}{\mathsf{add}}     % effect
\newcommand{\eff}{\mathsf{eff}}     % effect
\newcommand{\cond}{\mathsf{cond}}   % conditional effect
\newcommand{\true}{\mathsf{true}}   % true
\newcommand{\false}{\mathsf{false}} % false
\newcommand{\PE}{\mathrm{PE}}     % precondition
\newcommand{\strips}{\textsc{Strips}}

\newtheorem{mytheorem}{Theorem}
\newtheorem{mylemma}[mytheorem]{Lemma}
\newtheorem{mydefinition}[mytheorem]{Definition}

%%%%%%%%%%%%%





% the following package is optional:
%\usepackage{latexsym} 

% Following comment is from ijcai97-submit.tex:
% The preparation of these files was supported by Schlumberger Palo Alto
% Research, AT\&T Bell Laboratories, and Morgan Kaufmann Publishers.
% Shirley Jowell, of Morgan Kaufmann Publishers, and Peter F.
% Patel-Schneider, of AT\&T Bell Laboratories collaborated on their
% preparation.

% These instructions can be modified and used in other conferences as long
% as credit to the authors and supporting agencies is retained, this notice
% is not changed, and further modification or reuse is not restricted.
% Neither Shirley Jowell nor Peter F. Patel-Schneider can be listed as
% contacts for providing assistance without their prior permission.

% To use for other conferences, change references to files and the
% conference appropriate and use other authors, contacts, publishers, and
% organizations.
% Also change the deadline and address for returning papers and the length and
% page charge instructions.
% Put where the files are available in the appropriate places.

\title{Learning \strips\ action models from {\em state-invariants}}

% Single author syntax
%\author{
%    Sarit Kraus
%    \affiliations
%    Department of Computer Science, Bar-Ilan University, Israel \emails
%    pcchair@ijcai19.org
%}

% Multiple author syntax (remove the single-author syntax above and the \iffalse ... \fi here)
% Check the ijcai19-multiauthor.tex file for detailed instructions
\author{
Diego Aineto$^1$\and
Sergio Jim\'enez$^1$\and
Eva Onaindia$^1$
\affiliations
$^1${\small Departamento de Sistemas Inform\'aticos y Computaci\'on. Universitat Polit\`ecnica de Val\`encia. Valencia, Spain}
\emails
{\scriptsize \{dieaigar,serjice,onaindia\}@dsic.upv.es}}


\begin{document}
\maketitle


\begin{abstract}
This paper addresses the learning of action models from {\em deductive} knowledge with the form of {\em state-invariants} (i.e logic formulae that specify constraints about the possible states of a given domain) to reduce the negative effects of insufficient input observations. Given an action model, state-of-the-art planners use it to infer {\em state-invariants} that reduce the planner search space and make the planning process more efficient~\cite{helmert2009concise}. In this paper we follow the opposite direction and exploit {\em state-invariants} to learn a planning action model.
\end{abstract}

\section{Introduction}
\label{sec:introduction}

The specification of planning action models is a complex process that limits, too often, the application of {\em model-based planning} to real-world tasks~\cite{kambhampati:modellite:AAAI2007}. The {\em machine learning} of action models can relieve the {\em knowledge acquisition bottleneck} of planning and nowadays, there exists a wide range of effective approaches for learning action models~\cite{arora:amodels:ker2018}. Many of the most successful approaches for learning planning action models are however purely {\em inductive}~\cite{yang2007learning,pasula2007learning,mourao2010learning,zhuo2013action}, meaning that their performance depends on the {\em amount} and {\em quality} of the input examples (namely observations of plan executions generated by the aimed action model). 

This paper addresses the learning of action models from {\em deductive} knowledge with the form of {\em state-invariants} (i.e logic formulae that specify constraints about the possible states of a given domain) to reduce the negative effects of insufficient input observations. Given an action model, state-of-the-art planners use it to infer {\em state-invariants} that reduce the planner search space and make the planning process more efficient~\cite{helmert2009concise}. In this paper we follow the opposite direction and exploit {\em state-invariants} to learn a planning action model.

Our approach is built on top of the {\em classical planning} compilation for the learning of \strips\ action models~\cite{aineto2018learning}. This compilation is flexible to different kinds of input knowledge (e.g., partially/fully observations of actions of plan executions as well as partially/fully observed intermediate states) and outputs an action model that is {\em consistent} with the given input knowledge. In this paper we show that, in unfavorable scenarios where input observations are minimal (just an {\em initial state} and the {\em goals}), {\em state-invariant} help to learn better \strips\ models with the {\em classical planning} compilation. 



\section{Background}
\label{sec:background}
This section formalizes the {\em classical planning model} we follow in this work and the kind of {\em knowledge} that can be given as input to the task of learning \strips\ action models.  

\subsection{Classical planning with conditional effects}
Let $F$ be the set of  propositional state variables ({\em fluents}) describing a state. A {\em literal} $l$ is a valuation of a fluent $f\in F$; i.e. either~$l=f$ or $l=\neg f$. A set of literals $L$ represents a partial assignment of values to fluents (without loss of generality, we will assume that $L$ does not contain conflicting values). Given $L$, let $\neg L=\{\neg l:l\in L\}$ be its complement. We use $\mathcal{L}(F)$ to denote the set of all literal sets on $F$; i.e.~all partial assignments of values to fluents. A {\em state} $s$ is a full assignment of values to fluents; $|s|=|F|$.

A {\em classical planning action} $a\in A$ has: a precondition $\pre(a)\in\mathcal{L}(F)$, a set of effects $\eff(a)\in\mathcal{L}(F)$, and a positive action cost $cost(a)$. The semantics of actions $a\in A$ is specified with two functions: $\rho(s,a)$ denotes whether action $a$ is {\em applicable} in a state $s$ and $\theta(s,a)$ denotes the {\em successor state} that results of applying action $a$ in a state $s$. Then, $\rho(s,a)$ holds iff $\pre(a)\subseteq s$, i.e.~if its precondition holds in $s$. The result of executing an applicable action $a\in A$ in a state $s$ is a new state $\theta(s,a)=(s\setminus \neg\eff(a))\cup\eff(a)$. Subtracting the complement of $\eff(a)$ from $s$ ensures that $\theta(s,a)$ remains a well-defined state. The subset of action effects that assign a positive value to a state fluent is called {\em positive effects} and denoted by $\eff^+(a)\in \eff(a)$ while $\eff^-(a)\in \eff(a)$ denotes the {\em negative effects} of an action $a\in A$.

A {\em classical planning problem} is a tuple $P=\tup{F,A,I,G}$, where $I$ is the initial state and $G\in\mathcal{L}(F)$ is the set of goal conditions over the state variables. A {\em plan} $\pi$ is an action sequence $\pi=\tup{a_1, \ldots, a_n}$, with $|\pi|=n$ denoting its {\em plan length} and $cost(\pi)=\sum_{a\in\pi} cost(a)$ its {\em plan cost}. The execution of $\pi$ on the initial state of $P$ induces a {\em trajectory} $\tau(\pi,P)=\tup{s_0, a_1, s_1, \ldots, a_n, s_n}$ such that $s_0=I$ and, for each {\small $1\leq i\leq n$}, it holds $\rho(s_{i-1},a_i)$ and $s_i=\theta(s_{i-1},a_i)$. A plan $\pi$ solves $P$ iff the induced {\em trajectory} $\tau(\pi,P)$ reaches a final state $G \subseteq s_n$, where all goal conditions are met. A solution plan is {\em optimal} iff its cost is minimal.

We also define {\em actions with conditional effects} because they are useful to compactly formulate our approach for {\em goal recognition with unknown domain models}. An action $a_c\in A$ with conditional effects is a set of preconditions $\pre(a_c)\in\mathcal{L}(F)$ and a set of {\em conditional effects} $\cond(a_c)$. Each conditional effect $C\rhd E\in\cond(a_c)$ is composed of two sets of literals: $C\in\mathcal{L}(F)$, the {\em condition}, and $E\in\mathcal{L}(F)$, the {\em effect}. An action $a_c$ is applicable in a state $s$ if $\rho(s,a_c)$ is true, and the result of applying action $a_c$ in state $s$ is $\theta(s,a_c)=\{s\setminus\neg\eff_c(s,a)\cup\eff_c(s,a)\}$ where $\eff_c(s,a)$ are the {\em triggered effects} resulting from the action application (conditional effects whose conditions hold in $s$):
\[
\eff_c(s,a)=\bigcup_{C\rhd E\in\cond(a_c),C\subseteq s} E,
\]

\subsection{State-invaraints}
The notion of {\em state-constraint} is very general and has been used in different areas of AI and for different purposes.  If we restrict ourselves to planning, {\em state-constraints} are abstractions for compactly specifying sets of states. For instance, {\em state-constraints} in planning allow to specify the set of states where a given action is applicable, the set of states where a given {\em derived predicate} holds or the set of states that are considered goal states.

{\em State invariants} is a kind of state-constraints useful for computing more compact state representations~\cite{helmert2009concise} or making {\em satisfiability planning} and {\em backward search} more efficient~\cite{rintanen2014madagascar,alcazar2015reminder}. Given a classical planning problem $P=\tup{F,A,I,G}$, a {\em state invariant} is a formula $\phi$ that holds at the initial state of a given classical planning problem, $I\models \phi$, and at every state $s$, built from $F$, that is reachable from $I$ by applying actions in $A$.

The formula $\phi_{I,A}^*$ represents the {\em strongest invariant} and exactly characterizes the set of all states reachable from $I$ with the actions in $A$. For instance Figure~\ref{fig:strongest-invariant} shows five clauses that define the {\em strongest invariant} for the {\em blocksworld} planning domain~\cite{slaney2001blocks}. There are infinitely many strongest invariants, but they are all logically equivalent, and computing the strongest invariant is PSPACE-hard (as hard as testing plan existence~\cite{bylander:complexity:AIJ1994}).

\begin{figure}[hbt!]
  \begin{footnotesize}
$\forall x_1,x_2\ ontable(x_1)\leftrightarrow\neg on(x_1,x_2)$.\\
$\forall x_1,x_2\ clear(x_1)\leftrightarrow\neg on(x_2,x_1)$.\\
$\forall x_1,x_2,x_3\ \neg on(x_1,x_2)\vee\neg on(x_1,x_3)\ such\ that\ x_2\neq x_3$.\\
$\forall x_1,x_2,x_3\ \neg on(x_2,x_1)\vee\neg on(x_3,x_1)\ such\ that\ x_2\neq x_3$.\\
$\forall x_1,\ldots,x_n\ \neg(on(x_1,x_2)\wedge on(x_2,x_3)\wedge\ldots\wedge on(x_{n-1},x_n)\wedge on(x_n,x_1)).$
\end{footnotesize}
 \caption{\small {\em Strongest invariant} for the {\em blocksworld} domain.}
\label{fig:strongest-invariant}
\end{figure}

A {\em mutex} (mutually exclusive) is a state invariant that takes the form of a binary clause and indicates a pair of different properties that cannot be simultaneously true~\cite{kautz:mutex:IJCAI1999}. For instance in a three-block {\em blocksworld}, $\phi_1=\neg on(block_A,block_B)\vee \neg on(block_A,block_C)$ is a mutex because $block_A$ can only be on top of a single block.

A {\em domain invariant} is an instance-independent invariant, i.e. holds for any possible initial state and set of objects. Therefore, if a given state $s$ holds $s\nvDash \phi$ such that $\phi$ is a {\em domain invariant}, it means that $s$ is not a valid state. Domain invariants are often compactly defined as {\em lifted invariants} (also called schematic invariants)~\cite{rintanen:schematicInvariants:AAAI2017}. For instance, $\phi_2=\forall x:\ (\neg handempty\vee \neg holding(x))$, is a {\em domain mutex} for the {\em blocksworld} because the robot hand is never empty and holding a block at the same time.



\section{Learning \strips\ action models from {\em state-invaraints}}
\label{sec:learning}

We define the task of learning an action model from {\em state-invaraints} as a tuple $\Lambda=\tup{P,\Phi,M}$, where:
\begin{itemize}
\item $P=\tup{F,A[\cdot],I,G}$, is a {\em classical planning problem} where $A[\cdot]$ is a set of actions s.t., the semantics of each action $a\in A[\cdot]$ is unknown (i.e. the functions $\rho$ and/or $\theta$ of $a$ are undefined).
\item $\Phi$, a set of {\em state-invariants} that constrain the set of possible states in the given domain.
\item $M$ is the {\em space of possible action model} for the $A[\cdot]$ actions. This set is the {\em full} space of action models, when learning from scratch, or a {\em partially specified action model}, when some fragments of the aimed action model are a priori known. A set of action models can be defined {\em explicitly}, enumerating all the models that belong to the set or {\em implicitly}, enumerating all the constraints that must satisfy any model that belongs to the set. A {\em partially specified} \strips\ action model is then a formalism for the {\em implicit} representation of a set of \strips\ schemes~\cite{sreedharan2018handling}.
\end{itemize}

A {\em solution} to a $\Lambda=\tup{P,\Phi,M}$ learning task is a model $\mathcal{M}'\in M$ such that there exists a plan $\pi$ satisfying that any state traversed in the trajectory $\tau(\pi,P)$ satisfies all the {\em state-invariants} in $\Phi$. Inductive approaches for the learning planning action models compute an action model that maximizes some notion of {\em statistical consistency} over a set of observations of plan executions so output an action model in our case a solution to the addressed learning task is an action model that is {\em consistent} with the input knowledge.

Next we show that the set $M$ of possible action models can be encoded as a set of propositional variables and a set of constraints over those variables. Then, we show how to exploit this encoding to solve a $\Lambda=\tup{P,\Phi,M}$ learning task with an off-the-shelf classical planner.


\subsection{A propositional encoding for the space of STRIPS action models}
{\em A \strips\ action schema} $\xi$ is defined by four lists: A list of {\em parameters} $pars(\xi)$, and three list of predicates (namely $pre(\xi)$, $del(\xi)$ and $add(\xi)$) that shape the kind of fluents that can appear in the {\em preconditions}, {\em negative effects} and {\em positive effects} of the actions induced from that schema. Let be $\Psi$ the set of {\em predicates} that shape the propositional state variables $F$, and a list of {\em parameters} $pars(\xi)$. The set of elements that can appear in $pre(\xi)$, $del(\xi)$ and $add(\xi)$ of the \strips\ action schema $\xi$ is given by FOL interpretations of $\Psi$ over the parameters $pars(\xi)$ and is denoted as ${\mathcal I}_{\Psi,\xi}$.

For instance in a four-operator {\em blocksworld}~\cite{slaney2001blocks}, the ${\mathcal I}_{\Psi,\xi}$ set contains only five elements for the {\small \tt pickup($v_1$)} schemata, ${\mathcal I}_{\Psi,pickup}$={\small\tt\{handempty, holding($v_1$), clear($v_1$), ontable($v_1$), on($v_1,v_1$)\}} while it contains eleven elements for the {\small \tt stack($v_1$,$v_2$)} schemata, ${\mathcal I}_{\Psi,stack}$={\small\tt\{handempty, holding($v_1$), holding($v_2$), clear($v_1$), clear($v_2$), ontable($v_1$), ontable($v_2$), on($v_1,v_1$), on($v_1,v_2$), on($v_2,v_1$), on($v_2,v_2$)\}}. 

Despite any element of ${\mathcal I}_{\Psi,\xi}$ can {\em a priori} appear in the $pre(\xi)$, $del(\xi)$ and $add(\xi)$ of schema $\xi$, the actual space of possible \strips\ schemata is bounded by constraints of three kinds:
\begin{enumerate}
\item {\bf Syntactic constraints}. \strips\ constraints require $del(\xi)\subseteq pre(\xi)$, $del(\xi)\cap add(\xi)=\emptyset$ and $pre(\xi)\cap add(\xi)=\emptyset$. Considering exclusively these syntactic constraints, the size of the space of possible \strips\ schemata is given by $2^{2\times|{\mathcal I}_{\Psi,\xi}|}$. {\em Typing constraints} are also of this kind~\cite{mcdermott1998pddl}. 
\item {\bf Domain-specific constraints}. One can introduce domain-specific knowledge to constrain further the space of possible schemata. For instance, in the {\em blocksworld} one can argue that {\small\tt on($v_1$,$v_1$)} and {\small\tt on($v_2$,$v_2$)} will not appear in the $pre(\xi)$, $del(\xi)$ and $add(\xi)$ lists of an action schema $\xi$ because, in this specific domain, a block cannot be on top of itself. {\it State invariants} are also constraints of this kind. 
\item {\bf Observation constraints}. An observation $\mathcal{O}(\tau)$ depicts {\em semantic knowledge} that constraints further the space of possible action schemata.   
\end{enumerate}

\begin{figure}
  \begin{tiny}  
  \begin{verbatim}
(:action stack
   :parameters (?v1 ?v2)
   :precondition (and (holding ?v1) (clear ?v2))
   :effect (and (not (holding ?v1)) (not (clear ?v2))
                (clear ?v1) (handempty) (on ?v1 ?v2)))


(pre_holding_v1_stack) (pre_clear_v2_stack)
(eff_holding_v1_stack) (eff_clear_v2_stack)
(eff_clear_v1_stack) (eff_handempty_stack) (eff_on_v1_v2_stack)
  \end{verbatim}           
  \end{tiny}  
 \caption{\small PDDL encoding of the {\tt\small stack(?v1,?v2)} schema and our propositional representation for this same schema.}
\label{fig:propositional}
\end{figure}

In this work we introduce a propositional encoding of the {\em preconditions}, {\em negative}, and {\em positive} effects of a \strips\ action schema $\xi$ using only fluents of two kinds {\tt\small pre\_e\_$\xi$} and {\tt\small eff\_e\_$\xi$} (where $e\in{\mathcal I}_{\Psi,\xi}$). This encoding exploits the syntactic constraints of \strips\, so it is more compact that the one previously proposed by~\citeauthor{aineto2018learning}~\citeyear{aineto2018learning} for learning classical planning action models. In more detail, if {\tt\small pre\_e\_$\xi$} holds it means that $e\in{\mathcal I}_{\Psi,\xi}$ is a {\em precondition} in $\xi$. If {\tt\small pre\_e\_$\xi$} and {\tt\small eff\_e\_$\xi$} holds it means that $e\in{\mathcal I}_{\Psi,\xi}$ is a {\em negative effect} in $\xi$ while if $pre\_e\_\xi$ does not hold but {\tt\small eff\_e\_$\xi$} holds, it means that $e\in{\mathcal I}_{\Psi,\xi}$ is a {\em positive effect} in $\xi$. Figure~\ref{fig:propositional} shows the PDDL encoding of the {\tt\small stack(?v1,?v2)} schema and our propositional representation for this same schema with {\tt\small pre\_e\_stack} and {\tt\small eff\_e\_stack} fluents ($e\in{\mathcal I}_{\Psi,stack}$).


\subsection{Learning \strips\ action models with classical planning}
Given a $\Lambda=\tup{P,\Phi,M}$ where $\Phi$ is a set of {\em domain mutex} $\phi\in\Phi$, we create a classical planning problem $P'=\tup{F',A',I,G}$ such that:
\begin{itemize}
\item $F'$ extends $F$ with a fluent ${\tt inconsistent}$, to indicate whether an action model is {\em inconsistent} with the input {\em state-invariants}, a fluent $mode_{insert}$, to indicate whether action models are being programmed, and the fluents for the propositional encoding of the corresponding space of STRIPS action models. This is a set of fluents of the type $\{pre\_e\_\xi, eff\_e\_\xi\}_{\forall e\in{\mathcal I}_{\Psi,\xi}}$ such that $e\in{\mathcal I}_{\Psi,\xi}$ is a single element from the set of FOL interpretations of predicates $\Psi$ over the corresponding action parameters $pars(\xi)$. 

\item $A'$ replaces the actions in $A$ with two types of actions.
\begin{enumerate}
\item Actions for {\em inserting} a {\em precondition}, {\em positive} effect or {\em negative} effect in $\xi$ following the syntactic constraints of \strips\ models. In the particular case that $M$ is a {\em partially specified model} then only the actions for inserting a possible {\em precondition} or {\em effect} are necessary.
\begin{itemize}
\item Actions which support the addition of a {\em precondition} $p\in \Psi_{\xi}$ to the action model $\xi$. A precondition $p$ is inserted in $\xi$ when neither $pre_p$, $eff_p$ exist in $\xi$.

\begin{small}
\begin{align*}
\hspace*{7pt}\pre(\mathsf{insertPre_{p,\xi}})=&\{\neg pre_{p}(\xi), \neg eff_{p}(\xi), mode_{insert}\},\\
\cond(\mathsf{insertPre_{p,\xi}})=&\{\emptyset\}\rhd\{pre_{p}(\xi)\}.
\end{align*}
\end{small}

\item Actions which support the addition of a {\em negative} or {\em positive} effect $p\in \Psi_{\xi}$ to the action model $\xi$. 

\begin{small}
\begin{align*}
\hspace*{7pt}\pre(\mathsf{insertEff_{p,\xi}})=&\{\neg eff_{p}(\xi), mode_{insert}\},\\
\cond(\mathsf{insertEff_{p,\xi}})=&\{\emptyset\}\rhd\{eff_{p}(\xi)\}.
\end{align*}
\end{small}
\end{itemize}

\item Actions for {\em applying} an action model $\xi$ built by the {\em insert} actions and bounded to objects $\omega\subseteq\Omega^{|pars(\xi)|}$ (where $\Omega$ is the set of {\em objects} used to induce the fluents $F$ by assigning objects in $\Omega$ to the $\Psi$ predicates and $\Omega^k$ is the $k$-th Cartesian power of $\Omega$). The action parameters, $pars(\xi)$, are bound to the objects in $\omega$ that appear in the same position.
\end{enumerate}
\end{itemize}

\begin{small}
\begin{align*}
\pre(\mathsf{apply_{\xi,\omega}})=&\{\emptyset\},\\
\cond(\mathsf{apply_{\xi,\omega}})=&\{pre_{p}(\xi)\wedge \neg p(\omega)\}_{\forall p\in\Psi_\xi},\\
&\{pre_{p}(\xi)\wedge eff_{p}(\xi)\}\rhd\{\neg p(\omega)\}_{\forall p\in\Psi_\xi},\\
&\{\neg pre_{p}(\xi)\wedge eff_{p}(\xi)\}\rhd\{p(\omega)\}_{\forall p\in\Psi\xi}\},\\
&\{\emptyset\}\rhd\{\neg mode_{insert}\}.
\end{align*}
\end{small}



\section{Learning from {\em state-invariants} and observations of plan executions}
\label{sec:observations}
Most of the approachs for leanning planning action mdoels compute an action model that maximizes some notion of {\em statistical consistency} over a set of observations of plan executions so output an action model that is {\em consistent} with the input knowledge.

\subsection{The observation model}
Given a planning problem $P=\tup{F,A,I,G}$, a plan $\pi$ and a trajectory $\tau(\pi,P)$, we define the \emph{observation of the trajectory} as an interleaved combination of actions and states that represents the observation from the execution of $\pi$ in $P$. Formally, $\mathcal{O}(\tau)=\tup{s_0^o,a_1^o,s_1^o \ldots , a_l^o, s_m^o}$, $s_0^o=I$, and:

\begin{itemize}
\item The {\bf observed actions} are consistent with $\pi$, which means that $\tup{a_1^o, \ldots, a_l^o}$ is a sub-sequence of $\pi$. The number of observed actions, $l$, ranges from $0$ (fully unobserved action sequence) to $|\pi|$ (fully observed action sequence).
\item The {\bf observed states} $\tup{s_0^o, s_1^o, \ldots, s_m^o}$ is a sequence of possibly {\em partially observable states}, except for the initial state $s_0^o$, which is fully observed. A partially observable state $s_i^o$ is one in which $|s_i^o| < |F|$; i.e., a state in which at least a fluent of $F$ is not observable. Note that this definition also comprises the case $|s_i^o| = 0$, when the state is fully unobservable. Whatever the sequence of observed states of $\mathcal{O}(\tau)$ is, it must be consistent with the sequence of states of $\tau(\pi,P)$, meaning that $\forall i, s_i^o \subseteq s_i$. The number of observed states, $m$, range from 1 (the initial state, at least), to $|\pi|+1$, and each {\em observed} states comprises $[1,|F|]$ fluents (the observation can still miss intermediate states that are {\em unobserved}).
\end{itemize}

We assume a bijective monotone mapping between actions/states of trajectories and observations~\cite{ramirez2009plan}, thus also granting the inverse consistency relationship (the trajectory is a superset of the observation). Therefore, transiting between two consecutive observed states in $\mathcal{O}(\tau)$ may require the execution of more than a single action ($\theta(s_i^o,\tup{a_1,\ldots,a_k})=s_{i+1}^o$, where ${\small k\geq 1}$ is unknown but finite. In other words, having an input observation $\mathcal{O}(\tau)$ does not imply knowing the actual length of $\pi$.

\subsection{Extending the compilation to deal with {\em plan recognition as planning}}
The work on {\em plan recognition as planning} usually assumes an observation model that is referred only to logs of executed actions. However, the approach applies also to more expressive observation models that consider state observations as well, like the observation model defined above, with a simple three-fold extension:
\begin{itemize}
\item One fluent $\{validated_j\}_{0\leq j\leq m}$ to point at every $s_j^o\in\mathcal{O}(\tau)$ state observation. Two fluents, $at_i$ and $next_{i,i+1}$, {\small $1\leq i \leq n$}, to iterate through the $n$ observed actions of $\tau$. The former is used to ensure that actions are executed in the same order as they are observed in $\tau$. The latter is used to iterate to the next planning step when solving $P_{\Lambda}$.
\item Adding $at_1$ and $\{next_{i,i+1}\}$, {\small $1\leq i \leq n$} to the initial state and $validated_m$ to every possible goal $G\in G[\cdot]$ to constrain solution plans $\pi^\top$ to be consistent with all the state observations.

\item When the input plan trace contains observed actions, the extra conditional effects $\{at_{i},plan(name(a_i),\Omega^{ar(a_i)},i)\}\rhd\{\neg at_{i},at_{i+1}\}_{\forall i\in [1,n]}$ are included in the $\mathsf{apply_{\xi,\omega}}$ actions to ensure that actions are applied in the same order as they appear in $\tau$.\\

\item Actions for {\em validating} the partially observed state $s_j\in\tau$, {\tt\small $1\leq j< m$}. These actions are also part of the postfix of the solution plan $\pi_\Lambda$ and they are aimed at checking that the observable data of the input plan trace $\tau$ follows after the execution of the apply actions.


\item One $\mathsf{validate_{j}}$ action to constraint $\pi^\top$ to be consistent with the $s_j^o\in\mathcal{O}(\tau)$ input state observation, {\small $(1\leq j\leq m)$}.  
\end{itemize}
\begin{small}
\begin{align*}
\hspace*{7pt}\pre(\mathsf{validate_{j}})=&s_j^o\cup\{validated_{j-1}\},\\
\cond(\mathsf{validate_{j}})=&\{\emptyset\}\rhd\{\neg validated_{j-1}, validated_j\}.
\end{align*}
\end{small}








{\em domain mutex} are useful to reduce the amount of applicable actions for programming a precondition or an effect for a given action schema. For example given the {\em domain mutex} $\phi=(\neg f_1\vee \neg f_2)$ such that $f_1\in F_v(\xi)$ and $f_2\in F_v(\xi)$, we can redefine the corresponding programming actions for {\bf removing} the {\em precondition} $f_1\in F_v(\xi)$ from the action schema $\xi\in\mathcal{M}$ as:




\section{Evaluation}
\label{sec:evaluation}



\section{Conclusions}
\label{sec:conclusions}


%% The file named.bst is a bibliography style file for BibTeX 0.99c
\bibliographystyle{named}
\bibliography{planlearnbibliography}

\end{document}
