
\section{Evaluation model}
\label{evaluation}


The roles of two action schemes whose headers match or the roles of two action parameters that belong to the same type can be swapped by a $\Lambda$ learning task. For instance, the {\em blocksworld} operator {\small\tt stack} can be {\em learned} with the preconditions and effects of the {\small\tt unstack} operator and vice versa, or the parameters of the {\small\tt stack} operator can be swapped. Pure syntax-based metrics can report low scores for learned models that are actually good but correspond to {\em reformulations} of the actual model; i.e. a learned model semantically equivalent but syntactically different to the reference model.

Here we introduce an evaluation approach that is robust to role changes of this particular kind. The intuition of the approach is to assess how well a \strips\ action model $\mathcal{M}$ explains given observations of plan executions according to the amount of {\em edition} required by $\mathcal{M}$ to induce that observations. %In the extreme, if $\mathcal{M}$ perfectly explains $\mathcal{O}$, no model {\em edition} is necessary.

\subsection{The \strips\ edit distance}

We first define the two allowed \emph{operations} to edit a given \strips\ action model $\mathcal{M}$:
\begin{itemize}
\item {\em Deletion}. A fluent $pre_f(\xi)/del_f(\xi)/add_f(\xi)$ is removed from the operator schema $\xi\in\mathcal{M}$, such that $f\in F_v(\xi)$.
\item {\em Insertion}. A fluent $pre_f(\xi)/del_f(\xi)/add_f(\xi)$ is added to the operator schema $\xi\in\mathcal{M}$, s.t. $f\in F_v(\xi)$.
\end{itemize}

We can now formalize an {\em edit distance} that quantifies how similar two given \strips\ action models are. The distance is symmetric and meets the {\em metric axioms} provided that the two {\em edit operations}, deletion and insertion, have the same positive cost.

\begin{mydefinition}
  Let $\mathcal{M}$ and $\mathcal{M}'$ be two \strips\ action models, such that they are {\em comparable}. The {\bf edit distance}, denoted as $\delta(\mathcal{M},\mathcal{M}')$, is the minimum number of {\em edit operations} that is required to transform $\mathcal{M}$ into $\mathcal{M}'$.
\end{mydefinition}

Since $F_v$ is a bound set, the maximum number of edits that can be introduced to a given action model defined within $F_v$ is bound as well. In more detail, for an operator schema $\xi\in\mathcal{M}$ the maximum number of edits that can be introduced to their precondition set is $|F_v(\xi)|$ while the max number of edits that can be introduced to the effects is twice $|F_v(\xi)|$.
\begin{mydefinition}
The \textbf{maximum edit distance} of an \strips\ action model $\mathcal{M}$ built from the set of possible elements $F_v$ is $\delta(\mathcal{M},*)=\sum_{\xi\in\mathcal{M}} 3\times|F_v(\xi)|$.
\end{mydefinition}

Normally evaluating a given learned domain with respect to the actual generative model is not possible because the actual model is not available. As the ARMS system shows, the error of a learned action model can also be estimated with respect to a set of observations of plan executions~\cite{yang2007learning}. With this regard, we define now an edit distance to asses the quality of a learned action model with respect to a sequence of state observations.

\begin{mydefinition}
  Given $\mathcal{M}$, a \strips\ action model built from $F_v$, and the observations sequence $\mathcal{O}=\tup{s_0, s_1, \ldots, s_n}$ such that each state observation in $\mathcal{O}$ is built with fluents in $F$. The {\bf observation edit distance}, denoted by  $\delta(\mathcal{M},\mathcal{O})$, is the minimal edit distance from $\mathcal{M}$ to any {\em comparable} model $\mathcal{M}'$, such that $\mathcal{M}'$ can produce a valid plan $\pi=\tup{a_1, \ldots, a_n}$ that induces $\mathcal{O}$; \[\delta(\mathcal{M},\mathcal{O})=\min_{\forall \mathcal{M}' \rightarrow \mathcal{O}} \delta(\mathcal{M},\mathcal{M}')\]
\end{mydefinition}

Unlike the error function defined by {\sc ARMS}, the {\em observation edit distance} assess, with a single expression, the flaws in the preconditions and effects of a given learned model. This fact enables the recognition of \strips\ action models. The idea, taken from {\em plan recognition as planning}~\cite{ramirez2009plan}, is to map distances into likelihoods. This {\em edit distance} could be mapped into a likelihood with the following expression $P(\mathcal{O}|\mathcal{M})=1-\frac{\delta(\mathcal{M},\mathcal{O})}{\delta(\mathcal{M},*)}$.

The error of a learned action model could also be defined quantifying the amount of edition required by the observations of the plan execution to match the given model. This would imply defining {\em edit operations} that modify the fluents in the state observations instead of the {\em edit operations} that modify the action schemes. Our definition of the edit distance is more practical since normally, $F_v$ is smaller than $F$ because the number of {\em variable objects} is smaller than the number of objects in the state observations. Finally, the {\em edit distance} can also be defined with respect to a set of plans, if they are available.

\begin{mydefinition}
  Given an action model $\mathcal{M}$ and a set of valid plans $\Pi=\{\pi_1,\ldots,\pi_{\tau}\}$ that only contain actions built grounding the schemes in $\mathcal{M}$. The {\bf plans edit distance}, denoted by  $\delta(\mathcal{M},\Pi)$, is the minimal edit distance from $\mathcal{M}$ to any {\em comparable} model $\mathcal{M}'$, such that $\mathcal{M}'$ can produce all the plans $\pi_t\in \Pi$, {\tt\small $1\leq t\leq \tau$}; \[\delta(\mathcal{M},\Pi)=\min_{\forall \mathcal{M}' \rightarrow \Pi} \delta(\mathcal{M},\mathcal{M}')\]
\end{mydefinition}




\subsection{Computing the observations and plans edit distance}
Our compilation is extensible to compute the {\em observation edit distance} by simply considering that the input \strips\ model $\mathcal{M}$, given in a learning task $\Lambda=\tup{\mathcal{M},\mathcal{O},\Psi}$, is {\em non-empty}. In other words, now $\mathcal{M}$ is a set of given operator schemas, wherein each $\xi\in\mathcal{M}$ initially contains $head(\xi)$ but also the $pre(\xi)$, $del(\xi)$ and $add(\xi)$ sets. A solution to the planning task resulting from the extended compilation is a sequence of actions that:

\begin{enumerate}
\item {\bf Edits the action model $\mathcal{M}$ to build $\mathcal{M}'$}. A solution plan starts with a {\em prefix} that modifies the preconditions and effects of the action schemes in $\mathcal{M}$ using to the two {\em edit operations} defined above, {\em deletion} and {\em insertion}. In theory, we could implement a third edit operation for {\em substituting} a fluent from a given operator schema. However, and with the aim of keeping a tractable branching factor of the planning instances that result from our compilations, we only implement {\em deletion} and {\em insertion}.
\item {\bf Validates the edited model $\mathcal{M}'$ in observations of the plan executions}. The solution plan continues with a postfix that validates the edited model $\mathcal{M}'$ on the given observations $\mathcal{O}$ (and on $\Pi$ if available), as explained in Section~\ref{sec:Section5} for the models that are programmed from scratch.
\end{enumerate}

Now $\Lambda$ does not formalize a learning task but the task of editing $\mathcal{M}$ to produce the observations $\mathcal{O}$, which results in the edited model $\mathcal{M}'$. The output of the extended compilation is a classical planning task $P_{\Lambda}'=\tup{F_{\Lambda},A_{\Lambda}',I_{\Lambda}',G_{\Lambda}}$:

\begin{itemize}
\item $F_{\Lambda}$ and $G_{\Lambda}$ are defined as in the previous compilation.
\item $I_{\Lambda}'$ contains the fluents from $F$ that encode $s_0$ and $mode_{prog}$ set to true. In addition, the input action model $\mathcal{M}$ is now encoded in the initial state. This means that the fluents $pre_f(\xi)/del_f(\xi)/add_f(\xi)$, $f\in F_v(\xi)$, hold in the initial state iff they appear in $\mathcal{M}$.
\item $A_{\Lambda}'$, comprises the same three kinds of actions of $A_{\Lambda}$. The actions for {\em applying} an already programmed operator schema and the actions for {\em validating} an observation are defined exactly as in the previous compilation. The only difference here is that the actions for {\em programming} the operator schema now implement the two {\em edit operations} (i.e. include actions for {\em inserting} a precondition and for {\em deleting} a negative/positive effect).
\end{itemize}

To illustrate this, the plan of Figure~\ref{fig:plan-odistance} solves the classical planning task that corresponds to editing a \emph{blocksworld} action model where the positive effects {\tt\small (handempty)} and {\tt\small (clear ?v1)} of the {\tt\small stack} schema are missing. First, the plan edits the {\tt\small stack} schema, {\em inserting} these two positive effects, and then validates the edited action model in the five-observation sequence of Figure~\ref{fig:example-observations}.

\begin{figure}[hbt!]
{\tt\small
00 : (insert\_add\_handempty\_stack)\\
01 : (insert\_add\_clear\_stack\_var1)\\
02 : (apply\_unstack blockB blockA)\\
03 : (validate\_1)\\
04 : (apply\_putdown blockB)\\
05 : (validate\_2)\\
06 : (apply\_pickup blockA)\\
07 : (validate\_3)\\
08 : (apply\_stack blockA blockB)\\
09 : (validate\_4)\\
}
\caption{\small Plan for editing a given action model and validating it at the state observations shown in Figure~\ref{fig:example-observations}. The given action model is a four operator {\em blockswold} where the positive effects {\tt\small (handempty)} and {\tt\small (clear ?v1)} of the {\tt\small stack} schema are missing.}
\label{fig:plan-odistance}
\end{figure}

Our interest when computing the {\em observation edit distance} is not in the resulting action model $\mathcal{M}'$ but in the number of required {\em edit operations} for that $\mathcal{M}'$ is validated in the given observations, e.g. $\delta(\mathcal{M},\mathcal{O})=2$ for the example in Figure~\ref{fig:plan-odistance}. In this case $\delta(\mathcal{M},*)=3\times 2\times (11+5)$ since there are 4 action schemes ({\small\tt pickup}, {\small\tt putdown}, {\small\tt stack} and {\small\tt unstack}) and $|F_v|=|F_v(stack)|=|F_v(unstack)|=11$ while $|F_v(pickup)|=|F_v(putdown)|=5$  (as shown in Section~\ref{sec:Section3}). The {\em observation edit distance} is exactly computed if the classical planning task resulting from our compilation is optimally solved (according to the number of edit actions); is approximated if it is solved with a satisfying planner; and is a less accurate estimate (but faster to be computed) if the solved task is a relaxation of the classical planning task that results from our compilation~\cite{bonet2001planning}.

When the executed plans $\Pi$ are also available, the compilation can be adapted to compute the {\em plan edit distance} $\delta(\mathcal{M},\Pi)$. The modifications to the compilation explained in Section~\ref{sec:Section5} are also useful here to redefine the learning task as the task of editing $\mathcal{M}$ to produce the set of plans $\Pi$, which results in the edited model $\mathcal{M}'$. Figure~\ref{fig:plan-pdistance} shows the plan for editing a given {\em blockswold} action model where again the positive effects {\tt\small (handempty)} and {\tt\small (clear ?v1)} of the {\tt\small stack} schema are missing. In this case the edited action model is however validated at the plan shown in Figure~\ref{fig:example-plans}.

\begin{figure}[hbt!]
{\tt\small
00 : (insert\_add\_handempty\_stack)\\
01 : (insert\_add\_clear\_stack\_var1)\\
02 : (apply\_unstack blockB blockA i1 i2)\\
03 : (apply\_putdown blockB i2 i3)\\
04 : (apply\_pickup blockA i3 i4)\\
05 : (apply\_stack blockA blockB i4 i5)\\
06 : (validate\_1)
}
 \caption{\small Plan for editing a given {\em blockswold} schema and validating it at the plan shown in Figure~\ref{fig:example-plans}.}
\label{fig:plan-pdistance}
\end{figure}

Last but not least, this compilation is flexible to compute the {\em edit distance} between two {\em comparable} \strips\ action models, $\mathcal{M}$ and $\mathcal{M}'$. A solution to the planning task resulting from this compilation is a sequence of actions that edits the action model $\mathcal{M}$ to produce $\mathcal{M}'$ using to the two {\em edit operations}, deletion and insertion. In this case the edited model is not validated on a sequence of observations or plans but on the given action model $\mathcal{M}'$ that acts as a reference. The sets of fluents $F_{\Lambda}$ and $I_{\Lambda}$ are defined like in the previous compilation. With respect to the actions, $A_{\Lambda}$ again implement the two {\em edit operations} (i.e. include actions for {\em inserting} a precondition and for {\em deleting} a negative/positive effect) but does not contain apply actions because the \strips\ action model are not validated in any observation of plan executions. Finally, the goals are also different and are now defined by the set of fluents, $pre_f(\xi)/del_f(\xi)/add_f(\xi)$ that represent all the operator schema $\xi\in\mathcal{M'}$, such that $f\in F_v(\xi)$. To illustrate this, the plan of Figure~\ref{fig:plan-mdistance} solves the classical planning task that corresponds to computing the distance between a \emph{blocksworld} action model, where the positive effects {\tt\small (handempty)} and {\tt\small (clear ?v1)} of the {\tt\small stack} schema are missing, and the actual four-operator {\em blocksworld} model. The plan edits first the {\tt\small stack} schema, {\em inserting} these two positive effects. Again our interest is in the number of required {\em edit operations}, e.g. $\delta(\mathcal{M},\mathcal{M'})=2$.

\begin{figure}[hbt!]
{\tt\small
00 : (insert\_add\_handempty\_stack)\\
01 : (insert\_add\_clear\_stack\_var1)
}
 \caption{\small Plan for computing the distance between a \emph{blocksworld} action model, where the positive effects {\tt\small (handempty)} and {\tt\small (clear ?v1)} of the {\tt\small stack} schema are missing, and the actual four-operator {\em blocksworld} model.}
\label{fig:plan-mdistance}
\end{figure}



\subsection{Recognition of \strips\ action models}
\label{sec:Section7}
Given a set of possible \strips\ models and set of observations of plan executions, the {\em recognition of \strips\ models} is the task of identifying which model has the highest probability of producing the given observations.

According to the Bayes rule, the probability of an hypothesis $\mathcal{H}$ given the observations $\mathcal{O}$ can be computed with $P(\mathcal{H}|\mathcal{O})=\frac{P(\mathcal{O}|\mathcal{H})P(\mathcal{H})}{P(\mathcal{O})}$. In our scenario, the hypotheses are about the set of possible \strips\ action models. Given set of predicates $\Psi$ and a given a set of operator headers (in other words, given the $F_v(\xi)$ sets) the size of the set of possible \strips\ models set is $\prod_\xi 2^{2|F_v(\xi)|}$, as explained in Section~\ref{sec:Section3}. With respect to the observations, given $\Psi$ and a set of objects $\Omega$, the size of the possible state observations of length $n$, that is $\mathcal{O}=s_0,\ldots,s_n$ is given by $2^{n\times|F|}$.

With this regard, $P(\mathcal{M}|\mathcal{O})$, the probability distribution of the possible \strips\ models (within the $F_v(\xi)$ sets) given an observation sequence $\mathcal{O}$ could be computed by:
\begin{enumerate}
\item Computing the {\em observation edit distance} $\delta(\mathcal{M},\mathcal{O})$ for every possible model $\mathcal{M}$. If a set of plans $\Pi$ is available, this same strategy can be followed using the {\em plan edit distance} $\delta(\mathcal{M},\Pi)$.
\item Applying the resulting distances to the above $P(\mathcal{O}|\mathcal{M})$ formula to map these distances into likelihoods
\item Applying the Bayes rule to obtain the normalized posterior probabilities, these probabilities must sum 1.
\end{enumerate}





