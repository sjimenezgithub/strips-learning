\section{Evaluation of \strips\ Action Models with Classical Planning}
\label{evaluation}

If the actual model is available, the quality of a learned action model is quantifiable using the {\em precision} and {\em recall} metrics. These two metrics are frequently used in {\em pattern recognition}, {\em information retrieval} and {\em binary classification} and are more informative that simply counting the number of errors in the learned model or computing the {\em symmetric difference} between the learned and the reference model~\cite{davis2006relationship}. Intuitively, precision gives a notion of {\em soundness} while recall gives a notion of the {\em completeness} of the learned models:
\begin{itemize}
\item $Precision=\frac{tp}{tp+fp}$, where $tp$ is the number of {\em true positives} (predicates that correctly appear in the action model) and $fp$ is the number of {\em false positives} (predicates of the learned model that should not appear).
\item $Recall=\frac{tp}{tp+fn}$, where $fn$ is the number of {\em false negatives} (predicates that should appear in the learned model but are missing).
\end{itemize}

The roles of two {\em comparable} action schemes or the roles of two action parameters that belong to the same type can be swapped by a $\Lambda$ learning task when the input data given in $\mathcal{T}$ is insufficient. For instance, the {\em blocksworld} operator {\small\tt stack} can be {\em learned} with the preconditions and effects of the {\small\tt unstack} operator and vice versa, or the parameters of the {\small\tt stack} operator can be swapped. Pure syntax-based evaluation metrics (like {\em precision} and {\em recall}) can report low scores for learned models that are actually good but correspond to {\em reformulations} of the actual model; i.e. a learned model semantically equivalent but syntactically different to the reference model.

Here we introduce a novel evaluation approach that is robust to role changes of this particular kind. The intuition of the approach is to assess how well a \strips\ action model $\mathcal{M}$ explains given observations of plan executions according to the amount of {\em edition} required by $\mathcal{M}$ to induce that observations. Like our learning approach, our evaluation approach is also flexible to various amount and kind of available input knowledge.

\subsection{The \strips\ edit distance}

We first define the two allowed \emph{operations} to edit a given \strips\ action model $\mathcal{M}$:
\begin{itemize}
\item {\em Deletion}. A fluent $pre_f(\xi)/del_f(\xi)/add_f(\xi)$ is removed from the operator schema $\xi\in\mathcal{M}$, such that $f\in F_v(\xi)$.
\item {\em Insertion}. A fluent $pre_f(\xi)/del_f(\xi)/add_f(\xi)$ is added to the operator schema $\xi\in\mathcal{M}$, s.t. $f\in F_v(\xi)$.
\end{itemize}

We can now formalize an {\em edit distance} that quantifies how similar two given \strips\ action models are. The distance is symmetric and meets the {\em metric axioms} provided that the two {\em edit operations}, deletion and insertion, have the same positive cost.

\begin{mydefinition}
  Let $\mathcal{M}$ and $\mathcal{M}'$ be two \strips\ action models, such that they are {\em comparable}. The {\bf edit distance}, denoted as $\delta(\mathcal{M},\mathcal{M}')$, is the minimum number of {\em edit operations} that is required to transform $\mathcal{M}$ into $\mathcal{M}'$.
\end{mydefinition}
Note that the number of deletions indicate the number of {\em false positives} while the number of insertions represents the number of {\em false negatives}. As explained, these numbers can be normalized with the {\em true positives} and {\em true negatives} to compute the {\em precision} and {\em recall} of a learned model.   

Since $F_v$ is a bound set, the maximum number of edits that can be introduced to a given action model defined within $F_v$ is bound as well. In more detail, for an operator schema $\xi\in\mathcal{M}$ the maximum number of edits that can be introduced to their precondition set is $|F_v(\xi)|$ while the max number of edits that can be introduced to the effects is twice $|F_v(\xi)|$.
\begin{mydefinition}
The \textbf{maximum edit distance} of an \strips\ action model $\mathcal{M}$ built from the set of possible elements $F_v$ is $\delta(\mathcal{M},*)=\sum_{\xi\in\mathcal{M}} 3\times|F_v(\xi)|$.
\end{mydefinition}

Normally evaluating a given learned domain with respect to the actual generative model is not possible because the actual model is not available. As the ARMS system shows, the error of a learned action model can also be estimated with respect to a set of observations of plan executions~\cite{yang2007learning}. With this regard, we define now an edit distance to asses the quality of a learned action model with respect to a plan trace $\mathcal{T}=\tup{s_0,a,_1,s_1,\ldots,a_n,s_{n}}$ obtained watching the execution of a plan $\pi=\tup{a_1, \ldots, a_n}$ such that, for each {\small $1\leq i\leq n$}, $a_i$ is applicable in $s_{i-1}$ and generates the successor state $s_i=\theta(s_{i-1},a_i)$. 

\begin{mydefinition}
  Given $\mathcal{M}$, a \strips\ action model built from $F_v$, and a plan trace $\mathcal{T}=\tup{s_0,a,_1,s_1,\ldots,a_n,s_{n}}$ whose state observation are built with fluents in $F$. The {\bf observation edit distance}, denoted by  $\delta(\mathcal{M},\mathcal{T})$, is the minimal edit distance from $\mathcal{M}$ to any {\em comparable} model $\mathcal{M}'$, such that $\mathcal{M}'$ can produce a valid plan trace $\mathcal{T}$; \[\delta(\mathcal{M},\mathcal{T})=\min_{\forall \mathcal{M}' \rightarrow \mathcal{T}} \delta(\mathcal{M},\mathcal{M}')\]
\end{mydefinition}

The {\em observation edit distance} is mappable into a likelihood with the following expression $1-\frac{\delta(\mathcal{M},\mathcal{O})}{\delta(\mathcal{M},*)}$. Further, following the previous correspondances ({\em false positives} $\equiv$ {\em deletions} and {\em false negatives} $\equiv$ {\em insertions}) we can redefine the {\em precision} and {\em recall} of a learned model with respect to a partially observed plan trace $\mathcal{T}=\tup{s_0,a,_1,s_1,\ldots,a_n,s_{n}}$. In this case, for each action schema $\xi\in\mathcal{M}$, the corresponding number of {\em true positives} is given by $tp(\xi)=|pre(\xi)|+|del(\xi)|+|add(\xi)|-fp(\xi)$ while the {\em true negatives} are given by $tn(\xi)=3\times|F_v(\xi)|-tp(\xi)$

Note that the error of a learned action model could also be defined quantifying the amount of edition required by the observations of the plan execution to match the given model. This would imply defining {\em edit operations} that modify the fluents in the state observations instead of the {\em edit operations} that modify the action schemes~\cite{sohrabi:precognition:IJCAI2016}. Our definition of the {\em observation edit distance} is more practical since normally, $F_v$ is smaller than $F$ because the number of {\em variable objects} is smaller than the number of objects in the state observations. 


\subsection{Computing the edit distance with classical planning}
Our compilation is extensible to compute the {\em observation edit distance} by simply considering that the input \strips\ model $\mathcal{M}$, given in a learning task $\Lambda=\tup{\mathcal{M},\Psi,\mathcal{T}}$, is {\em non-empty}. In other words, now $\mathcal{M}$ is a set of given operator schemas, wherein each $\xi\in\mathcal{M}$ initially contains $head(\xi)$ but also the $pre(\xi)$, $del(\xi)$ and $add(\xi)$ sets. A solution to the planning task resulting from the extended compilation is a sequence of actions that:

\begin{enumerate}
\item {\bf Edits the action model $\mathcal{M}$ to build $\mathcal{M}'$}. A solution plan starts with a {\em prefix} that modifies the preconditions and effects of the action schemes in $\mathcal{M}$ using to the two {\em edit operations} defined above, {\em deletion} and {\em insertion}. In theory, we could implement a third edit operation for {\em substituting} a fluent from a given operator schema. However, and with the aim of keeping a tractable branching factor of the planning instances that result from our compilations, we only implement {\em deletion} and {\em insertion}.
\item {\bf Validates the edited model $\mathcal{M}'$ in the observed plan trace}. The solution plan continues with a postfix that validates the edited model $\mathcal{M}'$ on the given observations $\mathcal{T}$, as explained in Section~\ref{sec:Section5} for the models that are programmed from scratch.
\end{enumerate}

Now $\Lambda$ does not formalize a learning task but the task of editing $\mathcal{M}$ to produce the plan trace $\mathcal{T}$, which results in the edited model $\mathcal{M}'$. The output of the extended compilation is a classical planning task $P_{\Lambda}'=\tup{F_{\Lambda},A_{\Lambda}',I_{\Lambda}',G_{\Lambda}}$:

\begin{itemize}
\item $F_{\Lambda}$ and $G_{\Lambda}$ are defined as in the previous compilation.
\item $I_{\Lambda}'$ contains the fluents from $F$ that encode $s_0$ and $mode_{prog}$ set to true. In addition, the input action model $\mathcal{M}$ is now encoded in the initial state. This means that the fluents $pre_f(\xi)/del_f(\xi)/add_f(\xi)$, $f\in F_v(\xi)$, hold in the initial state iff they appear in $\mathcal{M}$.
\item $A_{\Lambda}'$, comprises the same three kinds of actions of $A_{\Lambda}$. The actions for {\em applying} an already programmed operator schema and the actions for {\em validating} an observation are defined exactly as in the previous compilation. The only difference here is that the actions for {\em programming} the operator schema now implement the two {\em edit operations} (i.e. include actions for {\em inserting} a precondition and for {\em deleting} a negative/positive effect).
\end{itemize}

To illustrate this, the plan of Figure~\ref{fig:plan-pdistance} shows the plan for editing a given {\em blockswold} action model where again the positive effects {\tt\small (handempty)} and {\tt\small (clear ?v1)} of the {\tt\small stack} schema are missing. In this case the edited action model is however validated at the plan shown in Figure~\ref{fig:example-plans}.

\begin{figure}[hbt!]
{\tt\small
00 : (insert\_add\_handempty\_stack)\\
01 : (insert\_add\_clear\_stack\_var1)\\
02 : (apply\_unstack blockB blockA i1 i2)\\
03 : (apply\_putdown blockB i2 i3)\\
04 : (apply\_pickup blockA i3 i4)\\
05 : (apply\_stack blockA blockB i4 i5)\\
06 : (validate\_1)
}
 \caption{\small Plan for editing a given {\em blockswold} schema and validating it at the plan shown in Figure~\ref{fig:example-plans}.}
\label{fig:plan-pdistance}
\end{figure}

Our interest when computing the {\em observation edit distance} is not in the resulting action model $\mathcal{M}'$ but in the number of required {\em edit operations} (insertions and deleitions) for that $\mathcal{M}'$ is validated in the given observations, e.g. $\delta(\mathcal{M},\mathcal{T})=2$ for the example in Figure~\ref{fig:plan-pdistance}. In this case $\delta(\mathcal{M},*)=3\times 2\times (11+5)$ since there are 4 action schemes ({\small\tt pickup}, {\small\tt putdown}, {\small\tt stack} and {\small\tt unstack}) and $|F_v|=|F_v(stack)|=|F_v(unstack)|=11$ while $|F_v(pickup)|=|F_v(putdown)|=5$  (as shown in Section~\ref{sec:Section3}). The {\em observation edit distance} is exactly computed if the classical planning task resulting from our compilation is optimally solved (according to the number of edit actions); is approximated if it is solved with a satisfying planner; and is a less accurate estimate (but faster to be computed) if the solved task is a relaxation of the classical planning task that results from our compilation~\cite{bonet2001planning}.

Last but not least, this compilation is flexible to compute the {\em edit distance} between two {\em comparable} \strips\ action models, $\mathcal{M}$ and $\mathcal{M}'$. A solution to the planning task resulting from this compilation is a sequence of actions that edits the action model $\mathcal{M}$ to produce $\mathcal{M}'$ using to the two {\em edit operations}, deletion and insertion. In this case the edited model is not validated on a sequence of observations or plans but on the given action model $\mathcal{M}'$ that acts as a reference. The sets of fluents $F_{\Lambda}$ and $I_{\Lambda}$ are defined like in the previous compilation. With respect to the actions, $A_{\Lambda}$ again implement the two {\em edit operations} (i.e. include actions for {\em inserting} a precondition and for {\em deleting} a negative/positive effect) but does not contain $\mathsf{apply_{\xi,\omega}}$ or $\mathsf{validate_{i}}$ actions because the \strips\ action model are not validated in any observation of plan executions. Finally, the goals are also different and are now defined by the set of fluents, $pre_f(\xi)/del_f(\xi)/add_f(\xi)$ that represent all the operator schema $\xi\in\mathcal{M'}$, such that $f\in F_v(\xi)$. To illustrate this, the plan of Figure~\ref{fig:plan-mdistance} solves the classical planning task that corresponds to computing the distance between a \emph{blocksworld} action model, where the positive effects {\tt\small (handempty)} and {\tt\small (clear ?v1)} of the {\tt\small stack} schema are missing, and the actual four-operator {\em blocksworld} model. The plan edits first the {\tt\small stack} schema, {\em inserting} these two positive effects. Again our interest is in the number of required {\em edit operations}, e.g. $\delta(\mathcal{M},\mathcal{M'})=2$.

\begin{figure}[hbt!]
{\tt\small
00 : (insert\_add\_handempty\_stack)\\
01 : (insert\_add\_clear\_stack\_var1)
}
 \caption{\small Plan for computing the distance between a \emph{blocksworld} action model, where the positive effects {\tt\small (handempty)} and {\tt\small (clear ?v1)} of the {\tt\small stack} schema are missing, and the actual four-operator {\em blocksworld} model.}
\label{fig:plan-mdistance}
\end{figure}









