% Template for Elsevier CRC journal article
% version 1.1 dated 16 March 2010

% This file (c) 2010 Elsevier Ltd.  Modifications may be freely made,
% provided the edited file is saved under a different name

% This file contains modifications for Procedia Computer Science
% but may easily be adapted to other journals

% Changes since version 1.0
% - elsarticle class option changed from 1p to 3p (to better reflect CRC layout)

%-----------------------------------------------------------------------------------

%% This template uses the elsarticle.cls document class and the extension package ecrc.sty
%% For full documentation on usage of elsarticle.cls, consult the documentation "elsdoc.pdf"
%% Further resources available at http://www.elsevier.com/latex

%-----------------------------------------------------------------------------------

%%%%%%%%%%%%%%%%%%%%%%%%%%%%%%%%%%%%%%%%%%%%%%
%%%%%%%%%%%%%%%%%%%%%%%%%%%%%%%%%%%%%%%%%%%%%%
%%                                          %%
%% Important note on usage                  %%
%% -----------------------                  %%
%% This file must be compiled with PDFLaTeX %%
%% Using standard LaTeX will not work!      %%
%%                                          %%
%%%%%%%%%%%%%%%%%%%%%%%%%%%%%%%%%%%%%%%%%%%%%%
%%%%%%%%%%%%%%%%%%%%%%%%%%%%%%%%%%%%%%%%%%%%%%

%% The '3p' and 'times' class options of elsarticle are used for Elsevier CRC
\documentclass[3p,times]{elsarticle}

%% The `ecrc' package must be called to make the CRC functionality available
\usepackage{ecrc}
%% The ecrc package defines commands needed for running heads and logos.
%% For running heads, you can set the journal name, the volume, the starting page and the authors

%% set the volume if you know. Otherwise `00'
\volume{00}

%% set the starting page if not 1
\firstpage{1}

%% Give the name of the journal
\journalname{Artificial Intelligence}

%% Give the author list to appear in the running head
%% Example \runauth{C.V. Radhakrishnan et al.}
\runauth{}

%% The choice of journal logo is determined by the \jid and \jnltitlelogo commands.
%% A user-supplied logo with the name <\jid>logo.pdf will be inserted if present.
%% e.g. if \jid{yspmi} the system will look for a file yspmilogo.pdf
%% Otherwise the content of \jnltitlelogo will be set between horizontal lines as a default logo

%% Give the abbreviation of the Journal.
\jid{procs}

%% Give a short journal name for the dummy logo (if needed)
\jnltitlelogo{Artificial Intelligence}

%% Hereafter the template follows `elsarticle'.
%% For more details see the existing template files elsarticle-template-harv.tex and elsarticle-template-num.tex.

%% Elsevier CRC generally uses a numbered reference style
%% For this, the conventions of elsarticle-template-num.tex should be followed (included below)
%% If using BibTeX, use the style file elsarticle-num.bst

%% End of ecrc-specific commands
%%%%%%%%%%%%%%%%%%%%%%%%%%%%%%%%%%%%%%%%%%%%%%%%%%%%%%%%%%%%%%%%%%%%%%%%%%

%% The amssymb package provides various useful mathematical symbols
\usepackage{amssymb}
%% The amsthm package provides extended theorem environments
%%\usepackage{amsthm}

%% The lineno packages adds line numbers. Start line numbering with
%% \begin{linenumbers}, end it with \end{linenumbers}. Or switch it on
%% for the whole article with \linenumbers after \end{frontmatter}.
%% \usepackage{lineno}

%% natbib.sty is loaded by default. However, natbib options can be
%% provided with \biboptions{...} command. Following options are
%% valid:

%%   round  -  round parentheses are used (default)
%%   square -  square brackets are used   [option]
%%   curly  -  curly braces are used      {option}
%%   angle  -  angle brackets are used    <option>
%%   semicolon  -  multiple citations separated by semi-colon
%%   colon  - same as semicolon, an earlier confusion
%%   comma  -  separated by comma
%%   numbers-  selects numerical citations
%%   super  -  numerical citations as superscripts
%%   sort   -  sorts multiple citations according to order in ref. list
%%   sort&compress   -  like sort, but also compresses numerical citations
%%   compress - compresses without sorting
%%
%% \biboptions{comma,round}

% \biboptions{}

% if you have landscape tables
\usepackage[figuresright]{rotating}

\usepackage{amsthm}
\usepackage{amsmath}
\usepackage{amssymb}
\usepackage{times}
\usepackage{helvet}
\usepackage{courier}
\usepackage{pstricks}
\usepackage{pst-node}
\usepackage{multirow}
\usepackage{listings}

\usepackage{pgf}
\usepackage{tikz}
\usetikzlibrary{calc,backgrounds,positioning,fit}
\usepackage{subcaption}
\usetikzlibrary{arrows,automata}
\usepackage{arydshln}



% put your own definitions here:
%   \newcommand{\cZ}{\cal{Z}}
%   \newtheorem{def}{Definition}[section]
%   ...

\newtheorem{mytheorem}{Theorem}
\newtheorem{mylemma}[mytheorem]{Lemma}
\newtheorem{mydefinition}[mytheorem]{Definition}
\newtheorem{myconstruction}{Construction}


\mathchardef\mh="2D
\newcommand{\pre}{\mathsf{pre}}  % precondition
\newcommand{\eff}{\mathsf{eff}}  % effect
\newcommand{\cond}{\mathsf{cond}}   % conditional effect
\newcommand{\add}{\mathsf{add}}  % add effect
\newcommand{\del}{\mathsf{del}}  % delete effect
\newcommand{\PE}{\mathrm{PE}}     % precondition
\newcommand{\PSPACE}{\mathrm{PSPACE}}     % precondition
\newcommand{\NPSPACE}{\mathrm{NPSPACE}}     % precondition
\newcommand{\strips}{\textsc{Strips}}     % precondition


\newcommand{\pbox}[1]{\makebox[2em][l]{#1}}

\newcommand{\tup}[1]{{\langle #1 \rangle}}

\lstset{
  basicstyle=\ttfamily,
  mathescape
}

% add words to TeX's hyphenation exception list
%\hyphenation{author another created financial paper re-commend-ed Post-Script}

% declarations for front matter

\begin{document}

\begin{frontmatter}

%% Title, authors and addresses

%% use the tnoteref command within \title for footnotes;
%% use the tnotetext command for the associated footnote;
%% use the fnref command within \author or \address for footnotes;
%% use the fntext command for the associated footnote;
%% use the corref command within \author for corresponding author footnotes;
%% use the cortext command for the associated footnote;
%% use the ead command for the email address,
%% and the form \ead[url] for the home page:
%%
%% \title{Title\tnoteref{label1}}
%% \tnotetext[label1]{}
%% \author{Name\corref{cor1}\fnref{label2}}
%% \ead{email address}
%% \ead[url]{home page}
%% \fntext[label2]{}
%% \cortext[cor1]{}
%% \address{Address\fnref{label3}}
%% \fntext[label3]{}

\dochead{}
%% Use \dochead if there is an article header, e.g. \dochead{Short communication}

\title{Learning, Evaluation and Recognition of \strips\ Action Models with Classical Planning}
\author[label1]{Diego Aineto}
\author[label1]{Sergio Jim\'{e}nez Celorrio}
\author[label1]{Eva Onaindia}
\address[label1]{Department of Computer Systems and Computation, Universitat Polit\`ecnica de Val\`encia. Spain}


%% use optional labels to link authors explicitly to addresses:
%% \author[label1,label2]{<author name>}
%% \address[label1]{<address>}
%% \address[label2]{<address>}


\begin{abstract}
  This paper presents a novel approach for learning \strips\ action models from observations of plan executions that compiles this learning task into classical planning. The compilation approach is flexible to various amount and kind of available input knowledge; learning examples can range from plans (with their corresponding initial state), sequences of state observations or just a set of initial and final states (where no intermediate action or state is a priori known). The compilation accepts also partially specified action models and can be used to validate whether the observation of a plan execution follows a given \strips\ action model, even if this model is not fully specified. What is more, the compilation is extensible to assess how well a given \strips\ action model matches a {\em test set} with observations of plan executions. This is relevant since it allows us to assessing the learned models with respect to the true models but with respect to test sets of observations of plan executions, which is a necessary step for the task of model recognition. The empirical performance of our compilation approach is evaluated learning action models for a wide range of classical planning domains from the International Planning Competition (IPC). 
\end{abstract}

\begin{keyword}
Classical planning\sep Learning action models\sep Generalized planning\sep Model recognition
%% keywords here, in the form: keyword \sep keyword

%% MSC codes here, in the form: \MSC code \sep code
%% or \MSC[2008] code \sep code (2000 is the default)
\end{keyword}

\end{frontmatter}

%%
%% Start line numbering here if you want
%%
% \linenumbers

%% main text

%% The Appendices part is started with the command \appendix;
%% appendix sections are then done as normal sections
%% \appendix

%% \section{}
%% \label{}

% HLP: Expressiveness is pushed when pure compilations are used. Otherwise we just use them.

\section{Introduction}
\label{sec:Section1}
Besides {\em plan synthesis}~\cite{ghallab2004automated}, planning action models are also useful for {\em plan/goal recognition}~\cite{ramirez2012plan}. At both planning tasks, an automated planner is required to reason about action models that correctly and completely capture the possible world transitions~\cite{geffner:book:2013}. Unfortunately, building planning action models is complex, even for planning experts, and this knowledge acquisition task is a bottleneck that limits the potential of AI planning~\cite{kambhampati:modellite:AAAI2007}.

Machine Learning (ML) has shown to be able to compute a wide range of different kinds of models from examples~\cite{michalski2013machine}. The application of inductive ML to the learning of \strips\ action models, the vanilla action model for automated planning~\cite{fikes1971strips}, is not straightforward though:
\begin{itemize}
\item The {\em input} to ML algorithms (the {\em learning/training} data) usually are finite vectors encoding the value of fixed features in a given set of objects. The input for learning planning action models are observations of plan executions (where each plan possibly has a different length and may involve a different number of objects).
\item The {\em output} of ML algorithms usually is a scalar value (an integer, in the case of {\em classification} tasks, or a real value, in the case of {\em regression} tasks). When learning \strips\ action models the output is, for each action, the sets of {\em preconditions}, {\em negative} and {\em positive effects} that define which state transitions are possible.
\end{itemize}

Motivated by recent advances on the synthesis of different kinds of generative models with classical planning~\cite{bonet2009automatic,segovia2016generalized,segovia2016hierarchical,segovia2017generating}, this paper introduces an innovative approach for learning \strips\ action models that can be defined as a classical planning compilation. A solution to the classical planning task that results from our compilation is a sequence of actions that determines the preconditions and effects of a \strips\ model such that this \strips\ model can satisfy all the observations of plan executions that are given as input.

The compilation approach is appealing by itself because it leverages off-the-shelf planners for learning and because its practicality allow us to report model learning results over a wide range of IPC planning domains. Moreover, it opens up a way towards \emph{bootstrapping} planning action models, enabling a planner to gradually learn/update its action model. Apart from these, the compilation approach presents the following contributions:
\begin{enumerate}
\item {\em Flexibility}. The compilation is flexible to various amount and kind of available input knowledge. Learning examples can range from a set of plans (with their corresponding initial state) or state observations, to just a set of initial and final states where no intermediate action or state is observed. Further, the compilation is robust to partially obervability of the intermediate and final states. Last but not least, the compilation also accepts previous knowledge about the structure of the actions in the form of partially specified action models, where some preconditions and effects are a priori known. % Also accepts background knowledge

\item {\em Model Evaluation}. The compilation can assess how well a given \strips\ action model matches a {\em test set} with observations of plan executions. This allows to assess learning performance without having the actual model and, like in the leaning task, the evaluation of strips model with our compilation is flexible to various amount and kind of available input knowledge. Further, our compilation is extensible to accept a learned model as input besides the observations of plan executions to transform the input model into a new model that induces the observations whilst assessing the amount of edition required by the input model to induce the given observations. 
\item {\em Model Recognition}. Our mechanism for model evaluation allows us to define the model recognition task. This task is relevant since different generative models like policies, programs, grammars or different forms of domain-specific control knowledge are STRIPS compilable. In this sense, our work poses a general framework to assess the validation of a generative model (provided that it is STRIPS compilable) with a given set of observations.
\end{enumerate}

A first description of the compilation previously appeared in the conference paper~\cite{aineto2018learning}. Compared to the conference paper, this work includes the following novel material:
\begin{itemize}
\item A unified formulation for learning and evaluating \strips\ action models from observed plans but also from state observations.
\item The formulation of {\em model recognition} as the task of selecting, from a given set, the action model that best matches a {\em test set} with observations of plan executions.
\item Leveraging {\em background knowledge} (given as planning constraints either in the form of {\em state constraints} or {\em trajectory constraints}) for learning/evaluating/recognizing \strips\ action models.
\item A more complete empirical evaluation of the compilation approach. Our evaluation analyses how the amount of input knowledge and how {\em partial state observability} (some of the fluents either with {\em positive} or {\em negative} value of the intermediate states are missing because they cannot be observed) affect to the performance of the compilation approach.
\end{itemize}  

Section~\ref{sec:Section2} reviews related work on learning planning action models. Section~\ref{sec:Section3} introduces the classical planning model with {\em conditional effects} (a requirement of the proposed compilation) and the \strips\ action model (the target of our learning/evaluation/recognition tasks). Section~\ref{sec:Section4} formalizes the learning of \strips\ action models with regard to different amount and kind of available input knowledge. Sections~\ref{sec:Section5} and ~\ref{sec:Section6} describe our compilation approach for addressing the formalized learning tasks its extension to evaluate and recognize \strips\ action models. Section~\ref{sec:Section7} reports the data collected in a two-fold empirical evaluation of our learning approach: First, the learned \strips\ action models are tested with observations of plan executions and second, the learned models are compared to the actual models. Finally, Section ~\ref{sec:Section8} discusses the strengths and weaknesses of the compilation approach and proposes several opportunities for future research.



\section{Related work}
\label{sec:Section2}
Back in the 90's various systems aimed learning operators mostly via interaction with the environment. {\sc LIVE} captured and formulated observable features of objects and used them to acquire and refine operators \cite{ShenS89}. {\sc OBSERVER} updated preconditions and effects by removing and adding facts, respectively, accordingly to observations \cite{Wang95learningby}. These early works were based on lifting the observed states supported by exploratory plans or external teachers, but none provided a theoretical justification for this second source of knowledge.

More recent work on learning planning action models \cite{WalshL08} shows that although learning \strips\ operators from pure interaction with the environment requires an exponential number of samples, access to an external teacher can provide solution traces on demand.

Whilst the aforementioned works deal with full state observability,action model learning has also been studied in domains where there is partial or missing state observability. {\sf ARMS} works when no partial intermediate state is given. It defines a set of weighted constraints that must hold for the plans to be correct, and solves the weighted propositional satisfiability problem with a MAX-SAT solver~\cite{yang2007learning}. In order to efficiently solve the large MAX-SAT representations, {\sf ARMS} implements a hill-climbing method that models the actions approximately. %and so it may output an inconsistent model
{\sc SLAF} also deals with partial observability~\cite{amir:alearning:JAIR08}. Given a formula representing the initial belief state, a sequence of executed actions and the corresponding partially observed states, it builds a complete explanation of observations by models of actions through a CNF formula. The learning algorithm updates the formula of the belief state with every action and observation in the sequence. This update makes sure that the new formula represents all the transition relations consistent with the actions and observations. The formula returned at the end includes all consistent models, which can then be retrieved with additional processing.

Unlike the previous approaches, the one described in \cite{MouraoZPS12} deals with both missing and noisy predicates in the observations. An action model is first learnt by constructing a set of kernel classifiers which tolerate noise and partial observability and then \strips rules are derived from the classifiers' parameters.

{\sf LOCM} only requires the example plans as input without need for providing information about predicates or states~\cite{cresswell2013acquiring}. This makes {\sf LOCM} be most likely the learning approach that works with the least information possible. The lack of available information is addressed by LOCM by exploiting assumptions about the kind of domain model it has to generate. Particularly, it assumes a domain consists of a collection of objects (sorts) whose defined set of states can be captured by a parameterized Finite State Machine (FSM). The intuitive assumptions of {\sf LOCM}
, like the continuity of object transitions or the association of parameters between consecutive actions in the training sequence,
yield a learning model heavily reliant on the kind of domain structure. The inability of {\sf LOCM} to properly derive domain theories where the state of a sort is subject to different FSMs is later overcome by {\sf LOCM2} by forming separate FSMs, each containing a subset of the full transition set for the sort~\cite{cresswell2011generalised}. {\sf LOP} ({\sf LOCM} with Optimized Plans ~\cite{gregory2015domain}), the last contribution of the {\sf LOCM} family, addresses the problem of inducing static predicates. Because {\sf LOCM} approaches induce similar models for domains with similar structures, they face problems at generating models for domains that are only distinguished by whether or not they contain static relations (e.g. {\em blocksworld} and {\em freecell}). In order to mitigate this drawback, {\sf LOP} applies a post-processing step after the {\sf LOCM} analysis which requires additional information about the plans, namely a set of optimal plans to be used in the learning phase.

Recenctly classical planning compilations have been defined to learn different kinds of generative models from examples. The existing compilations for computing FSCs for generalized planning follow a {\it top-down} approach that interleaves {\it programming} the FSC with validating it and hence, they tightly integrate planning and generalization. To keep the computation of FSCs tractable, they limit the space of possible solutions bounding the maximum size of the FSC. In addition, they impose that the instances to solve share, not only the domain theory (actions and predicates schemes) but the set of fluents~\cite{javi-Gplanning-ICAPS16} or a subset of {\it observable} fluents~\cite{Geffner:FSM:AAAI10}. Programs increase the readability of FSCs separating the control-flow structures from the primitive actions. Like FSCs, programs can also be computed following a {\it top-down} approach, e.g.~exploiting compilations that program and validate the program on instances with the same state and action space~\cite{javi-Gplanning-ICAPS16}. Since these {\it top-down} approaches search in the space of solutions, it is helpful to limit the set of different control-flow instructions. For instance using only {\it conditional gotos} that can both implement branching and loops~\cite{Jimenez15}. 



\section{Background}
\label{sec:Section3}
This section defines the planning model used on this work, {\em classical planning with conditional effects}, and the target of the learning/evaluation/recognition tasks addressed in the paper, a \strips\ action model.

\subsection{Classical planning with conditional effects}
We use $F$ to denote the set of {\em fluents} (propositional variables) describing a state. A {\em literal} $l$ is a valuation of a fluent $f\in F$, i.e. either~$l=f$ or $l=\neg f$. A set of literals $L$ represents a partial assignment of values to fluents (without loss of generality, we will assume that $L$ does not assign conflicting values to any fluent). We use $\mathcal{L}(F)$ to denote the set of all literal sets on $F$, i.e.~all partial assignments of values to fluents.

A {\em state} $s$ is a full assignment of values to fluents, i.e. $|s|=|F|$, so the size of the state space is $2^{|F|}$. Explicitly including negative literals $\neg f$ in states simplifies subsequent definitions but often, we will abuse notation by defining a state $s$ only in terms of the fluents that are true in $s$, as is common in \strips\ planning.

A {\em classical planning frame} is a tuple $\Phi=\tup{F,A}$, where $F$ is a set of fluents and $A$ is a set of actions. An action $a\in A$ is defined with {\em preconditions}, $\pre(a)\subseteq\mathcal{L}(F)$, {\em negative effects} $\eff^-(a)\subseteq\mathcal{L}(F)$ and {\em positive effects}, $\eff^+(a)\subseteq\mathcal{L}(F)$. We say that an action $a\in A$ is {\em applicable} in a state $s$ iff $\pre(a)\subseteq s$. The result of applying $a$ in $s$ is the {\em successor state} denoted by $\theta(s,a)=\{s\setminus\eff^-(a))\cup\eff^+(a)\}$.

%\begin{itemize}
%\item $\pre(a)\subseteq\mathcal{L}(F)$, called {\em preconditions}, the literals that must hold for the action $a\in A$ to be applicable.
%\item $\eff^+(a)\subseteq\mathcal{L}(F)$, called {\em positive effects}, that defines the fluents set to true by the application of the action $a\in A$.
%\item $\eff^-(a)\subseteq\mathcal{L}(F)$, called {\em negative effects}, that defines the fluents set to false by the action application.
%\end{itemize}


An action $a\in A$ with conditional effects is defined as a set of {\em preconditions} $\pre(a)\in\mathcal{L}(F)$ and a set of {\em conditional effects} $\cond(a)$. Each conditional effect $C\rhd E\in\cond(a)$ is composed of two sets of literals $C\in\mathcal{L}(F)$, the {\em condition}, and $E\in\mathcal{L}(F)$, the {\em effect}. An action $a\in A$ is {\em applicable} in a state $s$ if and only if $\pre(a)\subseteq s$, and the {\em triggered effects} resulting from the action application are the effects whose conditions hold in $s$:
\[
triggered(s,a)=\bigcup_{C\rhd E\in\cond(a),C\subseteq s} E,
\]

The result of applying action $a$ in state $s$ is the {\em successor} state $\theta(s,a)=\{s\setminus\eff_c^-(s,a))\cup\eff_c^+(s,a)\}$ where $\eff_c^-(s,a)\subseteq triggered(s,a)$ and $\eff_c^+(s,a)\subseteq triggered(s,a)$ are, respectively, the triggered {\em negative} and {\em positive} effects.


A {\em classical planning problem} is a tuple $P=\tup{F,A,I,G}$, where $I$ is an initial state and $G\subseteq\mathcal{L}(F)$ is a goal condition. A {\em plan} for $P$ is an action sequence $\pi=\tup{a_1, \ldots, a_n}$ that induces the {\em state trajectory} $\tup{s_0, s_1, \ldots, s_n}$ such that $s_0=I$ and, for each {\small $1\leq i\leq n$}, $a_i$ is applicable in $s_{i-1}$ and generates the successor state $s_i=\theta(s_{i-1},a_i)$. The {\em plan length} is denoted with $|\pi|=n$ . A plan $\pi$ {\em solves} $P$ iff $G\subseteq s_n$, i.e.~if the goal condition is satisfied at the last state reached after following the application of the plan $\pi$ in the initial state $I$.


\subsection{\strips\ action schemas}
This work addresses the learning of PDDL action schemas that follow the \strips\ requirement~\cite{mcdermott1998pddl,fox2003pddl2}. Figure~\ref{fig:stack} shows the {\em stack} action schema, coded in PDDL, from a four-operator {\em blocksworld}~\cite{slaney2001blocks}.

\begin{figure}[hbt!]
\begin{footnotesize}
\begin{verbatim}
(:action stack
 :parameters (?v1 ?v2 - object)
 :precondition (and (holding ?v1) (clear ?v2))
 :effect (and (not (holding ?v1)) (not (clear ?v2)) (handempty) (clear ?v1) (on ?v1 ?v2)))
\end{verbatim}
\end{footnotesize}
 \caption{\small \strips\ operator schema coding, in PDDL, the {\em stack} action from a four-operator {\em blocksworld}.}
\label{fig:stack}
\end{figure}

To formalize the output of the learning task, we assume that fluents $F$ are instantiated from a set of {\em predicates} $\Psi$, as in PDDL. Each predicate $p\in\Psi$ has an argument list of arity $ar(p)$. Given a set of {\em objects} $\Omega$, the set of fluents $F$ is induced by assigning objects in $\Omega$ to the arguments of predicates in $\Psi$, i.e.~$F=\{p(\omega):p\in\Psi,\omega\in\Omega^{ar(p)}\}$ s.t. $\Omega^k$ is the $k$-th Cartesian power of $\Omega$.

Let $\Omega_v=\{v_i\}_{i=1}^{\operatorname*{max}_{a\in A} ar(a)}$ be a new set of objects ($\Omega\cap\Omega_v=\emptyset$), denoted as {\em variable names}, and that is bound by the maximum arity of an action in a given planning frame. For instance, in a three-block {\em blocksworld} $\Omega=\{block_1, block_2, block_3\}$ while $\Omega_v=\{v_1, v_2\}$ because the operators with the maximum arity, {\small\tt stack} and {\small\tt unstack}, have arity two. We define $F_v$, a new set of fluents s.t. $F\cap F_v=\emptyset$, that results from instantiating $\Psi$ using only the objects in $\Omega_v$, i.e. the variable names, and that defines the elements that can appear in an action schema. For the {\em blocksworld}, $F_v$={\small\tt\{handempty, holding($v_1$), holding($v_2$), clear($v_1$), clear($v_2$), ontable($v_1$), ontable($v_2$), on($v_1,v_1$), on($v_1,v_2$), on($v_2,v_1$), on($v_2,v_2$)\}}.

For a given operator schema $\xi$, we define $F_v(\xi)\subseteq F_v$ as the subset of fluents that represent the elements that can appear in that action schema. For instance, for the {\em stack} action schema $F_v({\tt stack})=F_v$ while $F_v({\tt pickup})$={\small\tt\{handempty, holding($v_1$), clear($v_1$), ontable($v_1$), on($v_1,v_1$)\}} excludes the fluents from $F_v$ that involve $v_2$ because the action header {\small\tt pickup($v_1$)} contains the single parameter $v_1$.

We assume also that actions $a\in A$ are instantiated from \strips\ operator schemas $\xi=\tup{head(\xi),pre(\xi),add(\xi),del(\xi)}$ where:
\begin{itemize}
\item $head(\xi)=\tup{name(\xi),pars(\xi)}$, is the operator {\em header} defined by its name and the corresponding {\em variable names}, $pars(\xi)=\{v_i\}_{i=1}^{ar(\xi)}$. The headers of a four-operator {\em blocksworld} are {\small\tt pickup($v_1$), putdown($v_1$), stack($v_1,v_2$)} and {\small\tt unstack($v_1,v_2$)}.
\item The preconditions $pre(\xi)\subseteq F_v$, the negative effects $del(\xi)\subseteq F_v$, and the positive effects $add(\xi)\subseteq F_v$ such that, $del(\xi)\subseteq pre(\xi)$, $del(\xi)\cap add(\xi)=\emptyset$ and $pre(\xi)\cap add(\xi)=\emptyset$.
\end{itemize}
Therefore, given the set of predicates $\Psi$ and the header of the operator schema $\xi$, $2^{2|F_v(\xi)|}$ defines the size of space of the possible \strips\ models for that operator, given that the previous constraints require that negative effects appear as preconditions and that they cannot be positive effects and also, that a positive effect cannot appear as a precondition. For instance, this number is 4194304 for the blocksworld {\tt stack} operator while is only 1024 for the {\tt pickup} operator.

Last but not least, we say that two \strips\ operator schemes $\xi$ and $\xi'$ are {\em comparable} if both schemas have the same headers so they can be built from the same set of possible elements. Formally, iff $head(\xi)=head(\xi')$ and it holds that $F_v(\xi)=F_v(\xi')$. For instance we can claim that the {\tt stack} and {\tt unstack} blocksworld operators are {\em comparable} while  {\tt stack} and {\tt pickup} are not. Likewise we say that two \strips\ action models $\mathcal{M}$ and $\mathcal{M}'$ are {\em comparable} iff there is a bijective function $\mathcal{M} \mapsto \mathcal{M}^*$ that maps every $\xi\in\mathcal{M}$ to a comparable action schema $\xi'\in\mathcal{M'}$ and viceversa.



\section{Learning \strips\ action models}
\label{sec:Section4}
Learning \strips\ action models from fully available input knowledge, i.e. from plans where the {\em pre-} and {\em post-states} of every action are available, is straightforward~\cite{jimenez2012review}:
\begin{itemize}
  \item {\em Preconditions} are derived lifting the minimal set of literals that appears in all the pre-states of the corresponding actions. 
  \item {\em Effects} are derived lifting the literals that change between the pre and post-state of the corresponding action executions.
\end{itemize}
This section formalizes more challenging learning tasks, where less input knowledge is available.

\subsection{Learning from observations of plan executions}
The first learning task corresponds to observing an agent acting in the world but watching only the states that result of its actions, the actual executed actions are unobserved. This learning task is formalized as a tuple $\Lambda=\tup{\mathcal{M},\mathcal{O},\Psi}$:
\begin{itemize}
\item $\mathcal{M}$, the set of {\em empty} operator schemas, wherein each $\xi\in\mathcal{M}$ is only composed of $head(\xi)$. 
\item $\mathcal{O}=\tup{s_0,s_1,\ldots,s_{n}}$ is the sequence of {\em state observations} obtained watching the execution of an {\em unobserved} plan $\pi=\tup{a_1, \ldots, a_n}$ such that, for each {\small $1\leq i\leq n$}, $a_i$ is applicable in $s_{i-1}$ and generates the successor state $s_i=\theta(s_{i-1},a_i)$. We assume that state observations are noiseless, meaning that if the value of a fluent is observed it is correct. However, states in $\mathcal{O}$ are not assumed to be a fully observed states. States in $\mathcal{O}$ can be {\em partial states}, in the sense that some of the observed fluents are missing, because it is unknown whether their value is either positive or negative. 
\item $\Psi$ is the set of predicates that define the abstract state space of a given planning frame. This could be inferred from the state observations provided that at least a state in $\mathcal{O}$ is a full state.
\end{itemize}

A solution to a $\Lambda=\tup{\mathcal{M},\mathcal{O},\Psi}$ learning task is a set of operator schema $\mathcal{M}'$ compliant with the input model $\mathcal{M}$, the state observation sequence $\mathcal{O}$ and the predicates $\Psi$. In this learning scenario, a solution must not only determine a possible \strips\ action model but also the unobserved plan $\pi$, that explains the given observations with the learned \strips\ model. Figure~\ref{fig:example-observations} shows an example of a $\Lambda=\tup{\mathcal{M},\mathcal{O},\Psi}$ task for learning the {\em blocksworld} \strips\ action model from the five-state observations sequence that corresponds to inverting a 2-block tower. 

\begin{figure}[hbt!]
{\footnotesize\tt ;;;;;; Headers in $\mathcal{M}$}
\begin{footnotesize}
\begin{verbatim}
(pickup v1) (putdown v1) (stack v1 v2} (unstack v1 v2)
\end{verbatim}
\end{footnotesize}
\vspace{0.2cm}
{\footnotesize\tt ;;;;;; Predicates $\Psi$}
\begin{footnotesize}
\begin{verbatim}
(handempty) (holding ?o  - object) (clear ?o - object) (ontable ?o - object)
(on ?o1 - object ?o2 - object)
\end{verbatim}
\end{footnotesize}
\vspace{0.2cm}
{\footnotesize\tt ;;;;;; Observations $\mathcal{O}$}
\begin{footnotesize}
\begin{verbatim}
;;; observation #0
(clear B) (on B A) (ontable A) (handempty)

;;; observation #1
(holding B) (clear A) (ontable A)

;;; observation #2
(clear A) (ontable A) (clear B) (ontable B) (handempty)

;;; observation #3
(holding A) (clear B) (ontable B)

;;; observation #4
(clear A) (on A B) (ontable B) (handempty)
\end{verbatim}
\end{footnotesize}
 \caption{\small Task $\Lambda=\tup{\mathcal{M},\mathcal{O},\Psi}$ for learning a \strips\ action model in the {\em blocksworld} from a sequence of five state observations.}
\label{fig:example-observations}
\end{figure}

Here we redefine the learning task to cover the scenario where some of the actions executed by the observed agent are also available. In the case all actions in the plan are known, $\mathcal{O}$ should contain partial states because otherwise, the learning task becomes trivial as explained. The learning task is now formalized as $\Lambda=\tup{\mathcal{M},\mathcal{O},\Psi,\pi}$, where:
\begin{itemize}
\item The plan $\pi=\tup{a_1, \ldots, a_n}$, is an action sequence that produces the sequence of state observations given in $\mathcal{O}$. Again we assume that action observations are noiseless, meaning that if the value of an action is observed it is correct. However, some actions maybe unoserved. When the plan is {\em diverse} enough, i.e. it contains at least one ground action for each of the aimed action schemes, the set of {\em empty} operator schemas $\mathcal{M}$ can be inferred from the input plan $\pi$.
\end{itemize}

Figure~\ref{fig:example-plans} shows an example of a learning task $\Lambda=\tup{\mathcal{M},\mathcal{O},\Psi,\pi}$, that corresponds to observing the execution of a four-action plan for inverting a two-block tower. The exaple skips $\mathcal{M},\Psi$ since they are the same as in Figure~\ref{fig:example-observations}. In this example only the first and last states were observed and intermediate states $s_1$, $s_2$ and $s_3$ are fully unknown.

\begin{figure}[hbt!]
{\footnotesize\tt ;;;;;; Observations $\mathcal{O}$}
\begin{footnotesize}
\begin{verbatim}
;;; observation #0
(clear B) (on B A) (ontable A) (handempty)

;;; observation #4
(clear A) (on A B) (ontable B) (handempty)
\end{verbatim}
\end{footnotesize}
  
{\footnotesize\tt ;;; Plan $\pi$}
\begin{footnotesize}
\begin{verbatim}
0: (unstack B A)
1: (putdown B)
2: (pickup A)
3: (stack A B)
\end{verbatim}
\end{footnotesize}

 \caption{\small Task $\Lambda=\tup{\mathcal{M},\mathcal{O},\Psi,\pi}$ for learning a {\em blocksworld} \strips\ action model from a four-action plan and two state observations.}
\label{fig:example-plans}
\end{figure}

In some cases, we may not require to start learning from scratch that is, the operator schemas in $\mathcal{M}$ may be not {\em empty} but {\em partially specified} operator schemes (preconditions and some positive or negative effects are a priori known). Such scenario is relevant for {\em Policy learning}. A policy is function that maps states into actions and represent the conditions under which actions should be applied to achieve certain goals. Given an action model $\mathcal{M}$ the task of learning a policy that is compliant with a given set of observations of plan executions can be defined as the task of learning extra preconditions for each of the action schemes $\xi\in\mathcal{M}$. These extra preconditions capture when actions can be applied according to the policy. If for each reachable state the policy defines only a single applicable action the policy is a {\em full policy} otherwise, we say it is a {\em partial policy}. In the general case, learning a {\em full policy} for an arbitrary planning task is complex because it may require computing {\em high-level state features}~\cite{lotinac2016automatic}.

%\begin{itemize}
%\item {\em Goal learning}. The goal condition is the set of constraints that defines the subset of states that are considered goal states. Given an action model $\mathcal{M}$ the task of learning the goal conditions for a a given set of observations of plan executions can be defined as learning the preconditions of a extra action that is always applied in the last place and whose only effect is marking that the goals of the planning task are reached.
%\end{itemize}
  
The previous definitions formalize the learning of \strips\ action models from the observation of a single plan execution. These definitions are extensible to the more general case where the available input data is obtained from the execution of multiple plans:
\begin{itemize}
\item When learning from states observations, $\Lambda=\tup{\mathcal{M},\mathcal{O'},\Psi}$ is still a valid formalization by simply considering that now $\mathcal{O'}=\tup{s_0^1,s_1^1,\ldots,s_{n}^1,\ldots,s_0^t,s_1^t,\ldots,s_{n}^t\ldots,s_0^{\tau},s_1^{\tau},\ldots,s_{n}^{\tau}}$ is a sequence of {\em state observations} obtained watching the execution of a serie of {\em unobserved} plans $\pi^t=\tup{a_1, \ldots, a_n^t}$, {\tt\small $1\leq t\leq \tau$}, one after the other. This means that, for each {\small $1\leq i\leq n^t$}, $a_i^t\in \pi^t$ is applicable in $s_{i-1}^t$ and generates the successor state $s_i^t=\theta(s_{i-1}^t,a_i^t)$.
  \item When learning from a set of observed plans, the task is defined as $\Lambda=\tup{\mathcal{M},\Psi,\mathcal{O'},\Pi}$, where $\Pi=\{\pi_1,\ldots,\pi_{\tau}\}$ is the given sequence of example plans producing the corresponding sequence of state observations $\mathcal{O'}$.
\end{itemize}  


\section{Learning \strips\ action models with classical planning}
\label{sec:Section5}
Our approach for addressing a $\Lambda$ learning task is compiling it into a classical planning task $P_{\Lambda}$ with conditional effects. A planning compilation is a suitable approach because computing a solution for $\Lambda$ involves, not only determining the \strips\ action model $\mathcal{M}'$, but also the {\em unobserved} plans that explain the given inputs to the learning task. The intuition behind the compilation is that a solution to the resulting classical planning task is a sequence of actions that:

\begin{enumerate}
\item {\bf Programs the action model $\mathcal{M}'$}. A solution plan starts with a {\em prefix} that, for each $\xi\in\mathcal{M}$, determines which fluents $f\in F_v(\xi)$ belong to its $pre(\xi)$, $del(\xi)$ and $add(\xi)$ sets.
\item {\bf Validates the action model $\mathcal{M}'$}. The solution plan continues with a postfix that reproduces the given input knowledge (the available observations of the plan executions) but using the programmed action model $\mathcal{M}'$.
\end{enumerate}


\subsection{Learning from state observations}
Here we formalize the compilation for learning \strips\ action models with classical planning. Given a learning task $\Lambda=\tup{\mathcal{M},\mathcal{O},\Psi}$ the compilation outputs a classical planning task $P_{\Lambda}=\tup{F_{\Lambda},A_{\Lambda},I_{\Lambda},G_{\Lambda}}$:
\begin{itemize}
\item $F_{\Lambda}$ contains:
\begin{itemize}
\item The set of fluents $F$ built instantiating the predicates $\Psi$ with the objects $\Omega$ that appear in the input observations, i.e. {\tt\small block A} and {\tt\small block B} in Figure~\ref{fig:example-observations}. Formally, $\Omega=\bigcup_{\small s\in\mathcal{O}} obj(s)$, where $obj$ is a function that returns the objects that appear in a given state.
\item Fluents $pre_f(\xi)$, $del_f(\xi)$ and $add_f(\xi)$, for every $f\in F_v(\xi)$, that represent the programmed action model. If a fluent $pre_f(\xi)/del_f(\xi)/add_f(\xi)$ holds, it means that $f$ is a precondition/negative/positive effect in the schema $\xi\in \mathcal{M}'$. For instance, the preconditions of the $stack$ schema (Figure~\ref{fig:stack}) are represented by the pair of fluents {\small\tt pre\_holding\_stack\_$v_1$} and {\small\tt pre\_clear\_stack\_$v_2$} set to {\em True}.
\item The fluents $mode_{prog}$ and $mode_{val}$ to indicate whether the operator schemas are programmed or validated, and the fluents $\{test_i\}_{1\leq i\leq |\mathcal{O}|}$, indicating the observation in $\mathcal{O}$ where the action model is validated.
\end{itemize}
\item $I_{\Lambda}$ contains the fluents from $F$ that encode $s_0$ (the first observation) and $mode_{prog}$ set to true. Our compilation assumes that initially, operator schemas are programmed with every possible precondition (the most specific learning hypothesis), no negative effect and no positive effect. Therefore fluents $pre_f(\xi)$, for every $f\in F_v(\xi)$, hold also at the initial state. 

\item $G_{\Lambda}=\bigcup_{1\leq i\leq |\mathcal{O}|}\{test_i\}$, requires that the programmed action model is validated in all the input observations.
\item $A_{\Lambda}$ comprises three kinds of actions:
\begin{enumerate}
\item Actions for {\em programming} operator schema $\xi\in\mathcal{M}$:
\begin{itemize}
\item Actions for {\bf removing} a {\em precondition} $f\in F_v(\xi)$ from the action schema $\xi\in\mathcal{M}$.

\begin{small}
\begin{align*}
\hspace*{7pt}\pre(\mathsf{programPre_{f,\xi}})=&\{\neg del_{f}(\xi),\neg add_{f}(\xi), mode_{prog}, pre_{f}(\xi)\},\\
\cond(\mathsf{programPre_{f,\xi}})=&\{\emptyset\}\rhd\{\neg pre_{f}(\xi)\}.
\end{align*}
\end{small}

\item Actions for {\bf adding} a {\em negative} or {\em positive} effect $f\in F_v(\xi)$ to the action schema $\xi\in\mathcal{M}$.

\begin{small}
\begin{align*}
\hspace*{7pt}\pre(\mathsf{programEff_{f,\xi}})=&\{\neg del_{f}(\xi),\neg add_{f}(\xi), mode_{prog}\},\\
\cond(\mathsf{programEff_{f,\xi}})=&\{pre_{f}(\xi)\}\rhd\{del_{f}(\xi)\},\{\neg pre_{f}(\xi)\}\rhd\{add_{f}(\xi)\}.
\end{align*}
\end{small}
\end{itemize}

\item Actions for {\em applying} a programmed operator schema $\xi\in\mathcal{M}$ bound with objects $\omega\subseteq\Omega^{ar(\xi)}$. Since operators headers are given as input, the variables $pars(\xi)$ are bound to the objects in $\omega$ that appear at the same position. Figure~\ref{fig:compilation} shows the PDDL encoding of the action for applying a programmed operator $stack$ from {\em blocksworld}.
\begin{small}
\begin{align*}
\hspace*{7pt}\pre(\mathsf{apply_{\xi,\omega}})=&\{pre_{f}(\xi)\implies p(\omega)\}_{\forall p\in\Psi,f=p(pars(\xi))}\cup \{\neg mode_{val}\},\\
\cond(\mathsf{apply_{\xi,\omega}})=&\{del_{f}(\xi)\}\rhd\{\neg p(\omega)\}_{\forall p\in\Psi,f=p(pars(\xi))},\\
&\{add_{f}(\xi)\}\rhd\{p(\omega)\}_{\forall p\in\Psi,f=p(pars(\xi))},\\
&\{mode_{prog}\}\rhd\{\neg mode_{prog}\},\\
&\{\emptyset\}\rhd\{mode_{val}\}.
\end{align*}
\end{small}

\begin{figure}[hbt!]
\begin{scriptsize}
\begin{verbatim}
(:action apply_stack
  :parameters (?o1 - object ?o2 - object)
  :precondition
   (and (or (not (pre_on_stack_v1_v1)) (on ?o1 ?o1))
        (or (not (pre_on_stack_v1_v2)) (on ?o1 ?o2))
        (or (not (pre_on_stack_v2_v1)) (on ?o2 ?o1))
        (or (not (pre_on_stack_v2_v2)) (on ?o2 ?o2))
        (or (not (pre_ontable_stack_v1)) (ontable ?o1))
        (or (not (pre_ontable_stack_v2)) (ontable ?o2))
        (or (not (pre_clear_stack_v1)) (clear ?o1))
        (or (not (pre_clear_stack_v2)) (clear ?o2))
        (or (not (pre_holding_stack_v1)) (holding ?o1))
        (or (not (pre_holding_stack_v2)) (holding ?o2))
        (or (not (pre_handempty_stack)) (handempty)))
  :effect
   (and (when (del_on_stack_v1_v1) (not (on ?o1 ?o1)))
        (when (del_on_stack_v1_v2) (not (on ?o1 ?o2)))
        (when (del_on_stack_v2_v1) (not (on ?o2 ?o1)))
        (when (del_on_stack_v2_v2) (not (on ?o2 ?o2)))
        (when (del_ontable_stack_v1) (not (ontable ?o1)))
        (when (del_ontable_stack_v2) (not (ontable ?o2)))
        (when (del_clear_stack_v1) (not (clear ?o1)))
        (when (del_clear_stack_v2) (not (clear ?o2)))
        (when (del_holding_stack_v1) (not (holding ?o1)))
        (when (del_holding_stack_v2) (not (holding ?o2)))
        (when (del_handempty_stack) (not (handempty)))
        (when (add_on_stack_v1_v1) (on ?o1 ?o1))
        (when (add_on_stack_v1_v2) (on ?o1 ?o2))
        (when (add_on_stack_v2_v1) (on ?o2 ?o1))
        (when (add_on_stack_v2_v2) (on ?o2 ?o2))
        (when (add_ontable_stack_v1) (ontable ?o1))
        (when (add_ontable_stack_v2) (ontable ?o2))
        (when (add_clear_stack_v1) (clear ?o1))
        (when (add_clear_stack_v2) (clear ?o2))
        (when (add_holding_stack_v1) (holding ?o1))
        (when (add_holding_stack_v2) (holding ?o2))
        (when (add_handempty_stack) (handempty))
        (when (modeProg) (not (modeProg)))))
\end{verbatim}
\end{scriptsize}
 \caption{\small PDDL action for applying an already programmed schema $stack$ (implications coded as disjunctions).}
\label{fig:compilation}
\end{figure}

\item Actions for {\em validating} an observation {\tt\small $1\leq i\leq |\mathcal{O}|$}.
\begin{small}
\begin{align*}
\hspace*{7pt}\pre(\mathsf{validate_{i}})=&s_i\cup\{test_j\}_{j\in 1\leq j<i}\cup\{\neg test_j\}_{j\in i\leq j\leq |\mathcal{O}|}\cup \{mode_{val}\},\\
\cond(\mathsf{validate_{i}})=&\{\emptyset\}\rhd\{test_i,\neg mode_{val}\}.
\end{align*}
\end{small}
\end{enumerate}
\end{itemize}

As introduced, the compilation approach is flexible to different amount of input data. Known preconditions and effects (that is, a partially specified \strips\ action model) can be encoded as fluents $pre_f(\xi)$, $del_f(\xi)$ and $add_f(\xi)$ set to true at the initial state $I_{\Lambda}$. In this case, the corresponding programming actions, $\mathsf{programPre_{f,\xi}}$ and $\mathsf{programEff_{f,\xi}}$, become unnecessary and are removed from $A_{\Lambda}$ making the classical planning task $P_{\Lambda}$ easier to be solved. When a {\em fully} or {\em partially specified} \strips\ action model $\mathcal{M}$ is given, the compilation validates whether the given observations of the plan execution follows the given model. This feature is beyond the functionality of VAL~\cite{howey2004val} because the plan validation tool requires (1) a full action model and (2), a plan:
\begin{itemize}
\item $\mathcal{M}$ is proved to be a {\em valid} \strips\ action model for the given examples if a solution plan is found for $P_{\Lambda}$.
\item $\mathcal{M}$ is proved to be a {\em invalid} \strips\ action model for the given examples if $P_{\Lambda}$ is unsolvable since it means that $\mathcal{M}$ cannot be compliant with all the given observations. 
\end{itemize}
  
Figure~\ref{fig:plan-observations} illustrate how our compilation works and shows a plan that solves a classical planning task resulting from the compilation. The plan programs and validates the $stack$ schema (from {\em blocksworld}) using the five state observations shown in Figure~\ref{fig:example-observations} as well as previously specified operator schemes for $pickup$, $putdown$ and $unstack$. Plan steps $[0,8]$ program the preconditions of the {\tt\small stack} operator, steps $[9,13]$ program the operator effects and steps $[14,21]$ validate the programmed operators using the sequence of five state observations shown in the Figure~\ref{fig:example-observations}.

\begin{figure}[hbt!]
{\footnotesize\tt
     {\bf 00} : (program\_pre\_clear\_stack\_v1)\\
     01 : (program\_pre\_handempty\_stack)\\
     02 : (program\_pre\_holding\_stack\_v2)\\
     03 : (program\_pre\_on\_stack\_v1\_v1)\\
     04 : (program\_pre\_on\_stack\_v1\_v2)\\
     05 : (program\_pre\_on\_stack\_v2\_v1)\\
     06 : (program\_pre\_on\_stack\_v2\_v2)\\
     07 : (program\_pre\_ontable\_stack\_v1)\\
     08 : (program\_pre\_ontable\_stack\_v2)\\
     {\bf 09} : (program\_eff\_clear\_stack\_v1)\\
    10 : (program\_eff\_clear\_stack\_v2)\\
    11 : (program\_eff\_handempty\_stack)\\
    12 : (program\_eff\_holding\_stack\_v1)\\
    13 : (program\_eff\_on\_stack\_v1\_v2)\\
    {\bf 14} : (apply\_unstack blockB blockA)\\
    15 : (validate\_1)\\
    16 : (apply\_putdown blockB)\\
    17 : (validate\_2)\\
    18 : (apply\_pickup blockA)\\
    19 : (validate\_3)\\
    20 : (apply\_stack blockA blockB)\\
    21 : (validate\_4)
}
 \caption{\small Plan for programming and validating the $stack$ schema using the five state observations shown in Figure~\ref{fig:example-observations} as well as previously specified operator schemes for $pickup$, $putdown$ and $unstack$.}
\label{fig:plan-observations}
\end{figure}

Note that this compilation is valid for the particular scenario where the observed states are not full states but {\em partial states}, in the sense that some of the observed fluents (either with positive or negative value) are missing. If any reference to the $mode_{val}$ fluent is removed, the compilation becomes even more flexible and can learn \strips\ action models even if there are missing state observations in $\mathcal{O}$. On the other hand, if the $mode_{val}$ fluent is ignored, the resulting classical planning task $P_{\Lambda}$ becomes harder to be solved since the classical planner must now determine how many {\em apply} actions are necessary between any two observations, i.e. between the application of two {\em validate} actions.  


\subsection{Learning from plans}
Now we extend the compilation to consider observed actions. Given a learning task $\Lambda=\tup{\mathcal{M},\mathcal{O},\Psi,\pi}$, the compilation outputs a classical planning task $P_{\Lambda'}=\tup{F_{\Lambda'},A_{\Lambda'},I_{\Lambda'},G_{\Lambda}}$ such that:
\begin{itemize}
\item $F_{\Lambda'}$ extends $F_{\Lambda}$ with the $F_{\pi}=\{plan(name(a_j),\Omega^{ar(a_j)},j)\}_{\small 1\leq j\leq |\pi|}$ fluents to code the $j^{th}$ step of the observed plan $\pi=\tup{a_1, \ldots, a_n}$ that contains the action $a_j$. The static facts $next_{j,j+1}$ and the fluents $at_j$, {\small $1\leq j< |\pi|$}, are also added to represent the current plan step and iterate through the steps of the plan.
\item $I_{\Lambda'}$ extends $I_{\Lambda}$ with fluents $F_{\pi}$ plus fluents $at_1$ and $\{next_{j,j+1}\}$, {\small $1\leq j<|\pi|$}, for tracking the plan step where the action model is validated. Goals are as in the original compilation. In other words, the observed states are used for validation adding a $test_t$, ${\small 1\leq t\leq |\mathcal{O}|}$ fluent for each observation in $\mathcal{O}$.
\item With respect to the set of actions $A_{\Lambda'}$.
\begin{enumerate}
\item The actions for {\em programming} the preconditions/effects of a given operator schema $\xi\in\Xi$ and the actions for {\em validating} the programmed action model in a given state observation are the same as in the previous compilation.
\item The actions for {\em applying} an already programmed operator have an extra precondition $f\in F_{\pi}$ that encodes the current plan step, and extra conditional effects $\{at_{j}\}\rhd\{\neg at_{j},at_{j+1}\}_{\forall j\in [1,n]}$ for advancing to the next plan step. This mechanism ensures that these actions are applied, exclusively, in the same order as in the example plan $\pi$.
\end{enumerate}
\end{itemize}

To illustrate this, the classical plan of Figure~\ref{fig:plan-lplan} is a solution to a learning task $\Lambda=\tup{\mathcal{M},\mathcal{O},\Psi,\pi}$ for getting the {\em blocksworld} action model where operator schemes for {\tt\small pickup}, {\tt\small putdown} and {\tt\small unstack} are specified in $\mathcal{M}$. This plan programs and validates the operator schema {\tt\small stack} from {\em blocksworld}, using the plan $\pi$ and the two state observations $\mathcal{O}$ shown in Figure~\ref{fig:example-plans}. Plan steps $[0,8]$ program the preconditions of the {\tt\small stack} operator, steps $[9,13]$ program the operator effects and steps $[14,18]$ validate the programmed operators following the four-action plan $\pi$ shown in the Figure~\ref{fig:example-plans}.

\begin{figure}[hbt!]
{\footnotesize\tt
     {\bf 00} : (program\_pre\_clear\_stack\_v1)\\
     01 : (program\_pre\_handempty\_stack)\\
     02 : (program\_pre\_holding\_stack\_v2)\\
     03 : (program\_pre\_on\_stack\_v1\_v1)\\
     04 : (program\_pre\_on\_stack\_v1\_v2)\\
     05 : (program\_pre\_on\_stack\_v2\_v1)\\
     06 : (program\_pre\_on\_stack\_v2\_v2)\\
     07 : (program\_pre\_ontable\_stack\_v1)\\
     08 : (program\_pre\_ontable\_stack\_v2)\\
     {\bf 09} : (program\_eff\_clear\_stack\_v1)\\
    10 : (program\_eff\_clear\_stack\_v2)\\
    11 : (program\_eff\_handempty\_stack)\\
    12 : (program\_eff\_holding\_stack\_v1)\\
    13 : (program\_eff\_on\_stack\_v1\_v2)\\
    {\bf 14} : (apply\_unstack blockB blockA i1 i2)\\
    15 : (apply\_putdown blockB i2 i3)\\
    16 : (apply\_pickup blockA i3 i4)\\
    17 : (apply\_stack blockA blockB i4 i5)\\
    {\bf 18} : (validate\_1)
}
 \caption{\small Plan for programming and validating the $stack$ schema using plan $\pi$ and state observations $\mathcal{O}$ (shown in Figure~\ref{fig:example-plans}) as well as previously specified operator schemes for $pickup$, $putdown$ and $unstack$.}
\label{fig:plan-lplan}
\end{figure}

The compilation is flexible to the learning scenario where some $a_j$ actions in $\pi$ are missing. In such scenario the corresponding $plan(name(a_j),\Omega^{ar(a_j)},j)$ fluents are not coded in the initial state as well as they are not added as extra preconditions of the corresponding corresponding {\em apply} actions. Again, the resulting classical planning task that results from this modification of the compilation becomes harder to be solved since the planner must determine which are the {\em apply} action necessary for validating the learned action model with the given input knowledge.

Now we explain how to address learning \strips\ action models from the observation of the execution of multiple plans $\Pi=\{\pi_1,\ldots,\pi_{\tau}\}$, {\tt\small $1\leq t\leq \tau$}. Let us first define a set of classical planning instances $P_t=\tup{F,\emptyset,I_t,G_t}$ that belong to the same planning frame (i.e. same fluents and actions but different initial states and goals). The set of actions, $A=\emptyset$, is empty because the action model is initially unknown. Finally, the initial state $I_t$ is given by the state $s_0^t$ and the plan $\pi_t$, and the goals $G_t$ are defined by the state $s_n^t$. Addressing the learning task $\Lambda=\tup{\mathcal{M},\mathcal{O},\Psi,\Pi}$ with classical planning requires introducing a small modification to our compilation. In particular, the actions in $P_{\Lambda'}$ for {\em validating} the plan $\pi_t\in\Pi$, {\tt\small $1\leq t\leq \tau$} reset the current state, and the current plan, and are now defined as:
\begin{small}
\begin{align*}
\hspace*{7pt}\pre(\mathsf{validate_{t}})=&G_t\cup\{test_j\}_{1\leq j<t}\cup\{\neg test_j\}_{t\leq j\leq \tau}\cup \{\neg mode_{prog}\},\\
\cond(\mathsf{validate_{t}})=&\{\emptyset\}\rhd\{test_t\} \cup \{\neg f\}_{\forall f\in G_t, f \notin I_{t+1}}\cup \{f\}_{\forall f\in I_{t+1}, f \notin G_t}.
\end{align*}
\end{small}


\subsection{Compilation properties}

\begin{mylemma}
Soundness. Any classical plan $\pi$ that solves $P_{\Lambda}$ induces an action model $\mathcal{M}'$ that solves $\Lambda=\tup{\mathcal{M},\mathcal{O},\Psi,\Pi}$.
\end{mylemma}

\begin{proof}[Proof sketch]
\begin{small}
Once operator schemas $\mathcal{M}'$ are programmed, they can only be applied and validated, because of the $mode_{prog}$ fluent. In addition, $P_{\Lambda}$ is only solvable if fluents $\{test_i\}$, {\small $1\leq i\leq n$} hold at the last reached state. These goals can only be achieved executing an applicable sequence of programmed operator schemas that reaches every state $s_i\in\mathcal{O}$, starting from the corresponding initial state and following the sequence of actions defined by the plans in $\Pi$. This means that the programmed action model $\mathcal{M}'$ complies with the provided input knowledge and hence, solves $\Lambda$.
\end{small}
\end{proof}


\begin{mylemma}
Completeness. Any \strips\ action model $\mathcal{M}'$ that solves a $\Lambda=\tup{\mathcal{M},\Psi,\mathcal{O},\Pi}$ learning task, is computable solving the corresponding classical planning task $P_{\Lambda}$.
\end{mylemma}

\begin{proof}[Proof sketch]
\begin{small}
By definition, $F_v(\xi)\subseteq F_\Lambda$ fully captures the full set of elements that can appear in a \strips\ action schema $\xi\in\mathcal{M}$ given its header and the set of predicates $\Psi$. The compilation does not discard any possible \strips\ action schema definable within $F_v$ that satisfies the state trajectory constraint given by $\mathcal{O},\Pi$.
\end{small}
\end{proof}

The size of the classical planning task $P_{\Lambda}$ output by the compilation depends on:
\begin{itemize}
\item The arity of the actions headers in $\mathcal{M}$ and the predicates $\Psi$ that are given as input to the $\Lambda$ learning task. The larger these numbers, the larger the size of the $F_v(\xi)$ sets. This is the term that dominates the compilation size because it defines the $pre_f(\xi)/del_f(\xi)/add_f(\xi)$ fluents set and the corresponding set of {\em programming} actions. 
\item The number of given state observations. The larger $|\mathcal{O}|$, the more $test_i$ fluents and $\mathsf{validate_{i}}$ actions in $P_{\Lambda}$.
\end{itemize}



\subsection{Exploiting static predicates to optimize the compilation}
A {\em static predicate} $p \in \Psi$ is a predicate that does not appear in the effects of any action~\cite{fox:TIM:JAIR1998}. Therefore, one can get rid of the mechanism for programming these predicates in the effects of any action schema while keeping the compilation complete. Given a static predicate $p$:
\begin{itemize}
\item Fluents $del_f(\xi)$ and $add_f(\xi)$, such that $f\in F_v$ is an instantiation of the static predicate $p$ in the set of {\em variable objects} $\Omega_v$, can be discarded for every $\xi\in\Xi$.
\item Actions $\mathsf{programEff_{f,\xi}}$ (s.t. $f\in F_v$ is an instantiation of $p$ in $\Omega_v$) can also be discarded for every $\xi\in\Xi$.
\end{itemize}

Static predicates can also constrain the space of possible preconditions by looking at the given set of state observation $\mathcal{O}$. One can assume that if a precondition $f\in F_v$ (s.t. $f\in F_v$ is an instantiation of a static predicate in $\Omega_v$) is not compliant with the observations in $\mathcal{O}$ then, fluents $pre_f(\xi)$ and actions $\mathsf{programPre_{f,\xi}}$ can be discarded for every $\xi\in\mathcal{M}$. For instance, in the {\em zenotravel}~\cite{long20033rd} domain $pre\_next\_board\_v1\_v1$, $pre\_next\_debark\_v1\_v1$, $pre\_next\_fly\_v1\_v1$, $pre\_next\_zoom\_v1\_v1$, $pre\_next\_refuel\_v1\_v1$ can be discarded (and their corresponding programming actions) because a precondition {\tt\small(next ?v1 ?v1 - flevel)} will never hold at any state in $\mathcal{O}$.

Furthermore looking as well at the given example plans, fluents $pre_f(\xi)$ and actions $\mathsf{programPre_{f,\xi}}$ are also discardable for every $\xi\in\Xi$ if a precondition $f\in F_v$ (s.t. $f\in F_v$ is an instantiation of a static predicate in $\Omega_v$) is not possible according to $\Pi$ and $\mathcal{O}$. Back to the {\em zenotravel} domain, if an example plan $\pi_t\in \Pi$ contains the action {\tt\small (fly plane1 city2 city0 fl3 fl2)} and the corresponding state observations contain the static literal {\tt\small (next fl2 fl3)} but does not contain {\tt\small (next fl2 fl2)}, {\tt\small (next fl3 fl3)} or {\tt\small (next fl3 fl2)} the only possible precondition including the static predicate is $pre\_next\_fly\_v5\_v4$.



\section{Evaluating \strips\ action models}
\label{sec:Section6}
The roles of two action schemes whose headers match or the roles of two action parameters that belong to the same type can be swapped by a $\Lambda$ learning task. For instance, the {\em blocksworld} operator {\small\tt stack} can be {\em learned} with the preconditions and effects of the {\small\tt unstack} operator and vice versa, or the parameters of the {\small\tt stack} operator can be swapped. Pure syntax-based metrics can report low scores for learned models that are actually good but correspond to {\em reformulations} of the actual model; i.e. a learned model semantically equivalent but syntactically different to the reference model.

Here we introduce an evaluation approach that is robust to role changes of this particular kind. The intuition of the approach is to assess how well a \strips\ action model $\mathcal{M}$ explains given observations of plan executions according to the amount of {\em edition} required by $\mathcal{M}$ to induce that observations. %In the extreme, if $\mathcal{M}$ perfectly explains $\mathcal{O}$, no model {\em edition} is necessary.

\subsection{The \strips\ edit distance}

We first define the two allowed \emph{operations} to edit a given \strips\ action model $\mathcal{M}$:
\begin{itemize}
\item {\em Deletion}. A fluent $pre_f(\xi)/del_f(\xi)/add_f(\xi)$ is removed from the operator schema $\xi\in\mathcal{M}$, such that $f\in F_v(\xi)$.
\item {\em Insertion}. A fluent $pre_f(\xi)/del_f(\xi)/add_f(\xi)$ is added to the operator schema $\xi\in\mathcal{M}$, s.t. $f\in F_v(\xi)$.
\end{itemize}

We can now formalize an {\em edit distance} that quantifies how similar two given \strips\ action models are. The distance is symmetric and meets the {\em metric axioms} provided that the two {\em edit operations}, deletion and insertion, have the same positive cost.

\begin{mydefinition}
  Let $\mathcal{M}$ and $\mathcal{M}'$ be two \strips\ action models, such that they are {\em comparable}. The {\bf edit distance}, denoted as $\delta(\mathcal{M},\mathcal{M}')$, is the minimum number of {\em edit operations} that is required to transform $\mathcal{M}$ into $\mathcal{M}'$.
\end{mydefinition}

Since $F_v$ is a bound set, the maximum number of edits that can be introduced to a given action model defined within $F_v$ is bound as well. In more detail, for an operator schema $\xi\in\mathcal{M}$ the maximum number of edits that can be introduced to their precondition set is $|F_v(\xi)|$ while the max number of edits that can be introduced to the effects is twice $|F_v(\xi)|$. 
\begin{mydefinition}
The \textbf{maximum edit distance} of an \strips\ action model $\mathcal{M}$ built from the set of possible elements $F_v$ is $\delta(\mathcal{M},*)=\sum_{\xi\in\mathcal{M}} 3\times|F_v(\xi)|$.
\end{mydefinition}

Normally evaluating a given learned domain with respect to the actual generative model is not possible because the actual model is not available. As the ARMS system shows, the error of a learned action model can also be estimated with respect to a set of observations of plan executions~\cite{yang2007learning}. With this regard, we define now an edit distance to asses the quality of a learned action model with respect to a sequence of state observations.

\begin{mydefinition}
  Given $\mathcal{M}$, a \strips\ action model built from $F_v$, and the observations sequence $\mathcal{O}=\tup{s_0, s_1, \ldots, s_n}$ such that each state observation in $\mathcal{O}$ is built with fluents in $F$. The {\bf observation edit distance}, denoted by  $\delta(\mathcal{M},\mathcal{O})$, is the minimal edit distance from $\mathcal{M}$ to any {\em comparable} model $\mathcal{M}'$, such that $\mathcal{M}'$ can produce a valid plan $\pi=\tup{a_1, \ldots, a_n}$ that induces $\mathcal{O}$; \[\delta(\mathcal{M},\mathcal{O})=\min_{\forall \mathcal{M}' \rightarrow \mathcal{O}} \delta(\mathcal{M},\mathcal{M}')\]
\end{mydefinition}

Unlike the error function defined by {\sc ARMS}, the {\em observation edit distance} assess, with a single expression, the flaws in the preconditions and effects of a given learned model. This fact enables the recognition of \strips\ action models. The idea, taken from {\em plan recognition as planning}~\cite{ramirez2009plan}, is to map distances into likelihoods. This {\em edit distance} could be mapped into a likelihood with the following expression $P(\mathcal{O}|\mathcal{M})=1-\frac{\delta(\mathcal{M},\mathcal{O})}{\delta(\mathcal{M},*)}$.

The error of a learned action model could also be defined quantifying the amount of edition required by the observations of the plan execution to match the given model. This would imply defining {\em edit operations} that modify the fluents in the state observations instead of the {\em edit operations} that modify the action schemes. Our definition of the edit distance is more practical since normally, $F_v$ is smaller than $F$ because the number of {\em variable objects} is smaller than the number of objects in the state observations. Finally, the {\em edit distance} can also be defined with respect to a set of plans, if they are available. 

\begin{mydefinition}
  Given an action model $\mathcal{M}$ and a set of valid plans $\Pi=\{\pi_1,\ldots,\pi_{\tau}\}$ that only contain actions built grounding the schemes in $\mathcal{M}$. The {\bf plans edit distance}, denoted by  $\delta(\mathcal{M},\Pi)$, is the minimal edit distance from $\mathcal{M}$ to any {\em comparable} model $\mathcal{M}'$, such that $\mathcal{M}'$ can produce all the plans $\pi_t\in \Pi$, {\tt\small $1\leq t\leq \tau$}; \[\delta(\mathcal{M},\Pi)=\min_{\forall \mathcal{M}' \rightarrow \Pi} \delta(\mathcal{M},\mathcal{M}')\]
\end{mydefinition}




\subsection{Computing the observations and plans edit distance}
Our compilation is extensible to compute the {\em observation edit distance} by simply considering that the input \strips\ model $\mathcal{M}$, given in a learning task $\Lambda=\tup{\mathcal{M},\mathcal{O},\Psi}$, is {\em non-empty}. In other words, now $\mathcal{M}$ is a set of given operator schemas, wherein each $\xi\in\mathcal{M}$ initially contains $head(\xi)$ but also the $pre(\xi)$, $del(\xi)$ and $add(\xi)$ sets. A solution to the planning task resulting from the extended compilation is a sequence of actions that:

\begin{enumerate}
\item {\bf Edits the action model $\mathcal{M}$ to build $\mathcal{M}'$}. A solution plan starts with a {\em prefix} that modifies the preconditions and effects of the action schemes in $\mathcal{M}$ using to the two {\em edit operations} defined above, {\em deletion} and {\em insertion}. In theory, we could implement a third edit operation for {\em substituting} a fluent from a given operator schema. However, and with the aim of keeping a tractable branching factor of the planning instances that result from our compilations, we only implement {\em deletion} and {\em insertion}.
\item {\bf Validates the edited model $\mathcal{M}'$ in observations of the plan executions}. The solution plan continues with a postfix that validates the edited model $\mathcal{M}'$ on the given observations $\mathcal{O}$ (and on $\Pi$ if available), as explained in Section~\ref{sec:Section5} for the models that are programmed from scratch.
\end{enumerate}

Now $\Lambda$ does not formalize a learning task but the task of editing $\mathcal{M}$ to produce the observations $\mathcal{O}$, which results in the edited model $\mathcal{M}'$. The output of the extended compilation is a classical planning task $P_{\Lambda}'=\tup{F_{\Lambda},A_{\Lambda}',I_{\Lambda}',G_{\Lambda}}$:

\begin{itemize}
\item $F_{\Lambda}$ and $G_{\Lambda}$ are defined as in the previous compilation.
\item $I_{\Lambda}'$ contains the fluents from $F$ that encode $s_0$ and $mode_{prog}$ set to true. In addition, the input action model $\mathcal{M}$ is now encoded in the initial state. This means that the fluents $pre_f(\xi)/del_f(\xi)/add_f(\xi)$, $f\in F_v(\xi)$, hold in the initial state iff they appear in $\mathcal{M}$.
\item $A_{\Lambda}'$, comprises the same three kinds of actions of $A_{\Lambda}$. The actions for {\em applying} an already programmed operator schema and the actions for {\em validating} an observation are defined exactly as in the previous compilation. The only difference here is that the actions for {\em programming} the operator schema now implement the two {\em edit operations} (i.e. include actions for {\em inserting} a precondition and for {\em deleting} a negative/positive effect).
\end{itemize}

To illustrate this, the plan of Figure~\ref{fig:plan-odistance} solves the classical planning task that corresponds to editing a \emph{blocksworld} action model where the positive effects {\tt\small (handempty)} and {\tt\small (clear ?v1)} of the {\tt\small stack} schema are missing. First, the plan edits the {\tt\small stack} schema, {\em inserting} these two positive effects, and then validates the edited action model in the five-observation sequence of Figure~\ref{fig:example-observations}.

\begin{figure}[hbt!]
{\tt\small
00 : (insert\_add\_handempty\_stack)\\
01 : (insert\_add\_clear\_stack\_var1)\\
02 : (apply\_unstack blockB blockA)\\
03 : (validate\_1)\\
04 : (apply\_putdown blockB)\\
05 : (validate\_2)\\
06 : (apply\_pickup blockA)\\
07 : (validate\_3)\\
08 : (apply\_stack blockA blockB)\\
09 : (validate\_4)\\
}
\caption{\small Plan for editing a given action model and validating it at the state observations shown in Figure~\ref{fig:example-observations}. The given action model is a four operator {\em blockswold} where the positive effects {\tt\small (handempty)} and {\tt\small (clear ?v1)} of the {\tt\small stack} schema are missing.}
\label{fig:plan-odistance}
\end{figure}

Our interest when computing the {\em observation edit distance} is not in the resulting action model $\mathcal{M}'$ but in the number of required {\em edit operations} for that $\mathcal{M}'$ is validated in the given observations, e.g. $\delta(\mathcal{M},\mathcal{O})=2$ for the example in Figure~\ref{fig:plan-odistance}. In this case $\delta(\mathcal{M},*)=3\times 2\times (11+5)$ since there are 4 action schemes ({\small\tt pickup}, {\small\tt putdown}, {\small\tt stack} and {\small\tt unstack}) and $|F_v|=|F_v(stack)|=|F_v(unstack)|=11$ while $|F_v(pickup)|=|F_v(putdown)|=5$  (as shown in Section~\ref{sec:Section3}). The {\em observation edit distance} is exactly computed if the classical planning task resulting from our compilation is optimally solved (according to the number of edit actions); is approximated if it is solved with a satisfying planner; and is a less accurate estimate (but faster to be computed) if the solved task is a relaxation of the classical planning task that results from our compilation~\cite{bonet2001planning}.

When the executed plans $\Pi$ are also available, the compilation can be adapted to compute the {\em plan edit distance} $\delta(\mathcal{M},\Pi)$. The modifications to the compilation explained in Section~\ref{sec:Section5} are also useful here to redefine the learning task as the task of editing $\mathcal{M}$ to produce the set of plans $\Pi$, which results in the edited model $\mathcal{M}'$. Figure~\ref{fig:plan-pdistance} shows the plan for editing a given {\em blockswold} action model where again the positive effects {\tt\small (handempty)} and {\tt\small (clear ?v1)} of the {\tt\small stack} schema are missing. In this case the edited action model is however validated at the plan shown in Figure~\ref{fig:example-plans}.

\begin{figure}[hbt!]
{\tt\small
00 : (insert\_add\_handempty\_stack)\\
01 : (insert\_add\_clear\_stack\_var1)\\
02 : (apply\_unstack blockB blockA i1 i2)\\
03 : (apply\_putdown blockB i2 i3)\\
04 : (apply\_pickup blockA i3 i4)\\
05 : (apply\_stack blockA blockB i4 i5)\\
06 : (validate\_1)
}
 \caption{\small Plan for editing a given {\em blockswold} schema and validating it at the plan shown in Figure~\ref{fig:example-plans}.}
\label{fig:plan-pdistance}
\end{figure}

Last but not least, this compilation is flexible to compute the {\em edit distance} between two {\em comparable} \strips\ action models, $\mathcal{M}$ and $\mathcal{M}'$. A solution to the planning task resulting from this compilation is a sequence of actions that edits the action model $\mathcal{M}$ to produce $\mathcal{M}'$ using to the two {\em edit operations}, deletion and insertion. In this case the edited model is not validated on a sequence of observations or plans but on the given action model $\mathcal{M}'$ that acts as a reference. The sets of fluents $F_{\Lambda}$ and $I_{\Lambda}$ are defined like in the previous compilation. With respect to the actions, $A_{\Lambda}$ again implement the two {\em edit operations} (i.e. include actions for {\em inserting} a precondition and for {\em deleting} a negative/positive effect) but does not contain apply actions because the \strips\ action model are not validated in any observation of plan executions. Finally, the goals are also different and are now defined by the set of fluents, $pre_f(\xi)/del_f(\xi)/add_f(\xi)$ that represent all the operator schema $\xi\in\mathcal{M'}$, such that $f\in F_v(\xi)$. To illustrate this, the plan of Figure~\ref{fig:plan-mdistance} solves the classical planning task that corresponds to computing the distance between a \emph{blocksworld} action model, where the positive effects {\tt\small (handempty)} and {\tt\small (clear ?v1)} of the {\tt\small stack} schema are missing, and the actual four-operator {\em blocksworld} model. The plan edits first the {\tt\small stack} schema, {\em inserting} these two positive effects. Again our interest is in the number of required {\em edit operations}, e.g. $\delta(\mathcal{M},\mathcal{M'})=2$. 

\begin{figure}[hbt!]
{\tt\small
00 : (insert\_add\_handempty\_stack)\\
01 : (insert\_add\_clear\_stack\_var1)
}
 \caption{\small Plan for computing the distance between a \emph{blocksworld} action model, where the positive effects {\tt\small (handempty)} and {\tt\small (clear ?v1)} of the {\tt\small stack} schema are missing, and the actual four-operator {\em blocksworld} model.}
\label{fig:plan-mdistance}
\end{figure}



\section{Recognition of \strips\ action models}
\label{sec:Section7}
Given a set of possible \strips\ models and set of observations of plan executions, the {\em recognition of \strips\ models} is the task of identifying which model has the highest probability of producing the given observations.

According to the Bayes rule, the probability of an hypothesis $\mathcal{H}$ given the observations $\mathcal{O}$ can be computed with $P(\mathcal{H}|\mathcal{O})=\frac{P(\mathcal{O}|\mathcal{H})P(\mathcal{H})}{P(\mathcal{O})}$. In our scenario, the hypotheses are about the set of possible \strips\ action models. Given set of predicates $\Psi$ and a given a set of operator headers (in other words, given the $F_v(\xi)$ sets) the size of the set of possible \strips\ models set is $\prod_\xi 2^{2|F_v(\xi)|}$, as explained in Section~\ref{sec:Section3}. With respect to the observations, given $\Psi$ and a set of objects $\Omega$, the size of the possible state observations of length $n$, that is $\mathcal{O}=s_0,\ldots,s_n$ is given by $2^{n\times|F|}$. 

With this regard, $P(\mathcal{M}|\mathcal{O})$, the probability distribution of the possible \strips\ models (within the $F_v(\xi)$ sets) given an observation sequence $\mathcal{O}$ could be computed by:
\begin{enumerate}
\item Computing the {\em observation edit distance} $\delta(\mathcal{M},\mathcal{O})$ for every possible model $\mathcal{M}$. If a set of plans $\Pi$ is available, this same strategy can be followed using the {\em plan edit distance} $\delta(\mathcal{M},\Pi)$.
\item Applying the resulting distances to the above $P(\mathcal{O}|\mathcal{M})$ formula to map these distances into likelihoods
\item Applying the Bayes rule to obtain the normalized posterior probabilities, these probabilities must sum 1.
\end{enumerate}



\section{Learning \strips\ action models with background knowledge}
A distinctive feature of Inductive Logic Programming (ILP) is that ILP can leverage {\em background knowledge} to learn explanations from data~\cite{muggleton1994inductive}. Inspired by ILP, we show that our approach for the learning of \strips\ action models can also leverage {\em background knowledge} in this case, given as planning constraints, either in the form of {\em state constraints} or {\em trajectory constraints}.

\subsection{State constraints}
The notion of {\em state-constraint} is very general and has been used in different areas of AI and for different purposes.  If we restrict ourselves to planning, {\em state-constraints} are abstractions for compactly specifying sets of states. For instance, {\em state-constraints} in planning allow to specify the set of states where a given action is applicable, the set of states where a given {\em derived predicate} holds or the set of states that are considered goal states.  

{\em State invariants} is a kind of state-constraints useful for computing more compact state representations~\cite{helmert2009concise} or making {\em satisfiability planning} and {\em backward search} more efficient~\cite{rintanen2014madagascar,alcazar2015reminder}. Given a classical planning problem $P=\tup{F,A,I,G}$, a {\em state invariant} is a formula $\phi$ that holds at the initial state of a given classical planning problem, $I\models \phi$, and at every state $s$, built from $F$, that is reachable from $I$.

The formula $\phi_{I,A}^*$ represents the {\em strongest invariant} and exactly characterizes the set of all states reachable from $I$ with the actions in $A$. For instance Figure~\ref{fig:strongest-invariant} shows five clauses that define the {\em strongest invariant} for {\em blocksworld}. There are infinitely many strongest invariants, but they are all logically equivalent, and computing the strongest invariant is PSPACE-hard as hard as testing plan existence.

\begin{figure}[hbt!]
  \begin{footnotesize}
    \begin{center}
$\forall x_1,x_2\ ontable(x_1)\leftrightarrow\neg on(x_1,x_2)$.\\
$\forall x_1,x_2\ clear(x_1)\leftrightarrow\neg on(x_2,x_1)$.\\
$\forall x_1,x_2,x_3\ \neg on(x_1,x_2)\vee\neg on(x_1,x_3)\ such\ that\ x_2\neq x_3$.\\
$\forall x_1,x_2,x_3\ \neg on(x_2,x_1)\vee\neg on(x_3,x_1)\ such\ that\ x_2\neq x_3$.\\
$\forall x_1,\ldots,x_n\ \neg(on(x_1,x_2)\wedge on(x_2,x_3)\wedge\ldots\wedge on(x_{n-1},x_n)\wedge on(x_n,x_1)).$ 
    \end{center}
\end{footnotesize}
 \caption{\small An example of the strongest invariant for the {\em blocksworld} domain.}
\label{fig:strongest-invariant}
\end{figure}


A {\em mutex} (mutually exclusive) is a state invariant that takes the form of a binary clause and indicates a pair of different properties that cannot be simultaneously true~\cite{kautz:mutex:IJCAI1999}. For instance in a three-block {\em blocksworld}, $\phi_1=\neg on(block_A,block_B)\vee \neg on(block_A,block_C)$ is a mutex because $block_A$ can only be on top of a single block. 

A {\em domain invariant} is an instance-independent invariant, i.e. holds for any possible initial state and set of objects. Therefore, if a given state $s$ holds $s\nvDash \phi$ such that $\phi$ is a {\em domain invariant}, it means that $s$ is not a valid state. Domain invariants are often compactly defined as {\em lifted invariants} (also called schematic invariants)~\cite{rintanen:schematicInvariants:AAAI2017}. For instance, $\phi_2=\forall x:\ (\neg handempty\vee \neg holding(x))$, is a {\em domain mutex} for the {\em blocksworld} because the robot hand is never empty and holding a block at the same time.

\subsection{Trajectory constraints}
Instead of enumerating the full sequence of states included in a trajectory, {\em state trajectory constraints} can be implicitly defined with {\em Linear Temporal Logic} (LTL)~\cite{haslum:LTL:ecai10,bacchus:LTL:1998}. LTL is a modal temporal logic interpreted over sequences of states. LTL interpreted over finite sequences of states (also called traces) has received attention from the planning community and provided a formal description called $LTL_f$~\cite{Giacomo:LTLf:AAAI2014}.

$LTL_f$ is built up from a finite set of propositional variables $P$, the logical operators $\neg$ and $\vee$ (from which it is possible to define operators $\wedge$, $\rightarrow$ and $\leftrightarrow)$, and the temporal modal operators {\em next}($\bigcirc$), and {\em until}(${\mathcal U}$). An $LTL_f$ formula is then inductively defined over a set of propositions $P$:
\begin{itemize}
\item A proposition $p\in P$ is an $LTL_f$ formula,
\item if $\psi$ and $\chi$ are $LTL_f$ formulae, then so they are $\neg\psi$, ($\psi\vee\chi$), $\bigcirc\psi$, ($\psi\ {\mathcal U}\ \chi$).
\end{itemize}

Given a sequence of states $\mathcal{O}=(s_0,\ldots, s_n)$, we say that a given $LTL_f$ formula $\phi$ holds at instant $i$ (denoted by $\mathcal{O}, i\models\phi$) iff:
\begin{itemize}
\item $\mathcal{O}, i\models f$ for a propositional variable $f\in F$, iff $f\in s_i$,
\item $\mathcal{O}, i\models \neg\psi$ iff it is not the case that $\mathcal{O}, i\models\psi$,
\item $\mathcal{O}, i\models \psi\wedge\chi$ iff $\mathcal{O}, i\models\psi$ and $\mathcal{O}, i\models\chi$,
\item $\mathcal{O}, i\models \bigcirc\phi$ if $i<n$ and $\mathcal{O}, i+1\models\phi$,
\item $\mathcal{O}, i\models (\phi_1\ {\mathcal U}\ \phi_2)$ if exists some $j\in\{1,\ldots,n\}$ such that, $\tau, j\models \phi_2$ and for all $k\in\{i,\ldots,j-1\}$ it holds that $\tau, k\models \phi_1$.
\end{itemize}
We say that $\mathcal{O}$ is {\em valid} for a given $LTL_f$ formula $\phi$ (denoted by $\mathcal{O}\models\phi$) iff for every {\small $0\leq i\leq n$} holds that $\mathcal{O}, i\models\phi$. From the previous basic operators it is also possible to define the abbreviation $last$ (that denotes the last instant of a sequence) and the temporal modal operators {\em eventually} ($\lozenge$), {\em always} ($\square$), and {\em release} (${\mathcal R}$):
\begin{itemize}
\item $last$ is shorthand for $\neg\bigcirc true$ and holds only at the last state of the sequence. The achievement of classical planning goals $G$ can then be expressed as the $LTL_f$ formula $\phi=last\wedge G$.  
\item $\lozenge$ $\psi$ stands for $(true\ {\mathcal U}\ \psi)$, and says that $\psi$ will eventually hold before the last state (last state included).
\item $\square\psi$ stands for $\neg \lozenge \neg\psi$ and says that from the current state till the last one the formula will always hold.
\item $\psi\ {\mathcal R}\ \chi$ stands for $\neg(\neg\psi\ {\mathcal U}\ \neg\chi)$. This says that $\chi$ has to be true until and including the point where $\psi$ first becomes true; if $\psi$ never becomes true, $\chi$ must remain true forever.
\end{itemize}

$LTL$ interpreted over finite traces has the expressive power of First Order Logic over finite ordered traces. Satisfiability, validity and logical implication (as well as model checking) for $LTL_f$ are PSPACE-complete.

\subsection{Leveraging background knowledge}
We have shown how our approach is flexible to validate the learned action models with respect to a set of observations of executed plans and with respect to a model that acts as a reference. Here we show that the validation of an action model can also be done with {\em state constraints} or {\em trajectory constraints}. Given $\Phi$, a set of either state or trajectory constraints, our validate actions can be adapted to check that the learned model satisfy every constraint $\phi_i\in\Phi$:
\begin{small}
\begin{align*}
\hspace*{7pt}\pre(\mathsf{validate_{i}})=&\phi_i\cup\{test_j\}_{j\in 1\leq j<i}\cup\{\neg test_j\}_{j\in i\leq j\leq |\Phi|}\cup \{mode_{val}\},\\
\cond(\mathsf{validate_{i}})=&\{\emptyset\}\rhd\{test_i,\neg mode_{val}\}.
\end{align*}
\end{small}

This redefinition of validate actions applies also to trajectory constraints because LTL formulae can be represented using classical action preconditions and goals by encoding the Non-deterministic B\"{u}chi Automaton (NBA), that is equivalent to the corresponding LTL formula, as part of the classical planning tasks~\cite{baier2006planning}.

Because of the combinatorial nature of the search for a solution plan, the sooner unpromising nodes are pruned from the search the more efficient the computation of a solution plan. Constraints can be used to confine earlier the set of possible \strips\ action models and reduce then the learning hypothesis space. With regard to our compilation, {\em domain mutex} are useful to reduce the amount of applicable actions for programming a precondition or an effect for a given action schema. For example given the {\em domain mutex} $\phi=(\neg f_1\vee \neg f_2)$ such that $f_1\in F_v(\xi)$ and $f_2\in F_v(\xi)$, we can redefine the corresponding programming actions for {\bf removing} the {\em precondition} $f_1\in F_v(\xi)$ from the action schema $\xi\in\mathcal{M}$ as:

\begin{small}
\begin{align*}
\hspace*{7pt}\pre(\mathsf{programPre_{f_1,\xi}})=&\{\neg del_{f_1}(\xi),\neg add_{f_1}(\xi), mode_{prog}, pre_{f_1}(\xi), pre_{f_2}(\xi)\},\\
\cond(\mathsf{programPre_{f,\xi}})=&\{\emptyset\}\rhd\{\neg pre_{f_1}(\xi)\}.
\end{align*}
\end{small}



\section{Evaluation}
\label{sec:Section8}
This section evaluates the performance of our approach for learning \strips\ action models starting from different amounts and kinds of available input knowledge.

\subsection{Setup}
The domains used in the evaluation are IPC domains that satisfy the \strips\ requirement~\cite{fox2003pddl2}, taken from the {\sc planning.domains} repository~\cite{muise2016planning}. We only use 5 learning examples for each domain and they are fixed for all the experiments so we can evaluate the impact of the input knowledge in the quality of the learned models. All experiments are run on an Intel Core i5 3.10 GHz x 4 with 8 GB of RAM.
\begin{itemize}
\item {\bf Planner}. The classical planner we use to solve the instances that result from our compilations is {\sc Madagascar}~\cite{rintanen2014madagascar}. We use {\sc Madagascar} because its ability to deal with planning instances populated with dead-ends. In addition, SAT-based planners can apply the actions for programming preconditions in a single planning step (in parallel) because these actions do not interact. Actions for programming action effects can also be applied in a single planning step reducing significantly the planning horizon.
\item {\bf Reproducibility}. We make fully available the compilation source code, the evaluation scripts and the used benchmarks at this repository {\em https://github.com/sjimenezgithub/strips-learning} so any experimental data reported in the paper is fully reproducible. 
\end{itemize}

\subsection{Evaluating with a reference model}
Here we evaluate the learned models with respect to the actual generative model. Opposite to what usually happens in ML, this model is available when learning is applied to IPC domains. The quality of the learned models is quantified with the {\em precision} and {\em recall} metrics. These two metrics are frequently used in {\em pattern recognition}, {\em information retrieval} and {\em binary classification} and are more informative that simply counting the number of errors in the learned model or computing the {\em symmetric difference} between the learned and the reference model~\cite{davis2006relationship}. Intuitively, precision gives a notion of {\em soundness} while recall gives a notion of the {\em completeness} of the learned models:
\begin{itemize}
\item $Precision=\frac{tp}{tp+fp}$, where $tp$ is the number of {\em true positives} (predicates that correctly appear in the action model) and $fp$ is the number of {\em false positives} (predicates of the learned model that should not appear). 
\item $Recall=\frac{tp}{tp+fn}$, where $fn$ is the number of {\em false negatives} (predicates that should appear in the learned model but are missing). 
\end{itemize}


\subsubsection{Learning from plans}
We start evaluating our approach with $\Lambda=\tup{\mathcal{M},\mathcal{O},\Psi,\Pi}$ learning tasks, where the action of the executed plans are available and the observation sequence contains only the corresponding initial and goal states, $s_o^t$ and $s_n^t$, for every plan $\pi_t\in\Pi$. We then repeat the evaluation but exploiting potential \emph{static predicates} computed from the observed states in $\mathcal{O}$, which are the predicates that appear unaltered in the states that belong to the same plan. Static predicates are used to constrain the space of possible action models as explained in Section~\ref{sec:Section5}.

Table~\ref{tab:results_plans} shows the obtained results. Precision ({\bf P}) and recall ({\bf R}) are computed separately for the preconditions ({\bf Pre}), positive effects ({\bf Add}) and negative Effects ({\bf Del}), while the last two columns of each setting and the last row report averages values. We can observe that identifying static predicates leads to models with better precondition {\em recall}. This fact evidences that many of the missing preconditions corresponded to static predicates because there is no incentive to learn them as they always hold~\cite{gregory2015domain}.

\begin{table*}[hbt!]
		\resizebox{\textwidth}{!}{%
		\begin{tabular}{l|l|l|l|l|l|l||l|l||l|l|l|l|l|l||l|l|}
& \multicolumn{8}{|c||}{\bf No Static}& \multicolumn{8}{|c|}{\bf Static}\\\cline{2-17}
& \multicolumn{2}{|c|}{\bf Pre} & \multicolumn{2}{|c|}{\bf Add} & \multicolumn{2}{|c|}{\bf Del} & \multicolumn{2}{|c||}{\bf}& \multicolumn{2}{|c|}{\bf Pre} & \multicolumn{2}{|c|}{\bf Add} & \multicolumn{2}{|c|}{\bf Del} & \multicolumn{2}{|c|}{\bf}\\ 			
			  & \multicolumn{1}{|c|}{\bf P} & \multicolumn{1}{|c|}{\bf R} & \multicolumn{1}{|c|}{\bf P} & \multicolumn{1}{|c|}{\bf R} & \multicolumn{1}{|c|}{\bf P} & \multicolumn{1}{|c|}{\bf R} &  \multicolumn{1}{|c|}{\bf P} & \multicolumn{1}{|c||}{\bf R}& \multicolumn{1}{|c|}{\bf P} & \multicolumn{1}{|c|}{\bf R} & \multicolumn{1}{|c|}{\bf P} & \multicolumn{1}{|c|}{\bf R} & \multicolumn{1}{|c|}{\bf P} & \multicolumn{1}{|c|}{\bf R} &  \multicolumn{1}{|c|}{\bf P} & \multicolumn{1}{|c|}{\bf R} \\
                          \hline
			Blocks & 1.0 & 1.0 & 1.0 & 1.0 & 1.0 & 1.0 & 1.0 & 1.0 & 1.0 & 1.0 & 1.0 & 1.0 & 1.0 & 1.0 & 1.0 & 1.0\\
			Driverlog & 1.0 & 0.36 & 0.75 & 0.86 & 1.0 & 0.71 & 0.92 & 0.64 & 0.9 & 0.64 & 0.56 & 0.71 & 0.86 & 0.86 & 0.78 & 0.73\\
			Ferry & 1.0 & 0.57 & 1.0 & 1.0 & 1.0 & 1.0 & 1.0 & 0.86 & 1.0 & 0.57 & 1.0 & 1.0 & 1.0 & 1.0 & 1.0 & 0.86\\
			Floortile & 0.52 & 0.68 & 0.64 & 0.82 & 0.83 & 0.91 & 0.66 & 0.80 & 0.68 & 0.68 & 0.89 & 0.73 & 1.0 & 0.82 & 0.86 & 0.74\\
			Grid & 0.62 & 0.47 & 0.75 & 0.86 & 0.78 & 1.0 & 0.71 & 0.78 & 0.79 & 0.65 & 1.0 & 0.86 & 0.88 & 1.0 & 0.89 & 0.83 \\
			Gripper & 1.0 & 0.67 & 1.0 & 1.0 & 1.0 & 1.0 & 1.0 & 0.89 & 1.0 & 0.67 & 1.0 & 1.0 & 1.0 & 1.0 & 1.0 & 0.89\\
			Hanoi & 1.0 & 0.50 & 1.0 & 1.0 & 1.0 & 1.0 & 1.0 & 0.83 & 0.75 & 0.75 & 1.0 & 1.0 & 1.0 & 1.0 & 0.92 & 0.92\\
			Miconic & 0.75 & 0.33 & 0.50 & 0.50 & 0.75 & 1.0 & 0.67 & 0.61 & 0.89 & 0.89 & 1.0 & 0.75 & 0.75 & 1.0 & 0.88 & 0.88\\
			Satellite & 0.60 & 0.21 & 1.0 & 1.0 & 1.0 & 0.75 & 0.87 & 0.65 & 0.82 & 0.64 & 1.0 & 1.0 & 1.0 & 0.75 & 0.94 & 0.80\\
			Transport & 1.0 & 0.40 & 1.0 & 1.0 & 1.0 & 0.80 & 1.0 & 0.73 & 1.0 & 0.70 & 0.83 & 1.0 & 1.0 & 0.80 & 0.94 & 0.83\\
			Visitall & 1.0 & 0.50 & 1.0 & 1.0 & 1.0 & 1.0 & 1.0 & 0.83 & 1.0 & 1.0 & 1.0 & 1.0 & 1.0 & 1.0 & 1.0 & 1.0\\
			Zenotravel & 1.0 & 0.36 & 1.0 & 1.0 & 1.0 & 0.71 & 1.0 & 0.69 &1.0 & 0.64 & 0.88 & 1.0 & 1.0 & 0.71 & 0.96 & 0.79\\
			\hline
			\bf  & 0.88 & 0.50 & 0.88 & 0.92 & 0.95 & 0.91 & 0.90 & 0.78 & 0.90 & 0.74 & 0.93 & 0.92 & 0.96 & 0.91 & 0.93 & 0.86\\
		\end{tabular}
	}
\caption{\small {\em Precision} and {\em recall} scores for learning tasks from labeled plans without (left) and with (right) static predicates.}
\label{tab:results_plans}
\end{table*}

Table~\ref{tab:time_plans} reports the total planning time, the preprocessing time (in seconds) invested by {\sc Madagascar} to solve the planning instances that result from our compilation as well as the number of actions of the solution plans. All the learning tasks are solved in a few seconds. Interestingly, one can identify the domains with static predicates by just looking at the reported plan length. In these domains some of the preconditions that correspond to static predicates are directly derived from the learning examples and therefore fewer programming actions are required. When static predicates are identified, the resulting compilation is also much more compact and produces smaller planning/instantiation times.



\begin{table}
\begin{scriptsize}
	\begin{center}
		\begin{tabular}{l|c|c|c||c|c|c|}
                         & \multicolumn{3}{|c||}{\bf No Static}& \multicolumn{3}{|c|}{\bf Static}\\
			 & Total & Preprocess & Length  & Total & Preprocess &  Length\\
                         \hline
			Blocks & 0.04 & 0.00 & 72  & 0.03 & 0.00 & 72 \\
			Driverlog & 0.14 & 0.09 & 83 & 0.06 & 0.03 & 59 \\
			Ferry & 0.06 & 0.03 & 55 & 0.06 & 0.03 & 55 \\
			Floortile & 2.42 & 1.64 & 168 & 0.67 & 0.57 & 77 \\
			Grid & 4.82 & 4.75 & 88 & 3.39 & 3.35 & 72 \\
			Gripper & 0.03 & 0.01 & 43 & 0.01 & 0.00 & 43 \\
                        Hanoi & 0.12 & 0.06 & 48 & 0.09 & 0.06 & 39 \\
                        Miconic & 0.06 & 0.03 & 57 & 0.04 & 0.00 & 41 \\
			Satellite & 0.20 & 0.14 & 67 & 0.18 & 0.12 & 60 \\
			Transport & 0.59 & 0.53 & 61 & 0.39 & 0.35 & 48 \\
			Visitall & 0.21 & 0.15 & 40 & 0.17 & 0.15 & 36 \\
			Zenotravel & 2.07 & 2.04 & 71 & 1.01 & 1.00 & 55 \\			
		\end{tabular}
	\end{center}
        \end{scriptsize}
	\caption{\small Total planning time, preprocessing time and plan length for learning tasks from labeled plans without/with static predicates.}
	\label{tab:time_plans}	
\end{table}

We evaluate now the ability of our approach to support partially specified action models; that is, when the input model $\mathcal{M}$ is not empty because some preconditions and effects of the actions are initially known.  In this particular experiment, the model of half of the actions is given in $\mathcal{M}$ as an extra input of the learning task. Tables~\ref{tab:results_plans_partial} and~\ref{tab:time_plans_partial} summarize the obtained results, which include the identification of static predicates. We only report the {\em precision} and {\em recall} of the {\em unknown} actions since the values of the metrics of the {\em known} action models is 1.0. In this experiment, a low value of {\em precision} or {\em recall} has a greater impact than in the previous learning tasks because the evaluation is done only over half of the actions. This occurs, for instance, in the precondition \emph{recall} of domains such as {\em Floortile}, {\em Gripper} or {\em Satellite}.

Remarkably, the overall \emph{precision} is now $0.98$, which means that the contents of the learned models is highly reliable. The value of \emph{recall}, 0.87, is an indication that the learned models still miss some information (preconditions are again the component more difficult to be fully learned). Overall, the results confirm the previous trend: the more input knowledge of the task, the better the models and the less planning time. Additionally, the solution plans required for this task are smaller because it is only necessary to program half of the actions (the other half are included in the input knowledge $\mathcal{M}$). {\em Visitall} and {\em Hanoi} are excluded from this evaluation because they only contain one action schema.

\begin{table}[hbt!]
\begin{footnotesize}
	\begin{center}
		
		\begin{tabular}{l|l|l|l|l|l|l||l|l|}
			 & \multicolumn{2}{|c|}{\bf Pre} & \multicolumn{2}{|c|}{\bf Add} & \multicolumn{2}{|c||}{\bf Del} & \multicolumn{2}{|c}{\bf}\\ \cline{2-9}			
			  & \multicolumn{1}{|c|}{\bf P} & \multicolumn{1}{|c|}{\bf R} & \multicolumn{1}{|c|}{\bf P} & \multicolumn{1}{|c|}{\bf R} & \multicolumn{1}{|c|}{\bf P} & \multicolumn{1}{|c||}{\bf R} &  \multicolumn{1}{|c|}{\bf P} & \multicolumn{1}{|c|}{\bf R} \\
			\hline
				Blocks & 1.0 & 1.0 & 1.0 & 1.0 & 1.0 & 1.0 & 1.0 & 1.0 \\
				Driverlog & 1.0 & 0.71 & 1.0 & 1.0 & 1.0 & 1.0 & 1.0 & 0.90 \\
				Ferry & 1.0 & 0.67 & 1.0 & 1.0 & 1.0 & 1.0 & 1.0 & 0.89 \\
				Floortile & 0.75 & 0.60 & 1.0 & 0.80 & 1.0 & 0.80 & 0.92 & 0.73 \\
                Grid & 1.0 & 0.67 & 1.0 & 1.0 & 1.0 & 1.0 & 0.84 & 0.78 \\
				Gripper & 1.0 & 0.50 & 1.0 & 1.0 & 1.0 & 1.0 & 1.0 & 0.83 \\
				Miconic & 1.0 & 1.0 & 1.0 & 1.0 & 1.0 & 1.0 & 1.0 & 1.0 \\
				Satellite & 1.0 & 0.57 & 1.0 & 1.0 & 1.0 & 1.0 & 1.0 & 0.86 \\
				Transport & 1.0 & 0.75 & 1.0 & 1.0 & 1.0 & 1.0 & 1.0 & 0.92 \\
				Zenotravel & 1.0 & 0.67 & 1.0 & 1.0 & 1.0 & 0.67 & 1.0 & 0.78 \\
				\hline
				\bf  & 0.98 & 0.71 & 1.0 & 0.98 & 1.0 & 0.95 & 0.98 & 0.87 \\
			\end{tabular}
		
	\end{center}
\end{footnotesize}
\caption{\small {\em Precision} and {\em recall} scores for learning tasks with partially specified action models.}
\label{tab:results_plans_partial}
\end{table}

\begin{table}
\begin{footnotesize}
	\begin{center}
		\begin{tabular}{l|c|c|c|}			
			 & Total time & Preprocess & Plan length  \\
                         \hline
			Blocks & 0.07 & 0.01 & 54  \\
			Driverlog & 0.03 & 0.01 & 40 \\
			Ferry & 0.06 & 0.03 & 45 \\
			Floortile & 0.43 & 0.42 & 55 \\
                        Grid & 3.12 & 3.07 & 53 \\
			Gripper & 0.03 & 0.01 & 35 \\
			Miconic & 0.03 & 0.01 & 34  \\
			Satellite & 0.14 & 0.14 & 47 \\
			Transport & 0.23 & 0.21 & 37 \\
			Zenotravel & 0.90 & 0.89 & 40 \\
		\end{tabular}
	\end{center}
        \end{footnotesize}
	\caption{\small Time and plan length learning for learning tasks with partially specified action models.}
	\label{tab:time_plans_partial}	
\end{table}


\subsubsection{Learning from state observations}
Here we evaluate our approach with learning tasks of the kind $\Lambda=\tup{\mathcal{M},\mathcal{O},\Psi}$, where the action of the executed plans are not available but the initial and goal states are known. When input plans are not available, the planner must not only compute the action models but also the plans that satisfy the input observations. Table~\ref{tab:results_states} and ~\ref{tab:time_states} summarize the results obtained for this using static predicates and partially specified models. Values for the {\em Zenotravel} and {\em Grid} domains are not reported because {\sc Madagascar} was not able to solve the corresponding planning tasks within a 1000 sec. time bound. The values of \emph{precision} and \emph{recall} are significantly lower than in Table ~\ref{tab:results_plans}. Given that the learning hypothesis space is now fairly under-constrained, actions can be reformulated and still be compliant with the inputs (e.g. the {\em blocksworld} operator {\small\tt stack} can be {\em learned} with the preconditions and effects of the {\small\tt unstack} operator and vice versa). We tried to minimize this effect with the additional input knowledge (static predicates and partially specified action models) and yet the results are below the scores obtained when learning from labeled plans.


\begin{table}
\begin{footnotesize}
	\begin{center}
		\begin{tabular}{l|l|l|l|l|l|l||l|l|}
			 & \multicolumn{2}{|c|}{\bf Pre} & \multicolumn{2}{|c|}{\bf Add} & \multicolumn{2}{|c||}{\bf Del} & \multicolumn{2}{|c}{\bf}\\ \cline{2-9}			
			  & \multicolumn{1}{|c|}{\bf P} & \multicolumn{1}{|c|}{\bf R} & \multicolumn{1}{|c|}{\bf P} & \multicolumn{1}{|c|}{\bf R} & \multicolumn{1}{|c|}{\bf P} & \multicolumn{1}{|c||}{\bf R} &  \multicolumn{1}{|c|}{\bf P} & \multicolumn{1}{|c|}{\bf R} \\
			\hline
            Blocks & 0.33 & 0.33 & 0.75 & 0.50 & 0.33 & 0.33 & 0.47 & 0.39 \\
            Driverlog & 1.0 & 0.29 & 0.33 & 0.67 & 1.0 & 0.50 & 0.78 & 0.48 \\
            Ferry & 1.0 & 0.67 & 0.50 & 1.0 & 1.0 & 1.0 & 0.83 & 0.89 \\
            Floortile & 0.67 & 0.40 & 0.50 & 0.40 & 1.0 & 0.40 & 0.72 & 0.40 \\
            Grid & - & - & - & - & - & - & - & - \\
            Gripper & 1.0 & 0.50 & 1.0 & 1.0 & 1.0 & 1.0 & 1.0 & 0.83 \\
            Miconic & 0.0 & 0.0 & 0.33 & 0.50 & 0.0 & 0.0 & 0.11 & 0.17 \\
            Satellite & 1.0 & 0.14 & 0.67 & 1.0 & 1.0 & 1.0 & 0.89 & 0.71 \\
            Transport & 0.0 & 0.0 & 0.25 & 0.5 & 0.0 & 0.0 & 0.08 & 0.17 \\
            Zenotravel & - & - & - & - & - & - & - & - \\
            \hline
            & 0.63 & 0.29 & 0.54 & 0.70 & 0.67 & 0.53 & 0.61 & 0.51 \\			
			\end{tabular}
	\end{center}
\end{footnotesize}
\caption{\small {\em Precision} and {\em recall} scores for learning from (initial,final) state pairs.}
\label{tab:results_states}
\end{table}

\begin{table}
\begin{footnotesize}
	\begin{center}
		\begin{tabular}{l|c|c|c|}			
			 & Total time & Preprocess & Plan length  \\
			\hline
            Blocks & 2.14 & 0.00 & 58  \\
            Driverlog & 0.09 & 0.00 & 88 \\
            Ferry & 0.17 & 0.01 & 65 \\
            Floortile & 6.42 & 0.15 & 126 \\
            Grid & - & - & - \\
            Gripper & 0.03 & 0.00 & 47 \\
            Miconic & 0.04 & 0.00 & 68 \\
            Satellite & 4.34 & 0.10 & 126 \\
            Transport & 2.57 & 0.21 & 47 \\			
            Zenotravel & - & - & - \\
		\end{tabular}
	\end{center}
        \end{footnotesize}
	\caption{\small Time and plan length when learning from (initial,final) state pairs.}
	\label{tab:time_states}	
\end{table}

Now we evaluate our approach with learning tasks of the kind $\Lambda=\tup{\mathcal{M},\mathcal{O},\Psi}$, where the action of the executed plans are not available but where the observation sequence $\mathcal{O}$ contains all the intermediate states, not just the initial and final states. Table~\ref{fig:observationsnomap} shows the precision ({\bf P}) and recall ({\bf R}) computed separately for the preconditions ({\bf Pre}), positive effects ({\bf Add}) and negative Effects ({\bf Del}) while the last two columns report averages values. The reason why the scores in Table ~\ref{fig:observationsnomap} are still low, despite more state observations are available, is because the syntax-based nature of {\em precision} and {\em recall} make these two metrics report low scores for learned models that are semantically correct but correspond to {\em reformulations} of the actual model (changes in the roles of actions with matching headers or parameters with matching types).

\begin{table}[hbt!]
	\begin{center}
		\begin{scriptsize}
			\begin{tabular}{l|l|l|l|l|l|l||l|l|}
				& \multicolumn{2}{|c|}{\bf Pre} & \multicolumn{2}{|c|}{\bf Add} & \multicolumn{2}{|c||}{\bf Del} & \multicolumn{2}{|c}{\bf}\\ \cline{2-9}			
				& \multicolumn{1}{|c|}{\bf P} & \multicolumn{1}{|c|}{\bf R} & \multicolumn{1}{|c|}{\bf P} & \multicolumn{1}{|c|}{\bf R} & \multicolumn{1}{|c|}{\bf P} & \multicolumn{1}{|c||}{\bf R} &  \multicolumn{1}{|c|}{\bf P} & \multicolumn{1}{|c|}{\bf R} \\
				\hline

				blocks & 0.44 & 0.44 & 0.44 & 0.44 & 0.44 & 0.44 & 0.44 & 0.44 \\
				driverlog & 0.0 & 0.0 & 0.25 & 0.43 & 0.0 & 0.0 & 0.08 & 0.14 \\
				ferry & 1.0 & 0.71 & 1.0 & 1.0 & 1.0 & 1.0 & 1.0 & 0.9 \\
				floortile & 0.38 & 0.55 & 0.4 & 0.18 & 0.56 & 0.45 & 0.44 & 0.39 \\
				grid & 0.5 & 0.47 & 0.33 & 0.29 & 0.25 & 0.29 & 0.36 & 0.35 \\
				gripper & 0.83 & 0.83 & 0.75 & 0.75 & 0.75 & 0.75 & 0.78 & 0.78 \\
				hanoi & 0.5 & 0.25 & 0.5 & 0.5 & 0.0 & 0.0 & 0.33 & 0.25 \\
				hiking & 0.43 & 0.43 & 0.5 & 0.35 & 0.44 & 0.47 & 0.46 & 0.42 \\
				miconic & 0.6 & 0.33 & 0.33 & 0.25 & 0.33 & 0.33 & 0.42 & 0.31 \\
				npuzzle & 0.33 & 0.33 & 0.0 & 0.0 & 0.0 & 0.0 & 0.11 & 0.11 \\
				parking & 0.25 & 0.21 & 0.0 & 0.0 & 0.0 & 0.0 & 0.08 & 0.07 \\
				satellite & 0.6 & 0.21 & 0.8 & 0.8 & 1.0 & 0.5 & 0.8 & 0.5 \\
				transport & 1.0 & 0.3 & 0.8 & 0.8 & 1.0 & 0.6 & 0.93 & 0.57 \\
				visitall & 0.0 & 0.0 & 0.0 & 0.0 & 0.0 & 0.0 & 0.0 & 0.0 \\
				zenotravel & 0.67 & 0.29 & 0.33 & 0.29 & 0.33 & 0.14 & 0.44 & 0.24
			\end{tabular}
		\end{scriptsize}
	\end{center}
	\caption{\small Precision and recall values obtained without computing the $f_{P\&R}$ mapping with the reference model.}
	\label{fig:observationsnomap}
\end{table}

To give an insight of the actual quality of the learned models, we defined a method for computing {\em Precision} and {\em Recall} that is robust to the mentioned model {\em reformulations}. Precision and recall are often combined using the {\em harmonic mean}. This expression, called the {\em F-measure} or the balanced {\em F-score}, is defined as $F=2\times\frac{Precision\times Recall}{Precision+Recall}$. Given the learned action model $\mathcal{M}$ and the reference action model $\mathcal{M}^*$, the bijective function $f_{P\&R}:\mathcal{M} \mapsto \mathcal{M}^*$ is the mapping between the learned and the reference model that maximizes the accumulated {\em F-measure} (considering swaps in the actions with matching headers or parameters with matching types).

Table~\ref{fig:observationsmap} shows that significantly higher values of {\em precision} and {\em recall} are reported when a learned action schema, $\xi\in\mathcal{M}$, is compared to its corresponding reference schema given by the $f_{P\&R}$ mapping ($f_{P\&R}(\xi)\in \mathcal{M}^*$). The {\em blocksworld} and {\em gripper} domains are perfectly learned from only 25 state observations. These results evidence that in all of the evaluated domains, except for {\em ferry} and {\em satellite}, the learning task swaps the roles of some actions (or parameters) with respect to their role in the reference model. 

\begin{table}
        \begin{center}
          
		\begin{scriptsize}
			\begin{tabular}{l|l|l|l|l|l|l||l|l|}
				& \multicolumn{2}{|c|}{\bf Pre} & \multicolumn{2}{|c|}{\bf Add} & \multicolumn{2}{|c||}{\bf Del} & \multicolumn{2}{|c}{\bf}\\ \cline{2-9}			
				& \multicolumn{1}{|c|}{\bf P} & \multicolumn{1}{|c|}{\bf R} & \multicolumn{1}{|c|}{\bf P} & \multicolumn{1}{|c|}{\bf R} & \multicolumn{1}{|c|}{\bf P} & \multicolumn{1}{|c||}{\bf R} &  \multicolumn{1}{|c|}{\bf P} & \multicolumn{1}{|c|}{\bf R} \\
				\hline

				blocks & 1.0 & 1.0 & 1.0 & 1.0 & 1.0 & 1.0 & 1.0 & 1.0 \\
				driverlog & 0.67 & 0.14 & 0.33 & 0.57 & 0.67 & 0.29 & 0.56 & 0.33 \\
				ferry & 1.0 & 0.71 & 1.0 & 1.0 & 1.0 & 1.0 & 1.0 & 0.9 \\
				floortile & 0.44 & 0.64 & 1.0 & 0.45 & 0.89 & 0.73 & 0.78 & 0.61 \\
				grid & 0.63 & 0.59 & 0.67 & 0.57 & 0.63 & 0.71 & 0.64 & 0.62 \\
				gripper & 1.0 & 1.0 & 1.0 & 1.0 & 1.0 & 1.0 & 1.0 & 1.0 \\
				hanoi & 1.0 & 0.5 & 1.0 & 1.0 & 1.0 & 1.0 & 1.0 & 0.83 \\
				hiking & 0.78 & 0.6 & 0.93 & 0.82 & 0.88 & 0.88 & 0.87 & 0.77 \\
				miconic & 0.8 & 0.44 & 1.0 & 0.75 & 1.0 & 1.0 & 0.93 & 0.73 \\
				npuzzle & 0.67 & 0.67 & 1.0 & 1.0 & 1.0 & 1.0 & 0.89 & 0.89 \\
				parking & 0.56 & 0.36 & 0.5 & 0.33 & 0.5 & 0.33 & 0.52 & 0.34 \\
				satellite & 0.6 & 0.21 & 0.8 & 0.8 & 1.0 & 0.5 & 0.8 & 0.5 \\
				transport & 1.0 & 0.3 & 1.0 & 1.0 & 1.0 & 0.6 & 1.0 & 0.63 \\
				visitall & 0.67 & 1.0 & 1.0 & 1.0 & 1.0 & 1.0 & 0.89 & 1.0 \\
				zenotravel & 1.0 & 0.43 & 0.67 & 0.57 & 1.0 & 0.43 & 0.89 & 0.48
			\end{tabular}
		\end{scriptsize}
	\end{center}
	\caption{\small Precision and recall values obtained when computing the $f_{P\&R}$ mapping with the reference model.}
	\label{fig:observationsmap}
\end{table}


\subsection{Evaluating with a test set}

When a reference model is not available, the learned models are tested with an observation set. Table~\ref{fig:observationstest} summarizes the results obtained when evaluating the quality of the learned models with respect to a test set of state observations. Each test set comprises between 20 and 50 observations per domain and is generated executing the plans for various instances of the IPC domains and collecting the intermediate states. The table shows, for each domain, the {\em observation edit distance} (computed with our extended compilation), the {\em maximum edit distance}, and their ratio. The reported results show that, despite learning only from 25 state observations, 12 out of 15 learned domains yield ratios of $90\%$ or above. This fact evidences that the learned models require very small amounts of edition to match the observations of the given test set.

\begin{table}[hbt!]
		\begin{center}
                \begin{footnotesize}
			\begin{tabular}{l|r|r|c|}
				& $\delta(\mathcal{M},\mathcal{O})$ & $\delta(\mathcal{M},*)$ & $1-\frac{\delta(\mathcal{M},\mathcal{O})}{\delta(\mathcal{M},*)}$ \\
				\hline
				blocks & 0 & 90 & 1.0 \\
				driverlog & 5 & 144 & 0.97 \\
				ferry & 2 & 69 & 0.97 \\
				floortile & 34 & 342 & 0.90 \\
				grid & 42 & 153 & 0.73 \\
				gripper & 2 & 30 & 0.93 \\
				hanoi & 1 & 63 & 0.98 \\
				hiking & 69 & 174 & 0.60 \\
				miconic & 3 & 72 & 0.96 \\
				npuzzle & 2 & 24 & 0.92 \\
                                parking & 4 & 111 & 0.96 \\
				satellite & 24 & 75 & 0.68 \\
				transport & 4 & 78 & 0.95 \\
				visitall & 2 & 24 & 0.92 \\
				zenotravel & 3 & 63 & 0.95
			\end{tabular}
                        	\end{footnotesize}
		\end{center}
	\caption{\small Evaluation of the quality of the learned models with respect to an observations test set.}
	\label{fig:observationstest}
\end{table}

The learning scores of several domains in Table~\ref{fig:observationsmap} are above the ones reported in Table~\ref{fig:observationsnomap}. The reason lies in the particular observations comprised by the test sets. As an example, in the {\em driverlog} domain, the action schema {\small \tt disembark-truck} is missing from the learned model because this action is never induced from the observations in the training set; that is, such action never appears in the corresponding \emph{unobserved} plan. The same happens with the {\small \tt paint-down} action of the {\em floortile} domain or {\small \tt move-curb-to-curb} in the {\em parking} domain. Interestingly, these actions do not appear either in the test sets and so the learned action models are not penalized in Table~\ref{fig:observationstest}. Generating {\em informative} and {\em representative} observations for learning planning action models is an open issue. Planning actions include preconditions that are only satisfied by specific sequences of actions, often, with a low probability of being chosen by chance~\cite{fern2004learning}.

  
\section{Conclusions}
\label{sec:Section9}
We presented a novel approach for learning \strips\ action models from examples using classical planning. The approach is flexible to various amount and kind of input knowledge and accepts partially specified action models. Unlike extensive-data ML approaches, our work explores an alternative research direction to learn sound models from very small data sets. The action models of the {\em blocksworld} or {\em gripper} domains were perfectly learned from only 25 state observations. Moreover, in 12 out of the 15 domains, the learned models yield {\em Precision} values over $0.75$.

To the best of our knowledge, this is the first work on learning \strips\ action models from state observations, using exclusively an {\em off-the-shelf} classical planner, and evaluated over a wide range of different domains. Recently, the work in~\cite{SternJ17} proposes a planning compilation for learning action models from plan traces following the {\em finite domain} representation for the state variables. This is a theoretical study on the boundaries of the learned models and no experimental results are reported.

We also introduced the {\em precision} and {\em recall} metrics, widely used in ML, for evaluating the learned action models with respect to a given reference model. These two metrics measure the soundness and completeness of the learned models and facilitate the identification of model flaws.

When example plans are available, we can compute accurate action models from small sets of learning examples in little computation time, less than a second. In many applications, the actual actions executed by the observed agent are not available but, instead, the resulting states can be observed. With this regard, we extended our approach for learning also from state observations as it broadens the range of application to external observers and facilitates the representation of imperfect observability, as shown in plan recognition \cite{SohrabiRU16}, as well as learning from unstructured data, like state images \cite{AsaiF18}. When action plans are not available, our approach still produces action models that are compliant with the input information. In this case, since learning is not constrained by actions, operators can be reformulated changing their semantics, in which case the comparison with a reference model turns out to be tricky.

We also introduced a semantic method for evaluating the learned \strips\ action models with respect to observations of plan executions. Generating {\em informative} examples for learning planning action models is still an open issue. Planning actions include preconditions that are only satisfied by specific sequences of actions which have low probability of being chosen by chance~\cite{fern2004learning}. The success of recent algorithms for exploring planning tasks~\cite{FrancesRLG17} motivates the development of novel techniques that enable to autonomously collect informative learning examples. The combination of such exploration techniques with our learning approach is an appealing research direction that opens up the door to the bootstrapping of planning action models.


\subsection*{Acknowledgment}
This work is supported by the Spanish MINECO project TIN2017-88476-C2-1-R. Diego Aineto is partially supported by the {\it FPU16/03184} and Sergio Jim\'enez by the {\it RYC15/18009}, both programs funded by the Spanish government.


%% References
%%
%% Following citation commands can be used in the body text:
%% Usage of \cite is as follows:
%%   \cite{key}         ==>>  [#]
%%   \cite[chap. 2]{key} ==>> [#, chap. 2]
%%

%% References with BibTeX database:

\bibliographystyle{elsarticle-num}
\bibliography{planlearnbibliography.bib}

%% Authors are advised to use a BibTeX database file for their reference list.
%% The provided style file elsarticle-num.bst formats references in the required Procedia style

%% For references without a BibTeX database:

% \begin{thebibliography}{00}

%% \bibitem must have the following form:
%%   \bibitem{key}...
%%

% \bibitem{}

% \end{thebibliography}



\end{document}

%%
%% End of file `ecrc-template.tex'. 
