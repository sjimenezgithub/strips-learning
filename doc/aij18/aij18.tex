% Template for Elsevier CRC journal article
% version 1.1 dated 16 March 2010

% This file (c) 2010 Elsevier Ltd.  Modifications may be freely made,
% provided the edited file is saved under a different name

% This file contains modifications for Procedia Computer Science
% but may easily be adapted to other journals

% Changes since version 1.0
% - elsarticle class option changed from 1p to 3p (to better reflect CRC layout)

%-----------------------------------------------------------------------------------

%% This template uses the elsarticle.cls document class and the extension package ecrc.sty
%% For full documentation on usage of elsarticle.cls, consult the documentation "elsdoc.pdf"
%% Further resources available at http://www.elsevier.com/latex

%-----------------------------------------------------------------------------------

%%%%%%%%%%%%%%%%%%%%%%%%%%%%%%%%%%%%%%%%%%%%%%
%%%%%%%%%%%%%%%%%%%%%%%%%%%%%%%%%%%%%%%%%%%%%%
%%                                          %%
%% Important note on usage                  %%
%% -----------------------                  %%
%% This file must be compiled with PDFLaTeX %%
%% Using standard LaTeX will not work!      %%
%%                                          %%
%%%%%%%%%%%%%%%%%%%%%%%%%%%%%%%%%%%%%%%%%%%%%%
%%%%%%%%%%%%%%%%%%%%%%%%%%%%%%%%%%%%%%%%%%%%%%

%% The '3p' and 'times' class options of elsarticle are used for Elsevier CRC
\documentclass[3p,times]{elsarticle}

%% The `ecrc' package must be called to make the CRC functionality available
\usepackage{ecrc}
%% The ecrc package defines commands needed for running heads and logos.
%% For running heads, you can set the journal name, the volume, the starting page and the authors

%% set the volume if you know. Otherwise `00'
\volume{00}

%% set the starting page if not 1
\firstpage{1}

%% Give the name of the journal
\journalname{Artificial Intelligence}

%% Give the author list to appear in the running head
%% Example \runauth{C.V. Radhakrishnan et al.}
\runauth{}

%% The choice of journal logo is determined by the \jid and \jnltitlelogo commands.
%% A user-supplied logo with the name <\jid>logo.pdf will be inserted if present.
%% e.g. if \jid{yspmi} the system will look for a file yspmilogo.pdf
%% Otherwise the content of \jnltitlelogo will be set between horizontal lines as a default logo

%% Give the abbreviation of the Journal.
\jid{procs}

%% Give a short journal name for the dummy logo (if needed)
\jnltitlelogo{Artificial Intelligence}

%% Hereafter the template follows `elsarticle'.
%% For more details see the existing template files elsarticle-template-harv.tex and elsarticle-template-num.tex.

%% Elsevier CRC generally uses a numbered reference style
%% For this, the conventions of elsarticle-template-num.tex should be followed (included below)
%% If using BibTeX, use the style file elsarticle-num.bst

%% End of ecrc-specific commands
%%%%%%%%%%%%%%%%%%%%%%%%%%%%%%%%%%%%%%%%%%%%%%%%%%%%%%%%%%%%%%%%%%%%%%%%%%

%% The amssymb package provides various useful mathematical symbols
\usepackage{amssymb}
%% The amsthm package provides extended theorem environments
%%\usepackage{amsthm}

%% The lineno packages adds line numbers. Start line numbering with
%% \begin{linenumbers}, end it with \end{linenumbers}. Or switch it on
%% for the whole article with \linenumbers after \end{frontmatter}.
%% \usepackage{lineno}

%% natbib.sty is loaded by default. However, natbib options can be
%% provided with \biboptions{...} command. Following options are
%% valid:

%%   round  -  round parentheses are used (default)
%%   square -  square brackets are used   [option]
%%   curly  -  curly braces are used      {option}
%%   angle  -  angle brackets are used    <option>
%%   semicolon  -  multiple citations separated by semi-colon
%%   colon  - same as semicolon, an earlier confusion
%%   comma  -  separated by comma
%%   numbers-  selects numerical citations
%%   super  -  numerical citations as superscripts
%%   sort   -  sorts multiple citations according to order in ref. list
%%   sort&compress   -  like sort, but also compresses numerical citations
%%   compress - compresses without sorting
%%
%% \biboptions{comma,round}

% \biboptions{}

% if you have landscape tables
\usepackage[figuresright]{rotating}

\usepackage{amsthm}
\usepackage{amsmath}
\usepackage{amssymb}
\usepackage{times}
\usepackage{helvet}
\usepackage{courier}
\usepackage{pstricks}
\usepackage{pst-node}
\usepackage{multirow}
\usepackage{listings}
\usepackage{xspace}


\usepackage{pgf}
\usepackage{tikz}
\usetikzlibrary{calc,backgrounds,positioning,fit}
\usepackage{subcaption}
\usetikzlibrary{arrows,automata}
\usepackage{arydshln}



% put your own definitions here:
%   \newcommand{\cZ}{\cal{Z}}
%   \newtheorem{def}{Definition}[section]
%   ...

\newtheorem{mytheorem}{Theorem}
\newtheorem{mylemma}[mytheorem]{Lemma}
\newtheorem{mydefinition}[mytheorem]{Definition}
\newtheorem{myproposition}[mytheorem]{Proposition}
\newtheorem{myconstruction}{Construction}


\mathchardef\mh="2D
\newcommand{\pre}{\mathsf{pre}}  % precondition
\newcommand{\eff}{\mathsf{eff}}  % effect
\newcommand{\cond}{\mathsf{cond}}   % conditional effect
\newcommand{\add}{\mathsf{add}}  % add effect
\newcommand{\del}{\mathsf{del}}  % delete effect
\newcommand{\PE}{\mathrm{PE}}     % precondition
\newcommand{\PSPACE}{\mathrm{PSPACE}}     % precondition
\newcommand{\NPSPACE}{\mathrm{NPSPACE}}     % precondition
\newcommand{\strips}{\textsc{Strips}}     % precondition


\newcommand{\ARMS}{{\small {\sffamily ARMS}}\xspace}
\newcommand{\CAMA}{{\small {\sffamily CAMA}}\xspace}
\newcommand{\SLAF}{{\small {\sffamily SLAF}}\xspace}
\newcommand{\LAMP}{{\small {\sffamily LAMP}}\xspace}
\newcommand{\NOISTA}{{\small {\sffamily NOISTA}}\xspace}
\newcommand{\LOCM}{{\small {\sffamily LOCM}}\xspace}
\newcommand{\LOCMtwo}{{\small {\sffamily LOCM2}}\xspace}
\newcommand{\LOP}{{\small {\sffamily LOP}}\xspace}
\newcommand{\AMAN}{{\small {\sffamily AMAN}}\xspace}
\newcommand{\FAMA}{{\small {\sffamily FAMA}}\xspace}

\newcommand{\FO}{{\small {\sffamily FO}}\xspace}
\newcommand{\PO}{{\small {\sffamily PO}}\xspace}
\newcommand{\POstar}{{\small {\sffamily PO*}}\xspace}
\newcommand{\NO}{{\small {\sffamily NO}}\xspace}

\newcommand{\pbox}[1]{\makebox[2em][l]{#1}}

\newcommand{\tup}[1]{{\langle #1 \rangle}}

\lstset{
  basicstyle=\ttfamily,
  mathescape
}

% add words to TeX's hyphenation exception list
%\hyphenation{author another created financial paper re-commend-ed Post-Script}

% declarations for front matter

\begin{document}

\begin{frontmatter}

%% Title, authors and addresses

%% use the tnoteref command within \title for footnotes;
%% use the tnotetext command for the associated footnote;
%% use the fnref command within \author or \address for footnotes;
%% use the fntext command for the associated footnote;
%% use the corref command within \author for corresponding author footnotes;
%% use the cortext command for the associated footnote;
%% use the ead command for the email address,
%% and the form \ead[url] for the home page:
%%
%% \title{Title\tnoteref{label1}}
%% \tnotetext[label1]{}
%% \author{Name\corref{cor1}\fnref{label2}}
%% \ead{email address}
%% \ead[url]{home page}
%% \fntext[label2]{}
%% \cortext[cor1]{}
%% \address{Address\fnref{label3}}
%% \fntext[label3]{}

\dochead{}
%% Use \dochead if there is an article header, e.g. \dochead{Short communication}

\title{Learning action models with minimal observability}
\author[label1]{Diego Aineto}
\author[label1]{Sergio Jim\'{e}nez Celorrio}
\author[label1]{Eva Onaindia}
\address[label1]{Department of Computer Systems and Computation, Universitat Polit\`ecnica de Val\`encia. Spain}


%% use optional labels to link authors explicitly to addresses:
%% \author[label1,label2]{<author name>}
%% \address[label1]{<address>}
%% \address[label2]{<address>}


\begin{abstract}
  This paper presents \FAMA, a novel approach for learning \strips\ action models from observations of plan executions that compiles the learning task into a classical planning task. Unlike all existing learning systems, \FAMA is able to learn when the actions of the plan executions are partially or totally unobservable and information on intermediate states is partially provided. This flexibility makes \FAMA an ideal learning approach in domains where only sensoring data are accessible. Additionally, we leverage the compilation scheme and extend it to come up with an evaluation method that allows us to assess the quality of a learned model syntactically, that is, with respect to the actual model; and, semantically, that is, with respect to a set of observations of plan executions. We also show that the extended compilation scheme can be used to lay the foundations of a framework for action model comparison. \FAMA is exhaustively evaluated over a wide range of IPC domains and its performance is compared to \ARMS, a state-of-the-art benchmark in action model learning.
\end{abstract}

\begin{keyword}
Action model learning\sep AI planning\sep Machine Learning
%% keywords here, in the form: keyword \sep keyword

%% MSC codes here, in the form: \MSC code \sep code
%% or \MSC[2008] code \sep code (2000 is the default)
\end{keyword}

\end{frontmatter}

%%
%% Start line numbering here if you want
%%
% \linenumbers

%% main text

%% The Appendices part is started with the command \appendix;
%% appendix sections are then done as normal sections
%% \appendix

%% \section{}
%% \label{}

% HLP: Expressiveness is pushed when pure compilations are used. Otherwise we just use them.



\section{Introduction}
\label{sec:introduction}

There is common agreement in the planning community that the unavailability of an \textcolor[rgb]{1.00,0.00,0.00}{adequate} domain model is a bottleneck in the applicability of planning technology to many real-world domains~\cite{kambhampati:modellite:AAAI2007}. Motivated by the difficulty and cost of crafting action models, research in action-model learning has seen huge advances. Since the emergence of pioneer learning systems like ARMS~\cite{yang2007learning}, we have seen systems able to learn action models with quantifiers~\cite{AmirC08,ZhuoYHL10}, from noisy actions or noisy states~\cite{zhuo2013action,MouraoZPS12}, from null state information~\cite{cresswell2013acquiring}, from incomplete domain models~\cite{ZhuoNK13,ZhuoK17} and many more.

A system for learning planning action models receives as an input observations of the agent's plan execution and \textcolor[rgb]{1.00,0.00,0.00}{outputs an approximation of the actions that embody the physics of the real-world domain being modeled.} The primary underlying motivation for acquiring planning action models is to solve model-based planning tasks afterwards, \textcolor[rgb]{1.00,0.00,0.00}{but there exists as well a large variety of planning-related tasks that rely upon the existence of a planning model. Among these tasks, we might cite: \emph{goal and plan recognition} approaches based on a domain theory~\cite{ramirez2009plan,ramirez2012plan,SohrabiRU16}; \emph{transparent planning}, in which an agent implicitly communicates its true goal by making its intentions and its action selection transparent (recognizable) to observers~\cite{MacNallyLRP18}; or \emph{deceptive path-planning}, which draws on the definition of path-planning domains and aims at finding a path such that the probability of an observer identifying the final destination is minimised~\cite{MastersS17}. Planning models are also used in \emph{explainable AI planning} to form a common basis for communicating with users and facilitate the generation of transparent and explainable decisions~\cite{FoxLM17} as well as explanations in terms of the differences with a human mental model~\cite{ChakrabortiSK18}. \emph{Counterplanning} requires a model of the opponent agent in order to recognize its goals \cite{PozancoEFB18} and \emph{model reconciliation} aims to conform the models of two agents with respect to an observation of a plan computed with one of the two models~\cite{ChakrabortiSZK17}.}

\textcolor[rgb]{1.00,0.00,0.00}{Motivated by the requirement for a planning model in many different tasks and the recent advances on the use of classical planning for the generation of different types of planning models (regular automata, context-free grammars, finite-state machines, \strips)} \cite{bonet2009automatic,segovia2016generalized,segovia2016hierarchical,segovia2017generating}, in this paper we claim that a planning model is learnable even though an accurate representation of the agent's behavior is not available. Particularly, we present a novel learning algorithm, called \FAMA, capable of inferring the preconditions and effects of \strips\ action models, the vanilla action model for automated planning~\cite{fikes1971strips}, under minimal observability.


%Up to date, all existing learning approaches assume that a given input plan trace contains a fully observed sequence of the executed actions. This heavily restricts the applicability of the learning approach to contexts in which the behaviour of the agent is fully observable and a human annotator correctly labels the executed actions.

\textcolor[rgb]{1.00,0.00,0.00}{Current learning approaches assume that the observation of the agent's plan execution (plan trace) encompasses the fully observed sequence of the executed actions; i.e, they assume all the actions performed by the agent are observable. This heavily restricts the applicability of the learning approaches to contexts where the behaviour of the agent is fully observable, which also commonly entails a human annotator that correctly labels the executed actions. On the other hand, learning approaches accept a varying degree of observability in the states traversed in the plan trace, ranging from fully observable to fully unobservable states (see section \ref{related_work} for details).} In contrast, \FAMA allows for an incomplete or empty sequence of observable actions, and the minimum observability case acceptable by \FAMA is when the algorithm is only fed with the initial and final state of a plan trace. Like many Machine Learning (ML) techniques, \FAMA is able to operate with only input/output pairs of states and an unknown or a partially known model of the agent. Unlike ML algorithms, \FAMA requires a symbolic structured representation of the input knowledge. In this sense, recent investigations tackle the problem of learning symbolic representations from low-level sensing information and unstructured data~\cite{KonidarisKL18,AsaiF18}.

\textcolor{red}{\FAMA is a new learning approach, based on AI planning technology, that automatically compiles the task of learning \strips\ actions into a planning task which is then solved with a planner~\cite{aineto2018learning}. The construction of a \strips\ planning model starts out from a set of plan traces containing the observation of several plan executions. As mentioned above, a plan trace may comprise none of the actions executed by the agent but must include, at least, the initial state $s_0$ and final state $s_f$ of the execution. The compilation scheme lies in defining a planning task from the set of plan traces using a set of \emph{building actions} that insert the preconditions and effects of a learned action, and a set of \emph{validating actions} that validate the learned actions in the plan traces. Hence, a solution to this planning task is a plan that determines the preconditions and effects of the actions of a \strips\ planning model $M$ while ensuring consistency with the input traces. We say that a model $M$ is \emph{consistent} with a plan trace when $M$ can produce a solution $\pi$ to the planning problem $\tup{s_0,s_f}$ so that: (1) $\pi$ contains the observed actions of the plan trace, if any, and (2) the states generated by the application of $\pi$ to $s_0$ will encompass all the (possibly) partially observed states of the trace.}

\FAMA is thus a model-based approach that automatically builds its own planning model by logical inference from the input plan traces that contain the observations of the agent execution. This behaviour largely differs from ML techniques, which aim to minimize an error function on the training data. Moreover, \FAMA requires far less sample data (example plan traces) than typical ML algorithms, thus alleviating the dependency on the assumption that there are enough data for learning the action models~\cite{Zhuo15}.

A key aspect in action-model learning is the evaluation method to assess the quality and performance of the learning approach. The most common method is to use a syntax-based evaluation that compares the learned model with a reference model. \FAMA proposes instead two novel semantic evaluation metrics that build upon two well-known ML metrics, {\em precision} and {\em recall}~\cite{davis2006relationship}, to evaluate the learned action models with respect to observations of plan executions. Our semantic evaluation is generally more informative than counting the number of errors between two models and alleviates two important limitations of a purely syntax-based assessment: (a) that the learned model is syntactically different from the reference model but semantically correct and (b) that the learned model comprises correct though unnecessary preconditions in regards to the reference model. This latter issue is concerned with the qualification problem, which is defined as the actual impossibility of listing all the preconditions required for a real world action to have its intended effects~\cite{GinsbergS88}.

Our semantic evaluation method is built on the same compilation scheme for solving a learning task. In particular, \FAMA also accepts an input initial action model $M$ of the agent's behaviour, either complete or partially specified~\cite{ZhuoNK13,ZhuoK17}, alongside the observation of the agent execution. In this case, \FAMA returns a model $M'$ that follows the input model $M$ and is consistent with the observations. We designed an {\em edition} mechanism that serves to correct the input model to the output model, which in turn defines an assessment of the accuracy with which $M$ explains the observations. Interpreting the edition measure as a distance-based concept between two models can also be exploitable in model reconciliation~\cite{KulkarniCZVZK16}.


%Like other learning approaches, \FAMA also accepts as an input an initial action model of the agent's behaviour, either complete or partially specified~\cite{ZhuoNK13,ZhuoK17}. In this case, the output model of \FAMA is compliant with both the input model and with the observations. Following this behaviour, we can define a distance measure between the two models which yields an assessment of the accuracy with which the input model explains the observations. In other words, the distance measure provides a mechanism to explicitly correct or adjust a model to the observations. By using this distance-based concept, \FAMA contributes with two novel semantic evaluation metrics that are generally more informative than counting the number of errors between two models. Moreover, this new evaluation method opens up a way towards model reconciliation~\cite{KulkarniCZVZK16}.

%All in all, \FAMA is a planning-based solving scheme that outputs a \strips\ action model using a planning model that is automatically built from minimal input knowledge. Unlike extensive-data ML approaches, \FAMA only requires a small amount of input plan traces. Unlike most relevant action-model learning algorithms, \FAMA does not require the traces to contain any observed action executed by the agent.

\textcolor[rgb]{1.00,0.00,0.00}{In summary, \FAMA is a novel learning approach characterized by:
\begin{itemize}
\item compiling the task of learning a \strips\ planning model into a planning task that is automatically built from a set of input plan traces.
\item the plan traces are correct (no noise is considered in the observations) but may be incomplete in the number of observed actions as well as in the number and contents of the observed states.
\item the planning task resulting from the compilation comprises actions for programming the preconditions and effects of the \strips\ actions and actions for validating the learned \strips\ actions.
\item a semantic evaluation proposal that enables to assess a learned model beyond a merely syntactic comparison to a reference model.
\end{itemize} }

A first description of the \FAMA compilation scheme already appeared in our previous conference paper~\cite{aineto2018learning}. This paper brings the following contributions over the first version of the compilation:

\begin{itemize}
\item A unified formulation for learning and evaluating action models from observations of plan executions. In the case of minimum observability, these executions only comprise the initial and final state of the plan traces.
\item A thorough elaboration of two semantic evaluation metrics that build upon the notions of {\em precision} and {\em recall} to evaluate the output action models with respect to observations of plan executions.
\item An exhaustive empirical evaluation over 15 domains from the International Planning Competitions (IPCs). We include an analysis of the impact that the size of the input knowledge has in the performance of \FAMA, a comparison with \ARMS, and a detailed experimentation when \FAMA is executed with minimal input knowledge.
\end{itemize}


The paper is organized as follows. Section~\ref{sec:background} introduces classical planning concepts and reviews related work on learning planning action models. \textcolor{red}{Section~\ref{task_definition} formally defines the learning task and motivates our compilation-to-planning approach for learning action models.} Section~\ref{sec:learning} presents the compilation scheme, the core of \FAMA. Sections~\ref{sec:evaluation} explains the evaluation of a learned model with respect to a reference model (syntactic evaluation) and with respect to a set of plan traces (semantic evaluation). Section~\ref{sec:experiments} reports the results of the experimental evaluation and, finally, Section~\ref{sec:conclusions} discusses the strengths and weaknesses of the compilation approach and proposes several opportunities for future research.










\section{Background}
\label{sec:background}

This section serves two purposes; first, we introduce basic planning concepts and define the classical planning model we aim to learn; secondly, we summarize the most relevant existing approaches to learn classical planning action models.



\subsection{Basic planning concepts}
\label{basic_planning}


We use $F$ to denote the set of {\em fluents} (propositional variables) describing a state. A {\em literal} $l$ is a valuation of a fluent $f\in F$, i.e. either~$l=f$ or $l=\neg f$. A set of literals $L$ represents a partial assignment of values to fluents (without loss of generality, we will assume that $L$ does not assign conflicting values to any fluent). The complement of $L$ is defined as $\neg L=\{\neg l:l\in L\}$. We use $\mathcal{L}(F)$ to denote the set of all literal sets on $F$, i.e.~all partial assignments of values to fluents.

We will adopt the \emph{open world assumption}, that is, what is not known to be true in a state is unknown, to implicitly represent the unobserved literals of states. Consequently, states will explicitly include positive literals ($f$) and negative literals ($\neg f$) such that literals that are not in a state are unknown or unobserved. Hence, a {\em state} $s$ is a full assignment of values to fluents; i.e. $|s|=|F|$, so the size of the state space is $2^{|F|}$. Like in PDDL~\cite{fox2003pddl2}, we assume that fluents $F$ are instantiated from a set of {\em predicates} $\Psi$. Each predicate $p\in\Psi$ has an argument list of arity $ar(p)$. Given a set of {\em objects} $\Omega$, the set of fluents $F$ is induced by assigning objects in $\Omega$ to the arguments of predicates in $\Psi$; i.e.~$F=\{p(\omega):p\in\Psi,\omega\in\Omega^{ar(p)}\}$ such that $\Omega^k$ is the $k$-th Cartesian power of $\Omega$.

A {\em classical planning frame} is a tuple $\Phi=\tup{F,A}$, where $F$ is a set of fluents and $A$ is a set of actions. An action $a\in A$ has a set of preconditions $\pre(a)\in\mathcal{L}(F)$ and a set of effects $\eff(a)\in\mathcal{L}(F)$. An action $a\in A$ is applicable in a given state $s$ iff $pre(a)\subseteq s$, i.e.~if the literals $pre(a)$ hold in $s$. The result of executing an applicable action $a\in A$ in a state $s$ is a new state $\theta(s,a)=\{s\setminus \neg\eff(a)\cup\eff(a)\}$. Note that subtracting the complement of $\eff(a)$ from $s$ ensures that $\theta(s,a)$ remains a well-defined state with positive and negative literals. Then:

\begin{itemize}
\item $\eff^+(a)\in\mathcal{L}(F)$ is the {\em positive effects} of $a$, the subset of action effects that assert a positive literal in the state resulting after the application of $a$
\item $\eff^-(a)\in\mathcal{L}(F)$ is the {\em negative effects} of $a$, the subset of action effects that assert a negative literal in the state resulting after the application of $a$
\end{itemize}

Since we restrict our attention to \strips\ action models learning, we will assume the set of syntactic constraints imposed by \strips\ models, namely that $\eff^-(a)\subseteq \pre(a)$, $\eff^-(a)\cap \eff^+(a)=\emptyset$ and $\pre(a)\cap \eff^+(a)=\emptyset$. Additionally, actions $a\in A$ are instantiated from given action schemas, as in PDDL.


A {\em classical planning problem} is a tuple $P=\tup{F,A,I,G}$, where $I$ is an initial state and $G\in\mathcal{L}(F)$ is a goal condition. A {\em plan} for $P$ is an action sequence $\pi=\tup{a_1, \ldots, a_n}$ that induces the {\em state trajectory} $s=\tup{s_0, s_1, \ldots, s_n}$ such that $s_0=I$ and, for each {\small $1\leq i\leq n$}, $a_i$ is applicable in $s_{i-1}$ and generates the successor state $s_i=\theta(s_{i-1},a_i)$. The {\em plan length} is denoted with $|\pi|=n$ . A plan $\pi$ {\em solves} $P$ iff $G\subseteq s_n$, i.e.,~if the goal condition is satisfied at the last state reached after following the application of the plan $\pi$ in the initial state $I$. A solution plan for $P$ is {\em optimal} if it has minimum length.

In this work, the term \emph{plan trace} refers to the \emph{observation} of a plan execution that starts on a given initial state. A plan trace $\tau = \langle s_0, a_1, s_1, a_2, s_2, \ldots, a_n, s_n \rangle$ is generally defined as an interleaved combination of a sequence of executed actions $\tup{a_1, \ldots, a_n}$ and the induced state trajectory $\tup{s_0, s_1, \ldots, s_n}$. \emph{Plan traces} constitute the input knowledge of the learning tasks addressed in this paper.

Our approach copes with partial observability in the plan traces. Let $\pi=\tup{a_1, \ldots, a_n}$ be the plan executed by an agent that induces the state trajectory $s=\tup{s_0, s_1, \ldots, s_n}$, and let $\tau = \langle s_0, \ldots, a_i, \ldots, s_j, \ldots, s_n \rangle$ be the plan trace observed from the plan execution. With regards to the observed states of $\tau$, that we will refer to as $\tau_s$, we identify two general cases of observability:

\begin{enumerate}
\item We say that $\tau_s$ is a fully-observable (\FO) state trajectory if every observed intermediate state of $\tau_s$ is a full assignment of values to fluents, and there exists a single action that transitions from every state $s_i$ to state $s_{i+1}$ in $\tau_s$;  that is $\theta(s_i,\tup{a})=s_{i+1}$. This case clearly states that $\tau_s = s$, meaning that $\forall s_i \in \tau_s$, $s_i$ comprises all the literals of the corresponding state in the trajectory $s$ of the plan $\pi$. Formally, $\forall i, 1 \leq i  < n, |s_i| = |F|$.
\item We say that $\tau_s$ is a partially-observable (\PO) state trajectory if at least one intermediate state of $\tau_s$ is a partial assignment of values to fluents. Formally, $\exists i, 1 \leq i  < n, |s_i| < |F|$. This means that one or more literals are missing in the intermediate $s_i$, all of which may be missing. When all literals are missing, $s_i$ is a \emph{missing} or \emph{empty state} ($s_i = \emptyset$).
\end{enumerate}

The general definition of a \PO state trajectory gives rise to two special cases:

\begin{enumerate}
\item When \textbf{all} of the $n-1$ intermediate states of $s$ are \textbf{missing} in $\tau_s$, $\tau_s$ is a \emph{non-observable} (\NO) state trajectory. Formally, $\forall i, 1 \leq i < n,  s_i= \emptyset$; i.e., $|s_i| = 0$.
\item When \textbf{none} of the $n-1$ intermediate states of $s$ are \textbf{missing} in $\tau_s$, we will refer to $\tau_s$ as a \POstar state trajectory. Formally, $\exists i, 1 \leq i < n, s_i \neq \emptyset$; i.e., $0 < |s_i| < |F|$.
\end{enumerate}

Table \ref{tab:state_trajectory} summarizes the four types of state trajectories according to the observed information, which ultimately affects the number of observed intermediate states and the number of literals comprised in each intermediate state. \PO comprises both \POstar and \NO, and it thus encompasses trajectories with some missing state.

\begin{table}[hbt!]
\centering
\begin{tabular}{c|c|c|}
	     & {\bf \# intermediate states} & {\bf state type} \\ \hline
    \FO & $n-1$  & {\small $\forall i, 1 \leq i < n$}  \\  & & $s_i$ is a full assignment $|s_i|=|F|$ \\ \hline
    \multirow{1}{*}{\POstar} & $n-1$ & {\small $\exists i, 1 \leq i < n$}  \\ & & $s_i$ is a partial assignment $0 < |s_i|< |F|$\\ \hline
    \multirow{1}{*}{\PO} & $\leq n-1$ & {\small $\exists i, 1 \leq i < n$}   \\  & & $s_i$ is a partial assignment $|s_i|< |F|$\\ \hline
    \NO & 0 & {\small $\forall i, 1 \leq i < n$}  \\  & & $s_i$ is an empty state  $|s_i|=0$
\end{tabular}
\caption{Classification of state trajectories accordingly to the observed information.}
\label{tab:state_trajectory}
\end{table}


\FAMA can also deal with partial observability in the observed actions of $\tau$, that we will refer to as $\tau_a$. We identify three levels of observability, from the greatest to the lowest:

\begin{enumerate}
\item When \textbf{all} of the actions of $\pi$ appear in $\tau_a$, we say that $\tau_a$ is a fully-observable (\FO) action sequence; i.e.,  $\tau_a=\pi$. In this case, $\tau_a$ contains all the necessary actions to transit every state $s_{i-1}$ to its corresponding successor state $s_{i}$, from $s_0$ to $s_n$. This is the type of input trace accepted by all the existing learning approaches (see section \ref{related_work} for details).
\item When \textbf{some} of the actions of $\pi$  appear in $\tau_a$, we say that $\tau_a$ is a partially observable (\PO) action sequence. In this case, at least one of the necessary actions of the plan $\pi$ is missing in $\tau_a$. Formally, $\exists i, 1 \leq i \leq n, a_i \in \pi \wedge a_i \notin \tau_a$.
\item When \textbf{none} of the actions of $\pi$  appear in $\tau_a$, we say that $\tau_a$ is a non-observable (\NO) action sequence. Formally, $\forall i, 1 \leq i \leq n, a_i \in \pi \wedge a_i \notin \tau_a$. That is, $\tau_a = \emptyset$.
\end{enumerate}


Plan traces can be classified accordingly to the type of observed state trajectory (\FO, \POstar, \PO or \NO) and action sequence (\FO, \PO or \NO). In section \ref{task_definition}, we expose the impact of the combinations of observed state trajectories and observed action sequences when solving a learning task.



\subsection{Related work}
\label{related_work}

In this section we summarize the most recent and relevant approaches to learning action models found in the literature. Approaches will be examined according to the following parameters: the observability of the plan traces accepted by the system, the expressiveness of the learned action model and the principal technique used for learning the action model (Table \ref{table:models_comparison1}), as well as the characteristics of the evaluation method used to validate the learned models (Table \ref{table:models_comparison2}).

The first column of Table \ref{table:models_comparison1} shows the constraints imposed on the input plan traces with regard to observability. Since all approaches except ours deal only with \FO action sequences, constraints are exclusively concerned with the type of state trajectory. This directly affects the complexity of the task, which can be sorted from the least to the most constrained following this order: 1) \NO, 2) \PO, 3) \POstar, and 4) \FO. Note that \PO is less constrained than \POstar because \PO considers the possibility of having some missing state in the trajectory.

The task of learning from less constrained traces subsumes learning from more constrained ones. Consequently, approaches to learning from, say traces with \PO state trajectories, will also enable learning from traces with \POstar state trajectories. All the approaches analyzed in this work accept the more constrained definition of partial observations of intermediate states \POstar, two of them also allow the sequence of intermediate states to be empty (\PO) and the majority accept \NO state trajectories. Exceptionally, \LOCM is the only approach capable of learning from a fully-empty state trajectory, with neither initial nor final state.


The expressiveness of the learned action models varies across approaches (second column of Table \ref{table:models_comparison1}). All the presented systems are able to learn action models in a \textsc{Strips}  representation \cite{fikes1971strips} and some propose algorithms to learn more expressive action models that include quantifiers, logical implications or the type hierarchy of a PDDL domain.

Table \ref{table:models_comparison2} summarizes the main characteristics of the evaluation of the learned action models based on the type of evaluation method (first column of Table \ref{table:models_comparison2}), the metrics used in the evaluation (second column of Table \ref{table:models_comparison2}) and the number of tested domains alongside the size of the training dataset (third column of Table \ref{table:models_comparison2}). \textcolor[rgb]{1.00,0.00,0.00}{Regarding the evaluation method, almost all approaches rely on a comparison between the learned model and a {\em Ground-Truth Model} (GTM). By GTM we refer to an engineered model that knowledge modelers have adopted as being correct with respect to a domain, and that the planning community accepts as such.}

In the following, we present a comprehensive insight of the particularities of the seven systems presented in Table \ref{table:models_comparison1} and Table \ref{table:models_comparison2}. This exposition will also help us to highlight in section \ref{task_definition} the value of our contribution \FAMA.


\vspace{0.3cm}

The Action-Relation Modeling System (\textbf{\ARMS})~\cite{yang2007learning} is one of the first learning algorithms able to learn from plan traces with partial or null observations of intermediate states. \ARMS uncovers a number of constraints from the plan traces in the training data that must hold for the plans to be correct. These constraints are then used to build and solve a weighted propositional satisfiability problem with a MAX-SAT solver. Three types of constraints are considered: 1) constraints imposed by general axioms of correct \textsc{Strips} actions, 2) constraints extracted from the distribution of actions in the plan traces and 3) constraints obtained from the \PO states, if available. Frequent subsets of actions in which to apply the two latter types of constraints are found by means of frequent set mining.

\ARMS defines an error metric and a redundancy metric to measure the correctness and conciseness of an action model over the test set of input plan traces using a cross-validation evaluation. The model evaluation is posed as an optimization task that returns the model that best explains the input traces by minimizing the error and redundancy functions. This yields a model that is approximately correct (100\% correctness is not required so as to ensure generality and avoid overfitting), approximately concise (low redundancy rates), and that can explain as many examples as possible. Hence, there is no guarantee that the learned model of \ARMS explains all observed plans, not even that it correctly explains any of the plan traces of the test set.

The \ARMS system became a benchmark in action-model learning, showing empirically that is is feasible lo learn a model in a reasonably efficient way using a weighted MAX-SAT even with \NO state trajectories.


\begin{table}
	\small
	\centering
	\begin{tabular}{ l | c | c | c }
		& \multicolumn{1}{c|}{\bf Input plan traces}
        & \multicolumn{1}{c|}{\bf Learned action model}
        & \multicolumn{1}{c}{\bf Technique}     \\
		\hline			
		\multirow{2}{*}{\ARMS} & \NO states & \strips & MAX-SAT \\ & \FO actions & & \\
        \hline
        \multirow{2}{*}{\SLAF} & \POstar states  & universal quantifiers in $\eff$ & logical inference \\ & \FO actions &  & SAT solver \\
         \hline
		\multirow{2}{*}{\LAMP} & \PO states &  quantifiers &  Markov logic networks \\  & \FO actions & logical implications &  \\
         \hline
         \AMAN & \NO states & \strips & graphical model estimation \\ & noisy actions & & \\
         \hline
         \NOISTA & \POstar and noisy states & \strips &  classification \\ & \FO actions & & \strips \texttt{} rules derivation \\
         \hline
         \CAMA & \PO states &  \strips & crowdsourcing annotation\\ & \FO actions &  & MAX-SAT \\
         \hline
         \LOUGA & \NO states & \strips & Genetic algorithm \\ & \FO actions & negative preconditions & \\
         \hline
         \LOCMtwo & --- &  predicates and types & Finite State Machines \\ & \FO actions & & \\
         \hline
		\FAMA & \NO states & \strips &   compilation to planning\\ & \NO actions & & \\
         \hline
	\end{tabular}
	\caption{Characteristics of action-model learning approaches}
	\label{table:models_comparison1}
\end{table}	

A tractable and exact solution of action models in partially observable domains using a technique known as Simultaneous Learning and Filtering (\textbf{\SLAF}) is presented in~\cite{AmirC08}. \SLAF alongside \ARMS can be considered another of the precursors of the modern algorithms for action-model learning, able to learn from partially observable states. Given a formula representing the initial belief state, a sequence of executed actions and the corresponding partially observed states, \SLAF builds a complete explanation of observations by models of actions through a CNF formula. The learning algorithm updates the formula of the belief state with every action and observation in the sequence such that the new transition belief formula represents all possible transition relations consistent with the actions and observations at every time step.

\SLAF extracts all satisfying models of the learned formula with a SAT solver. For doing so, the training data set for each domain is composed of randomly generated action-observation sequences  (1,000 randomly selected actions and 10 fluents uniformly selected at random per observation). Additional processing in the form of replacement procedures or extra axioms are run into the SAT solver when finding the satisfying models. The experimentally tested \SLAF version is an algorithm that learns only effects for actions that have no conditional effects and assumes that actions in the sequences are all executed successfully (without failures). This algorithm cannot effectively learn the unknown preconditions of the actions and in the resulting models `\emph{one can see that the learned preconditions are often inaccurate}` \cite{AmirC08}. On the other hand, it does not report any statistical evaluation of measurement error other than a manually comparison of the learned models with a ground-truth model.

The Learning Action Models from Plan Traces (\textbf{\LAMP}) \cite{ZhuoYHL10} algorithm extends the expressiveness to learning models with universal and existential quantifiers as well as logical implications. The input to \LAMP is a set of plan traces with intermediate states, which are encoded by the algorithm into propositional formulas. \LAMP then uses the action headers and predicates to build a set of candidate formulas that are validated against the input set using a Markov Logic Network and effectively weighting each formula. The formulas with weights larger than a certain threshold are chosen to represent preconditions and effects of the learned action models.

\LAMP allows \PO state trajectories up to a minimum observability of 1/5 of non-empty states as well as \POstar state trajectories with different degrees of observability in the number of propositions in each state. It uses an error metric based on counting the differences in the number of precondition and effects between the ground-truth model and the learned model. In general, the results show that the accuracy of the learned models is fairly sensitive to the threshold chosen to learn the weights of the candidate formulas, and that domains that feature more conditional effects are harder to learn.


\begin{table}
	\small
	\centering
	\begin{tabular}{ l | c | c | c }
		%& \multicolumn{1}{c}{Evaluation task}
        & \textbf{Evaluation method} & \textbf{Metrics} & \textbf{\#tested domains/}   \\
        &   &   & \textbf{training data size} \\
		\hline			
		\multirow{1}{*}{\ARMS} & cross-validation with a test set & error counting of \#$\pre$ satisfaction  & 6  \\
        & of plan traces & and redundancy & 1,600-4,320 actions\\ & & & (160 plan traces) \\
        \hline
         \SLAF &  manual checking wrt GTM &  ---   & 4\\ & & & 1,000 actions\\
         \hline
		\multirow{1}{*}{\LAMP} & checking wrt GTM  & error counting of extra & 4\\ & & and missing \#$\pre$ and \#$\eff$ & 1,300-6,100 actions\\
           & & & (100-200 plan traces) \\
         \hline
         \AMAN & checking wrt GTM &  error counting of extra &  3 \\ & & and missing \#$\pre$ and \#$\eff$ & 40-200 plan traces\\
         \hline
         \NOISTA & checking wrt GTM  & error counting of extra & 5\\ & & and missing \#$\pre$ and \#$\eff$ & 5,000-20,000 actions\\
         \hline
         \CAMA & checking wrt GTM  &  error counting of extra & 3 \\ & & and missing \#$\pre$ and \#$\eff$ & 15-75 plan traces\\
         \hline
       \LOUGA & cross-validation with & redundant effects & 5 \\  & a test set of plan traces & differences wrt the test set & 800 - 3200 actions (160 traces)\\
         \hline
		\LOCMtwo & manual checking wrt GTM &  ---  &   --- \\
         \hline
		\FAMA & checking wrt GTM  & precision and recall & 15\\  & validation with a test set &  & 20-50 actions\\
         \hline
	\end{tabular}
	\caption{Evaluation of action models (GTM: ground-truth model)}
	\label{table:models_comparison2}
\end{table}	


The Action Model Acquisition from Noisy plan traces (\textbf{\AMAN}) \cite{zhuo2013action} introduces an algorithm able to learn action models from plan traces with \NO state sequences where actions have a probability of being observed incorrectly (noisy actions). The first step of the \AMAN algorithm is to build the set of candidate domain models that are consistent with the action headers and predicates. \AMAN then builds a graphical model to capture the domain physics; i.e., the relations between states, correct actions, observed actions and domain models. After that, the parameters of the graphical model are learned, computing at the same time the probability distribution of each candidate domain model. \AMAN finally returns the model that maximizes a reward function defined in terms of the percentage of actions successfully executed and the percentage of goal propositions achieved after the last successfully executed action.

\AMAN uses the same metric as \LAMP, namely counting the number of preconditions and effects that appear in the learned model and not in the ground-truth model (extra fluents) and viceversa (missing fluents). In a comparison between \AMAN and \ARMS on noiseless inputs, the results show that the accuracy of the learnt models are very close to each other and neither dominates the other. The convergence property of \AMAN guarantees that the accuracy of the learned model with noisy input traces becomes more and more close to the case  \emph{without noise} because the distribution of noise in the plan becomes gradually closer to real distribution with the number of iterations.


Another interesting approach that deals with noisy and incomplete observations of states is presented in \cite{MouraoZPS12}. We will refer to this approach as \textbf{\NOISTA} henceforth. In \NOISTA, actions are correctly observed but they can obviously be unsuccessfully executed in the possibly noisy application state. The basis of this approach consists of two parts: a) the application of a voted Perceptron classification method to predict the effects of the actions in vectorized state descriptions and b) the derivation of explicit \strips \texttt{} action rules to predict each fluent in isolation. Experimentally, the error rates in \NOISTA fall below 0.1 after 5,000 training samples for the five tested domains under a maximum of 5\% noise and a minimum of 10\% of observed fluents.

The Crowdsourced Action-Model Acquisition (\textbf{\CAMA}) \cite{Zhuo15} explores knowledge from both crowdsourcing (human annotators) and plan traces to learn action models for planning. \CAMA relies on the assumption that obtaining enough training samples is often difficult and costly because there is usually a limited number of plan traces available. In order to overcome this limitation, \CAMA builds on a set of soft constraints based on labels \texttt{true} or \texttt{false} given by the crowd and a set of soft constraints based on the input plan traces. Then it solves the constraint-based problem using a MAX-SAT solver and converts the solution to action models.

Plan traces in \CAMA are composed of 80\% of empty states and each partial state was selected by 50\% of propositions in the corresponding full state. An experimental comparison reveals that a manual crowdsourcing of \CAMA outperforms \ARMS and that as expected the difference becomes smaller as the number of plan traces becomes larger. The accuracy of \CAMA for a small number of plan traces (e.g., 30) is not less than 80\%, thus revealing that exploiting the knowledge of the crowd can help learning action models.


One of the latest incorporations to the family of action model learning algorithms is \LOUGA. This system uses a genetic algorithm to learn the effects of actions. In order to do this, each gene in the genome encodes whether a predicate is a positive effect, negative effect or none for a particular action, and the fitness of an individual is evaluated by reproducing the trace with the model encoded in the individual. After a solution for the effects is found, an ad-hoc algorithm is used to infer preconditions by finding those literals that are always present before the execution of an action. \LOUGA evaluates the learned models via cross-validation using the same metrics they use in their fitness function. In more detail, they measure (1) redundant positive and negative effects, (2) preconditions not met, and (3) literals observed in the input trace but not in the corresponding execution of the plan with the learned model.

\vspace{0.15cm}


Finally, we present the Learning Object-Centred Models (\textbf{\LOCM}), possibly the most distinctive learning system due to its ability of learning with minimal input knowledge. \textbf{\LOCM} only requires the \FO action sequence of the plan trace, without need for providing any information about the predicates or the state trajectory, not even the initial or final state~\cite{CresswellMW09,cresswell2013acquiring}. The lack of available state information is overcome by exploiting assumptions about the structure of the actions. Particularly, \LOCM assumes that objects found in the same position in the header of actions are grouped as a collection of objects named \emph{sort} whose defined set of states is captured by a parameterized Finite State Machine (FSM). The intuitive assumptions of \LOCM, like the continuity of object transitions or the association of parameters between consecutive actions in the training dataset, yield a learning model heavily reliant on the kind of domain structure. A later work, \textbf{\LOCMtwo}, extends the applicability of the \LOCM algorithm to a wider range of domains by introducing a richer representation that allows using multiple FSMs to represent the state of a \emph{sort}~\cite{cresswell2011generalised}.

\LOCMtwo is not experimentally evaluated, only the outcome of running the \LOCMtwo algorithm on several benchmark domains wrt to the reference model is reported in ~\cite{cresswell2011generalised}. It is worth noting the last contribution of the \LOCM family, called \textbf{\LOP} (\LOCM with Optimized Plans), addresses the problem of inducing static predicates~\cite{GregoryC16}. \LOP applies a post-processing step after the \LOCM analysis and it requires additional input information, particularly a set of optimal plans besides the suboptimal \FO action sequences.

The distinctive feature of the \LOCM family lies in the capacity to learn the state variables (fluents) because predicates are neither provided as input nor they are deducible from the plan trace as no state observability is allowed. In contrast, \FAMA and the rest of approaches either assume the set of predicates are provided alongside the input traces or assume they are extractable from the observed states of the plan trace, in which case the plan trace must comprise at least a grounded sample of every predicate. Similarly, the syntax of an action header (the action name and its parameters) is either extractable from the action sequence of the plan trace or it must be explicitly provided.








\section{Overview of \FAMA}
\label{overview}


\subsection{Problem definition}
\label{probkem_definition}

\subsection{Complexity of the learning tasks: a justification for using planning}
\label{complexity}

In this section, we will highlight the principal distinctive features of our approach \FAMA with respect to the related work reviewed in Section~\ref{related_work}.

\vspace{0.1cm}

%\subsection{Learning action models from plan traces of unbound horizon}
When learning action models from observations of plan executions there are two main sources of partial observability:
\begin{enumerate}
\item As many of the approaches in Section~\ref{sec:background} assume, there may be an unknown number of missing intermediate states in the trace because of the partial state observability (\PO and \NO). The assumption of having \FO state trajectories means that the sensors are able to capture every state change at every instant, which typically is unrealistic. Normally, the process for obtaining state feedback from sensors (or the processing of the sensor readings) is associated with a given sampling frequency that misses intermediate data between two subsequent sensor readings.

\item There may be also an unbound number of missing actions in the plan trace because of partial observability. The common assumption of having \FO action sequences in a learning task is unrealistic in many domains as it implies the existence of human observers that annotate the observed action sequences. In some real-world applications, the observed and collected data are sensory data (e.g., home automation, robotics) or images (e.g. traffic) and one cannot rely on human intervention for labeling actions. Actually, learning the executed actions can also be part of the action-model learning task. Learning, for instance, from unstructured data involves transforming the sensor or image information into a predicate-like format before applying the action-model learning approach, and it also requires the ability of identifying action symbols~\cite{AsaiF18}.
\end{enumerate}


\FAMA represents one step ahead towards learning action models without assuming observed actions. The main novelty of \FAMA with respect to other approaches lies in that our system is capable of handling \PO and \NO action sequences, which combined with \PO and \NO state trajectories, make the learning task more challenging. This essentially brings one key difference: the transition between two given observed states may now involve more than one action; i.e., $\theta(s_i,\tup{a_1,\ldots,a_k})=s_{i+1}$, with $k \geq 1$, $k$ unknown and unbound, and so the horizon of the input plan traces is no longer known now. Table \ref{tab:complex} shows that the worst case complexity of learning \strips\ action models becomes PSPACE-complete when combining \PO and \NO state trajectories and action sequences.

In this particular scenario, the actual number of plan traces associated to a given input observation is also unbound and grows exponentially with the actual length of the plan trace (that is now unknown). Otherwise, the learning task is SAT compilable, which is known to be a NP-complete task~\cite{russell2016artificial}. This is the reason why SAT solving is a common technique in the approaches presented in section \ref{sec:background}.


\begin{table}[ht]
\centering
\begin{tabular}{c|c|c|c|c|}
	& \multicolumn{4}{c|}{\emph{state observability}} \\ \cline{2-5}
	\multirow{1}{*}{\emph{action}} & \FO & \POstar & \PO & \NO\\ {\emph{observability}} & & & & \\ \hline
	\FO & - & NP-complete & NP-complete & NP-complete \\ \hline
	\PO & NP-complete & NP-complete & \textbf{PSPACE-complete} & \textbf{PSPACE-complete} \\ \hline
	\NO & NP-complete & NP-complete & \textbf{PSPACE-complete} & \textbf{PSPACE-complete} \\ \hline
\end{tabular}
\caption{Complexity of learning tasks according to the type of input trace}
\label{tab:complex}
\end{table}


When we assume partial observability in both actions and states, a complete approach must consider the length of the input plan traces to be unknown. \FAMA shows that classical planning is a complete approach for this particular scenario. Consequently, the new learning scenario features PSPACE-complete instead of NP-complete tasks, which motivates and justifies the use of planning, as our proposal of compiling the learning task to a classical planning problem.

When the plan trace is fully observed, learning \strips\ action models is straightforward~\cite{jimenez2012review}. In this case the {\em pre-} and {\em post-states} of every action are available and so action {\em effects} are derived lifting the literals that change between the pre and post-state of the corresponding action executions. Likewise {\em preconditions} are derived lifting the minimal set of literals that appears in all the pre-states of the corresponding action.


%In \FAMA we set a probability threshold of observability to each fluent and action, which determines the percentage of literals in each state and in turn the percentage of observed states (when the probability of every fluent is above the threshold, the result is a missing state). This way, \FAMA must always work under the assumption that the number and length of the input plan traces are unknown and so the task of learning an action model becomes as hard as solving a \strips planning problem. Consequently, the new learning scenario features PSPACE-complete instead of NP-complete tasks, which motivates and justifies the use of planning techniques, as our proposal of compiling the learning task to a planning problem.

%Thereby, solving this type of learning tasks justifies the use of techniques other than SAT solvers, as our proposal of compiling the learning task to a planning problem.


\vspace{0.1cm}
Regarding the evaluation of the learned action models, we can observe in Table \ref{table:models_comparison2} that most of the approaches use a similar syntax-based metric that consists in (1) counting the missing and extra fluents that appear in the learned model wrt the GTM and (2) normalizing this error by the the total number of all the possible preconditions and effects of an action model. This is an \emph{optimistic} metric since error rates are not normalized by the size of the actual GTM. The set of preconditions and effects of the GTM is usually smaller than the set of all possible preconditions and effects and thereby it turns out that these syntax-based metrics may output error rates below 100\% for totally wrong learned models. To overcome this limitation we propose to use two standard metrics from ML, {\em precision} and {\em recall}, that are frequently used in pattern recognition, information retrieval and binary classification~\cite{davis2006relationship}. %{\em Precision} and {\em recall} are more informative that simply counting the number of errors in the learned model or computing the {\em symmetric difference} between the learned and the reference model.

Pure syntax-based evaluation metrics, like the ones mentioned in the above paragraph, can report low scores for learned models that are actually {\em sound} and {\em complete} but syntactically different from the GTM. Semantic evaluation metrics add a distinctive value over the syntactic ones, which is that they evaluate the learned model with a set of observations of plan executions and hence they are appropriate for scenarios where the GTM is not available. In this sense, \FAMA also contributes with a novel semantic-based error measure that builds upon the {\em precision} and {\em recall} metrics. Unlike the semantic metric used in \ARMS~\cite{yang2007learning}, our semantic version of {\em precision} and {\em recall} is not sensitive to the repetition of one same flaw in the model evaluation.


%When learning action models from observations of plan executions, the roles of two {\em comparable} action schemes (or the roles of two action parameters with the same type) can be swapped. These role swaps typically happen when the observed input data is scarce. For instance the {\em blocksworld} operator {\small\tt stack} can be {\em learned} with the preconditions and effects of the {\small\tt unstack} operator and vice versa. Further, the roles of parameters of the {\small\tt stack} (or the {\small\tt unstack}) operator could be swapped and these learned models would still be semantically correct with respect to the given input observations.

%Pure syntax-based evaluation metrics can report low scores for learned models that are actually {\em sound} and {\em complete} but correspond to {\em reformulations} of the GTM model; i.e. a learned model semantically equivalent but syntactically different to the reference model. The ARMS system implemented semantic metrics for evaluating the learned action models with respect to observations of plan executions that act as a {\em test set}. This semantic evaluation approach is suitable for scenarios where the GTM is not available, e.g. the traditional ML setting. For this scenario our work proposes semantic versions of {\em precision} and {\em recall} that provide a notion of the soundness and completeness of the learned models with respect to input plan traces. Unlike the metrics defined by ARMS, our semantic metrics do not accumulate errors that are caused by the same flaw in the learned model.

%\subsection{Input data}

On the other hand, a striking figure of Table \ref{table:models_comparison2} that emphasizes a relevant feature of \FAMA is the small size of the training dataset it requires in comparison to other approaches. Unlike extensive-data ML approaches, our work explores an alternative research direction to learn sound models from small amounts of plan traces. This is an important advantage, particularly in domains in which it is costly or impossible to obtain a significant number of training samples. Unlike \CAMA, our approach does not require human intervention to label samples as it is able to learn from empty sequences of observed actions.

%very small datasets.

Finally, as it will be shown in section \ref{sec:experiments}, \FAMA is exhaustively evaluated, syntactically and semantically, over a wide range of domains (14 domains compared to the scarce number of tested domains of the rest of the approaches in Table \ref{table:models_comparison2}) and uses exclusively an \emph{off-the shelf} classical planner so it can benefit straightforward from the last advances in classical planning.




























\section{Learning task}
\label{learning_task}


\subsection{\strips\ action schemas}
This work addresses the learning of PDDL action schemas that follow the \strips\ requirement~\cite{mcdermott1998pddl,fox2003pddl2}. Figure~\ref{fig:stack} shows the {\em stack} action schema, coded in PDDL, from a four-operator {\em blocksworld}~\cite{slaney2001blocks}.

\begin{figure}[hbt!]
\begin{footnotesize}
\begin{verbatim}
(:action stack
 :parameters (?v1 ?v2 - object)
 :precondition (and (holding ?v1) (clear ?v2))
 :effect (and (not (holding ?v1)) (not (clear ?v2)) (handempty) (clear ?v1) (on ?v1 ?v2)))
\end{verbatim}
\end{footnotesize}
 \caption{\small \strips\ operator schema coding, in PDDL, the {\em stack} action from a four-operator {\em blocksworld}.}
\label{fig:stack}
\end{figure}

To formalize the output of the learning task, we assume that fluents $F$ are instantiated from a set of {\em predicates} $\Psi$, as in PDDL. Each predicate $p\in\Psi$ has an argument list of arity $ar(p)$. Given a set of {\em objects} $\Omega$, the set of fluents $F$ is induced by assigning objects in $\Omega$ to the arguments of predicates in $\Psi$, i.e.~$F=\{p(\omega):p\in\Psi,\omega\in\Omega^{ar(p)}\}$ s.t. $\Omega^k$ is the $k$-th Cartesian power of $\Omega$.

Let $\Omega_v=\{v_i\}_{i=1}^{\operatorname*{max}_{a\in A} ar(a)}$ be a new set of objects ($\Omega\cap\Omega_v=\emptyset$), denoted as {\em variable names}, and that is bound by the maximum arity of an action in a given planning frame. For instance, in a three-block {\em blocksworld} $\Omega=\{block_1, block_2, block_3\}$ while $\Omega_v=\{v_1, v_2\}$ because the operators with the maximum arity, {\small\tt stack} and {\small\tt unstack}, have arity two. We define $F_v$, a new set of fluents s.t. $F\cap F_v=\emptyset$, that results from instantiating $\Psi$ using only the objects in $\Omega_v$, i.e. the variable names, and that defines the elements that can appear in an action schema. For the {\em blocksworld}, $F_v$={\small\tt\{handempty, holding($v_1$), holding($v_2$), clear($v_1$), clear($v_2$), ontable($v_1$), ontable($v_2$), on($v_1,v_1$), on($v_1,v_2$), on($v_2,v_1$), on($v_2,v_2$)\}}.

For a given operator schema $\xi$, we define $F_v(\xi)\subseteq F_v$ as the subset of fluents that represent the elements that can appear in that action schema. For instance, for the {\em stack} action schema $F_v({\tt stack})=F_v$ while $F_v({\tt pickup})$={\small\tt\{handempty, holding($v_1$), clear($v_1$), ontable($v_1$), on($v_1,v_1$)\}} excludes the fluents from $F_v$ that involve $v_2$ because the action header {\small\tt pickup($v_1$)} contains the single parameter $v_1$.

We assume also that actions $a\in A$ are instantiated from \strips\ operator schemas $\xi=\tup{head(\xi),pre(\xi),add(\xi),del(\xi)}$ where:
\begin{itemize}
\item $head(\xi)=\tup{name(\xi),pars(\xi)}$, is the operator {\em header} defined by its name and the corresponding {\em variable names}, $pars(\xi)=\{v_i\}_{i=1}^{ar(\xi)}$. The headers of a four-operator {\em blocksworld} are {\small\tt pickup($v_1$), putdown($v_1$), stack($v_1,v_2$)} and {\small\tt unstack($v_1,v_2$)}.
\item The preconditions $pre(\xi)\subseteq F_v$, the negative effects $del(\xi)\subseteq F_v$, and the positive effects $add(\xi)\subseteq F_v$ such that, $del(\xi)\subseteq pre(\xi)$, $del(\xi)\cap add(\xi)=\emptyset$ and $pre(\xi)\cap add(\xi)=\emptyset$.
\end{itemize}
Therefore, given the set of predicates $\Psi$ and the header of the operator schema $\xi$, $2^{2|F_v(\xi)|}$ defines the size of space of the possible \strips\ models for that operator, given that the previous constraints require that negative effects appear as preconditions and that they cannot be positive effects and also, that a positive effect cannot appear as a precondition. For instance, this number is 4194304 for the blocksworld {\tt stack} operator while is only 1024 for the {\tt pickup} operator.

Last but not least, we say that two \strips\ operator schemes $\xi$ and $\xi'$ are {\em comparable} if both schemas have the same headers so they can be built from the same set of possible elements. Formally, iff $head(\xi)=head(\xi')$ so it also holds that $F_v(\xi)=F_v(\xi')$. For instance we can claim that the {\tt stack} and {\tt unstack} blocksworld operators are {\em comparable} while  {\tt stack} and {\tt pickup} are not. Likewise we say that two \strips\ action models $\mathcal{M}$ and $\mathcal{M}'$ are {\em comparable} iff there exists a bijective function $\mathcal{M} \mapsto \mathcal{M}^*$ that maps every $\xi\in\mathcal{M}$ to a comparable action schema $\xi'\in\mathcal{M'}$ and viceversa.



Learning \strips\ action models from fully available input knowledge, i.e. from plans where the {\em pre-} and {\em post-states} of every action are available, is straightforward~\cite{jimenez2012review}:
\begin{itemize}
  \item {\em Preconditions} are derived lifting the minimal set of literals that appears in all the pre-states of the corresponding action, that is any action that belongs to the same operator scheme.
  \item {\em Effects} are derived lifting the literals that change between the pre and post-state of the corresponding action executions.
\end{itemize}
This section formalizes more challenging learning tasks where less input knowledge is available for instance, because it cannot be observed.


\subsection{Learning from observations of plan executions}
The first learning task corresponds to observing an agent acting in the world but watching only the states that result of its actions, the actual executed actions are unobserved. This learning task is formalized as a tuple $\Lambda=\tup{\mathcal{M},\mathcal{O},\Psi}$:
\begin{itemize}
\item $\mathcal{M}$, the set of {\em empty} operator schemas, wherein each $\xi\in\mathcal{M}$ is only composed of $head(\xi)$.
\item $\mathcal{O}=\tup{s_0,s_1,\ldots,s_{n}}$ is the sequence of {\em state observations} obtained watching the execution of an {\em unobserved} plan $\pi=\tup{a_1, \ldots, a_n}$ such that, for each {\small $1\leq i\leq n$}, $a_i$ is applicable in $s_{i-1}$ and generates the successor state $s_i=\theta(s_{i-1},a_i)$. We assume that the initial state $s_0$ is {\em fully observable} while states $s_i$ s.t. {\small $1\leq i\leq |\mathcal{O}|$} can be {\em partially observable}, meaning that some fluents in $s_i$ are missing because it is unknown whether their value is either positive or negative. In the extreme, states $s_i$ {\small $1\leq i\leq |\mathcal{O}|$}, can be missing but we assume that any state observation is {\em noiseless}, meaning that if the value of a fluent is observed it is correct.
\item $\Psi$ is the set of predicates that define the abstract state space of a given planning frame. Note that $\Psi$ can be inferred from the state observations provided that at least a state $s\in \mathcal{O}$ is a full state, that is $|s|=|F|$.
\end{itemize}

A solution to a $\Lambda=\tup{\mathcal{M},\mathcal{O},\Psi}$ learning task is a set of operator schema $\mathcal{M}'$ that is compliant with the input model $\mathcal{M}$, the given state observations $\mathcal{O}$ and the predicates $\Psi$. Solving $\Lambda$ implies determining, not only the \strips\ action model $\mathcal{M}'$, but also the unobserved plan $\pi$, that explains the input observations with the learned \strips\ model. Figure~\ref{fig:example-observations} shows an example of a $\Lambda=\tup{\mathcal{M},\mathcal{O},\Psi}$ task for learning the {\em blocksworld} \strips\ action model from the five-state observations sequence obtained inverting a 2-block tower. Learning the action model from this example implies inferring the unobserved plan $\pi=\tup{\small\tt (unstack\ B\ A), (putdown\ B), (pickup\ A), (stack\ A\ B)}$.

\begin{figure}[hbt!]
{\footnotesize\tt ;;;;;; Headers in $\mathcal{M}$}
\begin{footnotesize}
\begin{verbatim}
(pickup v1) (putdown v1) (stack v1 v2} (unstack v1 v2)
\end{verbatim}
\end{footnotesize}
\vspace{0.2cm}
{\footnotesize\tt ;;;;;; Predicates $\Psi$}
\begin{footnotesize}
\begin{verbatim}
(handempty) (holding ?o  - object) (clear ?o - object) (ontable ?o - object)
(on ?o1 - object ?o2 - object)
\end{verbatim}
\end{footnotesize}
\vspace{0.2cm}
{\footnotesize\tt ;;;;;; Observations $\mathcal{O}$}
\begin{footnotesize}
\begin{verbatim}
;;; observation #0
(clear B) (on B A) (ontable A) (handempty)

;;; observation #1
(holding B) (clear A) (ontable A)

;;; observation #2
(clear A) (ontable A) (clear B) (ontable B) (handempty)

;;; observation #3
(holding A) (clear B) (ontable B)

;;; observation #4
(clear A) (on A B) (ontable B) (handempty)
\end{verbatim}
\end{footnotesize}
 \caption{\small Task $\Lambda=\tup{\mathcal{M},\mathcal{O},\Psi}$ for learning a \strips\ action model in the {\em blocksworld} from a sequence of five state observations.}
\label{fig:example-observations}
\end{figure}

In some cases we may not require to start learning from scratch because some preconditions and/or some positive or negative effects may be a priori known. The operator schemas in $\mathcal{M}$, that are given as input, may be not {\em empty} but {\em partially specified}. Such scenario is relevant for {\em policy learning}. A policy is function that maps states into actions and represents the conditions under which actions should be applied to achieve certain goals. Given an action model, the task of learning a policy that is compliant with a set of observations can be defined as learning extra preconditions for each action scheme in the model. These extra preconditions capture when actions can be applied according to the policy. A policy that exactly defines a single applicable action for each reachable state is a {\em full policy} otherwise, we say it is a {\em partial policy}. In the general case, learning a {\em full policy} for an arbitrary planning task is complex because a given action scheme may be applicable in different situations and also because the compact representation of this set of different situations may require the computation of {\em high-level state features}~\cite{lotinac2016automatic}.

The $\Lambda$ learning task can also be redefined to cover the scenario where some of the actions executed by the observed agent are available. If all the executed actions are known then states observations should be partial otherwise, the learning task is trivial. The learning task is now formalized as $\Lambda=\tup{\mathcal{M},\mathcal{O},\Psi,\pi}$, where:
\begin{itemize}
\item The plan $\pi=\tup{a_1, \ldots, a_n}$, is the action sequence that produces the sequence of state observations given in $\mathcal{O}$. Again we assume that action observations are noiseless, meaning that if the value of an action is observed it is correct. When the input plan is {\em diverse} enough, i.e. $\pi$ contains at least one ground action for each of the aimed action schemes, the set of {\em empty} operator schemas $\mathcal{M}$ can be inferred from $\pi$.
\end{itemize}

Figure~\ref{fig:example-plans} shows an example of a learning task $\Lambda=\tup{\mathcal{M},\mathcal{O},\Psi,\pi}$, that corresponds to observing the execution of a four-action plan for inverting a two-block tower. In this example $\mathcal{O}=\tup{s_0,s_4}$ which means that only the first and last states are observed and the three intermediate states $s_1$, $s_2$ and $s_3$ are fully unknown. $\mathcal{M},\Psi$ are skipped, since they are the same as in Figure~\ref{fig:example-observations}.

\begin{figure}[hbt!]
{\footnotesize\tt ;;;;;; Observations $\mathcal{O}$}
\begin{footnotesize}
\begin{verbatim}
;;; observation #0
(clear B) (on B A) (ontable A) (handempty)

;;; observation #4
(clear A) (on A B) (ontable B) (handempty)
\end{verbatim}
\end{footnotesize}

{\footnotesize\tt ;;; Plan $\pi$}
\begin{footnotesize}
\begin{verbatim}
0: (unstack B A)
1: (putdown B)
2: (pickup A)
3: (stack A B)
\end{verbatim}
\end{footnotesize}

 \caption{\small Task $\Lambda=\tup{\mathcal{M},\mathcal{O},\Psi,\pi}$ for learning a {\em blocksworld} \strips\ action model from a four-action plan and two state observations.}
\label{fig:example-plans}
\end{figure}

The previous definitions formalize the learning of \strips\ action models from the observation of a single plan execution. These definitions are extensible to the more general case where the execution of multiple plans is observed. The task is defined as $\Lambda=\tup{\mathcal{M},\Psi,\mathcal{O'},\Pi}$, where $\Pi=\{\pi_1,\ldots,\pi_{\tau}\}$ is the given sequence of example plans producing the corresponding sequence of state observations $\mathcal{O'}$. Now $\mathcal{O'}=\tup{s_0^1,s_1^1,\ldots,s_{n}^1,\ldots,s_0^t,s_1^t,\ldots,s_{n}^t\ldots,s_0^{\tau},s_1^{\tau},\ldots,s_{n}^{\tau}}$ is a sequence of {\em state observations} obtained watching the execution of a serie of {\em unobserved} plans $\pi^t=\tup{a_1, \ldots, a_n^t}$, {\tt\small $1\leq t\leq \tau$}, one after the other. This means that, for each {\small $1\leq i\leq n^t$}, $a_i^t\in \pi^t$ is applicable in $s_{i-1}^t$ and generates the successor state $s_i^t=\theta(s_{i-1}^t,a_i^t)$.




\section{Evaluation of action models}
\label{sec:evaluation}

In this section we introduce the metrics used by \FAMA to evaluate the action models that result from solving a learning task $\Lambda$. First, we will describe two standard syntactic metrics ({\em precision} and {\em recall}) and then section \ref{semantic_precision_recall} will define a semantic evaluation measure that builds upon {\em precision} and {\em recall}. Finally, section \ref{edit_distance} explains how \FAMA computes our novel semantic-based metrics.

\vspace{0.1cm}

When the planning reference model of the input observations (i.e., the GTM) is available, the quality of the learned action models is measurable using two well-studied syntax-based metrics, {\em precision} and {\em recall}, commonly used in tasks such as information retrieval and recommender systems~\cite{davis2006relationship}. These two syntactic metrics are generally more informative than counting the number of errors between the learned action models and the GTM. Intuitively, precision gives a notion of {\em soundness} while recall gives a notion of the {\em completeness} of the learned models:

\begin{itemize}
\item $Precision=\frac{tp}{tp+fp}$, where $tp$ is the number of {\em true positives} (in our particular case, predicates that correctly appear in the action model) and $fp$ is the number of {\em false positives} (predicates of the learned model that should not appear).
\item $Recall=\frac{tp}{tp+fn}$, where $fn$ is the number of {\em false negatives} (predicates that should appear in the learned model but are missing).
\end{itemize}


Introducing semantic-based evaluation metrics can be justified on two grounds:

\begin{enumerate}
\item When the GTM is unknown. This is the most common scenario in ML, where models are both learned and evaluated with respect to datasets.
\item When the GTM is known but a test-based evaluation on a dataset is preferable (or needed as complementary to a syntactic evaluation). As a rule of thumb, it is preferable to evaluate the learned models wrt a dataset because a learned model can be semantically correct though syntactically incorrect (different from the GTM). We refer to this phenomenon as \emph{model reformulation}.
\end{enumerate}

An example of \emph{model reformulation} is the swapping of the roles of two {\em comparable} action models. Two action models $\xi$ and $\xi'$ are comparable if both have the same parameters (iff $pars(\xi)=pars(\xi'$)) and so they share the same space of possible models. Hence, the {\em blocksworld} operator {\small\tt stack} could be {\em learned} with the preconditions and effects of the {\small\tt unstack} operator, and viceversa, because they are comparable. On the contrary, this reformulation will not happen between the {\tt stack} and {\tt pickup} because they are not comparable. In the same way, the roles of two action parameters that share the same type can also be swapped (e.g., interchanging the role of the two parameters of the operator {\small\tt stack} or the opreator {\small\tt unstack}) and yet the learned models would be semantically correct with respect to the given input observations. A more complex kind of reformulation occurs when two or more action models are learned in a single \emph{macro-action}.


These semantic alterations typically appear in the learned models when the observed input data given in $\tau$ is scarce. Defining a proper semantic evaluation is key because the application of syntax-based metrics may report low scores for learned models that are actually {\em sound} and {\em complete} but correspond to {\em reformulations} of the GTM. In the following sections, we introduce a novel evaluation metric that is robust to different types of reformulation.


\subsection{Semantic-based precision and recall}
\label{semantic_precision_recall}

The \ARMS system showed that a semantic evaluation can be done via validation of a set of plan traces with the learned model~\cite{yang2007learning}. The underlying idea is that an error indication of the learned action models is obtained by counting the number of preconditions that are not satisfied during the execution of the plan trace with the learned models, similarly to the functionality provided by the automatic validation tool VAL~\cite{howey2004val} used in the IPCs. This approach can be understood as modifying the plan trace (by adding the necessary preconditions to the intermediate states) so as to allow the execution of the observed actions using the learned models. In other words, modifying the plan trace to fit the model. Inspired by this approach, we present here an alternative evaluation that, instead, determines the modifications required by a learned model to explain the given plan traces.

The rationale behind our novel metrics lies in counting the \emph{edit operations} that need to be applied in a set of action models $\mathcal{M}$ to fit the plan traces. Given a set of action models $\mathcal{M}$, the two allowed edit operations are:

\begin{itemize}
\item {\em Deletion}. A fluent $pre_p(\xi)/del_p(\xi)/add_p(\xi)$ is removable from $\xi\in\mathcal{M}$.
\item {\em Insertion}. A fluent $pre_p(\xi)/del_p(\xi)/add_p(\xi)$ can be added to $\xi\in\mathcal{M}$.
\end{itemize}

We now provide formal definitions of $INS(\mathcal{M},\mathcal{M'})$ and $DEL(\mathcal{M},\mathcal{M'})$, the sets of insertions and deletions, respectively, that are needed to transform a set of action models $\mathcal{M}$ into a new set of action models $\mathcal{M'}$.

\begin{mydefinition}
	Let $PRE(\xi) = \underset{\forall p \in pre(\xi)}{\bigcup} pre_p(\xi)$, $ADD(\xi) = \underset{\forall p \in add(\xi)}{\bigcup} add_p(\xi)$, and $DEL(\xi) = \underset{\forall p \in del(\xi)}{\bigcup} del_p(\xi)$ be the set of propositional fluents that represent preconditions, positive and negative effects of a given action model $\xi$. We define:
	\begin{small}
		\begin{align*}
		INS(\mathcal{M}, \mathcal{M'})=&PRE(\xi') \backslash PRE(\xi) \cup\\
		&ADD(\xi') \backslash ADD(\xi) \cup \\
		&DEL(\xi') \backslash DEL(\xi),\forall \xi\in\mathcal{M}, \xi'\in\mathcal{M'} s.t.\ name(\xi) = name(\xi')\\
		\\
		DEL(\mathcal{M}, \mathcal{M'})=&PRE(\xi) \backslash PRE(\xi') \cup\\
		&ADD(\xi) \backslash ADD(\xi') \cup \\
		&DEL(\xi) \backslash DEL(\xi'),\forall \xi\in\mathcal{M}, \xi'\in\mathcal{M'} s.t.\ name(\xi) = name(\xi')\\
		\end{align*}
	\end{small}
\end{mydefinition}


With these ingredients in mind, we first adapt the definitions of syntactic precision and recall to sets of action models. Let $\mathcal{M}$ be a set of learned action models and let $\mathcal{M'}$ be the GTM. We know that $size(\mathcal{M}) = \left|pre(\xi)\right| + \left|add(\xi)\right| + \left|del(\xi)\right| \forall \xi \in \mathcal{M}$ and by definition the number of preconditions and effects of the learned action models is equal to the sum of {\em true positives} and {\em false positives}; that is, $size(\mathcal{M}) = tp + fp$.

The number of \emph{deletions} required to transform $\mathcal{M}$ into $\mathcal{M'}$ ($\left|DEL(\mathcal{M},\mathcal{M'})\right|$) matches our previous definition of the number of {\em false positives}; and $\left|INS(\mathcal{M},\mathcal{M'})\right|$, the number of \emph{insertions} required to transform $\mathcal{M}$ into $\mathcal{M'}$, corresponds to the number of {\em false negatives} of $\mathcal{M}$. Then we can affirm that $size(\mathcal{M'}) = size(\mathcal{M}) - \left|DEL(\mathcal{M},\mathcal{M'})\right| + \left|INS(\mathcal{M},\mathcal{M'})\right|$.

\begin{mydefinition} \label{syn-precision} The precision of $\mathcal{M}$ relative to the GTM is defined as the fraction of the common predicates and effects between $\mathcal{M}$ and the GTM among all predicates and effects of $\mathcal{M}$. This gives an intuitive measure of the soundness of $\mathcal{M}$.
\begin{small}
	\begin{align*}
     Precision=\frac{tp}{tp+fp}=\frac{size(\mathcal{M})- \left|DEL(\mathcal{M},GTM)\right|}{size(\mathcal{M})}
	\end{align*}
\end{small}
\end{mydefinition}



\begin{mydefinition} \label{syn-recall} The recall of $\mathcal{M}$ relative to the GTM is defined as the fraction of the common predicates and effects between $\mathcal{M}$ and the GTM among all predicates and effects of the GTM. This gives an intuitive measure of the completeness of $\mathcal{M}$.

\begin{small}
	\begin{align*}
     Recall= \frac{tp}{tp+fn}=
\frac{size(\mathcal{M})- \left|DEL(\mathcal{M},GTM)\right|}{size(\mathcal{M}) - \left|DEL(\mathcal{M},GTM)\right| + \left|INS(\mathcal{M},GTM)\right|}
	\end{align*}
\end{small}
\end{mydefinition}


Definitions \ref{syn-precision} and \ref{syn-recall} are syntax-based metrics to evaluate $\mathcal{M}$ with respect to the GTM. A semantic metric, on the other hand, evaluates $\mathcal{M}$ with respect to a given set of plan traces.

We interpret the semantic evaluation of action models as a learning task $\Lambda = \tup{\mathcal{M}, \mathcal{T}}$, where:

\begin{itemize}
	\item $\mathcal{M}$ is a \textbf{set of learned action models} obtained using any learning approach such as \FAMA; in general, $\mathcal{M}$ can be any given input set of action models even manually encoded.
	\item$\mathcal{T}$ is a set of plan traces  used for \textbf{testing}.
\end{itemize}


A solution to this task is an \textbf{edited set of action models} $\mathcal{M'}$ such that (1) $\mathcal{M'}$ is obtained by exclusively applying a finite sequence of \emph{deletion} and \emph{insertion} operations to $\mathcal{M}$ and (2) $\mathcal{M'}$ explains $\mathcal{T}$; i.e. $\mathcal{M'}$ is {\em compliant} with every plan trace $\tau\in\mathcal{T}$. It is always recommended for the test set to be different from the one used during learning and this is specially important for satisfying approaches such as \FAMA; otherwise $\mathcal{M'}$ = $\mathcal{M}$ since $\mathcal{M}$ would be able to explain $\mathcal{T}$ without any modification.

Since we are defining the semantic evaluation task in terms of a learning task $\Lambda$, there might exist potentially many edited models $\mathcal{M'}$ which are solution to this task. Although the actual GTM is included among the solution set, it is impossible to identify it, so we define the best solution based on its proximity to the input model.

\begin{mydefinition} \label{compliant}
  Given a set of action models $\mathcal{M}$, and all the sets of action models $\mathcal{M'}$ able to explain the plan traces $\mathcal{T}$. The {\bf closest compliant set of action models}, $\mathcal{M^*}$, is the comparable set of action models closest to $\mathcal{M}$ (in terms of editions) that is able to explain $\mathcal{T}$;
  \[\mathcal{M^*}=\underset{\forall \mathcal{M}' \rightarrow \mathcal{T}}{\arg\min} \ \left| INS(\mathcal{M},\mathcal{M'}) \cup DEL(\mathcal{M},\mathcal{M'}) \right|\]
\end{mydefinition}


The closest compliant set of action models $\mathcal{M^*}$ allows us to define a semantic version of {\em precision} and {\em recall} following definitions \ref{syn-precision} and \ref{syn-recall}.


\begin{small}
	\begin{align*}
	sem\text{-}Precision=&\frac{size(\mathcal{M})- \left|DEL(\mathcal{M},\mathcal{M^*})\right|}{size(\mathcal{M})}\\
    \vspace{0.5cm}
	sem\text{-}Recall=&\frac{size(\mathcal{M})- \left|DEL(\mathcal{M},\mathcal{M^*})\right|}{size(\mathcal{M}) - \left|DEL(\mathcal{M},\mathcal{M^*})\right| + \left|INS(\mathcal{M},\mathcal{M^*})\right|}
	\end{align*}
\end{small}




\begin{myproposition}
When the closest compliant set of action models $\mathcal{M^*}$ of an evaluation task $\Lambda = \tup{\mathcal{M}, \mathcal{T}}$ is the GTM, the syntactic and semantic evaluation of $\mathcal{M}$ return the same values; that is, $Precision=sem\text{-}Precision=$ and $Recall=sem\text{-}Recall$.
\end{myproposition}


The intuition behind this evaluation is to {\em semantically} assess how well the learned action models $\mathcal{M}$ explain a set of given observations of plan executions according to the amount of {\em edition} required by $\mathcal{M}$ to induce the observations. Unlike the semantic metric defined by ARMS, our novel semantic definitions of precision and recall are not sensitive to flaws that appear more than once in the plan traces since the flaws are corrected only once in the learned models instead of at every intermediate state of the plan traces.

%This semantic evaluation approach is again flexible to various amount and kind of available input knowledge.


\subsection{Semantic evaluation with classical planning}
\label{edit_distance}

%Our compilation is extensible to compute the {\em closest compliant set of action models} and hence, our semantic versions of the {\em precision} and {\em recall} metrics. This extension considers that the input models $\mathcal{M}$, is {\em non-empty} so instead of learning an action model from scratch we simply edit $\mathcal{M}$ until it satisfies the given input observations. In other words, now $\mathcal{M}$ is a set of given operator schemas, wherein each $\xi\in\mathcal{M}$ initially contains the $pre(\xi)$, $del(\xi)$ and $add(\xi)$ sets. A solution to the classical planning task resulting from the extended compilation is a sequence of actions that:

The compilation scheme presented in section \ref{compilation} is extensible to address the evaluation task $\Lambda=\tup{\mathcal{M}, \mathcal{T}}$ defined in section \ref{semantic_precision_recall}. In this extended task, $\mathcal{M}$ represents a set of previously learned action models; therefore, rather than learning the action models from scratch, we simply edit $\mathcal{M}$ until it satisfies the given test set of plan traces $\mathcal{T}$. A solution to the classical planning task resulting from the extended compilation is a plan that:

\begin{enumerate}
\item {\bf Edits the action models $\mathcal{M}$ to build $\mathcal{M}'$}. A solution plan starts with a prefix that modifies the preconditions and effects of the action schemes in $\mathcal{M}$ using the two {\em edit operations} defined above, {\em deletion} and {\em insertion}.
\item {\bf Validates the edited model $\mathcal{M}'$ in the observed plan traces}. The solution plan continues with a postfix that validates the edited model $\mathcal{M}'$ on the given observations $\mathcal{T}$, as explained in Section~\ref{compilation} for the models that are programmed from scratch.
\end{enumerate}

Given $\Lambda=\tup{\mathcal{M},\mathcal{T}}$, the output of the extended compilation is a planning task $P_{\Lambda}'=\tup{F_{\Lambda},A_{\Lambda}',I_{\Lambda},G_{\Lambda}}$ such that:

\begin{itemize}
\item $F_{\Lambda}$, $I_{\Lambda}$ and $G_{\Lambda}$ are defined as in the previous compilation. Note that, the input action model $\mathcal{M}$ is encoded in the initial state. This means that the fluents $pre_p(\xi)/del_p(\xi)/add_p(\xi)$, $p\in \Psi_\xi$, hold in $I_{\Lambda}$ iff they appear in $\mathcal{M}$.
\item $A_{\Lambda}'$, comprises the same three kinds of actions of $A_{\Lambda}$. The actions for {\em applying} an already programmed action model and the actions for {\em validating} an observation are defined exactly as in the previous compilation. The only difference here is that the {\em programming actions} now implement the two editing operations (i.e., they also include the actions for {\em deleting} a precondition or negative/positive effect from an action model).
\end{itemize}

Figure~\ref{fig:plan-pdistance} shows the plan for editing the action model of the operator {\tt\small stack} of the {\em blocksworld} domain where only the two positive effects {\tt\small (handempty)} and {\tt\small (clear ?v1)} are missing. In this case the edited action model is again validated in the plan trace shown in Figure~\ref{fig:example-plans}.

\begin{figure}[hbt!]
{\footnotesize\tt
  {\bf 00} : (insert\_add\_stack\_handempty)\\
  01 : (insert\_add\_stack\_clear\_var1)\\
  {\bf 02} : (apply\_unstack blockB blockA i1 i2)\\
  03 : (apply\_putdown blockB i2 i3)\\
  04 : (apply\_pickup blockA i3 i4)\\
  05 : (apply\_stack blockA blockB i4 i5)\\
  {\bf 06} : (validate\_1)
}
\caption{\small Plan for editing and validating the action model {\tt\small{stack}} in which the positive effects {\tt\small{(handempty)}} and {\tt\small{(clear ?v1)}} are missing.}
\label{fig:plan-pdistance}
\end{figure}

Assuming we are using an optimal planner to solve $P_{\Lambda}'$, the solution plan of this problem will induce the \emph{closest compliant set of action models} $\mathcal{M^*}$. Therefore, our compilation enables the straightforward computation of the semantic versions of \emph{precision} and \emph{recall}. An argument can be made, however, that solving optimally $P_{\Lambda}'$ may turn the evaluation process very time consuming. Considering this, $sem\text{-}Precision$ and $sem\text{-}Recall$ can be approximated if $P_{\Lambda}'$ is solved with a satisfying planner. In this case, no guarantees can be made that the edited models will be the closest compliant ones, but a classical planner will always try to minimize the solution plan length and hence the number of edit operations applied to the input models.
















\section{Experimental results}
\label{experiments}


\subsection{Setup}
The domains used in the evaluation are IPC domains that satisfy the \strips\ requirement~\cite{fox2003pddl2}, taken from the {\sc planning.domains} repository~\cite{muise2016planning}. We only use 5 learning examples for each domain and they are fixed for all the experiments so we can evaluate the impact of the input knowledge in the quality of the learned models. All experiments are run on an Intel Core i5 3.10 GHz x 4 with 8 GB of RAM.
\begin{itemize}
\item {\bf Planner}. The classical planner we use to solve the instances that result from our compilations is {\sc Madagascar}~\cite{rintanen2014madagascar}. We use {\sc Madagascar} because its ability to deal with planning instances populated with dead-ends. In addition, SAT-based planners can apply the actions for programming preconditions in a single planning step (in parallel) because these actions do not interact. Actions for programming action effects can also be applied in a single planning step reducing significantly the planning horizon.
\item {\bf Reproducibility}. We make fully available the compilation source code, the evaluation scripts and the used benchmarks at this repository {\em https://github.com/sjimenezgithub/strips-learning} so any experimental data reported in the paper is fully reproducible.
\end{itemize}

\subsection{Evaluating with a reference model}
Here we evaluate the learned models with respect to the actual generative model. 

\subsubsection{Learning from plans}
We start evaluating our approach with $\Lambda=\tup{\mathcal{M},\Psi,\mathcal{T}}$ learning tasks, where the action of the executed plans are available and the state observation sequence contains only the corresponding initial and goal states, $s_o^t$ and $s_n^t$, for every trace plan $t\in\mathcal{T}$. We then repeat the evaluation but exploiting potential \emph{static predicates} computed from the observed states in $\mathcal{T}$, which are the predicates that appear unaltered in the states that belong to the same plan. Static predicates are used to constrain the space of possible action models as explained in Section~\ref{sec:Section5}.

Table~\ref{tab:results_plans} shows the obtained results. Precision ({\bf P}) and recall ({\bf R}) are computed separately for the preconditions ({\bf Pre}), positive effects ({\bf Add}) and negative Effects ({\bf Del}), while the last two columns of each setting and the last row report averages values. We can observe that identifying static predicates leads to models with better precondition {\em recall}. This fact evidences that many of the missing preconditions corresponded to static predicates because there is no incentive to learn them as they always hold~\cite{gregory2015domain}.

\begin{table*}[hbt!]
		\resizebox{\textwidth}{!}{%
		\begin{tabular}{l|l|l|l|l|l|l||l|l||l|l|l|l|l|l||l|l|}
& \multicolumn{8}{|c||}{\bf No Static}& \multicolumn{8}{|c|}{\bf Static}\\\cline{2-17}
& \multicolumn{2}{|c|}{\bf Pre} & \multicolumn{2}{|c|}{\bf Add} & \multicolumn{2}{|c|}{\bf Del} & \multicolumn{2}{|c||}{\bf}& \multicolumn{2}{|c|}{\bf Pre} & \multicolumn{2}{|c|}{\bf Add} & \multicolumn{2}{|c|}{\bf Del} & \multicolumn{2}{|c|}{\bf}\\ 			
			  & \multicolumn{1}{|c|}{\bf P} & \multicolumn{1}{|c|}{\bf R} & \multicolumn{1}{|c|}{\bf P} & \multicolumn{1}{|c|}{\bf R} & \multicolumn{1}{|c|}{\bf P} & \multicolumn{1}{|c|}{\bf R} &  \multicolumn{1}{|c|}{\bf P} & \multicolumn{1}{|c||}{\bf R}& \multicolumn{1}{|c|}{\bf P} & \multicolumn{1}{|c|}{\bf R} & \multicolumn{1}{|c|}{\bf P} & \multicolumn{1}{|c|}{\bf R} & \multicolumn{1}{|c|}{\bf P} & \multicolumn{1}{|c|}{\bf R} &  \multicolumn{1}{|c|}{\bf P} & \multicolumn{1}{|c|}{\bf R} \\
                          \hline
			Blocks & 1.0 & 1.0 & 1.0 & 1.0 & 1.0 & 1.0 & 1.0 & 1.0 & 1.0 & 1.0 & 1.0 & 1.0 & 1.0 & 1.0 & 1.0 & 1.0\\
			Driverlog & 1.0 & 0.36 & 0.75 & 0.86 & 1.0 & 0.71 & 0.92 & 0.64 & 0.9 & 0.64 & 0.56 & 0.71 & 0.86 & 0.86 & 0.78 & 0.73\\
			Ferry & 1.0 & 0.57 & 1.0 & 1.0 & 1.0 & 1.0 & 1.0 & 0.86 & 1.0 & 0.57 & 1.0 & 1.0 & 1.0 & 1.0 & 1.0 & 0.86\\
			Floortile & 0.52 & 0.68 & 0.64 & 0.82 & 0.83 & 0.91 & 0.66 & 0.80 & 0.68 & 0.68 & 0.89 & 0.73 & 1.0 & 0.82 & 0.86 & 0.74\\
			Grid & 0.62 & 0.47 & 0.75 & 0.86 & 0.78 & 1.0 & 0.71 & 0.78 & 0.79 & 0.65 & 1.0 & 0.86 & 0.88 & 1.0 & 0.89 & 0.83 \\
			Gripper & 1.0 & 0.67 & 1.0 & 1.0 & 1.0 & 1.0 & 1.0 & 0.89 & 1.0 & 0.67 & 1.0 & 1.0 & 1.0 & 1.0 & 1.0 & 0.89\\
			Hanoi & 1.0 & 0.50 & 1.0 & 1.0 & 1.0 & 1.0 & 1.0 & 0.83 & 0.75 & 0.75 & 1.0 & 1.0 & 1.0 & 1.0 & 0.92 & 0.92\\
			Miconic & 0.75 & 0.33 & 0.50 & 0.50 & 0.75 & 1.0 & 0.67 & 0.61 & 0.89 & 0.89 & 1.0 & 0.75 & 0.75 & 1.0 & 0.88 & 0.88\\
			Satellite & 0.60 & 0.21 & 1.0 & 1.0 & 1.0 & 0.75 & 0.87 & 0.65 & 0.82 & 0.64 & 1.0 & 1.0 & 1.0 & 0.75 & 0.94 & 0.80\\
			Transport & 1.0 & 0.40 & 1.0 & 1.0 & 1.0 & 0.80 & 1.0 & 0.73 & 1.0 & 0.70 & 0.83 & 1.0 & 1.0 & 0.80 & 0.94 & 0.83\\
			Visitall & 1.0 & 0.50 & 1.0 & 1.0 & 1.0 & 1.0 & 1.0 & 0.83 & 1.0 & 1.0 & 1.0 & 1.0 & 1.0 & 1.0 & 1.0 & 1.0\\
			Zenotravel & 1.0 & 0.36 & 1.0 & 1.0 & 1.0 & 0.71 & 1.0 & 0.69 &1.0 & 0.64 & 0.88 & 1.0 & 1.0 & 0.71 & 0.96 & 0.79\\
			\hline
			\bf  & 0.88 & 0.50 & 0.88 & 0.92 & 0.95 & 0.91 & 0.90 & 0.78 & 0.90 & 0.74 & 0.93 & 0.92 & 0.96 & 0.91 & 0.93 & 0.86\\
		\end{tabular}
	}
\caption{\small {\em Precision} and {\em recall} scores for learning tasks from labeled plans without (left) and with (right) static predicates.}
\label{tab:results_plans}
\end{table*}

Table~\ref{tab:time_plans} reports the total planning time, the preprocessing time (in seconds) invested by {\sc Madagascar} to solve the planning instances that result from our compilation as well as the number of actions of the solution plans. All the learning tasks are solved in a few seconds. Interestingly, one can identify the domains with static predicates by just looking at the reported plan length. In these domains some of the preconditions that correspond to static predicates are directly derived from the learning examples and therefore fewer programming actions are required. When static predicates are identified, the resulting compilation is also much more compact and produces smaller planning/instantiation times.



\begin{table}
\begin{scriptsize}
	\begin{center}
		\begin{tabular}{l|c|c|c||c|c|c|}
                         & \multicolumn{3}{|c||}{\bf No Static}& \multicolumn{3}{|c|}{\bf Static}\\
			 & Total & Preprocess & Length  & Total & Preprocess &  Length\\
                         \hline
			Blocks & 0.04 & 0.00 & 72  & 0.03 & 0.00 & 72 \\
			Driverlog & 0.14 & 0.09 & 83 & 0.06 & 0.03 & 59 \\
			Ferry & 0.06 & 0.03 & 55 & 0.06 & 0.03 & 55 \\
			Floortile & 2.42 & 1.64 & 168 & 0.67 & 0.57 & 77 \\
			Grid & 4.82 & 4.75 & 88 & 3.39 & 3.35 & 72 \\
			Gripper & 0.03 & 0.01 & 43 & 0.01 & 0.00 & 43 \\
                        Hanoi & 0.12 & 0.06 & 48 & 0.09 & 0.06 & 39 \\
                        Miconic & 0.06 & 0.03 & 57 & 0.04 & 0.00 & 41 \\
			Satellite & 0.20 & 0.14 & 67 & 0.18 & 0.12 & 60 \\
			Transport & 0.59 & 0.53 & 61 & 0.39 & 0.35 & 48 \\
			Visitall & 0.21 & 0.15 & 40 & 0.17 & 0.15 & 36 \\
			Zenotravel & 2.07 & 2.04 & 71 & 1.01 & 1.00 & 55 \\			
		\end{tabular}
	\end{center}
        \end{scriptsize}
	\caption{\small Total planning time, preprocessing time and plan length for learning tasks from labeled plans without/with static predicates.}
	\label{tab:time_plans}	
\end{table}

We evaluate now the ability of our approach to support partially specified action models; that is, when the input model $\mathcal{M}$ is not empty because some preconditions and effects of the actions are initially known.  In this particular experiment, the model of half of the actions is given in $\mathcal{M}$ as an extra input of the learning task. Tables~\ref{tab:results_plans_partial} and~\ref{tab:time_plans_partial} summarize the obtained results, which include the identification of static predicates. We only report the {\em precision} and {\em recall} of the {\em unknown} actions since the values of the metrics of the {\em known} action models is 1.0. In this experiment, a low value of {\em precision} or {\em recall} has a greater impact than in the previous learning tasks because the evaluation is done only over half of the actions. This occurs, for instance, in the precondition \emph{recall} of domains such as {\em Floortile}, {\em Gripper} or {\em Satellite}.

Remarkably, the overall \emph{precision} is now $0.98$, which means that the contents of the learned models is highly reliable. The value of \emph{recall}, 0.87, is an indication that the learned models still miss some information (preconditions are again the component more difficult to be fully learned). Overall, the results confirm the previous trend: the more input knowledge of the task, the better the models and the less planning time. Additionally, the solution plans required for this task are smaller because it is only necessary to program half of the actions (the other half are included in the input knowledge $\mathcal{M}$). {\em Visitall} and {\em Hanoi} are excluded from this evaluation because they only contain one action schema.

\begin{table}[hbt!]
\begin{footnotesize}
	\begin{center}
		
		\begin{tabular}{l|l|l|l|l|l|l||l|l|}
			 & \multicolumn{2}{|c|}{\bf Pre} & \multicolumn{2}{|c|}{\bf Add} & \multicolumn{2}{|c||}{\bf Del} & \multicolumn{2}{|c}{\bf}\\ \cline{2-9}			
			  & \multicolumn{1}{|c|}{\bf P} & \multicolumn{1}{|c|}{\bf R} & \multicolumn{1}{|c|}{\bf P} & \multicolumn{1}{|c|}{\bf R} & \multicolumn{1}{|c|}{\bf P} & \multicolumn{1}{|c||}{\bf R} &  \multicolumn{1}{|c|}{\bf P} & \multicolumn{1}{|c|}{\bf R} \\
			\hline
				Blocks & 1.0 & 1.0 & 1.0 & 1.0 & 1.0 & 1.0 & 1.0 & 1.0 \\
				Driverlog & 1.0 & 0.71 & 1.0 & 1.0 & 1.0 & 1.0 & 1.0 & 0.90 \\
				Ferry & 1.0 & 0.67 & 1.0 & 1.0 & 1.0 & 1.0 & 1.0 & 0.89 \\
				Floortile & 0.75 & 0.60 & 1.0 & 0.80 & 1.0 & 0.80 & 0.92 & 0.73 \\
                Grid & 1.0 & 0.67 & 1.0 & 1.0 & 1.0 & 1.0 & 0.84 & 0.78 \\
				Gripper & 1.0 & 0.50 & 1.0 & 1.0 & 1.0 & 1.0 & 1.0 & 0.83 \\
				Miconic & 1.0 & 1.0 & 1.0 & 1.0 & 1.0 & 1.0 & 1.0 & 1.0 \\
				Satellite & 1.0 & 0.57 & 1.0 & 1.0 & 1.0 & 1.0 & 1.0 & 0.86 \\
				Transport & 1.0 & 0.75 & 1.0 & 1.0 & 1.0 & 1.0 & 1.0 & 0.92 \\
				Zenotravel & 1.0 & 0.67 & 1.0 & 1.0 & 1.0 & 0.67 & 1.0 & 0.78 \\
				\hline
				\bf  & 0.98 & 0.71 & 1.0 & 0.98 & 1.0 & 0.95 & 0.98 & 0.87 \\
			\end{tabular}
		
	\end{center}
\end{footnotesize}
\caption{\small {\em Precision} and {\em recall} scores for learning tasks with partially specified action models.}
\label{tab:results_plans_partial}
\end{table}

\begin{table}
\begin{footnotesize}
	\begin{center}
		\begin{tabular}{l|c|c|c|}			
			 & Total time & Preprocess & Plan length  \\
                         \hline
			Blocks & 0.07 & 0.01 & 54  \\
			Driverlog & 0.03 & 0.01 & 40 \\
			Ferry & 0.06 & 0.03 & 45 \\
			Floortile & 0.43 & 0.42 & 55 \\
                        Grid & 3.12 & 3.07 & 53 \\
			Gripper & 0.03 & 0.01 & 35 \\
			Miconic & 0.03 & 0.01 & 34  \\
			Satellite & 0.14 & 0.14 & 47 \\
			Transport & 0.23 & 0.21 & 37 \\
			Zenotravel & 0.90 & 0.89 & 40 \\
		\end{tabular}
	\end{center}
        \end{footnotesize}
	\caption{\small Time and plan length learning for learning tasks with partially specified action models.}
	\label{tab:time_plans_partial}	
\end{table}


\subsubsection{Learning from state observations}
Here we evaluate our approach with learning tasks of the kind $\Lambda=\tup{\mathcal{M},\Psi,\mathcal{T}}$, where the action of the executed plans are not available but the initial and goal states are known. When input plans are not available, the planner must not only compute the action models but also the plans that satisfy the input observations. Table~\ref{tab:results_states} and ~\ref{tab:time_states} summarize the results obtained for this using static predicates and partially specified models. Values for the {\em Zenotravel} and {\em Grid} domains are not reported because {\sc Madagascar} was not able to solve the corresponding planning tasks within a 1000 sec. time bound. The values of \emph{precision} and \emph{recall} are significantly lower than in Table ~\ref{tab:results_plans}. Given that the learning hypothesis space is now fairly under-constrained, actions can be reformulated and still be compliant with the inputs (e.g. the {\em blocksworld} operator {\small\tt stack} can be {\em learned} with the preconditions and effects of the {\small\tt unstack} operator and vice versa). We tried to minimize this effect with the additional input knowledge (static predicates and partially specified action models) and yet the results are below the scores obtained when learning from labeled plans.


\begin{table}
\begin{footnotesize}
	\begin{center}
		\begin{tabular}{l|l|l|l|l|l|l||l|l|}
			 & \multicolumn{2}{|c|}{\bf Pre} & \multicolumn{2}{|c|}{\bf Add} & \multicolumn{2}{|c||}{\bf Del} & \multicolumn{2}{|c}{\bf}\\ \cline{2-9}			
			  & \multicolumn{1}{|c|}{\bf P} & \multicolumn{1}{|c|}{\bf R} & \multicolumn{1}{|c|}{\bf P} & \multicolumn{1}{|c|}{\bf R} & \multicolumn{1}{|c|}{\bf P} & \multicolumn{1}{|c||}{\bf R} &  \multicolumn{1}{|c|}{\bf P} & \multicolumn{1}{|c|}{\bf R} \\
			\hline
            Blocks & 0.33 & 0.33 & 0.75 & 0.50 & 0.33 & 0.33 & 0.47 & 0.39 \\
            Driverlog & 1.0 & 0.29 & 0.33 & 0.67 & 1.0 & 0.50 & 0.78 & 0.48 \\
            Ferry & 1.0 & 0.67 & 0.50 & 1.0 & 1.0 & 1.0 & 0.83 & 0.89 \\
            Floortile & 0.67 & 0.40 & 0.50 & 0.40 & 1.0 & 0.40 & 0.72 & 0.40 \\
            Grid & - & - & - & - & - & - & - & - \\
            Gripper & 1.0 & 0.50 & 1.0 & 1.0 & 1.0 & 1.0 & 1.0 & 0.83 \\
            Miconic & 0.0 & 0.0 & 0.33 & 0.50 & 0.0 & 0.0 & 0.11 & 0.17 \\
            Satellite & 1.0 & 0.14 & 0.67 & 1.0 & 1.0 & 1.0 & 0.89 & 0.71 \\
            Transport & 0.0 & 0.0 & 0.25 & 0.5 & 0.0 & 0.0 & 0.08 & 0.17 \\
            Zenotravel & - & - & - & - & - & - & - & - \\
            \hline
            & 0.63 & 0.29 & 0.54 & 0.70 & 0.67 & 0.53 & 0.61 & 0.51 \\			
			\end{tabular}
	\end{center}
\end{footnotesize}
\caption{\small {\em Precision} and {\em recall} scores for learning from (initial,final) state pairs.}
\label{tab:results_states}
\end{table}

\begin{table}
\begin{footnotesize}
	\begin{center}
		\begin{tabular}{l|c|c|c|}			
			 & Total time & Preprocess & Plan length  \\
			\hline
            Blocks & 2.14 & 0.00 & 58  \\
            Driverlog & 0.09 & 0.00 & 88 \\
            Ferry & 0.17 & 0.01 & 65 \\
            Floortile & 6.42 & 0.15 & 126 \\
            Grid & - & - & - \\
            Gripper & 0.03 & 0.00 & 47 \\
            Miconic & 0.04 & 0.00 & 68 \\
            Satellite & 4.34 & 0.10 & 126 \\
            Transport & 2.57 & 0.21 & 47 \\			
            Zenotravel & - & - & - \\
		\end{tabular}
	\end{center}
        \end{footnotesize}
	\caption{\small Time and plan length when learning from (initial,final) state pairs.}
	\label{tab:time_states}	
\end{table}

Now we evaluate our approach with learning tasks of the kind $\Lambda=\tup{\mathcal{M},\Psi,\mathcal{T}}$, where the action of the executed plans are not available but where the plan trace $\mathcal{T}$ contains all the intermediate states, not just the initial and final states. Table~\ref{fig:observationsnomap} shows the precision ({\bf P}) and recall ({\bf R}) computed separately for the preconditions ({\bf Pre}), positive effects ({\bf Add}) and negative Effects ({\bf Del}) while the last two columns report averages values. The reason why the scores in Table ~\ref{fig:observationsnomap} are still low, despite more state observations are available, is because the syntax-based nature of {\em precision} and {\em recall} make these two metrics report low scores for learned models that are semantically correct but correspond to {\em reformulations} of the actual model (changes in the roles of actions with matching headers or parameters with matching types).

\begin{table}[hbt!]
	\begin{center}
		\begin{scriptsize}
			\begin{tabular}{l|l|l|l|l|l|l||l|l|}
				& \multicolumn{2}{|c|}{\bf Pre} & \multicolumn{2}{|c|}{\bf Add} & \multicolumn{2}{|c||}{\bf Del} & \multicolumn{2}{|c}{\bf}\\ \cline{2-9}			
				& \multicolumn{1}{|c|}{\bf P} & \multicolumn{1}{|c|}{\bf R} & \multicolumn{1}{|c|}{\bf P} & \multicolumn{1}{|c|}{\bf R} & \multicolumn{1}{|c|}{\bf P} & \multicolumn{1}{|c||}{\bf R} &  \multicolumn{1}{|c|}{\bf P} & \multicolumn{1}{|c|}{\bf R} \\
				\hline

				blocks & 0.44 & 0.44 & 0.44 & 0.44 & 0.44 & 0.44 & 0.44 & 0.44 \\
				driverlog & 0.0 & 0.0 & 0.25 & 0.43 & 0.0 & 0.0 & 0.08 & 0.14 \\
				ferry & 1.0 & 0.71 & 1.0 & 1.0 & 1.0 & 1.0 & 1.0 & 0.9 \\
				floortile & 0.38 & 0.55 & 0.4 & 0.18 & 0.56 & 0.45 & 0.44 & 0.39 \\
				grid & 0.5 & 0.47 & 0.33 & 0.29 & 0.25 & 0.29 & 0.36 & 0.35 \\
				gripper & 0.83 & 0.83 & 0.75 & 0.75 & 0.75 & 0.75 & 0.78 & 0.78 \\
				hanoi & 0.5 & 0.25 & 0.5 & 0.5 & 0.0 & 0.0 & 0.33 & 0.25 \\
				hiking & 0.43 & 0.43 & 0.5 & 0.35 & 0.44 & 0.47 & 0.46 & 0.42 \\
				miconic & 0.6 & 0.33 & 0.33 & 0.25 & 0.33 & 0.33 & 0.42 & 0.31 \\
				npuzzle & 0.33 & 0.33 & 0.0 & 0.0 & 0.0 & 0.0 & 0.11 & 0.11 \\
				parking & 0.25 & 0.21 & 0.0 & 0.0 & 0.0 & 0.0 & 0.08 & 0.07 \\
				satellite & 0.6 & 0.21 & 0.8 & 0.8 & 1.0 & 0.5 & 0.8 & 0.5 \\
				transport & 1.0 & 0.3 & 0.8 & 0.8 & 1.0 & 0.6 & 0.93 & 0.57 \\
				visitall & 0.0 & 0.0 & 0.0 & 0.0 & 0.0 & 0.0 & 0.0 & 0.0 \\
				zenotravel & 0.67 & 0.29 & 0.33 & 0.29 & 0.33 & 0.14 & 0.44 & 0.24
			\end{tabular}
		\end{scriptsize}
	\end{center}
	\caption{\small Precision and recall values obtained without computing the $f_{P\&R}$ mapping with the reference model.}
	\label{fig:observationsnomap}
\end{table}

To give an insight of the actual quality of the learned models, we defined a method for computing {\em Precision} and {\em Recall} that is robust to the mentioned model {\em reformulations}. Precision and recall are often combined using the {\em harmonic mean}. This expression, called the {\em F-measure} or the balanced {\em F-score}, is defined as $F=2\times\frac{Precision\times Recall}{Precision+Recall}$. Given the learned action model $\mathcal{M}$ and the reference action model $\mathcal{M}^*$, the bijective function $f_{P\&R}:\mathcal{M} \mapsto \mathcal{M}^*$ is the mapping between the learned and the reference model that maximizes the accumulated {\em F-measure} (considering swaps in the actions with matching headers or parameters with matching types).

Table~\ref{fig:observationsmap} shows that significantly higher values of {\em precision} and {\em recall} are reported when a learned action schema, $\xi\in\mathcal{M}$, is compared to its corresponding reference schema given by the $f_{P\&R}$ mapping ($f_{P\&R}(\xi)\in \mathcal{M}^*$). The {\em blocksworld} and {\em gripper} domains are perfectly learned from only 25 state observations. These results evidence that in all of the evaluated domains, except for {\em ferry} and {\em satellite}, the learning task swaps the roles of some actions (or parameters) with respect to their role in the reference model.

\begin{table}
        \begin{center}

		\begin{scriptsize}
			\begin{tabular}{l|l|l|l|l|l|l||l|l|}
				& \multicolumn{2}{|c|}{\bf Pre} & \multicolumn{2}{|c|}{\bf Add} & \multicolumn{2}{|c||}{\bf Del} & \multicolumn{2}{|c}{\bf}\\ \cline{2-9}			
				& \multicolumn{1}{|c|}{\bf P} & \multicolumn{1}{|c|}{\bf R} & \multicolumn{1}{|c|}{\bf P} & \multicolumn{1}{|c|}{\bf R} & \multicolumn{1}{|c|}{\bf P} & \multicolumn{1}{|c||}{\bf R} &  \multicolumn{1}{|c|}{\bf P} & \multicolumn{1}{|c|}{\bf R} \\
				\hline

				blocks & 1.0 & 1.0 & 1.0 & 1.0 & 1.0 & 1.0 & 1.0 & 1.0 \\
				driverlog & 0.67 & 0.14 & 0.33 & 0.57 & 0.67 & 0.29 & 0.56 & 0.33 \\
				ferry & 1.0 & 0.71 & 1.0 & 1.0 & 1.0 & 1.0 & 1.0 & 0.9 \\
				floortile & 0.44 & 0.64 & 1.0 & 0.45 & 0.89 & 0.73 & 0.78 & 0.61 \\
				grid & 0.63 & 0.59 & 0.67 & 0.57 & 0.63 & 0.71 & 0.64 & 0.62 \\
				gripper & 1.0 & 1.0 & 1.0 & 1.0 & 1.0 & 1.0 & 1.0 & 1.0 \\
				hanoi & 1.0 & 0.5 & 1.0 & 1.0 & 1.0 & 1.0 & 1.0 & 0.83 \\
				hiking & 0.78 & 0.6 & 0.93 & 0.82 & 0.88 & 0.88 & 0.87 & 0.77 \\
				miconic & 0.8 & 0.44 & 1.0 & 0.75 & 1.0 & 1.0 & 0.93 & 0.73 \\
				npuzzle & 0.67 & 0.67 & 1.0 & 1.0 & 1.0 & 1.0 & 0.89 & 0.89 \\
				parking & 0.56 & 0.36 & 0.5 & 0.33 & 0.5 & 0.33 & 0.52 & 0.34 \\
				satellite & 0.6 & 0.21 & 0.8 & 0.8 & 1.0 & 0.5 & 0.8 & 0.5 \\
				transport & 1.0 & 0.3 & 1.0 & 1.0 & 1.0 & 0.6 & 1.0 & 0.63 \\
				visitall & 0.67 & 1.0 & 1.0 & 1.0 & 1.0 & 1.0 & 0.89 & 1.0 \\
				zenotravel & 1.0 & 0.43 & 0.67 & 0.57 & 1.0 & 0.43 & 0.89 & 0.48
			\end{tabular}
		\end{scriptsize}
	\end{center}
	\caption{\small Precision and recall values obtained when computing the $f_{P\&R}$ mapping with the reference model.}
	\label{fig:observationsmap}
\end{table}


\subsection{Evaluating with a test set}

When a reference model is not available, the learned models are tested with an observation set. Table~\ref{fig:observationstest} summarizes the results obtained when evaluating the quality of the learned models with respect to a test set of state observations. Each test set comprises between 20 and 50 observations per domain and is generated executing the plans for various instances of the IPC domains and collecting the intermediate states. The table shows, for each domain, the {\em observation edit distance} (computed with our extended compilation), the {\em maximum edit distance}, and their ratio. The reported results show that, despite learning only from 25 state observations, 12 out of 15 learned domains yield ratios of $90\%$ or above. This fact evidences that the learned models require very small amounts of edition to match the observations of the given test set.

\begin{table}[hbt!]
		\begin{center}
                \begin{footnotesize}
			\begin{tabular}{l|r|r|c|}
				& $\delta(\mathcal{M},\mathcal{O})$ & $\delta(\mathcal{M},*)$ & $1-\frac{\delta(\mathcal{M},\mathcal{O})}{\delta(\mathcal{M},*)}$ \\
				\hline
				blocks & 0 & 90 & 1.0 \\
				driverlog & 5 & 144 & 0.97 \\
				ferry & 2 & 69 & 0.97 \\
				floortile & 34 & 342 & 0.90 \\
				grid & 42 & 153 & 0.73 \\
				gripper & 2 & 30 & 0.93 \\
				hanoi & 1 & 63 & 0.98 \\
				hiking & 69 & 174 & 0.60 \\
				miconic & 3 & 72 & 0.96 \\
				npuzzle & 2 & 24 & 0.92 \\
                                parking & 4 & 111 & 0.96 \\
				satellite & 24 & 75 & 0.68 \\
				transport & 4 & 78 & 0.95 \\
				visitall & 2 & 24 & 0.92 \\
				zenotravel & 3 & 63 & 0.95
			\end{tabular}
                        	\end{footnotesize}
		\end{center}
	\caption{\small Evaluation of the quality of the learned models with respect to an observations test set.}
	\label{fig:observationstest}
\end{table}

The learning scores of several domains in Table~\ref{fig:observationsmap} are above the ones reported in Table~\ref{fig:observationsnomap}. The reason lies in the particular observations comprised by the test sets. As an example, in the {\em driverlog} domain, the action schema {\small \tt disembark-truck} is missing from the learned model because this action is never induced from the observations in the training set; that is, such action never appears in the corresponding \emph{unobserved} plan. The same happens with the {\small \tt paint-down} action of the {\em floortile} domain or {\small \tt move-curb-to-curb} in the {\em parking} domain. Interestingly, these actions do not appear either in the test sets and so the learned action models are not penalized in Table~\ref{fig:observationstest}. Generating {\em informative} and {\em representative} observations for learning planning action models is an open issue. Planning actions include preconditions that are only satisfied by specific sequences of actions, often, with a low probability of being chosen by chance~\cite{fern2004learning}.



\section{Conclusions}
\label{sec:conclusions}


In this paper we have presented \FAMA, an approach for learning \strips\ action models based on the compilation of the learning task into a planning task. The distinctive characteristic of \FAMA over other state-of-the-art approaches is its ability to learn from minimal input knowledge and, more particularly, from minimal observability plan traces with no observed actions and no observed intermediate states. By lightening the input constraints, \FAMA opens up the way for action model learning to operate on real-world problems, as opposite to current approaches where the heavy input restrictions have limited their applicability to synthetic benchmarks. Our approach is thus very well suited for learning action models in data-based applications where the only observable information is a possibly incomplete sequence of partially observed states.

Besides its capacity of working with minimal observability, \FAMA is also able to learn from very small amounts of input knowledge, a clear advantage in domains where obtaining enough training samples is difficult or costly. While the experimental evaluation shows in general an exponential increase of the computation time as the number of training traces augments, \FAMA is able to learn action models more accurate than those of \ARMS with very limited input knowledge. Unlike extensive-data ML approaches, our work explores an alternative research direction that exploits logic reasoning to learn sound models from minimal input knowledge.

This great flexibility of \FAMA to different amounts of minimal input knowledge impacts the complexity of the planning tasks. When the length of the input plan traces is unknown, that is, we ignore the number of steps of the decision-making process of the agent, the planning task to be solved becomes a PSPACE-complete scenario. This is one of the reasons that justifies our compilation-to-planning scheme in contrast to the most extended SAT-solving scheme used in many existing learning systems. Additionally, our planning-based solving scheme allows us to leverage {\em off-the-shelf} classical planners and benefit from the multiple advances in classical planning research.

A key contribution of this work is the proposal of two novel metrics to semantically evaluate the learned action models. These two metrics build upon the well-known syntax-based metrics \emph{precision} and \emph{recall}. Our semantic evaluation mitigates the common issue known as \emph{reformulation} that appears when training sets of minimal observability are used. Due to the lack of observable information, \FAMA can learn semantically valid models that are syntactically different from the reference model. The application of the semantic-based precision and recall allows us to assess the validity of the learned models even in domains where a reference model is not available.
%Remarkably, the action models of the {\em gripper-strips}, {\em n-puzzle} and {\em grid-visit-all} domains were perfectly learned from only observations of the initial and final states.

We highlight that the semantic evaluation of \FAMA is done via the same compilation-to-planning scheme that we use to learn the action models. Given that the input of our learning task definition accepts an initial action model of the agent's behaviour, either complete or partially specified, this solving scheme is exploitable to computing a model that follows the initial model and is compliant with a test set of plan traces. In other words, \FAMA is applicable in model reconciliation tasks by defining a distance metric that measures how close the two models are~\cite{KulkarniCZVZK16}.

More importantly, \FAMA is applicable not only in environments where the domain model is unknown but also in environments where the executable actions are unknown as well. This ability  broadens the range of application to goal reasoning tasks such as plan recognition under imperfect observability \cite{SohrabiRU16}, planning for transparency~\cite{MacNallyLRP18} or counterplanning~\cite{PozancoEFB18}. The application of \FAMA to these tasks offers a plan-based solution where the assumption of a known domain model can be removed. In other words, \FAMA opens up the way towards domain-free goal reasoning.

Finally, we would like to add a note on the open issue of generating {\em informative} plan traces for learning planning action models. Planning actions include preconditions that are only satisfied by specific sequences of actions which have low probability of being chosen by chance~\cite{fern2004learning}. The success of recent algorithms for exploring planning tasks~\cite{FrancesRLG17} motivates the development of novel techniques that enable to autonomously collect informative learning examples. The combination of such exploration techniques with our learning approach is an appealing research direction towards the bootstrapping of planning action models.






%When action plans are not available, our approach still produces action models that are compliant with the input information. In this case, since learning is not constrained by actions, operators can be reformulated changing their semantics, in which case the comparison with a reference model turns out to be tricky.

%When example plans are available, our approach is also able to compute accurate action models from small sets of learning examples. Our experimental results report that \FAMA achieves {\em precision} and {\em recall} scores that are over 0.7 for ten out of the fourteen IPC domains, using only two plan traces (with ten actions each) and in little computation time (less than a second). Our experimental evaluation also shows that, in the particular setting of minimal input data, \FAMA outperforms \ARMS.

%On the other hand, this work also introduced a semantic version of the {\em precision} and {\em recall} metrics. These two metrics, widely used in ML, separately measure the soundness and completeness of the learned models, which facilitate the identification of model flaws. Our brand new semantic version of {\em precision} and {\em recall} do not require a GTM and evaluate \strips\ action models with respect to a given set of observations. In addition, when no reformulation occurs, these new metrics behave similarly to their syntactic counterparts.





\subsection*{Acknowledgment}
This work is supported by the Spanish MINECO project TIN2017-88476-C2-1-R. Diego Aineto is partially supported by the {\it FPU16/03184} and Sergio Jim\'enez by the {\it RYC15/18009}, both programs funded by the Spanish government.


%% References
%%
%% Following citation commands can be used in the body text:
%% Usage of \cite is as follows:
%%   \cite{key}         ==>>  [#]
%%   \cite[chap. 2]{key} ==>> [#, chap. 2]
%%

%% References with BibTeX database:

\bibliographystyle{elsarticle-num}
\bibliography{planlearnbibliography}

\section*{Appendix}
\label{sec:appendix}


\begin{figure}[hbtp!]
\begin{scriptsize}  
\begin{verbatim}
(define (domain BLOCKS)
  (:requirements :strips)
  (:predicates (on ?x ?y)(ontable ?x)(clear ?x)(handempty)(holding ?x))

  (:action pick-up
	     :parameters (?x)
	     :precondition (and (clear ?x) (ontable ?x) (handempty))
	     :effect (and (not (ontable ?x)) (not (clear ?x)) (not (handempty)) (holding ?x)))

  (:action put-down
	     :parameters (?x)
	     :precondition (holding ?x)
	     :effect (and (not (holding ?x)) (clear ?x) (handempty) (ontable ?x)))

  (:action stack
	     :parameters (?x ?y)
	     :precondition (and (holding ?x) (clear ?y))
	     :effect (and (not (holding ?x)) (not (clear ?y)) (clear ?x) (handempty) (on ?x ?y)))

  (:action unstack
	     :parameters (?x ?y)
	     :precondition (and (on ?x ?y) (clear ?x) (handempty))
	     :effect (and (holding ?x) (clear ?y) (not (clear ?x)) (not (handempty)) (not (on ?x ?y)))))
  \end{verbatim}
\end{scriptsize}  
\caption{\small PDDL domain file for the blocksworld domain.}
\label{fig:bw-domain}
\end{figure}


Compiled PDDL domain file for learning the blocksworld action models from two initial and final states.

\begin{scriptsize}  
\begin{verbatim}
(define (domain blocks)
 (:requirements :strips)
 (:types object - None )
 (:constants a -  object b -  object c -  object d -  object e -  object f -  object g -  object)
 (:predicates (add_clear_pick-up_var1) (add_clear_put-down_var1) (add_clear_stack_var1) (add_clear_stack_var2) 
(add_clear_unstack_var1) (add_clear_unstack_var2) (add_handempty_pick-up) (add_handempty_put-down) 
(add_handempty_stack) (add_handempty_unstack) (add_holding_pick-up_var1) (add_holding_put-down_var1) 
(add_holding_stack_var1) (add_holding_stack_var2) (add_holding_unstack_var1) (add_holding_unstack_var2) 
(add_on_stack_var1_var1) (add_on_stack_var1_var2) (add_on_stack_var2_var1) (add_on_stack_var2_var2) 
(add_on_unstack_var1_var1) (add_on_unstack_var1_var2) (add_on_unstack_var2_var1) (add_on_unstack_var2_var2) 
(add_ontable_pick-up_var1) (add_ontable_put-down_var1) (add_ontable_stack_var1) (add_ontable_stack_var2) 
(add_ontable_unstack_var1) (add_ontable_unstack_var2) (clear ?o1 - object) (del_clear_pick-up_var1) 
(del_clear_put-down_var1) (del_clear_stack_var1) (del_clear_stack_var2) (del_clear_unstack_var1) 
(del_clear_unstack_var2) (del_handempty_pick-up) (del_handempty_put-down) (del_handempty_stack) 
(del_handempty_unstack) (del_holding_pick-up_var1) (del_holding_put-down_var1) (del_holding_stack_var1) 
(del_holding_stack_var2) (del_holding_unstack_var1) (del_holding_unstack_var2) (del_on_stack_var1_var1) 
(del_on_stack_var1_var2) (del_on_stack_var2_var1) (del_on_stack_var2_var2) (del_on_unstack_var1_var1) 
(del_on_unstack_var1_var2) (del_on_unstack_var2_var1) (del_on_unstack_var2_var2) (del_ontable_pick-up_var1) 
(del_ontable_put-down_var1) (del_ontable_stack_var1) (del_ontable_stack_var2) (del_ontable_unstack_var1) 
(del_ontable_unstack_var2) (handempty) (holding ?o1 - object) (modeProg) (on ?o1 - object ?o2 - object) 
(ontable ?o1 - object) (pre_clear_pick-up_var1) (pre_clear_put-down_var1) (pre_clear_stack_var1) 
(pre_clear_stack_var2) (pre_clear_unstack_var1) (pre_clear_unstack_var2) (pre_handempty_pick-up) 
(pre_handempty_put-down) (pre_handempty_stack) (pre_handempty_unstack) (pre_holding_pick-up_var1) 
(pre_holding_put-down_var1) (pre_holding_stack_var1) (pre_holding_stack_var2) (pre_holding_unstack_var1) 
(pre_holding_unstack_var2) (pre_on_stack_var1_var1) (pre_on_stack_var1_var2) (pre_on_stack_var2_var1) 
(pre_on_stack_var2_var2) (pre_on_unstack_var1_var1) (pre_on_unstack_var1_var2) (pre_on_unstack_var2_var1) 
(pre_on_unstack_var2_var2) (pre_ontable_pick-up_var1) (pre_ontable_put-down_var1) (pre_ontable_stack_var1) 
(pre_ontable_stack_var2) (pre_ontable_unstack_var1) (pre_ontable_unstack_var2) (test1) (test2) (test3))

 (:action pick-up
   :parameters (?o1 - object)
   :precondition (and (not (modeProg ))(or (not (pre_ontable_pick-up_var1 ))(ontable ?o1))
(or (not (pre_clear_pick-up_var1 ))(clear ?o1))(or (not (pre_handempty_pick-up ))(handempty ))
(or (not (pre_holding_pick-up_var1 ))(holding ?o1)))
   :effect (and (when (and (del_ontable_pick-up_var1 ))(not (ontable ?o1)))
(when (and (add_ontable_pick-up_var1 ))(ontable ?o1))(when (and (del_clear_pick-up_var1 ))(not (clear ?o1)))
(when (and (add_clear_pick-up_var1 ))(clear ?o1))(when (and (del_handempty_pick-up ))(not (handempty )))
(when (and (add_handempty_pick-up ))(handempty ))(when (and (del_holding_pick-up_var1 ))(not (holding ?o1)))
(when (and (add_holding_pick-up_var1 ))(holding ?o1))))

 (:action put-down
   :parameters (?o1 - object)
   :precondition (and (not (modeProg ))(or (not (pre_ontable_put-down_var1 ))(ontable ?o1))
(or (not (pre_clear_put-down_var1 ))(clear ?o1))(or (not (pre_handempty_put-down ))(handempty ))
(or (not (pre_holding_put-down_var1 ))(holding ?o1)))
   :effect (and (when (and (del_ontable_put-down_var1 ))(not (ontable ?o1)))
(when (and (add_ontable_put-down_var1 ))(ontable ?o1))(when (and (del_clear_put-down_var1 ))(not (clear ?o1)))
(when (and (add_clear_put-down_var1 ))(clear ?o1))(when (and (del_handempty_put-down ))(not (handempty )))
(when (and (add_handempty_put-down ))(handempty ))(when (and (del_holding_put-down_var1 ))(not (holding ?o1)))
(when (and (add_holding_put-down_var1 ))(holding ?o1))))

 (:action stack
   :parameters (?o1 - object ?o2 - object)
   :precondition (and (not (modeProg ))(or (not (pre_on_stack_var1_var1 ))(on ?o1 ?o1))
(or (not (pre_on_stack_var1_var2 ))(on ?o1 ?o2))(or (not (pre_on_stack_var2_var1 ))(on ?o2 ?o1))
(or (not (pre_on_stack_var2_var2 ))(on ?o2 ?o2))(or (not (pre_ontable_stack_var1 ))(ontable ?o1))
(or (not (pre_ontable_stack_var2 ))(ontable ?o2))(or (not (pre_clear_stack_var1 ))(clear ?o1))
(or (not (pre_clear_stack_var2 ))(clear ?o2))(or (not (pre_handempty_stack ))(handempty ))
(or (not (pre_holding_stack_var1 ))(holding ?o1))(or (not (pre_holding_stack_var2 ))(holding ?o2)))
   :effect (and (when (and (del_on_stack_var1_var1 ))(not (on ?o1 ?o1)))(when (and (add_on_stack_var1_var1 ))(on ?o1 ?o1))
(when (and (del_on_stack_var1_var2 ))(not (on ?o1 ?o2)))(when (and (add_on_stack_var1_var2 ))(on ?o1 ?o2))
(when (and (del_on_stack_var2_var1 ))(not (on ?o2 ?o1)))(when (and (add_on_stack_var2_var1 ))(on ?o2 ?o1))
(when (and (del_on_stack_var2_var2 ))(not (on ?o2 ?o2)))(when (and (add_on_stack_var2_var2 ))(on ?o2 ?o2))
(when (and (del_ontable_stack_var1 ))(not (ontable ?o1)))(when (and (add_ontable_stack_var1 ))(ontable ?o1))
(when (and (del_ontable_stack_var2 ))(not (ontable ?o2)))(when (and (add_ontable_stack_var2 ))(ontable ?o2))
(when (and (del_clear_stack_var1 ))(not (clear ?o1)))(when (and (add_clear_stack_var1 ))(clear ?o1))
(when (and (del_clear_stack_var2 ))(not (clear ?o2)))(when (and (add_clear_stack_var2 ))(clear ?o2))
(when (and (del_handempty_stack ))(not (handempty )))(when (and (add_handempty_stack ))(handempty ))
(when (and (del_holding_stack_var1 ))(not (holding ?o1)))(when (and (add_holding_stack_var1 ))(holding ?o1))
(when (and (del_holding_stack_var2 ))(not (holding ?o2)))(when (and (add_holding_stack_var2 ))(holding ?o2))))

 (:action unstack
   :parameters (?o1 - object ?o2 - object)
   :precondition (and (not (modeProg ))(or (not (pre_on_unstack_var1_var1 ))(on ?o1 ?o1))
(or (not (pre_on_unstack_var1_var2 ))(on ?o1 ?o2))(or (not (pre_on_unstack_var2_var1 ))(on ?o2 ?o1))
(or (not (pre_on_unstack_var2_var2 ))(on ?o2 ?o2))(or (not (pre_ontable_unstack_var1 ))(ontable ?o1))
(or (not (pre_ontable_unstack_var2 ))(ontable ?o2))(or (not (pre_clear_unstack_var1 ))(clear ?o1))
(or (not (pre_clear_unstack_var2 ))(clear ?o2))(or (not (pre_handempty_unstack ))(handempty ))
(or (not (pre_holding_unstack_var1 ))(holding ?o1))(or (not (pre_holding_unstack_var2 ))(holding ?o2)))
   :effect (and (when (and (del_on_unstack_var1_var1 ))(not (on ?o1 ?o1)))
(when (and (add_on_unstack_var1_var1 ))(on ?o1 ?o1))(when (and (del_on_unstack_var1_var2 ))(not (on ?o1 ?o2)))
(when (and (add_on_unstack_var1_var2 ))(on ?o1 ?o2))(when (and (del_on_unstack_var2_var1 ))(not (on ?o2 ?o1)))
(when (and (add_on_unstack_var2_var1 ))(on ?o2 ?o1))(when (and (del_on_unstack_var2_var2 ))(not (on ?o2 ?o2)))
(when (and (add_on_unstack_var2_var2 ))(on ?o2 ?o2))(when (and (del_ontable_unstack_var1 ))(not (ontable ?o1)))
(when (and (add_ontable_unstack_var1 ))(ontable ?o1))(when (and (del_ontable_unstack_var2 ))(not (ontable ?o2)))
(when (and (add_ontable_unstack_var2 ))(ontable ?o2))(when (and (del_clear_unstack_var1 ))(not (clear ?o1)))
(when (and (add_clear_unstack_var1 ))(clear ?o1))(when (and (del_clear_unstack_var2 ))(not (clear ?o2)))
(when (and (add_clear_unstack_var2 ))(clear ?o2))(when (and (del_handempty_unstack ))(not (handempty )))
(when (and (add_handempty_unstack ))(handempty ))(when (and (del_holding_unstack_var1 ))(not (holding ?o1)))
(when (and (add_holding_unstack_var1 ))(holding ?o1))(when (and (del_holding_unstack_var2 ))(not (holding ?o2)))
(when (and (add_holding_unstack_var2 ))(holding ?o2))))

 (:action program_pre_ontable_pick-up_var1
   :parameters ()
  :precondition (and (modeProg )(not (pre_ontable_pick-up_var1 ))(not (del_ontable_pick-up_var1 ))
  (not (add_ontable_pick-up_var1 )))
   :effect (and (pre_ontable_pick-up_var1 )))

 (:action program_eff_ontable_pick-up_var1
   :parameters ()
   :precondition (and (modeProg )(not (del_ontable_pick-up_var1 ))(not (add_ontable_pick-up_var1 )))
  :effect (and (when (pre_ontable_pick-up_var1 )(del_ontable_pick-up_var1 ))
  (when (not (pre_ontable_pick-up_var1 ))(add_ontable_pick-up_var1 ))))

 (:action program_pre_clear_pick-up_var1
   :parameters ()
  :precondition (and (modeProg )(not (pre_clear_pick-up_var1 ))(not (del_clear_pick-up_var1 ))
  (not (add_clear_pick-up_var1 )))
   :effect (and (pre_clear_pick-up_var1 )))

 (:action program_eff_clear_pick-up_var1
   :parameters ()
   :precondition (and (modeProg )(not (del_clear_pick-up_var1 ))(not (add_clear_pick-up_var1 )))
  :effect (and (when (pre_clear_pick-up_var1 )(del_clear_pick-up_var1 ))
  (when (not (pre_clear_pick-up_var1 ))(add_clear_pick-up_var1 ))))

 (:action program_pre_handempty_pick-up
   :parameters ()
  :precondition (and (modeProg )(not (pre_handempty_pick-up ))(not (del_handempty_pick-up ))
  (not (add_handempty_pick-up )))
   :effect (and (pre_handempty_pick-up )))

 (:action program_eff_handempty_pick-up
   :parameters ()
   :precondition (and (modeProg )(not (del_handempty_pick-up ))(not (add_handempty_pick-up )))
  :effect (and (when (pre_handempty_pick-up )(del_handempty_pick-up ))
  (when (not (pre_handempty_pick-up ))(add_handempty_pick-up ))))

 (:action program_pre_holding_pick-up_var1
   :parameters ()
  :precondition (and (modeProg )(not (pre_holding_pick-up_var1 ))(not (del_holding_pick-up_var1 ))
  (not (add_holding_pick-up_var1 )))
   :effect (and (pre_holding_pick-up_var1 )))

 (:action program_eff_holding_pick-up_var1
   :parameters ()
   :precondition (and (modeProg )(not (del_holding_pick-up_var1 ))(not (add_holding_pick-up_var1 )))
  :effect (and (when (pre_holding_pick-up_var1 )(del_holding_pick-up_var1 ))
  (when (not (pre_holding_pick-up_var1 ))(add_holding_pick-up_var1 ))))

 (:action program_pre_ontable_put-down_var1
   :parameters ()
  :precondition (and (modeProg )(not (pre_ontable_put-down_var1 ))(not (del_ontable_put-down_var1 ))
  (not (add_ontable_put-down_var1 )))
   :effect (and (pre_ontable_put-down_var1 )))

 (:action program_eff_ontable_put-down_var1
   :parameters ()
   :precondition (and (modeProg )(not (del_ontable_put-down_var1 ))(not (add_ontable_put-down_var1 )))
  :effect (and (when (pre_ontable_put-down_var1 )(del_ontable_put-down_var1 ))
  (when (not (pre_ontable_put-down_var1 ))(add_ontable_put-down_var1 ))))

 (:action program_pre_clear_put-down_var1
   :parameters ()
  :precondition (and (modeProg )(not (pre_clear_put-down_var1 ))(not (del_clear_put-down_var1 ))
  (not (add_clear_put-down_var1 )))
   :effect (and (pre_clear_put-down_var1 )))

 (:action program_eff_clear_put-down_var1
   :parameters ()
   :precondition (and (modeProg )(not (del_clear_put-down_var1 ))(not (add_clear_put-down_var1 )))
  :effect (and (when (pre_clear_put-down_var1 )(del_clear_put-down_var1 ))
  (when (not (pre_clear_put-down_var1 ))(add_clear_put-down_var1 ))))

 (:action program_pre_handempty_put-down
   :parameters ()
  :precondition (and (modeProg )(not (pre_handempty_put-down ))(not (del_handempty_put-down ))
  (not (add_handempty_put-down )))
   :effect (and (pre_handempty_put-down )))

 (:action program_eff_handempty_put-down
   :parameters ()
   :precondition (and (modeProg )(not (del_handempty_put-down ))(not (add_handempty_put-down )))
  :effect (and (when (pre_handempty_put-down )(del_handempty_put-down ))
  (when (not (pre_handempty_put-down ))(add_handempty_put-down ))))

 (:action program_pre_holding_put-down_var1
   :parameters ()
  :precondition (and (modeProg )(not (pre_holding_put-down_var1 ))(not (del_holding_put-down_var1 ))
  (not (add_holding_put-down_var1 )))
   :effect (and (pre_holding_put-down_var1 )))

 (:action program_eff_holding_put-down_var1
   :parameters ()
   :precondition (and (modeProg )(not (del_holding_put-down_var1 ))(not (add_holding_put-down_var1 )))
  :effect (and (when (pre_holding_put-down_var1 )(del_holding_put-down_var1 ))
  (when (not (pre_holding_put-down_var1 ))(add_holding_put-down_var1 ))))

 (:action program_pre_on_stack_var1_var1
   :parameters ()
  :precondition (and (modeProg )(not (pre_on_stack_var1_var1 ))(not (del_on_stack_var1_var1 ))
  (not (add_on_stack_var1_var1 )))
   :effect (and (pre_on_stack_var1_var1 )))

 (:action program_eff_on_stack_var1_var1
   :parameters ()
   :precondition (and (modeProg )(not (del_on_stack_var1_var1 ))(not (add_on_stack_var1_var1 )))
  :effect (and (when (pre_on_stack_var1_var1 )(del_on_stack_var1_var1 ))
  (when (not (pre_on_stack_var1_var1 ))(add_on_stack_var1_var1 ))))

 (:action program_pre_on_stack_var1_var2
   :parameters ()
  :precondition (and (modeProg )(not (pre_on_stack_var1_var2 ))(not (del_on_stack_var1_var2 ))
  (not (add_on_stack_var1_var2 )))
   :effect (and (pre_on_stack_var1_var2 )))

 (:action program_eff_on_stack_var1_var2
   :parameters ()
   :precondition (and (modeProg )(not (del_on_stack_var1_var2 ))(not (add_on_stack_var1_var2 )))
  :effect (and (when (pre_on_stack_var1_var2 )(del_on_stack_var1_var2 ))
  (when (not (pre_on_stack_var1_var2 ))(add_on_stack_var1_var2 ))))

 (:action program_pre_on_stack_var2_var1
   :parameters ()
  :precondition (and (modeProg )(not (pre_on_stack_var2_var1 ))(not (del_on_stack_var2_var1 ))
  (not (add_on_stack_var2_var1 )))
   :effect (and (pre_on_stack_var2_var1 )))

 (:action program_eff_on_stack_var2_var1
   :parameters ()
   :precondition (and (modeProg )(not (del_on_stack_var2_var1 ))(not (add_on_stack_var2_var1 )))
  :effect (and (when (pre_on_stack_var2_var1 )(del_on_stack_var2_var1 ))
  (when (not (pre_on_stack_var2_var1 ))(add_on_stack_var2_var1 ))))

 (:action program_pre_on_stack_var2_var2
   :parameters ()
  :precondition (and (modeProg )(not (pre_on_stack_var2_var2 ))(not (del_on_stack_var2_var2 ))
  (not (add_on_stack_var2_var2 )))
   :effect (and (pre_on_stack_var2_var2 )))

 (:action program_eff_on_stack_var2_var2
   :parameters ()
   :precondition (and (modeProg )(not (del_on_stack_var2_var2 ))(not (add_on_stack_var2_var2 )))
  :effect (and (when (pre_on_stack_var2_var2 )(del_on_stack_var2_var2 ))
  (when (not (pre_on_stack_var2_var2 ))(add_on_stack_var2_var2 ))))

 (:action program_pre_ontable_stack_var1
   :parameters ()
  :precondition (and (modeProg )(not (pre_ontable_stack_var1 ))(not (del_ontable_stack_var1 ))
  (not (add_ontable_stack_var1 )))
   :effect (and (pre_ontable_stack_var1 )))

 (:action program_eff_ontable_stack_var1
   :parameters ()
   :precondition (and (modeProg )(not (del_ontable_stack_var1 ))(not (add_ontable_stack_var1 )))
  :effect (and (when (pre_ontable_stack_var1 )(del_ontable_stack_var1 ))
  (when (not (pre_ontable_stack_var1 ))(add_ontable_stack_var1 ))))

 (:action program_pre_ontable_stack_var2
   :parameters ()
  :precondition (and (modeProg )(not (pre_ontable_stack_var2 ))(not (del_ontable_stack_var2 ))
  (not (add_ontable_stack_var2 )))
   :effect (and (pre_ontable_stack_var2 )))

 (:action program_eff_ontable_stack_var2
   :parameters ()
   :precondition (and (modeProg )(not (del_ontable_stack_var2 ))(not (add_ontable_stack_var2 )))
  :effect (and (when (pre_ontable_stack_var2 )(del_ontable_stack_var2 ))
  (when (not (pre_ontable_stack_var2 ))(add_ontable_stack_var2 ))))

 (:action program_pre_clear_stack_var1
   :parameters ()
  :precondition (and (modeProg )(not (pre_clear_stack_var1 ))(not (del_clear_stack_var1 ))
  (not (add_clear_stack_var1 )))
   :effect (and (pre_clear_stack_var1 )))

 (:action program_eff_clear_stack_var1
   :parameters ()
   :precondition (and (modeProg )(not (del_clear_stack_var1 ))(not (add_clear_stack_var1 )))
  :effect (and (when (pre_clear_stack_var1 )(del_clear_stack_var1 ))
  (when (not (pre_clear_stack_var1 ))(add_clear_stack_var1 ))))

 (:action program_pre_clear_stack_var2
   :parameters ()
  :precondition (and (modeProg )(not (pre_clear_stack_var2 ))(not (del_clear_stack_var2 ))
  (not (add_clear_stack_var2 )))
   :effect (and (pre_clear_stack_var2 )))

 (:action program_eff_clear_stack_var2
   :parameters ()
   :precondition (and (modeProg )(not (del_clear_stack_var2 ))(not (add_clear_stack_var2 )))
  :effect (and (when (pre_clear_stack_var2 )(del_clear_stack_var2 ))
  (when (not (pre_clear_stack_var2 ))(add_clear_stack_var2 ))))

 (:action program_pre_handempty_stack
   :parameters ()
  :precondition (and (modeProg )(not (pre_handempty_stack ))(not (del_handempty_stack ))
  (not (add_handempty_stack )))
   :effect (and (pre_handempty_stack )))

 (:action program_eff_handempty_stack
   :parameters ()
   :precondition (and (modeProg )(not (del_handempty_stack ))(not (add_handempty_stack )))
  :effect (and (when (pre_handempty_stack )(del_handempty_stack ))
  (when (not (pre_handempty_stack ))(add_handempty_stack ))))

 (:action program_pre_holding_stack_var1
   :parameters ()
  :precondition (and (modeProg )(not (pre_holding_stack_var1 ))(not (del_holding_stack_var1 ))
  (not (add_holding_stack_var1 )))
   :effect (and (pre_holding_stack_var1 )))

 (:action program_eff_holding_stack_var1
   :parameters ()
   :precondition (and (modeProg )(not (del_holding_stack_var1 ))(not (add_holding_stack_var1 )))
  :effect (and (when (pre_holding_stack_var1 )(del_holding_stack_var1 ))
  (when (not (pre_holding_stack_var1 ))(add_holding_stack_var1 ))))

 (:action program_pre_holding_stack_var2
   :parameters ()
  :precondition (and (modeProg )(not (pre_holding_stack_var2 ))(not (del_holding_stack_var2 ))
  (not (add_holding_stack_var2 )))
   :effect (and (pre_holding_stack_var2 )))

 (:action program_eff_holding_stack_var2
   :parameters ()
   :precondition (and (modeProg )(not (del_holding_stack_var2 ))(not (add_holding_stack_var2 )))
  :effect (and (when (pre_holding_stack_var2 )(del_holding_stack_var2 ))
  (when (not (pre_holding_stack_var2 ))(add_holding_stack_var2 ))))

 (:action program_pre_on_unstack_var1_var1
   :parameters ()
  :precondition (and (modeProg )(not (pre_on_unstack_var1_var1 ))(not (del_on_unstack_var1_var1 ))
  (not (add_on_unstack_var1_var1 )))
   :effect (and (pre_on_unstack_var1_var1 )))

 (:action program_eff_on_unstack_var1_var1
   :parameters ()
   :precondition (and (modeProg )(not (del_on_unstack_var1_var1 ))(not (add_on_unstack_var1_var1 )))
  :effect (and (when (pre_on_unstack_var1_var1 )(del_on_unstack_var1_var1 ))
  (when (not (pre_on_unstack_var1_var1 ))(add_on_unstack_var1_var1 ))))

 (:action program_pre_on_unstack_var1_var2
   :parameters ()
  :precondition (and (modeProg )(not (pre_on_unstack_var1_var2 ))(not (del_on_unstack_var1_var2 ))
  (not (add_on_unstack_var1_var2 )))
   :effect (and (pre_on_unstack_var1_var2 )))

 (:action program_eff_on_unstack_var1_var2
   :parameters ()
   :precondition (and (modeProg )(not (del_on_unstack_var1_var2 ))(not (add_on_unstack_var1_var2 )))
  :effect (and (when (pre_on_unstack_var1_var2 )(del_on_unstack_var1_var2 ))
  (when (not (pre_on_unstack_var1_var2 ))(add_on_unstack_var1_var2 ))))

 (:action program_pre_on_unstack_var2_var1
   :parameters ()
  :precondition (and (modeProg )(not (pre_on_unstack_var2_var1 ))(not (del_on_unstack_var2_var1 ))
  (not (add_on_unstack_var2_var1 )))
   :effect (and (pre_on_unstack_var2_var1 )))

 (:action program_eff_on_unstack_var2_var1
   :parameters ()
   :precondition (and (modeProg )(not (del_on_unstack_var2_var1 ))(not (add_on_unstack_var2_var1 )))
  :effect (and (when (pre_on_unstack_var2_var1 )(del_on_unstack_var2_var1 ))
  (when (not (pre_on_unstack_var2_var1 ))(add_on_unstack_var2_var1 ))))

 (:action program_pre_on_unstack_var2_var2
   :parameters ()
  :precondition (and (modeProg )(not (pre_on_unstack_var2_var2 ))(not (del_on_unstack_var2_var2 ))
  (not (add_on_unstack_var2_var2 )))
   :effect (and (pre_on_unstack_var2_var2 )))

 (:action program_eff_on_unstack_var2_var2
   :parameters ()
   :precondition (and (modeProg )(not (del_on_unstack_var2_var2 ))(not (add_on_unstack_var2_var2 )))
  :effect (and (when (pre_on_unstack_var2_var2 )(del_on_unstack_var2_var2 ))
  (when (not (pre_on_unstack_var2_var2 ))(add_on_unstack_var2_var2 ))))

 (:action program_pre_ontable_unstack_var1
   :parameters ()
  :precondition (and (modeProg )(not (pre_ontable_unstack_var1 ))(not (del_ontable_unstack_var1 ))
  (not (add_ontable_unstack_var1 )))
   :effect (and (pre_ontable_unstack_var1 )))

 (:action program_eff_ontable_unstack_var1
   :parameters ()
   :precondition (and (modeProg )(not (del_ontable_unstack_var1 ))(not (add_ontable_unstack_var1 )))
  :effect (and (when (pre_ontable_unstack_var1 )(del_ontable_unstack_var1 ))
  (when (not (pre_ontable_unstack_var1 ))(add_ontable_unstack_var1 ))))

 (:action program_pre_ontable_unstack_var2
   :parameters ()
  :precondition (and (modeProg )(not (pre_ontable_unstack_var2 ))(not (del_ontable_unstack_var2 ))
  (not (add_ontable_unstack_var2 )))
   :effect (and (pre_ontable_unstack_var2 )))

 (:action program_eff_ontable_unstack_var2
   :parameters ()
   :precondition (and (modeProg )(not (del_ontable_unstack_var2 ))(not (add_ontable_unstack_var2 )))
  :effect (and (when (pre_ontable_unstack_var2 )(del_ontable_unstack_var2 ))
  (when (not (pre_ontable_unstack_var2 ))(add_ontable_unstack_var2 ))))

 (:action program_pre_clear_unstack_var1
   :parameters ()
  :precondition (and (modeProg )(not (pre_clear_unstack_var1 ))(not (del_clear_unstack_var1 ))
  (not (add_clear_unstack_var1 )))
   :effect (and (pre_clear_unstack_var1 )))

 (:action program_eff_clear_unstack_var1
   :parameters ()
   :precondition (and (modeProg )(not (del_clear_unstack_var1 ))(not (add_clear_unstack_var1 )))
  :effect (and (when (pre_clear_unstack_var1 )(del_clear_unstack_var1 ))
  (when (not (pre_clear_unstack_var1 ))(add_clear_unstack_var1 ))))

 (:action program_pre_clear_unstack_var2
   :parameters ()
  :precondition (and (modeProg )(not (pre_clear_unstack_var2 ))(not (del_clear_unstack_var2 ))
  (not (add_clear_unstack_var2 )))
   :effect (and (pre_clear_unstack_var2 )))

 (:action program_eff_clear_unstack_var2
   :parameters ()
   :precondition (and (modeProg )(not (del_clear_unstack_var2 ))(not (add_clear_unstack_var2 )))
  :effect (and (when (pre_clear_unstack_var2 )(del_clear_unstack_var2 ))
  (when (not (pre_clear_unstack_var2 ))(add_clear_unstack_var2 ))))

 (:action program_pre_handempty_unstack
   :parameters ()
  :precondition (and (modeProg )(not (pre_handempty_unstack ))(not (del_handempty_unstack ))
  (not (add_handempty_unstack )))
   :effect (and (pre_handempty_unstack )))

 (:action program_eff_handempty_unstack
   :parameters ()
   :precondition (and (modeProg )(not (del_handempty_unstack ))(not (add_handempty_unstack )))
  :effect (and (when (pre_handempty_unstack )(del_handempty_unstack ))
  (when (not (pre_handempty_unstack ))(add_handempty_unstack ))))

 (:action program_pre_holding_unstack_var1
   :parameters ()
  :precondition (and (modeProg )(not (pre_holding_unstack_var1 ))(not (del_holding_unstack_var1 ))
  (not (add_holding_unstack_var1 )))
   :effect (and (pre_holding_unstack_var1 )))

 (:action program_eff_holding_unstack_var1
   :parameters ()
   :precondition (and (modeProg )(not (del_holding_unstack_var1 ))(not (add_holding_unstack_var1 )))
  :effect (and (when (pre_holding_unstack_var1 )(del_holding_unstack_var1 ))
  (when (not (pre_holding_unstack_var1 ))(add_holding_unstack_var1 ))))

 (:action program_pre_holding_unstack_var2
   :parameters ()
  :precondition (and (modeProg )(not (pre_holding_unstack_var2 ))(not (del_holding_unstack_var2 ))
  (not (add_holding_unstack_var2 )))
   :effect (and (pre_holding_unstack_var2 )))

 (:action program_eff_holding_unstack_var2
   :parameters ()
   :precondition (and (modeProg )(not (del_holding_unstack_var2 ))(not (add_holding_unstack_var2 )))
  :effect (and (when (pre_holding_unstack_var2 )(del_holding_unstack_var2 ))
  (when (not (pre_holding_unstack_var2 ))(add_holding_unstack_var2 ))))

 (:action validate_1
   :parameters ()
   :precondition (and (modeProg ))
  :effect (and (not (modeProg ))(clear a)(not (clear b))(clear c)(not (clear d))(not (clear e))(not (clear f))
  (not (clear g))(handempty )(not (holding a))(not (holding b))(not (holding c))(not (holding d))
  (not (holding e))(not (holding f))(not (holding g))(not (on a a))(not (on a b))(not (on a c))(not (on a d))
  (not (on a e))(not (on a f))(on a g)(not (on b a))(not (on b b))(not (on b c))(not (on b d))(on b e)
  (not (on b f))(not (on b g))(not (on c a))(not (on c b))(not (on c c))(on c d)(not (on c e))(not (on c f))
  (not (on c g))(not (on d a))(on d b)(not (on d c))(not (on d d))(not (on d e))(not (on d f))(not (on d g))
  (not (on e a))(not (on e b))(not (on e c))(not (on e d))(not (on e e))(on e f)(not (on e g))(not (on f a))
  (not (on f b))(not (on f c))(not (on f d))(not (on f e))(not (on f f))(not (on f g))(not (on g a))
  (not (on g b))(not (on g c))(not (on g d))(not (on g e))(not (on g f))(not (on g g))(not (ontable a))
  (not (ontable b))(not (ontable c))(not (ontable d))(not (ontable e))(ontable f)(ontable g)(test1 )))

 (:action validate_2
   :parameters ()
  :precondition (and (not (modeProg ))(clear a)(clear b)(clear c)(clear d)(clear e)(not (clear f))(not (clear g))
  (handempty )(not (holding a))(not (holding b))(not (holding c))(not (holding d))(not (holding e))
  (not (holding f))(not (holding g))(not (on a a))(not (on a b))(not (on a c))(not (on a d))(not (on a e))
  (not (on a f))(on a g)(not (on b a))(not (on b b))(not (on b c))(not (on b d))(not (on b e))(not (on b f))
  (not (on b g))(not (on c a))(not (on c b))(not (on c c))(not (on c d))(not (on c e))(not (on c f))(not (on c g))
  (not (on d a))(not (on d b))(not (on d c))(not (on d d))(not (on d e))(not (on d f))(not (on d g))(not (on e a))
  (not (on e b))(not (on e c))(not (on e d))(not (on e e))(on e f)(not (on e g))(not (on f a))(not (on f b))
  (not (on f c))(not (on f d))(not (on f e))(not (on f f))(not (on f g))(not (on g a))(not (on g b))(not (on g c))
  (not (on g d))(not (on g e))(not (on g f))(not (on g g))(not (ontable a))(ontable b)(ontable c)(ontable d)
  (not (ontable e))(ontable f)(ontable g)(test1 ))
   :effect (and (test2 )(not (test1 ))))

 (:action validate_3
   :parameters ()
  :precondition (and (not (modeProg ))(clear a)(clear b)(clear c)(clear d)(clear e)(clear f)(clear g)(handempty )
  (not (holding a))(not (holding b))(not (holding c))(not (holding d))(not (holding e))(not (holding f))
  (not (holding g))(not (on a a))(not (on a b))(not (on a c))(not (on a d))(not (on a e))(not (on a f))
  (not (on a g))(not (on b a))(not (on b b))(not (on b c))(not (on b d))(not (on b e))(not (on b f))(not (on b g))
  (not (on c a))(not (on c b))(not (on c c))(not (on c d))(not (on c e))(not (on c f))(not (on c g))(not (on d a))
  (not (on d b))(not (on d c))(not (on d d))(not (on d e))(not (on d f))(not (on d g))(not (on e a))(not (on e b))
  (not (on e c))(not (on e d))(not (on e e))(not (on e f))(not (on e g))(not (on f a))(not (on f b))(not (on f c))
  (not (on f d))(not (on f e))(not (on f f))(not (on f g))(not (on g a))(not (on g b))(not (on g c))(not (on g d))
  (not (on g e))(not (on g f))(not (on g g))(ontable a)(ontable b)(ontable c)(ontable d)(ontable e)(ontable f)
  (ontable g)(test2 ))
   :effect (and (not (test2 ))(test3 )))

)

  \end{verbatim}
\end{scriptsize}  



\begin{figure}[hbtp!]
\begin{scriptsize}  
  \begin{verbatim}
(define (problem learning_problem)
  (:domain blocks)
  (:objects g -  object c -  object d -  object e -  object a -  object b -  object f -  object )
  (:init (modeProg ) )
  (:goal (and (test3 ))))
  \end{verbatim}
\end{scriptsize}  
\caption{\small Compiled PDDL problem file for learning the blocksworld action models from two initial and final states.}
\label{fig:compiled-problem}
\end{figure}


%% Authors are advised to use a BibTeX database file for their reference list.
%% The provided style file elsarticle-num.bst formats references in the required Procedia style

%% For references without a BibTeX database:

% \begin{thebibliography}{00}

%% \bibitem must have the following form:
%%   \bibitem{key}...
%%

% \bibitem{}

% \end{thebibliography}



\end{document}

%%
%% End of file `ecrc-template.tex'.
