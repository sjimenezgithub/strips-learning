%%%% ijcai19.tex

\typeout{IJCAI-19 Instructions for Authors}

% These are the instructions for authors for IJCAI-19.

\documentclass{article}
\pdfpagewidth=8.5in
\pdfpageheight=11in
% The file ijcai19.sty is NOT the same than previous years'
\usepackage{ijcai19}

% Use the postscript times font!
\usepackage{times}
\usepackage{soul}
\usepackage{url}
\usepackage[hidelinks]{hyperref}
\usepackage[utf8]{inputenc}
\usepackage[small]{caption}
\usepackage{graphicx}
\usepackage{amsmath}
\usepackage{booktabs}
\usepackage{algorithm}
\usepackage{algorithmic}
\urlstyle{same}

%%%%%%%%%%%%%%%%%% Added for this paper
\usepackage{ wasysym }
\newcommand{\tup}[1]{{\langle #1 \rangle}}
\newcommand{\pre}{\mathsf{pre}}     % precondition
\newcommand{\del}{\mathsf{del}}     % effect
\newcommand{\add}{\mathsf{add}}     % effect
\newcommand{\eff}{\mathsf{eff}}     % effect
\newcommand{\cond}{\mathsf{cond}}   % conditional effect
\newcommand{\true}{\mathsf{true}}   % true
\newcommand{\false}{\mathsf{false}} % false
\newcommand{\PE}{\mathrm{PE}}     % precondition
\newcommand{\strips}{\textsc{Strips}}

\newtheorem{theorem}{Theorem}
\newtheorem{lemma}[theorem]{Lemma}
\newtheorem{definition}[theorem]{Definition}

%%%%%%%%%%%%%55


% the following package is optional:
%\usepackage{latexsym} 

% Following comment is from ijcai97-submit.tex:
% The preparation of these files was supported by Schlumberger Palo Alto
% Research, AT\&T Bell Laboratories, and Morgan Kaufmann Publishers.
% Shirley Jowell, of Morgan Kaufmann Publishers, and Peter F.
% Patel-Schneider, of AT\&T Bell Laboratories collaborated on their
% preparation.

% These instructions can be modified and used in other conferences as long
% as credit to the authors and supporting agencies is retained, this notice
% is not changed, and further modification or reuse is not restricted.
% Neither Shirley Jowell nor Peter F. Patel-Schneider can be listed as
% contacts for providing assistance without their prior permission.

% To use for other conferences, change references to files and the
% conference appropriate and use other authors, contacts, publishers, and
% organizations.
% Also change the deadline and address for returning papers and the length and
% page charge instructions.
% Put where the files are available in the appropriate places.

\title{Computing the {\em least-commitment} action model from state observations}

% Single author syntax
%\author{
%    Sarit Kraus
%    \affiliations
%    Department of Computer Science, Bar-Ilan University, Israel \emails
%    pcchair@ijcai19.org
%}

% Multiple author syntax (remove the single-author syntax above and the \iffalse ... \fi here)
% Check the ijcai19-multiauthor.tex file for detailed instructions
\author{
Diego Aineto$^1$\and
Sergio Jim\'enez$^1$\and
Eva Onaindia$^1$\And
\and
Blai Bonet$^2$
\affiliations
$^1${\small Departamento de Sistemas Inform\'aticos y Computaci\'on. Universitat Polit\`ecnica de Val\`encia. Valencia, Spain}\\
$^2${\small Departamento de Computaci\'on. Universidad Sim\'on Bolívar. Caracas, Venezuela}
\emails
{\scriptsize \{dieaigar,serjice,onaindia\}@dsic.upv.es, bonet@usb.ve}}



\begin{document}

\maketitle

\begin{abstract}
  
\end{abstract}

\section{Introduction}
Given a sequence of partially observed states, this paper formalizes the task of computing the {\em least-commitment} action model that is able to {\em explain} the given observation. This task is of interest because it allows the incremental learning of action models from arbitrary large sets of partial observations.

In addition, the paper introduces a new method to compute the {\em least-commitment} action model from a sequence of partially observed states. The method assumes that action models are specified as \strips\ action schemata and it is built on top of off-the-shelf algorithms for {\em conformant planning}.



\section{Background}
This section formalizes the planning models we use in the paper as well as the kind of observations that are given as input for the computation of the {\em least-commitment} action model.  

\subsection{Classical planning with conditional effects}
Let $F$ be the set of {\em fluents} or {\em state variables} (propositional variables). A {\em literal} $l$ is a valuation of a fluent $f\in F$, i.e. either~$l=f$ or $l=\neg f$. $L$ is a set of literals that represents a partial assignment of values to fluents, and $\mathcal{L}(F)$ is the set of all literals sets on $F$, i.e.~all partial assignments of values to fluents. A {\em state} $s$ is a full assignment of values to fluents. We explicitly include negative literals $\neg f$ in states s.t. $|s|=|F|$ and the size of the state space is $2^{|F|}$.

A {\em classical planning frame} is a tuple $\Phi=\tup{F,A}$, where $F$ is a set of fluents and $A$ is a set of \emph{actions}. An action $a\in A$ is defined with {\em preconditions}, $\pre(a)\in\mathcal{L}(F)$, {\em positive effects}, $\eff^+(a)\in\mathcal{L}(F)$, and {\em negative effects} $\eff^-(a)\in\mathcal{L}(F)$. The semantics of actions $a\in A$ is specified with two functions: $\rho(s,a)$ denotes whether action $a$ is {\em applicable} in a state $s$ and $\theta(s,a)$ denotes the {\em successor state} that results of applying action $a$ in a state $s$. Then, $\rho(s,a)$ holds iff $\pre(a)\subseteq s$. And the result of applying $a$ in $s$ is $\theta(s,a)=\{s\setminus\eff^-(a))\cup\eff^+(a)\}$.

A {\em classical planning problem} is a tuple $P=\tup{F,A,I,G}$, where $I$ is the initial state and $G\in\mathcal{L}(F)$ is the set of goal conditions over the state variables. A {\em plan} $\pi$ is an action sequence $\pi=\tup{a_1, \ldots, a_n}$, with $|\pi|=n$ denoting its {\em plan length}. The execution of $\pi$ on the initial state $I$ of $P$ induces a {\em trajectory} $\tau(\pi,s_0)=\tup{s_0, a_1, s_1, \ldots, a_n, s_n}$ such that $s_0=I$ and, for each {\small $1\leq i\leq n$}, it holds $\rho(s_{i-1},a_i)$ and $s_i=\theta(s_{i-1},a_i)$. A plan $\pi$ solves $P$ iff the induced {\em trajectory} $\tau(\pi,s_0)$ reaches a final state $G \subseteq s_n$, where all goal conditions are met. A solution plan is optimal iff its length is minimal.

An action $a_c\in A$ with conditional effects is defined as a set of preconditions $\pre(a_c)\in\mathcal{L}(F)$ and a set of {\em conditional effects} $\cond(a_c)$. Each conditional effect $C\rhd E\in\cond(a_c)$ is composed of two sets of literals: $C\in\mathcal{L}(F)$, the {\em condition}, and $E\in\mathcal{L}(F)$, the {\em effect}. An action $a_c$ is applicable in a state $s$ if $\rho(s,a_c)$ is true, and the {\em triggered effects} resulting from the action application are the effects whose conditions hold in $s$:
\[
triggered(s,a_c)=\bigcup_{C\rhd E\in\cond(a_c),C\subseteq s} E,
\]

The result of applying action $a_c$ in state $s$ is $\theta(s,a_c)=\{s\setminus\eff_c^-(s,a))\cup\eff_c^+(s,a)\}$, where $\eff_c^-(s,a)\subseteq triggered(s,a)$ and $\eff_c^+(s,a)\subseteq triggered(s,a)$ are, respectively, the triggered {\em negative} and {\em positive} effects.


\subsection{The observation model}
Given a classical planning problem $P=\tup{F,A,I,G}$, a plan $\pi$ and a trajectory $\tau(\pi,s_0)$, we define the \emph{observation of the trajectory} as a sequence of partial states that results from observing the execution of $\pi$ on $I$. Formally, $\mathcal{O}(\tau)=\tup{s_0^o,s_1^o \ldots , s_m^o}$ where $s_0^o=I$.

A partially observable state $s_i^o$ is one in which $|s_i^o| < |F|$; i.e., a state in which at least a fluent of $F$ is not observable. Note that this definition also comprises the case $|s_i^o| = 0$, when the state is fully unobservable. Whatever the sequence of observed states of $\mathcal{O}(\tau)$ is, it must be consistent with the sequence of states of $\tau(\pi,s_0)$, meaning that $\forall i, s_i^o \subseteq s_i$. In practice, the number of observed states, $m$, ranges from 1 (the initial state, at least), to $|\pi|+1$, and the observed intermediate states will comprise a number of fluents between $[1,|F|]$.

In other words, we assume there is a bijective monotone mapping between trajectories and observations~\cite{ramirez2009plan}, thus also granting the inverse consistency relationship (the trajectory is a superset of the observation). Therefore, transiting between two consecutive observed states in $\mathcal{O}(\tau)$ may require the execution of more than a single action ($\theta(s_i^o,\tup{a_1,\ldots,a_k})=s_{i+1}^o$, where ${\small k\geq 1}$ is unknown but finite. In other words, having $\mathcal{O}(\tau)$ does not imply knowing the actual length of $\pi$.

\begin{definition}[Explaning a $\mathcal{O}(\tau)$ observation]
Given a {\em classical planning problem} $P$ and a sequence of partially observed states $\mathcal{O}(\tau)$, we say that $P$ {\em explains the observation} (denoted $P\mapsto\mathcal{O}(\tau)$) iff there exists a solution plan $\pi$ for $P$ that is consistent with $\mathcal{O}(\tau)$. If $\pi$ is also optimal, we say that $\pi$ is the {\em best explanation} for $\mathcal{O}(\tau)$. 
\end{definition}

\subsection{Conformant planning}
{\em Conformant planning} is planning with incomplete information about the initial state, no sensing, and validating that goals are achieved with certainty (despite the uncertainty of the initial state)~\cite{goldman1996expressive,smith1998conformant,bonet2000planning}.

Syntactically, conformant planning problems are expressed in compact form through a set of state variables. A {\em conformant planning problem} can be defined as a tuple $P_c=\tup{F,A,\Upsilon,G}$ where $F$, $A$ and $G$ are the set of {\em fluents}, {\em actions} and {\em goals} (as previously defined for {\em classical planning}). Now $\Upsilon$ is a set of clauses over literals $l=f$ or $l=\neg f$ (for $f\in F$) that define the set of possible initial states. 

A solution to a conformant planning problem is an action sequence that maps each possible initial state into a goal state. More precisely, an action sequence $\pi=\tup{a_1, \ldots, a_n}$ is a {\em conformant plan} for $P_c$ iff, for each possible {\em trajectory} $\tau(\pi,s_0)=\tup{s_0, a_1, s_1, \ldots, a_n, s_n}$ s.t. $s_0$ is a valuation of the fluents in $F$ that satisfies $\Upsilon$, then $\tau(\pi,s_0)$ reaches a final state $G \subseteq s_n$.



\section{Computing the {\em least-commitment} action model from state observations}
First, this section formalizes the notion of the {\em least-commitment} action model that is able to {\em explain} a sequence of partially observed states. Next, the section describes our approach to compute such model via {\em conformant planning}. 

\subsection{The {\em least-commitment} action model}
The task of computing the {\em least-commitment} action model from a sequence of state observations is defined as $\tup{P,\mathcal{O}(\tau)}$:
\begin{itemize}
\item $P=\tup{F,A[\cdot],I,G}$ is a classical planning problem where $A[\cdot]$ is a set of actions s.t. the semantics of each $a\in A[\cdot]$ is unknown; i.e. the corresponding $\tup{\rho,\theta}$ functions are undefined. The set of goals $G$ can also be unknown.
\item $\mathcal{O}(\tau)$ is a sequence of partial states that results from the observation of a trajectory $\tau(\pi,s_0)$ produced by the execution of certain unknown plan $\pi$ that solves $P$.
\end{itemize}

Before formalizing the solution to this task, i.e. the {\em least-commitment} action model, we introduce several necessary definitions. We first start defining a {\em partially specified action model} inspired by the notion of {\em incomplete (annotated) model}~\cite{sreedharan2018handling}.
\begin{definition}[Partially specified action model]
Given a set of actions $A[\cdot]$ and a set of fluents $F$ then, a {\em partially specified action model} $M$ is a set of possible models for the actions in $A[\cdot]$ such that: (1), any model $\mathcal{M}\in M$ is a definition of the $\tup{\rho,\theta}$ functions of every action in $A[\cdot]$ and (2), for every $\mathcal{M}\in M$ the $\tup{\rho,\theta}$ functions are defined in the set of state variables $F$. (Note that if $M$ is a singleton it represents a {\em fully specified action model}).
\end{definition}

Now we are ready to define the {\em least-commitment} action model for an observation $\mathcal{O}(\tau)$.
\begin{definition}[The {\em least-commitment} action model]
Given a $\tup{P,\mathcal{O}(\tau)}$ task and the {\em partially specified action model} $M$ that represents the full space of possible action models for the actions in $A[\cdot]\in P$, then the {\em least-commitment} action model is another {\em partially specified action model} that represents the largest subset of models $M^*\subseteq M$ such that every model $\mathcal{M}\in M^*$ {\em explains} the input observation.
\end{definition}

\subsection{The space of \strips\ action models}
Despite previous definitions are general, this work focuses on the particular kind of action models that are specified as \strips\ action schemata. 

{\em A \strips\ action schema} $\xi$ is defined by four lists: A list of {\em parameters} $pars(\xi)$, and three list of predicates (namely $pre(\xi)$, $del(\xi)$ and $add(\xi)$) that shape the kind of fluents that can appear in the {\em preconditions}, {\em negative effects} and {\em positive effects} of the actions induced from that schema. Let be $\Psi$ the set of {\em predicates} that shape the propositional state variables $F$, and a list of {\em parameters} $pars(\xi)$. The set of elements that can appear in $pre(\xi)$, $del(\xi)$ and $add(\xi)$ of the \strips\ action schema $\xi$ is given by FOL interpretations of $\Psi$ over the parameters $pars(\xi)$. We denote this set of FOL interpretations as ${\mathcal I}_{\Psi,\xi}$. For instance, in the {\em blocksworld} the ${\mathcal I}_{\Psi,\xi}$ set contain eleven elements for the {\tt stack} schemata, ${\mathcal I}_{\Psi,stack}$={\small\tt\{handempty, holding($v_1$), holding($v_2$), clear($v_1$), clear($v_2$), ontable($v_1$), ontable($v_2$), on($v_1,v_1$), on($v_1,v_2$), on($v_2,v_1$), on($v_2,v_2$)\}}. 

Despite any element of ${\mathcal I}_{\Psi,\xi}$ can {\em a priori} appear in the $pre(\xi)$, $del(\xi)$ and $add(\xi)$ of schema $\xi$, the space of possible \strips\ schemata is constrained by constraints of three kinds:
\begin{enumerate}
\item {\em Syntactic constraints}. \strips\ constraints require $del(\xi)\subseteq pre(\xi)$, $del(\xi)\cap add(\xi)=\emptyset$ and $pre(\xi)\cap add(\xi)=\emptyset$. Considering exclusively these syntactic constraints, the size of the space of possible \strips\ schemata is given by $2^{2\times|{\mathcal I}_{\Psi,\xi}|}$. 
\item {\em Domain-specific constraints}. One can introduce domain-specific knowledge to constrain further the space of possible schemata. For instance, in the {\em blocksworld} one can argue that {\small\tt on($v_1$,$v_1$)} and {\small\tt on($v_2$,$v_2$)} will not appear in the $pre(\xi)$, $del(\xi)$ and $add(\xi)$ lists of an action schema $\xi$ because, in this particular domain, a block cannot be on top of itself. As a rule of thumb, {\it state invariants} constraining the possible states of a given planning domain belong to this second class of constraints~\cite{fox1998automatic}. 
\item {\em Observation constraints}. A sequence of state observations $\mathcal{O}(\tau)$ is \emph{semantic knowledge} that constraints further the space of possible action schemata. 
\end{enumerate}


\subsection{Computing the {\em least-commitment} model via conformant planning}
Given the set of actions $A[\cdot]$ (with unknown $\tup{\rho,\theta}$ functions) and a sequence of partial states $\mathcal{O}(\tau)=\tup{s_0^o,s_1^o \ldots , s_m^o}$, we can build the classical planning problem $P_\mathcal{O}=\tup{F,A[\cdot],s_0^o,s_m^o}$. In other words, a classical planning problem such that $\mathcal{O}(\tau)$ is the observation of a $\tau(\pi,I)$ trajectory that solves that problem. 

In this section we show that starting from a $\tup{P_\mathcal{O},\mathcal{O}(\tau)}$ task we can build a {\em conformant planning problem} $P_c$ whose solution induces the {\em least-commitment} action model for the input observation $\mathcal{O}(\tau)$. In more detail, we build a {\em conformant planning problem} $P_c=\tup{F_c,A_c,\Upsilon,G}$ such that:
\begin{itemize}
\item The set of fluents $F_c$ extends $F$ with two new sets of fluents:
      \begin{itemize}
      \item $\{test_j\}_{1\leq j\leq m}$, indicating the state observation $s_j\in\mathcal{O}(\tau)$ where the action model is validated
      \item Fluents {\tt\small pre\_e\_$\xi$} and {\tt\small eff\_e\_$\xi$} (where $e\in{\mathcal I}_{\Psi,\xi}$) implementing a propositional encoding of the {\em preconditions}, {\em negative}, and {\em positive} effects of an action schema $\xi$. Our encoding exploits the syntactic constraint of \strips\ so, if {\tt\small pre\_e\_$\xi$} and {\tt\small eff\_e\_$\xi$} holds it means that $e\in{\mathcal I}_{\Psi,\xi}$ is a negative effect in $\xi$ while if $pre\_e\_\xi$ does not hold but {\tt\small eff\_e\_$\xi$} holds, it means that $e\in{\mathcal I}_{\Psi,\xi}$ is a positive effect in $\xi$. Figure~\ref{fig:propositional} shows the PDDL encoding of the {\tt\small stack(?v1,?v2)} schema and our propositional representation for this same schema.
      \end {itemize}

\begin{figure}
  \begin{tiny}  
  \begin{verbatim}
(:action stack
   :parameters (?v1 ?v2)
   :precondition (and (holding ?v1) (clear ?v2))
   :effect (and (not (holding ?v1)) (not (clear ?v2))
                (clear ?v1) (handempty) (on ?v1 ?v2)))


(pre_holding_v1_stack) (pre_clear_v2_stack)
(eff_holding_v1_stack) (eff_clear_v2_stack)
(eff_clear_v1_stack) (eff_handempty_stack) (eff_on_v1_v2_stack)
  \end{verbatim}           
  \end{tiny}  
 \caption{\small PDDL encoding of the {\tt\small stack(?v1,?v2)} schema and our propositional representation for this same schema.}
\label{fig:propositional}
\end{figure}

      \item The set of actions $A_c$ contains now actions of three different kinds:
\begin{itemize}
      \item Actions for {\em committing} {\tt\small pre\_e\_$\xi$} fluents to a positive/negative value (similar actions are also defined for {\em committing} {\tt\small eff\_e\_$\xi$} fluents to a positive/negative value).
\begin{small}
\begin{align*}
\hspace*{7pt}\pre(\mathsf{commit\top\_pre\_e\_\xi})=&\{mode_{commit}\},\\
\cond(\mathsf{commit\top\_pre\_e\_\xi})=&\{pre\_e\_\xi\}\rhd\{pre\_e\_\xi\},\\
                                    &\{\neg pre\_e\_\xi\}\rhd\{pre\_e\_\xi\}.\\\\
\hspace*{7pt}\pre(\mathsf{commit\bot\_pre\_e\_\xi})=&\{mode_{commit}\},\\
\cond(\mathsf{commit\bot\_pre\_e\_\xi})=&\{pre\_e\_\xi\}\rhd\{\neg pre\_e\_\xi\},\\
                                    &\{\neg pre\_e\_\xi\}\rhd\{\neg pre\_e\_\xi\}.                                    
\end{align*}
\end{small}

      \item Actions for {\em validating} that committed models explain the $s_j$ observed states, {\tt\small $0\leq j< m$}.
\begin{small}
\begin{align*}
\hspace*{7pt}\pre(\mathsf{validate_{j}})=&s_j\cup\{test_{j-1}\},\\
\cond(\mathsf{validate_{j}})=&\{\emptyset\}\rhd\{\neg test_{j-1}, test_j,\\
                            &\{mode_{commit}\}\rhd\{\neg mode_{commit}, mode_{val}\}.
\end{align*}
\end{small}

      \item {\em Editable} actions whose semantics is given by the value of {\tt\small pre\_e\_$\xi$}, {\tt\small eff\_e\_$\xi$} fluents at the current state. Figure~\ref{fig:editable} shows the PDDL encoding of an {\em editable} {\tt\small stack(?v1,?v2)} schema. Note that this editable schema when {\tiny\tt (pre\_holding\_v1\_stack) (pre\_clear\_v2\_stack) (eff\_holding\_v1\_stack) (eff\_clear\_v2\_stack) (eff\_clear\_v1\_stack) (eff\_handempty\_stack) (eff\_on\_v1\_v2\_stack)} hold, then it behaves exactly as the original PDDL schema defined in Figure~\ref{fig:propositional}. Formally, given an operator schema $\xi\in\mathcal{M}$ its {\em editable} version is:
\begin{small}  
\begin{align*}
\hspace*{7pt}\pre(\mathsf{editable_{\xi}})=&\{pre\_e\_\xi\implies e\}_{\forall e\in{\mathcal I}_{\Psi,\xi}}\\
\cond(\mathsf{editable_{\xi}})=&\{pre\_e\_\xi, eff\_e\_\xi\}\rhd\{\neg e\}_{\forall e\in{\mathcal I}_{\Psi,\xi}},\\
&\{\neg pre\_e\_\xi, eff\_e\_\xi\}\rhd\{e\}_{\forall e\in{\mathcal I}_{\Psi,\xi}}.
\end{align*}
\end{small}

\end{itemize}

\item The clauses in $\Upsilon$ comprises:
      \begin{enumerate}
      \item The {\em unit clauses} given by the fluents that hold in the initial state $I=s_0$ and $mode_{commit}$ set to true.
      \item The clauses representing that the actual value of fluents {\tt\small pre\_e\_$\xi$}, {\tt\small eff\_e\_$\xi$} is unknown. In other words, that any model from the \strips\ space of models (following the previously mentioned syntactic constraints) can initially be part of the {\em least-commitment} action model. Formally, for every $\xi$ and $e\in{\mathcal I}_{\Psi,\xi}$, then $\Upsilon$ includes these two clauses:
            \begin{itemize}
            \item {\tt\small pre\_e\_$\xi$} xor {\tt\small $\neg$pre\_e\_$\xi$}.
            \item {\tt\small eff\_e\_$\xi$} xor {\tt\small $\neg$eff\_e\_$\xi$}.           
            \end{itemize}
      \end{enumerate}
One can also add here clauses that encode {\em domain-specific constraints} (as mentioned in the previous section) to make the conformant planning problem easier to be solved for a particular domain. 
\item The new goals are $G_c=\{test_m\}$.
\end{itemize}


\begin{figure}
  \begin{tiny}  
  \begin{verbatim}
(:action stack
 :parameters (?o1 - object ?o2 - object)
 :precondition
   (and (or (not (pre_on_v1_v1_stack)) (on ?o1 ?o1))
        (or (not (pre_on_v1_v2_stack)) (on ?o1 ?o2))
        (or (not (pre_on_v2_v1_stack)) (on ?o2 ?o1))
        (or (not (pre_on_v2_v2_stack)) (on ?o2 ?o2))
        (or (not (pre_ontable_v1_stack)) (ontable ?o1))
        (or (not (pre_ontable_v2_stack)) (ontable ?o2))
        (or (not (pre_clear_v1_stack)) (clear ?o1))
        (or (not (pre_clear_v2_stack)) (clear ?o2))
        (or (not (pre_holding_v1_stack)) (holding ?o1))
        (or (not (pre_holding_v2_stack)) (holding ?o2))
        (or (not (pre_handempty_stack)) (handempty)))
 :effect (and
   (when (and (pre_on_v1_v1_stack)(eff_on_v1_v1_stack)) (not (on ?o1 ?o1)))
   (when (and (pre_on_v1_v2_stack)(eff_on_v1_v2_stack)) (not (on ?o1 ?o2)))
   (when (and (pre_on_v2_v1_stack)(eff_on_v2_v1_stack)) (not (on ?o2 ?o1)))
   (when (and (pre_on_v2_v2_stack)(eff_on_v2_v2_stack)) (not (on ?o2 ?o2)))
   (when (and (pre_ontable_v1_stack)(eff_ontable_v1_stack)) (not (ontable ?o1)))
   (when (and (pre_ontable_v2_stack)(eff_ontable_v2_stack)) (not (ontable ?o2)))
   (when (and (pre_clear_v1_stack)(eff_clear_v1_stack)) (not (clear ?o1)))
   (when (and (pre_clear_v2_stack)(eff_clear_v2_stack)) (not (clear ?o2)))
   (when (and (pre_holding_v1_stack)(eff_holding_v1_stack)) (not (holding ?o1)))
   (when (and (pre_holding_v2_stack)(eff_holding_v2_stack)) (not (holding ?o2)))
   (when (and (pre_handempty_stack)(eff_handempty_stack)) (not (handempty)))
   (when (and (not(pre_on_v1_v1_stack))(eff_on_v1_v1_stack)) (on ?o1 ?o1))
   (when (and (not(pre_on_v1_v2_stack))(eff_on_v1_v2_stack)) (on ?o1 ?o2))
   (when (and (not(pre_on_v2_v1_stack))(eff_on_v2_v1_stack)) (on ?o2 ?o1))
   (when (and (not(pre_on_v2_v2_stack))(eff_on_v2_v2_stack)) (on ?o2 ?o2))
   (when (and (not(pre_ontable_v1_stack))(eff_ontable_v1_stack)) (ontable ?o1))
   (when (and (not(pre_ontable_v2_stack))(eff_ontable_v2_stack)) (ontable ?o2))
   (when (and (not(pre_clear_v1_stack))(eff_clear_v1_stack)) (clear ?o1))
   (when (and (not(pre_clear_v2_stack))(eff_clear_v2_stack)) (clear ?o2))
   (when (and (not(pre_holding_v1_stack))(eff_holding_v1_stack)) (holding ?o1))
   (when (and (not(pre_holding_v2_stack))(eff_holding_v2_stack)) (holding ?o2))
   (when (and (not(pre_handempty_stack))(eff_handempty_stack)) (handempty))))
  \end{verbatim}           
  \end{tiny}  
 \caption{\small PDDL encoding of the editable version of the {\tt\small stack(?v1,?v2)} schema.}
\label{fig:editable}
\end{figure}


\subsection{Compilation properties}


\section{Evaluation}



\section{Conclusions}
~\cite{SternJ17}


%% The file named.bst is a bibliography style file for BibTeX 0.99c
\bibliographystyle{named}
\bibliography{planlearnbibliography.bib}

\end{document}

