%%%% ijcai19.tex

\typeout{IJCAI-19 Instructions for Authors}

% These are the instructions for authors for IJCAI-19.

\documentclass{article}
\pdfpagewidth=8.5in
\pdfpageheight=11in
% The file ijcai19.sty is NOT the same than previous years'
\usepackage{ijcai19}

% Use the postscript times font!
\usepackage{times}
\usepackage{soul}
\usepackage{url}
\usepackage[hidelinks]{hyperref}
\usepackage[utf8]{inputenc}
\usepackage[small]{caption}
\usepackage{graphicx}
\usepackage{amsmath}
\usepackage{booktabs}
\usepackage{algorithm}
\usepackage{algorithmic}
\urlstyle{same}

%%%%%%%%%%%%%%%%%% Added for this paper
\usepackage{listings}% http://ctan.org/pkg/listings
\lstset{
  basicstyle=\ttfamily,
  mathescape
}
\usepackage{ wasysym }
\newcommand{\tup}[1]{{\langle #1 \rangle}}
\newcommand{\pre}{\mathsf{pre}}     % precondition
\newcommand{\del}{\mathsf{del}}     % effect
\newcommand{\add}{\mathsf{add}}     % effect
\newcommand{\eff}{\mathsf{eff}}     % effect
\newcommand{\cond}{\mathsf{cond}}   % conditional effect
\newcommand{\true}{\mathsf{true}}   % true
\newcommand{\false}{\mathsf{false}} % false
\newcommand{\PE}{\mathrm{PE}}     % precondition
\newcommand{\strips}{\textsc{Strips}}

\newtheorem{theorem}{Theorem}
\newtheorem{lemma}[theorem]{Lemma}
\newtheorem{definition}[theorem]{Definition}

%%%%%%%%%%%%%55


% the following package is optional:
%\usepackage{latexsym} 

% Following comment is from ijcai97-submit.tex:
% The preparation of these files was supported by Schlumberger Palo Alto
% Research, AT\&T Bell Laboratories, and Morgan Kaufmann Publishers.
% Shirley Jowell, of Morgan Kaufmann Publishers, and Peter F.
% Patel-Schneider, of AT\&T Bell Laboratories collaborated on their
% preparation.

% These instructions can be modified and used in other conferences as long
% as credit to the authors and supporting agencies is retained, this notice
% is not changed, and further modification or reuse is not restricted.
% Neither Shirley Jowell nor Peter F. Patel-Schneider can be listed as
% contacts for providing assistance without their prior permission.

% To use for other conferences, change references to files and the
% conference appropriate and use other authors, contacts, publishers, and
% organizations.
% Also change the deadline and address for returning papers and the length and
% page charge instructions.
% Put where the files are available in the appropriate places.

\title{Computing the {\em least-commitment} action model from state observations}

% Single author syntax
%\author{
%    Sarit Kraus
%    \affiliations
%    Department of Computer Science, Bar-Ilan University, Israel \emails
%    pcchair@ijcai19.org
%}

% Multiple author syntax (remove the single-author syntax above and the \iffalse ... \fi here)
% Check the ijcai19-multiauthor.tex file for detailed instructions
\author{
Diego Aineto$^1$\and
Sergio Jim\'enez$^1$\and
Eva Onaindia$^1$\And
\and
Blai Bonet$^2$
\affiliations
$^1${\small Departamento de Sistemas Inform\'aticos y Computaci\'on. Universitat Polit\`ecnica de Val\`encia. Valencia, Spain}\\
$^2${\small Departamento de Computaci\'on. Universidad Sim\'on Bolívar. Caracas, Venezuela}
\emails
{\scriptsize \{dieaigar,serjice,onaindia\}@dsic.upv.es, bonet@usb.ve}}



\begin{document}

\maketitle

\begin{abstract}
  
\end{abstract}

\section{Introduction}
Given an input sequence of partially observed states, this paper formalizes the task of computing the {\em least-commitment} action model that is able to {\em explain} the given observation. This task is of interest because it allows the incremental learning of action models from arbitrary large sets of partial observations.

In addition, the paper introduces a new method to compute the {\em least-commitment} action model for an input sequence of partially observed states. The method assumes that action models are specified as \strips\ action schemata and it is built on top of off-the-shelf algorithms for {\em classical planning}.



\section{Background}
This section formalizes the {\em planning models} we use in the paper as well as the kind of state {\em observations} that are given as input for computing the {\em least-commitment} action model.  

\subsection{Classical planning with conditional effects}
Let $F$ be the set of {\em fluents} or {\em state variables} (propositional variables) describing a state. A {\em literal} $l$ is a valuation of a fluent $f\in F$; i.e. either~$l=f$ or $l=\neg f$. A set of literals $L$ represents a partial assignment of values to fluents (without loss of generality, we will assume that $L$ does not contain conflicting values). Given $L$, let $\neg L=\{\neg l:l\in L\}$ be its complement. We use $\mathcal{L}(F)$ to denote the set of all literal sets on $F$; i.e.~all partial assignments of values to fluents. A {\em state} $s$ is a full assignment of values to fluents; $|s|=|F|$.

A {\em classical planning frame} is a tuple $\Phi=\tup{F,A}$, where $F$ is a set of fluents and $A$ is a set of \emph{actions}. Each classical planning action $a\in A$ has a precondition $\pre(a)\in\mathcal{L}(F)$, a set of effects $\eff(a)\in\mathcal{L}(F)$, and a positive action cost $c(a)$. The semantics of actions $a\in A$ is specified with two functions: $\rho(s,a)$ denotes whether action $a$ is {\em applicable} in a state $s$ and $\theta(s,a)$ denotes the {\em successor state} that results of applying action $a$ in a state $s$. Then, $\rho(s,a)$ holds iff $\pre(a)\subseteq s$, i.e.~if its precondition holds in $s$. The result of executing an applicable action $a\in A$ in a state $s$ is a new state $\theta(s,a)=(s\setminus \neg\eff(a))\cup\eff(a)$. Subtracting the complement of $\eff(a)$ from $s$ ensures that $\theta(s,a)$ remains a well-defined state. The subset of action effects that assign a positive value to a state fluent is called {\em positive effects} and denoted by $\eff^+(a)\in \eff(a)$ while $\eff^-(a)\in \eff(a)$ denotes the {\em negative effects} of an action $a\in A$.

A {\em classical planning problem} is a tuple $P=\tup{F,A,I,G}$, where $I$ is the initial state and $G\in\mathcal{L}(F)$ is the set of goal conditions over the state variables. A {\em plan} $\pi$ is an action sequence $\pi=\tup{a_1, \ldots, a_n}$, with $|\pi|=n$ denoting its {\em plan length}. The execution of $\pi$ on the initial state $I$ of $P$ induces a {\em trajectory} $\tau(\pi,s_0)=\tup{s_0, a_1, s_1, \ldots, a_n, s_n}$ such that $s_0=I$ and, for each {\small $1\leq i\leq n$}, it holds $\rho(s_{i-1},a_i)$ and $s_i=\theta(s_{i-1},a_i)$. A plan $\pi$ solves $P$ iff the induced {\em trajectory} $\tau(\pi,s_0)$ reaches a final state $G \subseteq s_n$, where all goal conditions are met. A solution plan is {\em optimal} iff it minimizes the sum of action costs.

An action $a_c\in A$ with conditional effects is defined as a set of preconditions $\pre(a_c)\in\mathcal{L}(F)$ and a set of {\em conditional effects} $\cond(a_c)$. Each conditional effect $C\rhd E\in\cond(a_c)$ is composed of two sets of literals: $C\in\mathcal{L}(F)$, the {\em condition}, and $E\in\mathcal{L}(F)$, the {\em effect}. An action $a_c$ is applicable in a state $s$ if $\rho(s,a_c)$ is true, and the result of applying action $a_c$ in state $s$ is $\theta(s,a_c)=\{s\setminus\neg\eff_c(s,a)\cup\eff_c(s,a)\}$ where $\eff_c(s,a)$ are the {\em triggered effects} resulting from the action application (conditional effects whose conditions hold in $s$):
\[
\eff_c(s,a)=\bigcup_{C\rhd E\in\cond(a_c),C\subseteq s} E,
\]

\subsection{The observation model}
Given a classical planning problem $P=\tup{F,A,I,G}$, a plan $\pi$ and a trajectory $\tau(\pi,s_0)$, we define the \emph{observation of the trajectory} as a sequence of partial states that results from observing the execution of $\pi$ on $I$. Formally, $\mathcal{O}(\tau)=\tup{s_0^o,s_1^o \ldots , s_m^o}$ where $s_0^o=I$.

A {\em partial state} $s_i^o$, {\small $0<i<m$}, is one in which $|s_i^o| < |F|$; i.e., a state in which at least a fluent of $F$ is not observable. Note that this definition also comprises the case $|s_i^o| = 0$, when the state is fully unobservable. Whatever the sequence of observed states of $\mathcal{O}(\tau)$ is, it must be {\em consistent} with the sequence of states of $\tau(\pi,s_0)$, meaning that $\forall i, s_i^o \subseteq s_i$. In practice, the number of observed states $m$, ranges from 1 (the initial state, at least), to $|\pi|+1$, and the observed intermediate states will comprise a number of fluents between $[1,|F|]$.

\newpage

We are assuming then that there is a {\em bijective monotone mapping} between trajectories and observations~\cite{ramirez2009plan}, thus also granting the inverse consistency relationship (the trajectory is a superset of the observation). Therefore, transiting between two consecutive observed states in $\mathcal{O}(\tau)$ may require the execution of more than a single action ($\theta(s_i^o,\tup{a_1,\ldots,a_k})=s_{i+1}^o$, where ${\small k\geq 1}$ is unknown but finite. In other words, having $\mathcal{O}(\tau)$ does not imply knowing the actual length of $\pi$.

\begin{definition}[Plan explanation]
Given a {\em classical planning problem} $P$ and an observation $\mathcal{O}(\tau)$, a plan $\pi$ {\em explains} $\mathcal{O}(\tau)$ (denoted $\pi\mapsto\mathcal{O}(\tau)$) iff $\pi$ is a solution for $P$ that is {\em consistent} with the state trajectory constraints imposed by the sequence of partial states $\mathcal{O}(\tau)$.  
\end{definition}
If $\pi$ is also optimal, we say that $\pi$ is the {\em best explanation} for the input observation $\mathcal{O}(\tau)$.


\subsection{Conformant planning}
{\em Conformant planning} is planning with incomplete information about the initial state, no sensing, and validating that goals are achieved with certainty (despite the uncertainty of the initial state)~\cite{goldman1996expressive,smith1998conformant,bonet2000planning}.

Syntactically, conformant planning problems are expressed in compact form through a set of state variables. A {\em conformant planning problem} can be defined as a tuple $P_c=\tup{F,A,\Upsilon,G}$ where $F$, $A$ and $G$ are the set of {\em fluents}, {\em actions} and {\em goals} (as previously defined for {\em classical planning}). Now $\Upsilon$ is a set of clauses over literals $l=f$ or $l=\neg f$ (for $f\in F$) that define the set of possible initial states. 

A solution to a conformant planning problem is an action sequence that maps each possible initial state into a goal state. More precisely, an action sequence $\pi=\tup{a_1, \ldots, a_n}$ is a {\em conformant plan} for $P_c$ iff, for each possible {\em trajectory} $\tau(\pi,s_0)=\tup{s_0, a_1, s_1, \ldots, a_n, s_n}$ s.t. $s_0$ is a valuation of the fluents in $F$ that satisfies $\Upsilon$, then $\tau(\pi,s_0)$ reaches a final state $G \subseteq s_n$. 


\section{Computing the {\em least-commitment} action model from observations}
First, this section formalizes the notion of the {\em least-commitment} action model that is able to {\em explain} an observation. Next, the section describes our approach to compute such model. 

\subsection{The {\em least-commitment} action model}
The task of computing the {\em least-commitment} action model from a sequence of state observations is defined as $\tup{\Phi,\mathcal{O}(\tau)}$:
\begin{itemize}
\item $\Phi=\tup{F,A[\cdot]}$ is a {\em classical planning frame} where the semantics of each action $a\in A[\cdot]$ is unknown; i.e. the corresponding $\tup{\rho,\theta}$ functions are undefined. We say that an {\em action model} $\mathcal{M}$ is a definition of the $\tup{\rho,\theta}$ functions of every action in $A[\cdot]$. 
\item $\mathcal{O}(\tau)$ is a sequence of partial states that results from the partial observation of a trajectory $\tau(\pi,s_0)$ within the {\em classical planning frame} $\Phi$.
\end{itemize}

Given a {\em classical planning frame} $\Phi=\tup{F,A[\cdot]}$ and an observation $\mathcal{O}(\tau)=\tup{s_0^o,s_1^o \ldots , s_m^o}$ let $P_\mathcal{O}$ be the classical planning problem $P_\mathcal{O}=\tup{F,A[\cdot],s_0^o,s_m^o}$.
\begin{definition}[Model explanation]
A model $\mathcal{M}$ {\em explains} an observation $\mathcal{O}(\tau)$ iff, when the $\tup{\rho,\theta}$ functions of the actions in $P_\mathcal{O}$ are given by $\mathcal{M}$, there exists a solution plan for $P_\mathcal{O}$ that is the {\em best explanation} for $\mathcal{O}(\tau)$.  
\end{definition}

\begin{definition}[The {\em least-commitment} action model]
Given a $\tup{\Phi,\mathcal{O}(\tau)}$ task (and let $M$ be the set of action models that represents the full space of possible action models for the actions in $A[\cdot]\in \Phi$), the {\em least-commitment} action model is the largest subset of models $M^*\subseteq M$ such that every model $\mathcal{M}\in M^*$ {\em explains} the input observation.
\end{definition}

This work focuses on the particular task of computing the {\em least-commitment} action model when action models are specified as \strips\ action schemata. 

\subsection{The space of \strips\ action models}
{\em A \strips\ action schema} $\xi$ is defined by four lists: A list of {\em parameters} $pars(\xi)$, and three list of predicates (namely $pre(\xi)$, $del(\xi)$ and $add(\xi)$) that shape the kind of fluents that can appear in the {\em preconditions}, {\em negative effects} and {\em positive effects} of the actions induced from that schema. Let be $\Psi$ the set of {\em predicates} that shape the propositional state variables $F$, and a list of {\em parameters} $pars(\xi)$. The set of elements that can appear in $pre(\xi)$, $del(\xi)$ and $add(\xi)$ of the \strips\ action schema $\xi$ is given by FOL interpretations of $\Psi$ over the parameters $pars(\xi)$. We denote this set of FOL interpretations as ${\mathcal I}_{\Psi,\xi}$. For instance, in the {\em blocksworld} the ${\mathcal I}_{\Psi,\xi}$ set contain eleven elements for the {\small \tt stack($v_1$,$v_2$)} schemata, ${\mathcal I}_{\Psi,stack}$={\small\tt\{handempty, holding($v_1$), holding($v_2$), clear($v_1$), clear($v_2$), ontable($v_1$), ontable($v_2$), on($v_1,v_1$), on($v_1,v_2$), on($v_2,v_1$), on($v_2,v_2$)\}}. 

Despite any element of ${\mathcal I}_{\Psi,\xi}$ can {\em a priori} appear in the $pre(\xi)$, $del(\xi)$ and $add(\xi)$ of schema $\xi$, the space of possible \strips\ schemata is bounded by constraints of three kinds:
\begin{enumerate}
\item {\em Syntactic constraints}. \strips\ constraints require $del(\xi)\subseteq pre(\xi)$, $del(\xi)\cap add(\xi)=\emptyset$ and $pre(\xi)\cap add(\xi)=\emptyset$. Considering exclusively these syntactic constraints, the size of the space of possible \strips\ schemata is given by $2^{2\times|{\mathcal I}_{\Psi,\xi}|}$. {\em Typing constraints} are also of this kind~\cite{mcdermott1998pddl}. 
\item {\em Domain-specific constraints}. One can introduce domain-specific knowledge to constrain further the space of possible schemata. For instance, in the {\em blocksworld} one can argue that {\small\tt on($v_1$,$v_1$)} and {\small\tt on($v_2$,$v_2$)} will not appear in the $pre(\xi)$, $del(\xi)$ and $add(\xi)$ lists of an action schema $\xi$ because, in this specific domain, a block cannot be on top of itself. {\it State invariants} are also constraints of this kind~\cite{fox1998automatic}. 
\item {\em Observation constraints}. An observations $\mathcal{O}(\tau)$ depicts {\em semantic knowledge} that constraints further the space of possible action schemata.   
\end{enumerate}

\begin{figure}
  \begin{tiny}  
  \begin{verbatim}
(:action stack
   :parameters (?v1 ?v2)
   :precondition (and (holding ?v1) (clear ?v2))
   :effect (and (not (holding ?v1)) (not (clear ?v2))
                (clear ?v1) (handempty) (on ?v1 ?v2)))


(pre_holding_v1_stack) (pre_clear_v2_stack)
(eff_holding_v1_stack) (eff_clear_v2_stack)
(eff_clear_v1_stack) (eff_handempty_stack) (eff_on_v1_v2_stack)
  \end{verbatim}           
  \end{tiny}  
 \caption{\small PDDL encoding of the {\tt\small stack(?v1,?v2)} schema and our propositional representation for this same schema.}
\label{fig:propositional}
\end{figure}

In this work we introduce a propositional encoding of the {\em preconditions}, {\em negative}, and {\em positive} effects of a \strips\ action schema $\xi$ using only fluents of two kinds {\tt\small pre\_e\_$\xi$} and {\tt\small eff\_e\_$\xi$} (where $e\in{\mathcal I}_{\Psi,\xi}$). This encoding exploits the syntactic constraints of \strips\ so is more compact that the one previuosly proposed by~\citeauthor{aineto2018learning}~\citeyear{aineto2018learning}. In more detail, if {\tt\small pre\_e\_$\xi$} and {\tt\small eff\_e\_$\xi$} holds it means that $e\in{\mathcal I}_{\Psi,\xi}$ is a negative effect in $\xi$ while if $pre\_e\_\xi$ does not hold but {\tt\small eff\_e\_$\xi$} holds, it means that $e\in{\mathcal I}_{\Psi,\xi}$ is a positive effect in $\xi$. Figure~\ref{fig:propositional} shows the PDDL encoding of the {\tt\small stack(?v1,?v2)} schema and our propositional representation for this same schema. 

\subsection{{\em Partially specified} \strips\ action models}
A set of action models can be defined {\em explicitely}, by enumerating all the models that belong to the set or {\em implictely}, by enumerating the constraints that must satisfy any model that belongs to the set. Inspired by the notion of {\em incomplete (annotated) model}~\cite{sreedharan2018handling}, we introduce here {\em partially specified} \strips\ action models, a formalism for the {\em implicit} representation of a set of \strips\ schema.

We show now that extending to the {\em knwoledge level} our propositional encoding of \strips\ action schemes, we can compactly represent a set of \strips\ schema. The extension defines, for each proposition {\tt\small $pre\_e\_\xi$}, with two propositions {\tt\small $K\pre\_e\_\xi$} and {\tt\small $K\neg\pre\_e\_\xi$}, meaning that is {\em known} that $e\in{\mathcal I}_{\Psi,\xi}$ is a precondition of $\xi$ and that is {\em known} that $e\in{\mathcal I}_{\Psi,\xi}$ is not a precondition of $\xi$. Likewise for each {\tt\small $eff\_e\_\xi$} proposition we define these two, {\tt\small $K\eff\_e\_\xi$} and {\tt\small $K\neg\eff\_e\_\xi$}, meaning that is {\em known} that $e\in{\mathcal I}_{\Psi,\xi}$ is an effect of $\xi$ and that is {\em known} that $e\in{\mathcal I}_{\Psi,\xi}$ is not an effect of $\xi$. With respect to this formalization, the {\em full space} of possible \strips\ schemas for $\xi$ is compactly represented as {\tt\small $\bigcap_{e\in{\mathcal I}_{\Psi,\xi}} \neg K\pre\_e\_\xi\wedge \neg K\neg\pre\_e\_\xi \wedge\neg K\eff\_e\_\xi\wedge \neg K\neg\eff\_e\_\xi$}.

Figure~\ref{fig:partial} shows a {\em partially specified} \strips\ model that represents of a set of action models for the {\tt\small stack(?v1,?v2)} schema. This {\em partially specified} \strips\ model represents a set of $2^5$ models since (1), the actual five effects of the action are {\em unkwnon}, (2) preconditions are known to appear as in the actual definition of the {\tt\small stack(?v1,?v2)} schema for the blocksworld and (3), it is {\em known} that the preconditions and effects that are not part of the actual definition of the {\tt\small stack(?v1,?v2)} schema are not part of this {\em partially specified} \strips\ model.

\begin{figure}
  \begin{tiny}  
 \begin{lstlisting}
(K$\pre$_holding_v1_stack) (K$\pre$_clear_v2_stack)
(K$\neg\pre$_holding_v2_stack) (K$\neg\pre$_clear_v1_stack)
(K$\neg\pre$_handempty_stack) 
(K$\neg\pre$_ontable_v1_stack) (K$\neg\pre$_ontable_v2_stack)
(K$\neg\pre$_on_v1_v2_stack) (K$\neg\pre$_on_v2_v1_stack)
(K$\neg\pre$_on_v1_v1_stack) (K$\neg\pre$_on_v2_v2_stack) 

(K$\neg\eff$_holding_v2_stack) (K$\neg\eff$_clear_v2_stack)
(K$\neg\eff$_clear_v2_stack) (K$\neg\eff$_on_v2_v2_stack)
(K$\neg\eff$_on_v1_v1_stack) (K$\neg\eff$_on_v2_v1_stack)
  \end{lstlisting}           
  \end{tiny}  
 \caption{\small {\em Partially specified} \strips\ model that represents of a set of action models for the {\tt\small stack(?v1,?v2)} schema.}
\label{fig:partial}
\end{figure}

\begin{definition}[Partially specified model explanation]
A partially specified action schema $\xi[\cdot]$ (that defines the set of action models $M_\xi$) {\em explains} an observation $\mathcal{O}(\tau)$, iff every model $\mathcal{M}\in M_{\xi}$ {\em explains} the input observation.
\end{definition}


\subsection{Computing the {\em least-commitment} model via classical planning}
Inspired by the {\em classical planning compilation} $K_0$ for conformant planning~\cite{palacios-conformant-JAIR09}, this section shows that we can build a {\em classical planning problem} $P=\tup{F',A',I',G'}$ whose solution induces the {\em least-commitment} action model for an observation $\mathcal{O}(\tau)$: 
\begin{itemize}
\item The set of fluents $F'$ extends $F$ with two new sets of fluents:
      \begin{itemize}
      \item $\{test_j\}_{1\leq j\leq m}$, indicating the state observation $s_j\in\mathcal{O}(\tau)$ where the action model is validated
      \item The {\em knowledge level} fluents {\tt\small $K\pre\_e\_\xi$}, {\tt\small $K\neg\pre\_e\_\xi$}, {\tt\small $K\eff\_e\_\xi$} and {\tt\small $K\neg\eff\_e\_\xi$} encoding the space of possible {\em partially specified} action models.
      \end {itemize}
\item The set of actions $A'$ contains now actions of three different kinds:
\begin{itemize}
      \item Actions for {\em committing} {\tt\small pre\_e\_$\xi$} to a positive/negative value. Similar actions are also defined for {\em committing} {\tt\small eff\_e\_$\xi$} to a positive/negative value but the value of {\tt\small eff\_e\_$\xi$} can only be committed once the value of the corresponding {\tt\small pre\_e\_$\xi$} is committed (i.e. once either $\mathsf{K\pre\_e\_\xi}$ or $\mathsf{K\neg\pre\_e\_\xi}$ holds in the current state).
\begin{small}
\begin{align*}
\hspace*{7pt}\pre(\mathsf{commit\top\_pre\_e\_\xi})=&\{mode_{commit}, \\
&\mathsf{\neg K\pre\_e\_\xi}, \mathsf{\neg K\neg\pre\_e\_\xi}\},\\
\cond(\mathsf{commit\top\_pre\_e\_\xi})=&\{\emptyset\}\rhd\{\mathsf{K\pre\_e\_\xi}\}.\\\\
\hspace*{7pt}\pre(\mathsf{commit\bot\_pre\_e\_\xi})=&\{mode_{commit}, \\
&\mathsf{\neg K\pre\_e\_\xi},\mathsf{\neg K\neg\pre\_e\_\xi}\},\\
\cond(\mathsf{commit\bot\_pre\_e\_\xi})=&\{\emptyset\}\rhd\{\mathsf{K\neg\pre\_e\_\xi}\}.
\end{align*}
\end{small}

      \item Actions for {\em validating} that committed models explain the $s_j$ observed states, {\tt\small $0\leq j< m$}.
\begin{small}
\begin{align*}
\hspace*{7pt}\pre(\mathsf{validate_{j}})=&s_j\cup\{test_{j-1}\},\\
\cond(\mathsf{validate_{j}})=&\{\emptyset\}\rhd\{\neg test_{j-1}, test_j\},\\
                            &\{mode_{commit}\}\rhd\{\neg mode_{commit}, mode_{val}\}.
\end{align*}
\end{small}

      \item {\em Editable} actions whose semantics is given by the value of the {\em knowledge level} fluents ({\tt\small $K\pre\_e\_\xi$}, {\tt\small $K\neg\pre\_e\_\xi$}, {\tt\small $K\eff\_e\_\xi$} and {\tt\small $K\neg\eff\_e\_\xi$}) at the current state. Figure~\ref{fig:editable} shows the PDDL encoding of an {\em editable} {\tt\small stack(?v1,?v2)} schema. This editable schema behaves exactly as the original PDDL schema defined in Figure~\ref{fig:propositional} when the set of fluents {\tiny\tt (Kpre\_holding\_v1\_stack) (Kpre\_clear\_v2\_stack) (Keff\_holding\_v1\_stack) (Keff\_clear\_v2\_stack) (Keff\_clear\_v1\_stack) (Keff\_handempty\_stack) (Keff\_on\_v1\_v2\_stack)} hold at the current state as well as the remaining {\tt\small $K\neg\pre\_e\_\xi$} and {\tt\small $K\neg\eff\_e\_\xi$} for all the preconditions and effects that are not part of the actual {\tt\small stack(?v1,?v2)} schema. Formally, given an operator schema $\xi\in\mathcal{M}$ its {\em editable} version is:
\begin{small}  
\begin{align*}
\hspace*{7pt}\pre(\mathsf{editable_{\xi}})=&\{\neg K\neg\pre\_e\_\xi\implies e\}_{\forall e\in{\mathcal I}_{\Psi,\xi}}\\
\cond(\mathsf{editable_{\xi}})=&\{K\pre\_e\_\xi, \neg K\neg\eff\_e\_\xi\}\rhd\{\neg e\}_{\forall e\in{\mathcal I}_{\Psi,\xi}},\\
&\{K\neg\pre\_e\_\xi, \neg K\neg\eff\_e\_\xi\}\rhd\{e\}_{\forall e\in{\mathcal I}_{\Psi,\xi}}.
\end{align*}
\end{small}

\end{itemize}

\item The new initial state $I'=I \cup\{mode_{commit}\}$ while the new goals are $G'=s_m\cup\{test_m\}$.
\end{itemize}


\begin{figure}
  \begin{tiny}  
  \begin{verbatim}
(:action stack
 :parameters (?o1 - object ?o2 - object)
 :precondition
   (and (or (Knotpre_on_v1_v1_stack) (on ?o1 ?o1))
        (or (Knotpre_on_v1_v2_stack) (on ?o1 ?o2))
        (or (Knotpre_on_v2_v1_stack) (on ?o2 ?o1))
        (or (Knotpre_on_v2_v2_stack) (on ?o2 ?o2))
        (or (Knotpre_ontable_v1_stack) (ontable ?o1))
        (or (Knotpre_ontable_v2_stack) (ontable ?o2))
        (or (Knotpre_clear_v1_stack) (clear ?o1))
        (or (Knotpre_clear_v2_stack) (clear ?o2))
        (or (Knotpre_holding_v1_stack) (holding ?o1))
        (or (Knotpre_holding_v2_stack) (holding ?o2))
        (or (Knotpre_handempty_stack) (handempty)))
 :effect (and
   (when (and (Kpre_on_v1_v1_stack)(not(Knot_eff_on_v1_v1_stack))) (not (on ?o1 ?o1)))
   (when (and (Kpre_on_v1_v2_stack)(not(Knot_eff_on_v1_v2_stack))) (not (on ?o1 ?o2)))
   (when (and (Kpre_on_v2_v1_stack)(not(Knot_eff_on_v2_v1_stack))) (not (on ?o2 ?o1)))
   (when (and (Kpre_on_v2_v2_stack)(not(Knot_eff_on_v2_v2_stack))) (not (on ?o2 ?o2)))
   (when (and (Kpre_ontable_v1_stack)(not(Knot_eff_ontable_v1_stack))) (not (ontable ?o1)))
   (when (and (Kpre_ontable_v2_stack)(not(Knot_eff_ontable_v2_stack))) (not (ontable ?o2)))
   (when (and (Kpre_clear_v1_stack)(not(Knot_eff_clear_v1_stack))) (not (clear ?o1)))
   (when (and (Kpre_clear_v2_stack)(not(Knot_eff_clear_v2_stack))) (not (clear ?o2)))
   (when (and (Kpre_holding_v1_stack)(not(Knot_eff_holding_v1_stack))) (not (holding ?o1)))
   (when (and (Kpre_holding_v2_stack)(not(Knot_eff_holding_v2_stack))) (not (holding ?o2)))
   (when (and (Kpre_handempty_stack)(not(Knot_eff_handempty_stack))) (not (handempty)))
   (when (and (Knot_pre_on_v1_v1_stack)(not(Knot_eff_on_v1_v1_stack))) (on ?o1 ?o1))
   (when (and (Knot_pre_on_v1_v2_stack)(not(Knot_eff_on_v1_v2_stack))) (on ?o1 ?o2))
   (when (and (Knot_pre_on_v2_v1_stack)(not(Knot_eff_on_v2_v1_stack))) (on ?o2 ?o1))
   (when (and (Knot_pre_on_v2_v2_stack)(not(Knot_eff_on_v2_v2_stack))) (on ?o2 ?o2))
   (when (and (Knot_pre_ontable_v1_stack)(not(Knot_eff_ontable_v1_stack))) (ontable ?o1))
   (when (and (Knot_pre_ontable_v2_stack)(not(Knot_eff_ontable_v2_stack))) (ontable ?o2))
   (when (and (Knot_pre_clear_v1_stack)(not(Knot_eff_clear_v1_stack))) (clear ?o1))
   (when (and (Knot_pre_clear_v2_stack)(not(Knot_eff_clear_v2_stack))) (clear ?o2))
   (when (and (Knot_pre_holding_v1_stack)(not(Knot_eff_holding_v1_stack))) (holding ?o1))
   (when (and (Knot_pre_holding_v2_stack)(not(Knot_eff_holding_v2_stack))) (holding ?o2))
   (when (and (Knot_pre_handempty_stack)(not(Knot_eff_handempty_stack))) (handempty))))
  \end{verbatim}           
  \end{tiny}  
 \caption{\small PDDL encoding of the editable version of the {\tt\small stack(?v1,?v2)} schema.}
\label{fig:editable}
\end{figure}


\subsection{Compilation properties}



\section{Evaluation}



\section{Conclusions}
Related work~\cite{SternJ17}.


%% The file named.bst is a bibliography style file for BibTeX 0.99c
\bibliographystyle{named}
\bibliography{planlearnbibliography}

\end{document}

