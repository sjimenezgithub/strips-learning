%%%% ijcai19.tex

\typeout{IJCAI-19 Instructions for Authors}

% These are the instructions for authors for IJCAI-19.

\documentclass{article}
\pdfpagewidth=8.5in
\pdfpageheight=11in
% The file ijcai19.sty is NOT the same than previous years'
\usepackage{ijcai19}

% Use the postscript times font!
\usepackage{times}
\usepackage{soul}
\usepackage{url}
\usepackage[hidelinks]{hyperref}
\usepackage[utf8]{inputenc}
\usepackage[small]{caption}
\usepackage{graphicx}
\usepackage{amsmath}
\usepackage{booktabs}
\usepackage{algorithm}
\usepackage{algorithmic}
\urlstyle{same}

%%%%%%%%%%%%%%%%%% Added for this paper
\usepackage{listings}% http://ctan.org/pkg/listings
\lstset{
  basicstyle=\ttfamily,
  mathescape
}
\usepackage{ wasysym }
\newcommand{\tup}[1]{{\langle #1 \rangle}}
\newcommand{\pre}{\mathsf{pre}}     % precondition
\newcommand{\del}{\mathsf{del}}     % effect
\newcommand{\add}{\mathsf{add}}     % effect
\newcommand{\eff}{\mathsf{eff}}     % effect
\newcommand{\cond}{\mathsf{cond}}   % conditional effect
\newcommand{\true}{\mathsf{true}}   % true
\newcommand{\false}{\mathsf{false}} % false
\newcommand{\PE}{\mathrm{PE}}     % precondition
\newcommand{\strips}{\textsc{Strips}}

\newtheorem{theorem}{Theorem}
\newtheorem{lemma}[theorem]{Lemma}
\newtheorem{definition}[theorem]{Definition}

%%%%%%%%%%%%%





% the following package is optional:
%\usepackage{latexsym} 

% Following comment is from ijcai97-submit.tex:
% The preparation of these files was supported by Schlumberger Palo Alto
% Research, AT\&T Bell Laboratories, and Morgan Kaufmann Publishers.
% Shirley Jowell, of Morgan Kaufmann Publishers, and Peter F.
% Patel-Schneider, of AT\&T Bell Laboratories collaborated on their
% preparation.

% These instructions can be modified and used in other conferences as long
% as credit to the authors and supporting agencies is retained, this notice
% is not changed, and further modification or reuse is not restricted.
% Neither Shirley Jowell nor Peter F. Patel-Schneider can be listed as
% contacts for providing assistance without their prior permission.

% To use for other conferences, change references to files and the
% conference appropriate and use other authors, contacts, publishers, and
% organizations.
% Also change the deadline and address for returning papers and the length and
% page charge instructions.
% Put where the files are available in the appropriate places.

\title{Goal Recognition as Planning with Unknown Domain Models}

% Single author syntax
%\author{
%    Sarit Kraus
%    \affiliations
%    Department of Computer Science, Bar-Ilan University, Israel \emails
%    pcchair@ijcai19.org
%}

% Multiple author syntax (remove the single-author syntax above and the \iffalse ... \fi here)
% Check the ijcai19-multiauthor.tex file for detailed instructions
\author{
Diego Aineto$^1$\and
Sergio Jim\'enez$^1$\and
Eva Onaindia$^1$\And
\and
Miquel Ram\'irez$^2$
\affiliations
$^1${\small Departamento de Sistemas Inform\'aticos y Computaci\'on. Universitat Polit\`ecnica de Val\`encia. Valencia, Spain}\\
$^2${\small School of Computing and Information Systems. The University of Melbourne. Melbourne, Victoria. Australia}
\emails
{\scriptsize \{dieaigar,serjice,onaindia\}@dsic.upv.es, miquel.ramirez@unimelb.edu.au}}


\begin{document}
\maketitle

\begin{abstract}
The paper shows how to relax one key assumption of the {\em plan recognition as planning} approach for {\em goal recognition} that is knowing the action model of the observed agent. The paper introduces a novel formulation that fits together the {\em learning of planning action models} with {\em plan recognition as planning}. The empirical evaluation evidences that this novel formulation allows to solve standard goal recognition benchmarks without {\em a priori} knowing the action model of the observed agent.  
\end{abstract}

\section{Introduction}
\label{sec:introduction}
{\em Goal recognition} is a particular classification task in which each class represents a different goal and each example is an observation of an agent acting to achieve one of that goals. Despite there exists a wide range of diffenrent approaches for {\em goal recognition}, {\em plan recognition as planning}~\cite{ramirez2009plan,ramirez2012plan} is one of the most appealing since it is at the core of various activity recognition tasks such as, {\em goal recognition design}~\cite{KerenGK14}, {\em deceptive planning}~\cite{masters2017deceptive}, {\em planning for transparency}~\cite{macnally2018action} or {\em counter-planning}~\cite{PozancoEFB18}.

{\em Plan recognition as planning} leverages the action model of the observed agent and an off-the-shelf classical planner to compute the most likely goal of that agent. In this paper we show that we can relax the key assumption of the {\em plan recognition as planning} approach for {\em goal recognition} that is having an action model of the observed agent. In particular, the paper introduces a novel formulation that fits together the {\em learning of planning action models} with the {\em plan recognition as planning} approach. The evaluation of our formulation evidences that it allows to solve goal recognition tasks, even when the action model of the observed is unknown, using an off-the-shelf classical planner.



\section{Background}
\label{sec:background}
This section formalizes the {\em planning model} we follow as well as the kind of {\em observations} that are given as input to the {\em goal recognition} task.  

\subsection{Classical planning with conditional effects}
Let $F$ be the set of  propositional state variables ({\em fluents}) describing a state. A {\em literal} $l$ is a valuation of a fluent $f\in F$; i.e. either~$l=f$ or $l=\neg f$. A set of literals $L$ represents a partial assignment of values to fluents (without loss of generality, we will assume that $L$ does not contain conflicting values). Given $L$, let $\neg L=\{\neg l:l\in L\}$ be its complement. We use $\mathcal{L}(F)$ to denote the set of all literal sets on $F$; i.e.~all partial assignments of values to fluents. A {\em state} $s$ is a full assignment of values to fluents; $|s|=|F|$.

A {\em classical planning frame} is a tuple $\Phi=\tup{F,A}$, where $F$ is a set of fluents and $A$ is a set of \emph{actions}. Each classical planning action $a\in A$ has a precondition $\pre(a)\in\mathcal{L}(F)$, a set of effects $\eff(a)\in\mathcal{L}(F)$, and a positive action cost $cost(a)$. The semantics of actions $a\in A$ is specified with two functions: $\rho(s,a)$ denotes whether action $a$ is {\em applicable} in a state $s$ and $\theta(s,a)$ denotes the {\em successor state} that results of applying action $a$ in a state $s$. Then, $\rho(s,a)$ holds iff $\pre(a)\subseteq s$, i.e.~if its precondition holds in $s$. The result of executing an applicable action $a\in A$ in a state $s$ is a new state $\theta(s,a)=(s\setminus \neg\eff(a))\cup\eff(a)$. Subtracting the complement of $\eff(a)$ from $s$ ensures that $\theta(s,a)$ remains a well-defined state. The subset of action effects that assign a positive value to a state fluent is called {\em positive effects} and denoted by $\eff^+(a)\in \eff(a)$ while $\eff^-(a)\in \eff(a)$ denotes the {\em negative effects} of an action $a\in A$.

A {\em classical planning problem} is a tuple $P=\tup{F,A,I,G}$, where $I$ is the initial state and $G\in\mathcal{L}(F)$ is the set of goal conditions over the state variables. A {\em plan} $\pi$ is an action sequence $\pi=\tup{a_1, \ldots, a_n}$, with $|\pi|=n$ denoting its {\em plan length} and $cost(\pi)=\sum_{a\in\pi} cost(a)$ its {\em plan cost}. The execution of $\pi$ on the initial state $I$ of $P$ induces a {\em trajectory} $\tau(\pi,s_0)=\tup{s_0, a_1, s_1, \ldots, a_n, s_n}$ such that $s_0=I$ and, for each {\small $1\leq i\leq n$}, it holds $\rho(s_{i-1},a_i)$ and $s_i=\theta(s_{i-1},a_i)$. A plan $\pi$ solves $P$ iff the induced {\em trajectory} $\tau(\pi,s_0)$ reaches a final state $G \subseteq s_n$, where all goal conditions are met. A solution plan is {\em optimal} iff its cost is minimal.

An {\em action with conditional effects} $a_c\in A$ is defined as a set of preconditions $\pre(a_c)\in\mathcal{L}(F)$ and a set of {\em conditional effects} $\cond(a_c)$. Each conditional effect $C\rhd E\in\cond(a_c)$ is composed of two sets of literals: $C\in\mathcal{L}(F)$, the {\em condition}, and $E\in\mathcal{L}(F)$, the {\em effect}. An action $a_c$ is applicable in a state $s$ if $\rho(s,a_c)$ is true, and the result of applying action $a_c$ in state $s$ is $\theta(s,a_c)=\{s\setminus\neg\eff_c(s,a)\cup\eff_c(s,a)\}$ where $\eff_c(s,a)$ are the {\em triggered effects} resulting from the action application (conditional effects whose conditions hold in $s$):
\[
\eff_c(s,a)=\bigcup_{C\rhd E\in\cond(a_c),C\subseteq s} E,
\]

\subsection{The observation model}
Given a planning problem $P=\tup{F,A,I,G}$, a plan $\pi$ and a trajectory $\tau(\pi,P)$, we define the \emph{observation of the trajectory} as an interleaved combination of actions and states that represents the observation from the execution of $\pi$ in $P$. Formally, $\mathcal{O}(\tau)=\tup{s_0^o,a_1^o,s_1^o \ldots , a_l^o, s_m^o}$, $s_0^o=I$, and:

%which indicates that we observe $l$ actions, $1\leq l\leq |\pi|$, and $m$ states, $1\leq m \leq |\pi|+1$, from $\tau(\pi,P)$:

\begin{itemize}
\item The {\bf observed actions} are consistent with $\pi$, which means that $\tup{a_1^o, \ldots, a_l^o}$ is a sub-sequence of $\pi$. Specifically, the number of observed actions, $l$, can range from $0$ (fully unobservable action sequence) to $|\pi|$ (fully observable action sequence).
\item The {\bf observed states} $\tup{s_0^o, s_1^o, \ldots, s_m^o}$ is a sequence of possibly {\em partially observable states}, except for the initial state $s_0^o$, which is fully observable. A partially observable state $s_i^o$ is one in which $|s_i^o| < |F|$; i.e., a state in which at least a fluent of $F$ is not observable. Note that this definition also comprises the case $|s_i^o| = 0$, when the state is fully unobservable. Whatever the sequence of observed states of $\mathcal{O}(\tau)$ is, it must be consistent with the sequence of states of $\tau(\pi,P)$, meaning that $\forall i, s_i^o \subseteq s_i$. In practice, the number of observed states, $m$, range from 1 (the initial state, at least), to $|\pi|+1$, and the observed intermediate states will comprise a number of fluents between $[1,|F|]$.
    %Exceptionally, $s_m^o$ cannot be fully unobservable for the purpose of our task (we will elaborate on this issue later on).
\end{itemize}

We assume a bijective monotone mapping between actions/states of trajectories and observations~\cite{ramirez2009plan}, thus also granting the inverse consistency relationship (the trajectory is a superset of the observation). Therefore, transiting between two consecutive observed states in $\mathcal{O}(\tau)$ may require the execution of more than a single action ($\theta(s_i^o,\tup{a_1,\ldots,a_k})=s_{i+1}^o$, where ${\small k\geq 1}$ is unknown but finite. In other words, having $\mathcal{O}(\tau)$ does not imply knowing the actual length of $\pi$.

\subsection{Goal recognition as classical planning}
{\em Goal recognition} is a particular classification task in which each class represents a different goal $g\in G[\cdot]$ and each example is an $\mathcal{O}(\tau)$ observation of an agent acting to achieve one of the input goals in $G[\cdot]$. Following the {\em naive Bayes classifier}, the {\em solution} to the {\em goal recognition} task is the subset of goals in $G[\cdot]$ that maximizes this expression.

\begin{align}
argmax_{g\in G[\cdot]} P(\mathcal{O}|g) P(g).
\end{align}

The {\em plan recognition as planning} approach shows how to compute estimates of the $P(\mathcal{O}|g)$ likelihood leveraging the action model of the observed agent and an off-the-shelf classical planner. More precisely, given a {\em classical planning problem} $P=\tup{F,A,I,G[\cdot]}$, where $G[\cdot]$ represents the set of possible goals, then the {\em plan recognition as planning} approach estimates the $P(\mathcal{O}|g)$ by computing the cost difference between of solution plans to these two different classical planning problems:
\begin{itemize}
\item $P'_{\top}$, that constrains problem $P$ to achieve $g\in G[\cdot]$ through a plan $\pi_\top$ {\em consistent} with the input observation $\mathcal{O}(\tau)$.
\item $P'_{\bot}$, that constrains problem $P$ to achieve $g\in G[\cdot]$ through a plan $\pi_\bot$ {\em inconsistent} with $\mathcal{O}(\tau)$.
\end{itemize}

The higher the value of this cost difference $\Delta(cost(\pi_\top),cost(\pi_\bot))$, the better $g\in G[\cdot]$ predicts $\mathcal{O}(\tau)$ and hence, the higher $P(\mathcal{O}|g)$ likelihood. The function used by the {\em plan recognition as planning} approach for mapping the previous cost difference into likelihoods is the sigmoid function:

\begin{align}
P(\mathcal{O}|g) = \frac{1}{1+e^{-\beta\Delta(cost(\pi_\top),cost(\pi_\bot))}}
\end{align}

This expression is derived from the assumption that while the observed agent is not perfectly rational, he is more likely to follow cheaper plans, according to a Boltzmann distribution. The larger the value of $\beta$, the more rational the agent, and the less likely that he will follow suboptimal plans.

Recent works show that faster estimates of this $P(\mathcal{O}|g)$ likelihood can also be computed using relaxations of the $P'_{\top}$ and $P'_{\bot}$ classical planning tasks~\cite{pereira2017landmark}.



\section{Goal Recognition as Planning with Unknown Domain Models}
\label{sec:recognition}
This section shows that $cost(\pi_\top)$ and $cost(\pi_\bot)$, and hence an approximation to the $P(\mathcal{O}|g)$ likelihood, can also be computed with classical planing when the action model of the observed agent is {\em unknown}. 




\section{Evaluation}
\label{sec:evaluation}



\section{Conclusions}
\label{sec:conclusions}


%% The file named.bst is a bibliography style file for BibTeX 0.99c
\bibliographystyle{named}
\bibliography{planlearnbibliography}

\end{document}

