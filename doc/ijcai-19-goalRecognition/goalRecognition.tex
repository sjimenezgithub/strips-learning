%%%% ijcai19.tex

\typeout{IJCAI-19 Instructions for Authors}

% These are the instructions for authors for IJCAI-19.

\documentclass{article}
\pdfpagewidth=8.5in
\pdfpageheight=11in
% The file ijcai19.sty is NOT the same than previous years'
\usepackage{ijcai19}

% Use the postscript times font!
\usepackage{times}
\usepackage{soul}
\usepackage{url}
\usepackage[hidelinks]{hyperref}
\usepackage[utf8]{inputenc}
\usepackage[small]{caption}
\usepackage{graphicx}
\usepackage{amsmath}
\usepackage{booktabs}
\usepackage{algorithm}
\usepackage{algorithmic}
\urlstyle{same}

%%%%%%%%%%%%%%%%%% Added for this paper
\usepackage{listings}% http://ctan.org/pkg/listings
\lstset{
  basicstyle=\ttfamily,
  mathescape
}
\usepackage{ wasysym }
\newcommand{\tup}[1]{{\langle #1 \rangle}}
\newcommand{\pre}{\mathsf{pre}}     % precondition
\newcommand{\del}{\mathsf{del}}     % effect
\newcommand{\add}{\mathsf{add}}     % effect
\newcommand{\eff}{\mathsf{eff}}     % effect
\newcommand{\cond}{\mathsf{cond}}   % conditional effect
\newcommand{\true}{\mathsf{true}}   % true
\newcommand{\false}{\mathsf{false}} % false
\newcommand{\PE}{\mathrm{PE}}     % precondition
\newcommand{\strips}{\textsc{Strips}}

\newtheorem{theorem}{Theorem}
\newtheorem{lemma}[theorem]{Lemma}
\newtheorem{definition}[theorem]{Definition}

%%%%%%%%%%%%%





% the following package is optional:
%\usepackage{latexsym} 

% Following comment is from ijcai97-submit.tex:
% The preparation of these files was supported by Schlumberger Palo Alto
% Research, AT\&T Bell Laboratories, and Morgan Kaufmann Publishers.
% Shirley Jowell, of Morgan Kaufmann Publishers, and Peter F.
% Patel-Schneider, of AT\&T Bell Laboratories collaborated on their
% preparation.

% These instructions can be modified and used in other conferences as long
% as credit to the authors and supporting agencies is retained, this notice
% is not changed, and further modification or reuse is not restricted.
% Neither Shirley Jowell nor Peter F. Patel-Schneider can be listed as
% contacts for providing assistance without their prior permission.

% To use for other conferences, change references to files and the
% conference appropriate and use other authors, contacts, publishers, and
% organizations.
% Also change the deadline and address for returning papers and the length and
% page charge instructions.
% Put where the files are available in the appropriate places.

\title{Model-Based Goal Recognition with Unknown Domain Models}

% Single author syntax
%\author{
%    Sarit Kraus
%    \affiliations
%    Department of Computer Science, Bar-Ilan University, Israel \emails
%    pcchair@ijcai19.org
%}

% Multiple author syntax (remove the single-author syntax above and the \iffalse ... \fi here)
% Check the ijcai19-multiauthor.tex file for detailed instructions
\author{
Diego Aineto$^1$\and
Sergio Jim\'enez$^1$\and
Eva Onaindia$^1$\And
\and
Miquel Ram\'irez$^2$
\affiliations
$^1${\small Departamento de Sistemas Inform\'aticos y Computaci\'on. Universitat Polit\`ecnica de Val\`encia. Valencia, Spain}\\
$^2${\small School of Computing and Information Systems. The University of Melbourne. Melbourne, Victoria. Australia}
\emails
{\scriptsize \{dieaigar,serjice,onaindia\}@dsic.upv.es, miquel.ramirez@unimelb.edu.au}}


\begin{document}
\maketitle

\begin{abstract}
The paper shows how to relax one key assumption of the {\em plan recognition as planning} for {\em goal recognition} that is knowing the action model of the observed agent. The paper introduces a novel formulation that fits together the {\em learning of planning action models} with the {\em plan recognition as planning} approach. The empirical evaluation evidences that our novel formulation allows to solve standard goal recognition benchmarks without having knowing the action model of the observed agent.  
\end{abstract}

\section{Introduction}
\label{sec:introduction}
{\em Goal recognition} is a particular classification task in which each class represents a different goal and each example is an observation of an agent acting to achieve one of that goals. Despite there is a wide range of diffenrent approaches for {\em goal recognition}, {\em plan recognition as planning}~\cite{ramirez2009plan,ramirez2012plan} is one of the most popular since it is at the core of several interesting tasks such as, {\em goal recognition design}~\cite{KerenGK14}, {\em deceptive planning}~\cite{masters2017deceptive}, {\em planning for transparency}~\cite{macnally2018action} or {\em counter-planning}~\cite{PozancoEFB18}.

{\em Plan recognition as planning} leverages the action model of the observed agent and an off-the-shelf classical planner to compute the most likely goal of the agent. In this paper we show that we can relax the key assumption of {\em plan recognition as planning} for {\em Goal recognition} that is knowing the action model of the observed agent. In particular, the paper introduces a novel formulation that fits together the {\em learning of planning action models} with the {\em plan recognition as planning} approach. The evaluation of our formulation evidences that it allows to solve goal recognition tasks, even when the action model of the observed is unknown.





\section{Background}
\label{sec:background}
This section formalizes the {\em planning model} we follow as well as the kind of {\em observations} that are given as classification examples for the {\em goal recognition} task.  

\subsection{Classical planning with conditional effects}
Let $F$ be the set of {\em fluents} or {\em state variables} (propositional variables) describing a state. A {\em literal} $l$ is a valuation of a fluent $f\in F$; i.e. either~$l=f$ or $l=\neg f$. A set of literals $L$ represents a partial assignment of values to fluents (without loss of generality, we will assume that $L$ does not contain conflicting values). Given $L$, let $\neg L=\{\neg l:l\in L\}$ be its complement. We use $\mathcal{L}(F)$ to denote the set of all literal sets on $F$; i.e.~all partial assignments of values to fluents. A {\em state} $s$ is a full assignment of values to fluents; $|s|=|F|$.

A {\em classical planning frame} is a tuple $\Phi=\tup{F,A}$, where $F$ is a set of fluents and $A$ is a set of \emph{actions}. Each classical planning action $a\in A$ has a precondition $\pre(a)\in\mathcal{L}(F)$, a set of effects $\eff(a)\in\mathcal{L}(F)$, and a positive action cost $cost(a)$. The semantics of actions $a\in A$ is specified with two functions: $\rho(s,a)$ denotes whether action $a$ is {\em applicable} in a state $s$ and $\theta(s,a)$ denotes the {\em successor state} that results of applying action $a$ in a state $s$. Then, $\rho(s,a)$ holds iff $\pre(a)\subseteq s$, i.e.~if its precondition holds in $s$. The result of executing an applicable action $a\in A$ in a state $s$ is a new state $\theta(s,a)=(s\setminus \neg\eff(a))\cup\eff(a)$. Subtracting the complement of $\eff(a)$ from $s$ ensures that $\theta(s,a)$ remains a well-defined state. The subset of action effects that assign a positive value to a state fluent is called {\em positive effects} and denoted by $\eff^+(a)\in \eff(a)$ while $\eff^-(a)\in \eff(a)$ denotes the {\em negative effects} of an action $a\in A$.

A {\em classical planning problem} is a tuple $P=\tup{F,A,I,G}$, where $I$ is the initial state and $G\in\mathcal{L}(F)$ is the set of goal conditions over the state variables. A {\em plan} $\pi$ is an action sequence $\pi=\tup{a_1, \ldots, a_n}$, with $|\pi|=n$ denoting its {\em plan length} and $cost(\pi)=\sum_{a\in\pi} cost(a)$ its {\em plan cost}. The execution of $\pi$ on the initial state $I$ of $P$ induces a {\em trajectory} $\tau(\pi,s_0)=\tup{s_0, a_1, s_1, \ldots, a_n, s_n}$ such that $s_0=I$ and, for each {\small $1\leq i\leq n$}, it holds $\rho(s_{i-1},a_i)$ and $s_i=\theta(s_{i-1},a_i)$. A plan $\pi$ solves $P$ iff the induced {\em trajectory} $\tau(\pi,s_0)$ reaches a final state $G \subseteq s_n$, where all goal conditions are met. A solution plan is {\em optimal} iff it minimizes the sum of action costs.

An {\em action with conditional effects} $a_c\in A$ is defined as a set of preconditions $\pre(a_c)\in\mathcal{L}(F)$ and a set of {\em conditional effects} $\cond(a_c)$. Each conditional effect $C\rhd E\in\cond(a_c)$ is composed of two sets of literals: $C\in\mathcal{L}(F)$, the {\em condition}, and $E\in\mathcal{L}(F)$, the {\em effect}. An action $a_c$ is applicable in a state $s$ if $\rho(s,a_c)$ is true, and the result of applying action $a_c$ in state $s$ is $\theta(s,a_c)=\{s\setminus\neg\eff_c(s,a)\cup\eff_c(s,a)\}$ where $\eff_c(s,a)$ are the {\em triggered effects} resulting from the action application (conditional effects whose conditions hold in $s$):
\[
\eff_c(s,a)=\bigcup_{C\rhd E\in\cond(a_c),C\subseteq s} E,
\]

\subsection{The observation model}
Given a planning problem $P=\tup{F,A,I,G}$, a plan $\pi$ and a trajectory $\tau(\pi,P)$, we define the \emph{observation of the trajectory} as an interleaved combination of actions and states that represents the observation from the execution of $\pi$ in $P$. Formally, $\mathcal{O}(\tau)=\tup{s_0^o,a_1^o,s_1^o \ldots , a_l^o, s_m^o}$, $s_0^o=I$, and:

%which indicates that we observe $l$ actions, $1\leq l\leq |\pi|$, and $m$ states, $1\leq m \leq |\pi|+1$, from $\tau(\pi,P)$:

\begin{itemize}
\item The {\bf observed actions} are consistent with $\pi$, which means that $\tup{a_1^o, \ldots, a_l^o}$ is a sub-sequence of $\pi$. Specifically, the number of observed actions, $l$, can range from $0$ (fully unobservable action sequence) to $|\pi|$ (fully observable action sequence).
\item The {\bf observed states} $\tup{s_0^o, s_1^o, \ldots, s_m^o}$ is a sequence of possibly {\em partially observable states}, except for the initial state $s_0^o$, which is fully observable. A partially observable state $s_i^o$ is one in which $|s_i^o| < |F|$; i.e., a state in which at least a fluent of $F$ is not observable. Note that this definition also comprises the case $|s_i^o| = 0$, when the state is fully unobservable. Whatever the sequence of observed states of $\mathcal{O}(\tau)$ is, it must be consistent with the sequence of states of $\tau(\pi,P)$, meaning that $\forall i, s_i^o \subseteq s_i$. In practice, the number of observed states, $m$, range from 1 (the initial state, at least), to $|\pi|+1$, and the observed intermediate states will comprise a number of fluents between $[1,|F|]$.
    %Exceptionally, $s_m^o$ cannot be fully unobservable for the purpose of our task (we will elaborate on this issue later on).
\end{itemize}

We assume a bijective monotone mapping between actions/states of trajectories and observations~\cite{ramirez2009plan}, thus also granting the inverse consistency relationship (the trajectory is a superset of the observation). Therefore, transiting between two consecutive observed states in $\mathcal{O}(\tau)$ may require the execution of more than a single action ($\theta(s_i^o,\tup{a_1,\ldots,a_k})=s_{i+1}^o$, where ${\small k\geq 1}$ is unknown but finite. In other words, having $\mathcal{O}(\tau)$ does not imply knowing the actual length of $\pi$.


\subsection{Model-Based Goal Recognition}
We formalize the {\em goal recognition} task following the standard  formalization of Ram\'irez and Geffner for plan recognition~\cite{ramirez2012plan}.

The {\em model-based goal recognition} is a classification task defined by a tuple $\tup{P,\mathcal{O}}$, where:
\begin{itemize} 
\item $P=\tup{F,A,I,G[\cdot]}$ is a planning problem where $G[\cdot]$ is a set of possible goals.
\item $\mathcal{O}(\tau)$ is an observation of a trajectory $\tau(\pi,P)$ produced by the execution of an unknown plan $\pi$ that solves the planning problem $P$.
\end{itemize}

Following the {\em naive Bayes classifier}, the {\em solution} to the {\em model-based goal recognition} task is the subset of goals in $G[\cdot]$ that maximizes this expression.

\begin{align}
argmax_{g\in G[\cdot]} P(\mathcal{O}|g) P(g).
\end{align}

The {\em Plan recognition as planning} approach shows how to compute estimates of the $P(\mathcal{O}|g)$ likelyhood using an off-the-shelf classical planner. Recent works show that faster, but less accurate estimates, of this $P(\mathcal{O}|g)$ likelyhood can also be computed using relaxations of the classical planning task.



\section{Model-Based Goal Recognition with Unknown Domain Models}
\label{sec:recognition}

\subsection{Well-defined \strips\ action schemata}
\strips\ action schemata provide a compact representation for specifying classical planning models.{\em A \strips\ action schema} $\xi$ is defined by four lists: A list of {\em parameters} $pars(\xi)$, and three list of predicates (namely $pre(\xi)$, $del(\xi)$ and $add(\xi)$) that shape the kind of fluents that can appear in the {\em preconditions}, {\em negative effects} and {\em positive effects} of the actions induced from that schema.

Let be $\Psi$ the set of {\em predicates} that shape the propositional state variables $F$, and a list of {\em parameters} $pars(\xi)$. The set of elements that can appear in $pre(\xi)$, $del(\xi)$ and $add(\xi)$ of the \strips\ action schema $\xi$ is given by FOL interpretations of $\Psi$ over the parameters $pars(\xi)$ and is denoted as ${\mathcal I}_{\Psi,\xi}$. For instance, in the {\em blocksworld} the ${\mathcal I}_{\Psi,\xi}$ set contains five elements for a {\small \tt pickup($v_1$)} schemata, ${\mathcal I}_{\Psi,pickup}$={\small\tt\{handempty, holding($v_1$), clear($v_1$), ontable($v_1$), on($v_1,v_1$)\}} while it contains eleven elements for a {\small \tt stack($v_1$,$v_2$)} schemata, ${\mathcal I}_{\Psi,stack}$={\small\tt\{handempty, holding($v_1$), holding($v_2$), clear($v_1$), clear($v_2$), ontable($v_1$), ontable($v_2$), on($v_1,v_1$), on($v_1,v_2$), on($v_2,v_1$), on($v_2,v_2$)\}}. 

Despite any element of ${\mathcal I}_{\Psi,\xi}$ can {\em a priori} appear in the $pre(\xi)$, $del(\xi)$ and $add(\xi)$ of schema $\xi$, the space of possible \strips\ schemata is bounded by a set of constraints ${\mathcal C}$ of three kinds:
\begin{enumerate}
\item {\em Syntactic constraints}. \strips\ constraints require $del(\xi)\subseteq pre(\xi)$, $del(\xi)\cap add(\xi)=\emptyset$ and $pre(\xi)\cap add(\xi)=\emptyset$. Considering exclusively these syntactic constraints, the size of the space of possible \strips\ schemata is given by $2^{2\times|{\mathcal I}_{\Psi,\xi}|}$. {\em Typing constraints} are also of this kind~\cite{mcdermott1998pddl}. 
\item {\em Domain-specific constraints}. One can introduce domain-specific knowledge to constrain further the space of possible schemata. For instance, in the {\em blocksworld} one can argue that {\small\tt on($v_1$,$v_1$)} and {\small\tt on($v_2$,$v_2$)} will not appear in the $pre(\xi)$, $del(\xi)$ and $add(\xi)$ lists of an action schema $\xi$ because, in this specific domain, a block cannot be on top of itself. {\it State invariants} are also constraints of this kind~\cite{fox1998automatic}. 
\item {\em Observation constraints}. An observations $\mathcal{O}(\tau)$ depicts {\em semantic knowledge} that constraints further the space of possible action schemata.   
\end{enumerate}

\begin{definition}[Well-defined \strips\ action schemata]
Given a set of {\em predicates} $\Psi$, a list of action {\em parameters} $pars(\xi)$, and set of FOL constraints ${\mathcal C}$, $\xi$ is a {\bf well-defined \strips\ action schema} iff its three lists $pre(\xi)\subseteq {\mathcal I}_{\Psi,\xi}$, $del(\xi)\subseteq{\mathcal I}_{\Psi,\xi}$ and $add(\xi)\subseteq{\mathcal I}_{\Psi,\xi}$ only contain elements in ${\mathcal I}_{\Psi,\xi}$ and they satisfy all the constraints in ${\mathcal C}$.
\end{definition}

We say a planning model $\mathcal{M}$ is {\em well-defined} if all its \strips\ action schemata are {\em well-defined}.




\subsection{Edit distances for \strips\ planning models}
First, we define the two edit \emph{operations} on a schema $\xi$ that belongs to a \strips\ model $\mathcal{M}\in M$:

\begin{itemize}
\item {\em Deletion}. Given $\xi\in\mathcal{M}$, an element from any of the lists $pre(\xi)$/$del(\xi)$/$add(\xi)$ is removed such that the result is a {\em well-defined} \strips\ action schema.
\item {\em Insertion}. Given $\xi\in\mathcal{M}$, an element in ${\mathcal I}_{\Psi,\xi}$ is added to any of the lists $pre(\xi)$/$del(\xi)$/$add(\xi)$ such that the result is a {\em well-defined} action schema.
\end{itemize}

Second, let us define when to action models are comparable.  For instance, we claim that the {\small\tt stack(?v1,?v2)} and {\small\tt unstack(?v1,?v2)} actions schemata from a four operator {\em blocksworld}~\cite{slaney2001blocks} are comparable while, the {\small\tt stack(?v1,?v2)} and {\small\tt pick-up(?v1)} schemata are not. Last but not least, we say that two \strips\ models $\mathcal{M}$ and $\mathcal{M}'$ are {\em comparable} iff there exists a bijective function $\mathcal{M} \mapsto \mathcal{M}^*$ that maps every action schema $\xi\in\mathcal{M}$ to a comparable schemata $\xi'\in\mathcal{M'}$ and vice versa.

\begin{definition}[Comparable \strips\ action schemata]
Two \strips\ schemata $\xi$ and $\xi'$ are {\bf comparable} iff $pars(\xi)=pars(\xi')$, i.e, both share the same list of parameters.\footnote{In \strips\ models, $pars(\xi)=pars(\xi')$ implies the number of parameters must be the same. For other planning models that allow object typing, the equality implies that parameters share the same type}
\end{definition}

We are now ready to formalize an {\em edit distance} that quantifies how similar two given \strips\ models are. The distance is symmetric and meets the {\em metric axioms} provided that the two edit operations, {\em deletion} and {\em insertion}, have the same positive cost.

\begin{definition}[Edit distance]
  Let $\mathcal{M}$ and $\mathcal{M}'$ be two {\em comparable} and {\em well-defined} \strips\ planning models within the same set of predicates $\Psi$. The {\bf edit distance} $\delta(\mathcal{M},\mathcal{M}')$ is the minimum number of {\em edit operations} that is required to transform $\mathcal{M}$ into $\mathcal{M}'$.
\end{definition}

Since ${\mathcal I}_{\Psi,\xi}$ is a bounded set, the maximum number of edits that can be introduced to an action schema is bounded as well. The \textbf{maximum edit distance} of a \strips\ model $\mathcal{M}$ built with predicates $\Psi$ is $\delta(\mathcal{M},*)=\sum_{\xi\in\mathcal{M}} 3\times|{\mathcal I}_{\Psi,\xi}|$ (note that if we consider the set of syntactic constraints then $\delta(\mathcal{M},*)=\sum_{\xi\in\mathcal{M}} 2\times|{\mathcal I}_{\Psi,\xi}|$).

\vspace{0.02cm}

An observation of the execution of a plan generated with $\mathcal{M}$ further constraints the space of possible action schemata of $\mathcal{M}$. The \emph{semantic knowledge} included in the observations introduce a third type of constraints, that we will call {\em observation constraints}, and that can be added to the set $\mathcal{C}$. In addition, {\em observation constraints} allow us to define an edit distance to elicit the value of $P(\mathcal{O}|\mathcal{M})$. It can be argued that the shorter this distance the better the given model explains the given observation.

\begin{definition}[Observation edit distance]
Given a planning problem $P$, an observation $\mathcal{O}(\tau)$ of the execution of a plan that solves $P$ and a \strips\ planning model $\mathcal{M}$ (all defined within the same set of predicates $\Psi$). The {\bf observation edit distance}, $\delta^o(\mathcal{M},\mathcal{O})$, is the minimal edit distance from $\mathcal{M}$ to any {\em comparable} and well-defined model $\mathcal{M}'$ s.t. $\mathcal{M}'$ produces a trajectory $\tau(\pi,P)$ that reaches the goals in $P$ and is {\em consistent} with $\mathcal{O}(\tau)$; \[\delta^o(\mathcal{M},\mathcal{O})=\min_{\forall \mathcal{M}' \rightarrow \mathcal{O}} \delta(\mathcal{M},\mathcal{M}')\]
\end{definition}

$\delta^o(\mathcal{M},\mathcal{O})$ can also be defined through the edition that the observation $\mathcal{O}(\tau)$ requires to fit $\mathcal{M}$. This implies defining {\em edit operations} that modify the observation $\mathcal{O}(\tau)$ instead of the model $\mathcal{M}$~\cite{yang2007learning,SohrabiRU16}. Our definition of {\em observation edit distance} is more practical since the size of ${\mathcal I}_{\Psi,\xi}$ is usually much smaller than $F$ (the number of variables in the action schemata should normally be lower than the number of objects in a planning problem).

\begin{definition}[{\em Closest consistent models}] \label{consistent}
Given a model $\mathcal{M}$, the set $M^*$ of the {\bf closest consistent models} is the set of models $\mathcal{M}'$ that: (1) produce a trajectory $\tau(\pi,P)$ that reaches the goals in $P$ and is {\em consistent} with $\mathcal{O}(\tau)$ and (2) their {\em edit distance} to $\mathcal{M}$ is minimal;
  \[\underset{\forall \mathcal{M}' \rightarrow \mathcal{O}}{\arg\min}\ \delta(\mathcal{M},\mathcal{M}') \]
\end{definition}

\section{Evaluation}
\label{sec:evaluation}



\section{Conclusions}
\label{sec:conclusions}


%% The file named.bst is a bibliography style file for BibTeX 0.99c
\bibliographystyle{named}
\bibliography{planlearnbibliography}

\end{document}

